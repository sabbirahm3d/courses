%%%%%%%%%%%%%%%%%%%%%%%%%%%%%%%%%%%%%%%%%
% Template
% LaTeX Template
% Version 1.0 (December 8 2014)
%
% This template has been downloaded from:
% http://www.LaTeXTemplates.com
%
% Original author:
% Brandon Fryslie
% With extensive modifications by:
% Vel (vel@latextemplates.com)
%
% License:
% CC BY-NC-SA 3.0 (http://creativecommons.org/licenses/by-nc-sa/3.0/)
%
% Authors:
% Sabbir Ahmed
%
%%%%%%%%%%%%%%%%%%%%%%%%%%%%%%%%%%%%%%%%%

\documentclass[paper=usletter, fontsize=12pt]{article}
%%%%%%%%%%%%%%%%%%%%%%%%%%%%%%%%%%%%%%%%%
% Contract Structural Definitions File Version 1.0 (December 8 2014)
%
% Created by: Vel (vel@latextemplates.com)
% 
% This file has been downloaded from: http://www.LaTeXTemplates.com
%
% License: CC BY-NC-SA 3.0 (http://creativecommons.org/licenses/by-nc-sa/3.0/)
%
%%%%%%%%%%%%%%%%%%%%%%%%%%%%%%%%%%%%%%%%%

\usepackage{geometry} % Required to modify the page layout
\usepackage{multicol}
\usepackage{amsmath}
\usepackage{amssymb}

\usepackage[pdftex]{graphicx}
\usepackage{wrapfig}
\usepackage[font=scriptsize, labelfont=bf]{caption}
\usepackage[utf8]{inputenc} % Required for including letters with accents
\usepackage[T1]{fontenc} % Use 8-bit encoding that has 256 glyphs

\usepackage{avant} % Use the Avantgarde font for headings
\usepackage{xparse}
\usepackage{xcolor}
\usepackage{listings}  % for code verbatim and console outputs

\setlength{\textwidth}{16cm} % Width of the text on the page
\setlength{\textheight}{23cm} % Height of the text on the page
\setlength{\oddsidemargin}{0cm} % Width of the margin - negative to move text left, positive to move it right
\setlength{\topmargin}{-1.25cm} % Reduce the top margin

\setlength{\parindent}{0mm} % Don't indent paragraphs
\setlength{\parskip}{2.5mm} % Whitespace between paragraphs
\renewcommand{\baselinestretch}{1.2}

\renewcommand\familydefault{\sfdefault}  % default font for entire document

\definecolor{green}{rgb}{0.18, 0.55, 0.34}

\graphicspath{ {figures/} }
\captionsetup[table]{skip=10pt}

\lstset{language=C, keywordstyle={\bfseries \color{black}}}

% defines algorithm counter for chapter-level
\newcounter{nalg}[section]

%defines appearance of the algorithm counter
\renewcommand{\thenalg}{\thesection .\arabic{nalg}}

% defines a new caption label as Algorithm x.y
\DeclareCaptionLabelFormat{algocaption}{Algorithm \thenalg}

%defines the algorithm listing environment
\lstnewenvironment{pseudocode}[1][] {
    \refstepcounter{nalg} %increments algorithm number

    \captionsetup{labelformat=algocaption,labelsep=colon}
    \lstset{
        mathescape=true,
        frame=tB,
        numbers=left,
        numberstyle=\tiny,
        basicstyle=\scriptsize,
        keywordstyle=\color{black}\bfseries\em,
        keywords={,input, output, return, datatype, function, in, if, else, foreach, while, begin, end, },
        xleftmargin=.04\textwidth,
        #1
    }
}{}
 % specifies the document layout and style
\allowdisplaybreaks

%------------------------------------------------------------------------------
% document info command
\newcommand{\documentinfo}[5]{
    \begin{centering}
        \parbox{2in}{
        \begin{spacing}{1}
            \begin{flushleft}
                \begin{tabular}{l l}
                    #1 \\
                    #2 \\
                    #3 \\
                \end{tabular}\\
                \rule{\textwidth}{1pt}
            \end{flushleft}
        \end{spacing}
        }
    \end{centering}
}

\newcommand{\ans}{\textbf{Answer} \ }

\begin{document}

    \documentinfo{Sabbir Ahmed}{\textbf{DATE:} \today}{\textbf{CMPE 320} HW 04}
    \vspace{-0.2in}

    \begin{enumerate}

        % 1
        \item
        \begin{proof}[\unskip\nopunct]
            Given,\\
            Probability of not losing the first game: $p_1=0.4$\\
            Probability of losing the first game: $p_1^c=1-0.4=0.6$\\
            Probability of not losing the second game: $p_2=0.7$\\
            Probability of losing the second game: $p_2^c=1-0.7=0.3$\\

            Therefore, the $pmf(X)$ where $X=0,1,2,4$ represents the number of
            points earned over the weekend:
            \begin{align*}
                P(X=0) & = p_1^c \cdot p_2^c\\
                & = 0.6 \cdot 0.3\\
                & = 0.18
            \end{align*}
            \begingroup
            \addtolength{\jot}{1em}
            \begin{align*}
                P(X=1) & = \frac{p_1^c \cdot p_2}{2} + \frac{p_1 \cdot
                p_2^c}{2}\\
                & = \frac{0.6 \cdot 0.7}{2} + \frac{0.4 \cdot 0.3}{2}\\
                & = 0.27
            \end{align*}
            \endgroup
        \end{proof}
        \vspace{0.2in}

        % 2
        \item
        \begin{proof}[\unskip\nopunct]
            Given $p = 1/649640$.
            Therefore,
            \begingroup
            \addtolength{\jot}{1em}
            \begin{align*}
                P(X \ge 1) & = 1 - P(X = 0)\\
                & = 1 - \Bigg(\frac{649640-1}{649640}\Bigg)^{649640}\\
                & = 1 - \Bigg(1 - \frac{1}{649640}\Bigg)^{649640}\\
            \end{align*}
            If $n = 649640$
            \begin{align*}
                P(X \ge 1) & = 1 - \Bigg(1 - \frac{1}{n}\Bigg)^{n}\\
                & = 1 - \lim_{n \rightarrow \infty}\Bigg(1 -
                \frac{1}{n}\Bigg)^{n}\\
                & = 1 - \frac{1}{e} \qedhere
            \end{align*}
            \endgroup
        \end{proof}
        \vspace{0.2in}

        % 3
        \item
        \begin{proof}[\unskip\nopunct]
            A claim is first filed with the probability\\
            $pq = (0.05)(1 - 0.05)^{n-1}=(0.05)(0.095)^{n-1}$
        \end{proof}
        \vspace{0.2in}

        % 4
        \item
        \begin{proof}[\unskip\nopunct]

            \begin{enumerate}

                \item $Y = X \ (mod \ 3)$\\
                \begin{align*}
                    P(Y = 0) & = P(X = \{0,3,6,9\})\\
                    & = \frac{4}{10}\\
                    & = 0.4
                \end{align*}
                \begin{align*}
                    P(Y = 0) & = P(X = \{0,3,6,9\})\\
                    & = \frac{4}{10}\\
                    & = 0.4
                \end{align*}
                \begin{align*}
                    P(Y = 1) & = P(X = \{1,4,7\})\\
                    & = \frac{3}{10}\\
                    & = 0.3
                \end{align*}
                \begin{align*}
                    P(Y = 2) & = P(X = \{2,5,8\})\\
                    & = \frac{3}{10}\\
                    & = 0.3
                \end{align*}

                \item $Y = 5 \ (mod \ X + 1)$\\
                \begin{align*}
                    P(Y = 0) & = P(X = \{0,4\})\\
                    & = \frac{2}{10}\\
                    & = 0.2
                \end{align*}
                \begin{align*}
                    P(Y = 1) & = P(X = \{1,5\})\\
                    & = \frac{2}{10}\\
                    & = 0.2
                \end{align*}
                \begin{align*}
                    P(Y = 2) & = P(X = \{2\})\\
                    & = \frac{1}{10}\\
                    & = 0.1
                \end{align*}
                \begin{align*}
                    P(Y = 5) & = P(X = \{5,6,7,8,9\})\\
                    & = \frac{5}{10}\\
                    & = 0.5 \qedhere
                \end{align*}

            \end{enumerate}

        \end{proof}
        \vspace{0.2in}

        % 5
        \item
        \begin{proof}[\unskip\nopunct]
            Since $X$ is uniformly distributed over $[a, b]$,
            \begin{align*}
                p_X(k) & =
                \begin{cases}
                    \frac{1}{b-a+1}, & \text{if}\ k \in [a,b],\\
                    0, & \text{otherwise}
                \end{cases}
            \end{align*}
            and
            \begin{align*}
                max\{0,X\} & =
                \begin{cases}
                    X, & \text{if}\ X > 0\\
                    0, & \text{if}\ X \le 0
                \end{cases}
            \end{align*}
            Then,
            \begingroup
            \addtolength{\jot}{1em}
            \begin{align*}
                P(max\{0,X\} = 0) & = P(X \le 0)\\
                & = \frac{|a| + 1}{b - a + 1}
            \end{align*}
            \endgroup
            Similarly, for $min\{0,X\}$
            \begingroup
            \addtolength{\jot}{1em}
            \begin{align*}
                P(min\{0,X\} = 0) & = P(X \ge 0)\\
                & = \frac{b + 1}{b - a + 1}
            \end{align*}
            \endgroup
            For $k > 0$,
            \begingroup
            \addtolength{\jot}{1em}
            \begin{align*}
                P(max\{0,X\} = k) & = P(max\{0,X\} = k)\\
                & = P(X = k)\\
                & = \frac{1}{b - a + 1} \qedhere
            \end{align*}
            \endgroup
        \end{proof}
        \vspace{0.2in}

        % 6
        \item
        \begin{proof}[\unskip\nopunct]
        \end{proof}
        \vspace{0.2in}

        % 7
        \item
        \begin{proof}[\unskip\nopunct]
            Since $P_x(X) = sin(X\pi) = 0$ for $X \in \mathbb{Z}$:
            \begin{align*}
                E[sin(X\pi)] & = \sum_{k \in \mathbb{Z}} kP_x(k)\\
                & = 0
            \end{align*}
            Since $P_x(X) = cos(X\pi) = 1$ for $X \in \mathbb{Z}$:
            \begin{align*}
                E[cos(X\pi)] & = \sum_{k \in \mathbb{Z}} kP_x(k)\\
                & = 1 \qedhere
            \end{align*}
        \end{proof}
        \vspace{0.2in}

        % 8
        \item
        \begin{proof}[\unskip\nopunct]
            \begin{enumerate}

                \item Since the event where Fischer wins is independent, and a
                win is determined by a win in the $(n+1)$th until $n$ ties:
                \begin{equation*}
                    \sum_{n \ge 0} (1-p-q)^{n-1}(p) = \frac{p}{p+q}
                \end{equation*}

                \item The PMF of the geometric probability
                \begin{equation*}
                    p_X(k) = (1-p-q)^{k-1}(p+q) \text{, for } k \ge 0
                \end{equation*}
                The mean of the geometric probability
                \begin{equation*}
                    E[p_X] = \frac{1}{p+q}
                \end{equation*}
                The variance of the geometric probability
                \begin{equation*}
                    var[p_X] = \frac{1-(p+q)}{(p+q)^2} \qedhere
                \end{equation*}

            \end{enumerate}
        \end{proof}
        \vspace{0.2in}

        % 9
        \item
        \begin{proof}[\unskip\nopunct]
        \end{proof}
        \vspace{0.2in}

        % 10
        \item
        \begin{proof}[\unskip\nopunct]
        \end{proof}
        \vspace{0.2in}

    \end{enumerate}

\end{document}
