%%%%%%%%%%%%%%%%%%%%%%%%%%%%%%%%%%%%%%%%%%%%%%%%%%%%%%%%%%%%%%%%%%%%%%%%%%%%%%
% Template LaTeX Template Version 1.0 (December 8 2014)
%
% This template has been downloaded from: http://www.LaTeXTemplates.com
%
% Original author: Brandon Fryslie With extensive modifications by: Vel
% (vel@latextemplates.com)
%
% License: CC BY-NC-SA 3.0 (http://creativecommons.org/licenses/by-nc-sa/3.0/)
%
% Authors: Sabbir Ahmed
% 
%%%%%%%%%%%%%%%%%%%%%%%%%%%%%%%%%%%%%%%%%%%%%%%%%%%%%%%%%%%%%%%%%%%%%%%%%%%%%%

\documentclass[10pt]{extarticle}
%%%%%%%%%%%%%%%%%%%%%%%%%%%%%%%%%%%%%%%%%
% Contract Structural Definitions File Version 1.0 (December 8 2014)
%
% Created by: Vel (vel@latextemplates.com)
% 
% This file has been downloaded from: http://www.LaTeXTemplates.com
%
% License: CC BY-NC-SA 3.0 (http://creativecommons.org/licenses/by-nc-sa/3.0/)
%
%%%%%%%%%%%%%%%%%%%%%%%%%%%%%%%%%%%%%%%%%

\usepackage{geometry} % Required to modify the page layout
\usepackage{multicol}
\usepackage{amsmath}
\usepackage{amssymb}

\usepackage[pdftex]{graphicx}
\usepackage{wrapfig}
\usepackage[font=scriptsize, labelfont=bf]{caption}
\usepackage[utf8]{inputenc} % Required for including letters with accents
\usepackage[T1]{fontenc} % Use 8-bit encoding that has 256 glyphs

\usepackage{avant} % Use the Avantgarde font for headings
\usepackage{courier}
\usepackage{xparse}
\usepackage{xcolor}
\usepackage{listings}  % for code verbatim and console outputs

\setlength{\textwidth}{16cm} % Width of the text on the page
\setlength{\textheight}{23cm} % Height of the text on the page
\setlength{\oddsidemargin}{0cm} % Width of the margin - negative to move text left, positive to move it right
\setlength{\topmargin}{-1.25cm} % Reduce the top margin

\setlength{\parindent}{0mm} % Don't indent paragraphs
\setlength{\parskip}{2.5mm} % Whitespace between paragraphs
\renewcommand{\baselinestretch}{1.5}

\definecolor{green}{rgb}{0.18, 0.55, 0.34}

\graphicspath{ {figures/} }
\captionsetup[table]{skip=10pt}

\lstset{language=C, keywordstyle={\bfseries \color{black}}}

% defines algorithm counter for chapter-level
\newcounter{nalg}[section]

%defines appearance of the algorithm counter
\renewcommand{\thenalg}{\thesection .\arabic{nalg}}

% defines a new caption label as Algorithm x.y
\DeclareCaptionLabelFormat{algocaption}{Algorithm \thenalg}

% defines the algorithm listing environment
\lstnewenvironment{pseudocode}[1][] {
    \refstepcounter{nalg}  % increments algorithm number
    \captionsetup{font=normalsize, labelformat=algocaption, labelsep=colon}
    \lstset{
        breaklines=true,
        mathescape=true,
        numbers=left,
        numberstyle=\scriptsize,
        basicstyle=\footnotesize\ttfamily,
        keywordstyle=\color{black}\bfseries,
        keywords={input, output, return, parallel, function, for, to, in, if,
        else, foreach, while, and, or, new, print},
        xleftmargin=.04\textwidth,
        #1
    }
}{}

\renewcommand{\familydefault}{\sfdefault}  % default font for entire document
 % specifies the document layout and style

% document info command
\newcommand{\documentinfo}[5]{
    \begin{centering}
        \parbox{2in}{
        \begin{spacing}{1}
            \begin{flushleft}
                \begin{tabular}{l l} \textbf{#1} \\ \textbf{#2} \\ #3 \\
                \end{tabular}\\
                \rule{\textwidth}{1pt}
            \end{flushleft}
        \end{spacing} }
    \end{centering} }

\begin{document}

    \documentinfo{Sabbir Ahmed}
    {CMSC 411 - HW 05}
    {\textbf{DATE:} \today}
    \vspace{-0.3in}

    \begin{enumerate}

        \item The table below lists parameters for different direct-mapped cache designs.

        \begin{table}[h]
            \centering
            \begin{tabular*}{200pt}{@{\extracolsep{\fill}} ccc}
                    & \textbf{Cache Data Size}  & \textbf{Cache Block Size} \\
                \hline
                i)  & 64 kB                     & 1 word                    \\
                ii) & 64 kB                     & 2 words                   \\
            \end{tabular*}
        \end{table}

        \begin{enumerate}

            \item \textbf{Question} Calculate the total number of bits required
            for the cache listed in the table, assuming a 32-bit address.

            \textbf{Answer}

            Using: 
                \[ 2^{n} \times (\text{valid field size} + \text{tag
                size} + \text{block size}) \]
            where
            \begin{equation*}
                \begin{split}
                    \text{tag size} & = 32 - (n + m + 2) \\
                    n & = \log_2(\text{cache data size}) \\
                    m & = \log_2(\text{block size}) \\
                    \text{valid bits} & = 1 \\
                \end{split}
            \end{equation*}

            Total number of bits required for design i):
            \begin{equation*}
                \begin{split}
                    n & = \log_2(\text{64 kB}) = 16 \\
                    m & = \log_2(\text{1}) = 0 \\
                    \text{tag size} & = 32 - (16 + 0 + 2) = 12 \\
                    \text{valid bits} & = 1 \\
                    \text{total bits} & = 2^{16} \times (1 + 12 + 2) \\
                    & = 15 \times 2^{16} \text{ tag bits} \\
                \end{split}
            \end{equation*}

            Total number of bits required for design ii):
            \begin{equation*}
                \begin{split}
                    n & = \log_2(\text{64 kB}) = 16 \\
                    m & = \log_2(\text{2}) = 1 \\
                    \text{tag size} & = 32 - (16 + 1 + 2) = 13 \\
                    \text{valid bits} & = 1 \\
                    \text{total bits} & = 2^{16} \times (1 + 13 + 2) \\
                    & = 2^{20} \text{ tag bits} \\
                \end{split}
            \end{equation*}

            \item \textbf{Question} What is the total number of bits if the
            cache is organized as a 4-way associative with one word blocks?

            \textbf{Answer}

            Since there are 16 ($2^1$) bytes per block, a 32-bit address yields
            32 - 1 = 31 bits to be used for index and tag. The number of sets:
            \begin{equation*}
                \begin{split}
                    \text{number of sets} & = \log_2(\text{64 kB}) = 16 \\
                \end{split}
            \end{equation*}

            Each degree of associativity decreases the number of sets by a
            factor of 2 and thus decreases the number of bits used to index the
            cache by 1 and increases the number of bits in the tag by 1.
            \begin{equation*}
                \begin{split}
                    \frac{64 \text{ kB}}{4} & = 16 \text{ kB} \\
                \end{split}
            \end{equation*}

            \begin{equation*}
                \begin{split}
                    \text{total bits} & = (31 - 10) \times 1 \times 16 \text{ kB} \\
                    & = \text{21} \times \text{16 K tag bits}
                \end{split}
            \end{equation*}

        \newpage
        \end{enumerate}

            \item For a pipeline with a perfect CPI = 1 if no memory-access
            related stall, consider the following program and cache behaviors.

            \begin{table}[h]
                \centering
                \begin{tabular}{p{2cm}p{2cm}p{2cm}p{2cm}p{1.5cm}}
                    \textbf{Data Reads Per 1000 Instructions} & \textbf{Data
                    Writes Per 1000 Instructions} & \textbf{Instruction Cache
                    Miss Rate} & \textbf{Data Cache Miss Rate} & \textbf{Block
                    Size (Byte)} \\
                    \hline
                    200 & 160 & 0.20\% & 2\% & 8 \\
                \end{tabular}
            \end{table}

        \begin{enumerate}

            \item \textbf{Question} For a write-through, write-allocate cache
            with sufficiently large write buffer (i.e., no buffer caused
            stalls), what’s the minimum read and write bandwidths (measured by
            byte-per-cycle) needed to achieve a CPI of 2?

            \textbf{Answer} \\ Let $I$ be the number of instructions and $W$ be
            the read/write bandwidth.

            \begin{equation*}
                \begin{split}
                    \text{Data cache read miss penalty} &
                    = I \times \frac{200}{1000} \times \frac{2}{100} \times
                    \bigg(\frac{8}{W} + 1\bigg) \\
                    & = 0.004I \times \bigg(\frac{8}{W} + 1\bigg) \\
                \end{split}
            \end{equation*}

            \begin{equation*}
                \begin{split}
                    \text{Data cache write miss penalty} &
                    = I \times \frac{160}{1000} \times \frac{2}{100} \times
                    \bigg(\frac{8}{W} + 1\bigg) \\
                    & = 0.0032I \times \bigg(\frac{8}{W} + 1\bigg) \\
                \end{split}
            \end{equation*}

            \begin{equation*}
                \begin{split}
                    \text{Instruction cache read miss penalty} &
                    = I \times \frac{0.20}{100} \times \bigg(\frac{8}{W} +
                    1\bigg) \\
                    & = 0.002I \times \bigg(\frac{8}{W} + 1\bigg) \\
                \end{split}
            \end{equation*}

            For CPI = 2,
            \begin{equation*}
                \begin{split}
                    I \times 2 & = \text{Hit time + miss penalty} \\
                    & = I + I \times (0.004 + 0.0032 + 0.002) \times
                    \bigg(\frac{8}{W} + 1\bigg) \\
                    & = 0.0092I \times \bigg(\frac{8}{W} + 1\bigg) \\
                   \Rightarrow I \times 2 & = 0.0092I \times \bigg(\frac{8}{W}
                   + 1\bigg) \\
                   2 & = 0.0092 \times \bigg(\frac{8}{W} + 1\bigg) \\
                   \therefore W & \approx 0.037 \text{ byte per cycle} \\
                \end{split}
            \end{equation*}

            \item \textbf{Question} For a write-back, write-allocate cache,
            assuming 30\% of replaced data cache blocks are dirty, what’s the
            minimal read and write bandwidths needed for a CPI of 2?

            \textbf{Answer} \\ Let $I$ be the number of instructions and $W$ be
            the read/write bandwidth.

            \begin{equation*}
                \begin{split}
                    \text{Data cache read miss penalty} &
                    = I \times \frac{200}{1000} \times \frac{2}{100} \times
                    \bigg( 1+\frac{30}{100} \bigg) \times \bigg( \frac{8}{W} +
                    1 \bigg) \\
                    & = 0.0052I \times \bigg(\frac{8}{W} + 1\bigg) \\
                \end{split}
            \end{equation*}

            \begin{equation*}
                \begin{split}
                    \text{Data cache write miss penalty} &
                    = I \times \frac{160}{1000} \times \frac{2}{100} \times
                    \bigg( 1+\frac{30}{100} \bigg) \times \bigg( \frac{8}{W} +
                    1 \bigg) \\
                    & = 0.00416I \times \bigg(\frac{8}{W} + 1\bigg) \\
                \end{split}
            \end{equation*}

            \begin{equation*}
                \begin{split}
                    \text{Instruction cache read miss penalty} &
                    = I \times \frac{0.20}{100} \times \bigg(\frac{8}{W} +
                    1\bigg) \\
                    & = 0.002I \times \bigg(\frac{8}{W} + 1\bigg) \\
                \end{split}
            \end{equation*}

            For CPI = 2,
            \begin{equation*}
                \begin{split}
                    I \times 2 & = \text{Hit time + miss penalty} \\
                    & = I + I \times (0.0052 + 0.00416 + 0.002) \times
                    \bigg(\frac{8}{W} + 1\bigg) \\
                    & = 0.01136I \times \bigg(\frac{8}{W} + 1\bigg) \\
                   \Rightarrow I \times 2 & = 0.01136I \times \bigg(\frac{8}{W}
                   + 1\bigg) \\
                   2 & = 0.01136 \times \bigg(\frac{8}{W} + 1\bigg) \\
                   \therefore W & \approx 0.046 \text{ byte per cycle} \\
                \end{split}
            \end{equation*}

        \end{enumerate}

        \newpage
        \item Using the sequences of 32-bit memory read references, given as
        word addresses in the following table:

        \begin{table}[h]
            \centering
            \begin{tabular*}{300pt}{@{\extracolsep{\fill}}
            |c|c|c|c|c|c|c|c|c|c|c|c|}
            \hline
            6 & 214 & 175 & 214 & 6 & 84 & 65 & 174 & 64 & 105 & 85 & 215 \\
            \hline
            \end{tabular*}
        \end{table}

        For each of these read accesses, identify the binary address, the tag,
        the index, and whether it experiences a hit or a miss, for each of the
        following cache configurations. Assume the cache is initially empty.

        \begin{enumerate}

            \item \textbf{Question} A direct-mapped cache with 16 one-word
            blocks.

            \textbf{Answer}

            $\because$ 8 blocks, 1 word per block, 4 bytes per word,

            $\therefore$ log$_{2}$(8) = 3-bit offset, 3-bit index and (16-3-3)
            = 10 bits tag

            \begin{table}[h]
                \centering
                \caption{A Direct-mapped Cache With 16 One-word Blocks}
                \begin{tabular*}{300pt}{@{\extracolsep{\fill}} ccccc}
                    \textbf{Memory} & \textbf{Binary} & \textbf{Tag} &
                    \textbf{Index} & \textbf{Hit / Miss} \\
                    \hline
                    6   & 0000000000000110 & 0000000000 & 000 & miss  \\
                    214 & 0000000011010110 & 0000000011 & 010 & miss  \\
                    175 & 0000000010101111 & 0000000010 & 101 & miss  \\
                    214 & 0000000011010110 & 0000000011 & 010 & hit   \\
                    6   & 0000000000000110 & 0000000000 & 000 & hit   \\
                    84  & 0000000001010100 & 0000000001 & 010 & miss  \\
                    65  & 0000000001000001 & 0000000001 & 000 & miss  \\
                    174 & 0000000010101110 & 0000000010 & 101 & miss  \\
                    64  & 0000000001000000 & 0000000001 & 000 & miss  \\
                    105 & 0000000001101001 & 0000000001 & 101 & miss  \\
                    85  & 0000000001010101 & 0000000001 & 010 & miss  \\
                    215 & 0000000011010111 & 0000000011 & 010 & miss  \\
                \end{tabular*}
            \end{table}
            \newpage

            \item \textbf{Question} A direct-mapped cache with two-word blocks
            and a total size of 8 blocks.

            \textbf{Answer}

            $\because$ 8 blocks, 2 words per block, 4 bytes per word,

            $\therefore$ log$_{2}$(8) = 3-bit offset, 3-bit index and (16-3-3)
            = 10 bits tag

            \begin{table}[h]
                \centering
                \caption{A Direct-mapped Cache With Two-word Blocks And A Total
                Size Of 8 Blocks}
                \begin{tabular*}{300pt}{@{\extracolsep{\fill}} ccccc}
                    \textbf{Memory} & \textbf{Binary} & \textbf{Tag} &
                    \textbf{Index} & \textbf{Hit / Miss} \\
                    \hline
                    6   & 0000000000000110 & 0000000000 & 000 & miss  \\
                    214 & 0000000011010110 & 0000000011 & 010 & miss  \\
                    175 & 0000000010101111 & 0000000010 & 101 & miss  \\
                    214 & 0000000011010110 & 0000000011 & 010 & hit   \\
                    6   & 0000000000000110 & 0000000000 & 000 & hit   \\
                    84  & 0000000001010100 & 0000000001 & 010 & miss  \\
                    65  & 0000000001000001 & 0000000001 & 000 & miss  \\
                    174 & 0000000010101110 & 0000000010 & 101 & hit   \\
                    64  & 0000000001000000 & 0000000001 & 000 & hit   \\
                    105 & 0000000001101001 & 0000000001 & 101 & miss  \\
                    85  & 0000000001010101 & 0000000001 & 010 & hit   \\
                    215 & 0000000011010111 & 0000000011 & 010 & miss  \\
                \end{tabular*}
            \end{table}
            \newpage

            \item \textbf{Question} A fully associative cache with two-word
            blocks and a total size of 8 words. Use LRU replacement.

            \textbf{Answer}

            $\because$ 8 blocks, 2 words per block, 4 tags

            $\therefore$ log$_{2}$(8) = 3-bit offset, 0-bit index and (16-0-3)
            = 13 bits tag

            \begin{table}[h]
                \centering
                \caption{A Fully Associative Cache With Two-word Blocks And A
                Total Size Of 8 Words With LRU Replacement. The Integers Next to The Tags Indicate the Age of the Content}
                \begin{tabular*}{400pt}{@{\extracolsep{\fill}} cp{2cm}ccccccccc}
                    \textbf{Memory} & \textbf{00000000 + Binary} & \textbf{Tag 1} &&
                    \textbf{Tag 2} && \textbf{Tag 3} && \textbf{Tag 4} &&  \textbf{Hit / Miss} \\
                    \hline

    6   & 00000110 & 00000 & 1 &       &   &       &   &       &   & miss \\
    214 & 11010110 & 00000 & 2 & 11010 & 1 &       &   &       &   & miss \\
    175 & 10101111 & 00000 & 3 & 11010 & 2 & 10101 & 1 &       &   & miss \\
    214 & 11010110 & 00000 & 4 & 11010 & 1 & 10101 & 2 &       &   & hit  \\
    6   & 00000110 & 00000 & 1 & 11010 & 2 & 10101 & 3 &       &   & hit  \\
    84  & 01010100 & 00000 & 2 & 11010 & 3 & 10101 & 4 & 01010 & 1 & miss \\
    65  & 01000001 & 00000 & 3 & 11010 & 4 & 01000 & 1 & 01010 & 2 & miss \\
    174 & 10101110 & 00000 & 4 & 11010 & 5 & 10101 & 1 & 01010 & 3 & miss \\
    64  & 01000000 & 00000 & 5 & 11010 & 6 & 01000 & 1 & 01010 & 4 & miss \\
    105 & 01101001 & 00000 & 6 & 01101 & 1 & 01000 & 2 & 01010 & 5 & miss \\
    85  & 01010101 & 00000 & 7 & 01101 & 2 & 01000 & 3 & 01010 & 1 & hit \\
    215 & 11010111 & 11010 & 1 & 01101 & 3 & 01000 & 4 & 01010 & 2 & miss \\

                \end{tabular*}
            \end{table}
            \newpage

            \item \textbf{Question} A 2-way set associative cache with one-word
            block size and total size of 8 words, while applying LRU
            replacement policy.

            \textbf{Answer}

            $\because$ 8 blocks, 1 word per block, 4 bytes per word, 2 way-
            set,

            $\because \text{ The number of sets} =
            \frac{\text{8 words}}{\text{2 blocks per set}} \text{ = 4 sets}$

            $\therefore$ log$_{2}$(8) = 3-bit offset, 2-bit index and (16-2-3)
            = 11 bits tag

            \begin{table}[h]
                \centering
                \caption{A 2-way Set Associative Cache With One-word Block Size
                And Total Size Of 8 Words With LRU Replacement Policy}
                \begin{tabular*}{300pt}{@{\extracolsep{\fill}} ccccc}
                    \textbf{Memory} & \textbf{Binary} & \textbf{Tag} &
                    \textbf{Index} & \textbf{Hit / Miss} \\
                    \hline
                    6   & 0000000000000110 & 0000000000 & 00 & miss  \\
                    214 & 0000000011010110 & 0000000011 & 10 & miss  \\
                    175 & 0000000010101111 & 0000000010 & 01 & miss  \\
                    214 & 0000000011010110 & 0000000011 & 10 & hit   \\
                    6   & 0000000000000110 & 0000000000 & 00 & hit   \\
                    84  & 0000000001010100 & 0000000001 & 10 & miss  \\
                    65  & 0000000001000001 & 0000000001 & 00 & miss  \\
                    174 & 0000000010101110 & 0000000010 & 01 & miss  \\
                    64  & 0000000001000000 & 0000000001 & 00 & miss  \\
                    105 & 0000000001101001 & 0000000001 & 01 & miss  \\
                    85  & 0000000001010101 & 0000000001 & 10 & miss  \\
                    215 & 0000000011010111 & 0000000011 & 10 & miss  \\
                \end{tabular*}
            \end{table}

            \begin{table}[h]
                \centering
                \caption{Contents of the Cache After Each Addition}
                \begin{tabular*}{200pt}{@{\extracolsep{\fill}}cccc}
                00 & 01 & 10 & 11 \\
                \hline
                6 & & &  \\
                6 & & 214 &  \\
                6 & 175 & 214 &  \\
                6 & 175 & 214, 84 &  \\
                6, 65 & 175 & 214, 84 &  \\
                65, 64 & 174 & 214, 84 &  \\
                65, 64 & 174, 105 & 214, 84 &  \\
                65, 64 & 174, 105 & 84, 85 &  \\
                65, 64 & 174, 105 & 85, 215 &  \\
                \end{tabular*}
            \end{table}

            \newpage

        \end{enumerate}

    \end{enumerate}

\end{document}
