%%%%%%%%%%%%%%%%%%%%%%%%%%%%%%%%%%%%%%%%%
% Report
% LaTeX Template
% Version 1.0 (December 8 2014)
%
% This template has been downloaded from:
% http://www.LaTeXTemplates.com
%
% Original author:
% Brandon Fryslie
% With extensive modifications by:
% Vel (vel@latextemplates.com)
%
% License:
% CC BY-NC-SA 3.0 (http://creativecommons.org/licenses/by-nc-sa/3.0/)
%
%%%%%%%%%%%%%%%%%%%%%%%%%%%%%%%%%%%%%%%%%

\documentclass[usletter, 12pt]{article}
%%%%%%%%%%%%%%%%%%%%%%%%%%%%%%%%%%%%%%%%%
% Contract Structural Definitions File Version 1.0 (December 8 2014)
%
% Created by: Vel (vel@latextemplates.com)
% 
% This file has been downloaded from: http://www.LaTeXTemplates.com
%
% License: CC BY-NC-SA 3.0 (http://creativecommons.org/licenses/by-nc-sa/3.0/)
%
%%%%%%%%%%%%%%%%%%%%%%%%%%%%%%%%%%%%%%%%%

\usepackage{geometry} % Required to modify the page layout
\usepackage{multicol}
\usepackage{amsmath}
\usepackage{amssymb}

\usepackage[pdftex]{graphicx}
\usepackage{wrapfig}
\usepackage[font=scriptsize, labelfont=bf]{caption}
\usepackage[utf8]{inputenc} % Required for including letters with accents
\usepackage[T1]{fontenc} % Use 8-bit encoding that has 256 glyphs

\usepackage{avant} % Use the Avantgarde font for headings
\usepackage{xparse}
\usepackage{xcolor}
\usepackage{listings}  % for code verbatim and console outputs

\setlength{\textwidth}{16cm} % Width of the text on the page
\setlength{\textheight}{23cm} % Height of the text on the page
\setlength{\oddsidemargin}{0cm} % Width of the margin - negative to move text left, positive to move it right
\setlength{\topmargin}{-1.25cm} % Reduce the top margin

\setlength{\parindent}{0mm} % Don't indent paragraphs
\setlength{\parskip}{2.5mm} % Whitespace between paragraphs
\renewcommand{\baselinestretch}{1.2}

\renewcommand\familydefault{\sfdefault}  % default font for entire document

\definecolor{green}{rgb}{0.18, 0.55, 0.34}

\graphicspath{ {figures/} }
\captionsetup[table]{skip=10pt}

\lstset{language=C, keywordstyle={\bfseries \color{black}}}

% defines algorithm counter for chapter-level
\newcounter{nalg}[section]

%defines appearance of the algorithm counter
\renewcommand{\thenalg}{\thesection .\arabic{nalg}}

% defines a new caption label as Algorithm x.y
\DeclareCaptionLabelFormat{algocaption}{Algorithm \thenalg}

%defines the algorithm listing environment
\lstnewenvironment{pseudocode}[1][] {
    \refstepcounter{nalg} %increments algorithm number

    \captionsetup{labelformat=algocaption,labelsep=colon}
    \lstset{
        mathescape=true,
        frame=tB,
        numbers=left,
        numberstyle=\tiny,
        basicstyle=\scriptsize,
        keywordstyle=\color{black}\bfseries\em,
        keywords={,input, output, return, datatype, function, in, if, else, foreach, while, begin, end, },
        xleftmargin=.04\textwidth,
        #1
    }
}{}
  % document layout and style

% Member's information
\newcommand{\project}{Project 1: Divide and Conquer}
\newcommand{\members}{Sabbir Ahmed \& Zafar Mamarakhimov}

%----------------------------------------------------------------------------------------

\begin{document}

    \begin{titlepage}

        \vspace*{\fill} % Add whitespace above to center the title page content
        \begin{center}

            {\LARGE \project~Analysis Report}\\ [1.5cm]

            \today
            
            \vspace*{\fill}

            \members

        \end{center}
        \vspace*{\fill} % Add whitespace below to center the title page content

    \end{titlepage}

    \section{Description}
    A recursive, divide-and-conquer algorithm was developed and analyzed to multiply together lists of complex numbers. Two different multiplication methods were used to compute the same products to analyze the crossover point.

        \subsection{Background}
        A complex number z is given by a real part $x$ and an imaginary part $y$,
            \[ z=x+iy, \]
        where \textit{i} is the imaginary unit $\sqrt{-1}$.

        Multiplying two complex numbers are similar to multiplying polynomials. Let $z_{1}=x_{1}+iy_{1}$ and $z_{2}=x_{2}+iy_{2}$ be two complex numbers. Then their product is
            \[ z_{1}z_{2}=(x_{1}+iy_{1})(x_{2}+iy_{2})=(x_{1}x_{2}-y_{1}y_{2})+i(x_{1}y2+y_{1}x_{2}) \]

        That is, the real part of $z_{1}z_{2}$ is $x_{1}x_{2}-y_{1}y_{2}$ and the imaginary part is $x_{1}y_{2}+y_{1}x_{2}$. The computation of a single complex product requires four real products and two real additions (subtraction is just addition with one operand negative and the "$+i$" does not count as an addition as this is really just notation to separate the real and imaginary parts).

        As it turns out, there is a way to reduce the number of real multiplications needed to compute a complex product. It is based on the following observation, which is similar to how Karatsuba's method for multiprecision multiplication was derived:
            \[ (x_{1}+y_{1})(x_{2}+y_{2})=x_{1}x_{2}+(x_{1}y_{2}+y_{1}x_{2})+y_{1}y_{2} \]

        Let $t$ denote the product $(x_{1}+y_{1})(x_{2}+y_{2})$; then if the real products $r=x_{1}x_{2}$ and $s=y_{1}y_{2}$ are computed, the complex product is just
            \[ z_{1}z_{2}=(r-s)+i(t-r-s) \]

        Now, this computation only requires three real multiplications, but increases the number of additions to five. Since multiplication is the more expensive operation, it is expected for the reduction in the number of multiplies to pay-off, at least if the numbers are large enough.

        Simply multiplying two complex numbers would not require a divide-and-conquer solution. However, to multiply a list of $n$ complex numbers, there is a natural recursive divide-and-conquer solution, which is to recursively multiply the left and right halves of the list, each of length approximately $\frac{n}{2}$, and then multiply together the results of the two recursive calls. The base case is a list of length one, for which the function simply returns the single value in the list.

        When multiplying a list of numbers, the difference between three or four real multiplications per complex multiplication can make a significant difference in the running time, especially if individual real multiplications are expensive. Multiprecision arithmetic is required to handle the growth of the product as the numbers get multiplied. Since the cost to perform a single real multiplication will increase per iteration, the use of the "three-multiply" complex multiplication is expected to be faster than the "four-multiply" version. However, since the three-multiply version requires more additions, it may not pay-off until the numbers are large or the list of numbers is long.

    \section{Crossover Point}

        The point at which the asymptotically better algorithm becomes faster is called the \textit{crossover point}.

        \subsection{Theoretical Results}

    \section{Implementation}
    The project was written in C++11 and built with GCC v5.4.0. The GMP library, along with its C++ wrapper, GMPXX, were used to handle the multiprecision arithmetic. The recursive divide-and-conquer functions for both the three- and four-multiplication methods were implemented identically. The algorithm of the implementation is described in the pseudocode snippet provided in Algorithm \thesection.\ref{alg1}.

\begin{pseudocode}[caption={Divide and Conquer Multiplication}, label={alg1}]
/*
Uses a divide and conquer method to recursively multiply all the elements in
the complex array using either of the multiplication methods

Inputs:
    - complex_array: vector of GMP integer pairs
    - first, last: first and last indices of the subarray

Outputs:
    - cmulx() outputs: final complex product
*/
function cmulx_list(complex_array, first, last):

    // if length of the array is 1
    if (first == last):
        return complex_array[first]

    mid = (first + last) / 2
    left_half = cmulx_list(complex_array, first, mid)
    right_half = cmulx_list(complex_array, mid + 1, last)

    return cmulx(left_half, right_half)

\end{pseudocode}

    \section{Testing and Timing}

        \subsection{Platform Specifications}
        Before generating the statistics used in the document, information on the CPU and memory usage of the hosting machine were obtained. The following tables details the specifications captured before initializing the testing procedure.

        \begin{table}[h]
            \caption{Information about the CPU Architecture, Generated by \textit{\$ lscpu}}
            \centering
            \begin{tabular*}{300pt}{@{\extracolsep{\fill}} p{5cm} p{5cm}}

            \textbf{Component} & \textbf{Specification} \\
            \hline
            Architecture:          & x86\_64 \\
            CPU op-mode(s):        & 32-bit, 64-bit \\
            Byte Order:            & Little Endian \\
            CPU(s):                & 4 \\
            On-line CPU(s) list:   & 0-3 \\
            Thread(s) per core:    & 2 \\
            Core(s) per socket:    & 2 \\
            Socket(s):             & 1 \\
            NUMA node(s):          & 1 \\
            Vendor ID:             & GenuineIntel \\
            CPU family:            & 6 \\
            Model:                 & 142 \\
            Model name:            & Intel(R) Core(TM) i5-7200U CPU @ 2.50GHz \\
            Stepping:              & 9 \\
            CPU MHz:               & 1824.398 \\
            CPU max MHz:           & 3100.0000 \\
            CPU min MHz:           & 400.0000 \\
            BogoMIPS:              & 5423.89 \\
            Virtualization:        & VT-x \\
            L1d cache:             & 32K \\
            L1i cache:             & 32K \\
            L2 cache:              & 256K \\
            L3 cache:              & 3072K \\
            NUMA node0 CPU(s):     & 0-3 \\
            \end{tabular*}
        \end{table}

        \begin{table}[h]
            \caption{Information about the Memory, Generated by \textit{\$ free -gh}}
            \centering
            \begin{tabular*}{400pt}{@{\extracolsep{\fill}} ccccccc}

            \textbf{Component} & \textbf{total} & \textbf{used} & \textbf{free} & \textbf{shared} & \textbf{buff/cache} & \textbf{available} \\
            \hline
            Mem: & 3.7G & 2.0G & 344M & 291M & 1.4G & 555M \\
            Swap: & 7.8G & 1.1G & 6.7G & & & \\
            \end{tabular*}
        \end{table}


        \subsection{Testing Methodology}
        The divide and conquer functions of individual multiplication methods were timed. An example of the output time is provided in Figure \ref{time_sample}.

        \begin{figure}[ht]
            \begin{center}
                \includegraphics[width=0.3\textwidth]{time_sample.png}
                \caption{Sample Output of The Program Generated by cmplx\_numbers\_32\_8bit.txt} \label{time_sample}
            \end{center}
        \end{figure}

        The program, however, generates its output after a single iteration of the function call. The output times may vary on each execution due to the hosting machine fluctuating on its core and memory usage from other running processes. Therefore, further steps were taken to compute additional statistics on the timings.

        An additional Makefile has been provided to generate timings on multiple iterations. 200 iterations were used to compute the current statistics. Furthermore, each iteration recompiles before running the program. A clean compilation prevents any external influences and overhead from the system on the individual timings. All the outputs for a single input are dumped to a buffer file, which is later parsed by an external helper script to generate the following statistics:

        \begin{itemize}

            \item Mean
            \item Median
            \item Standard deviation
            \item Differences of the timing of the two methods

        \end{itemize}

        \subsection{Results}
        \begin{figure}[ht]
            \begin{center}
                \includegraphics[width=1\textwidth]{diffs.png}
                \caption{Distribution of Differences in The Multiplication Methods} \label{diffs}
            \end{center}
        \end{figure}

\end{document}
