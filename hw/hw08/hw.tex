%%%%%%%%%%%%%%%%%%%%%%%%%%%%%%%%%%%%%%%%%
% Template
% LaTeX Template
% Version 1.0 (December 8 2014)
%
% This template has been downloaded from:
% http://www.LaTeXTemplates.com
%
% Original author:
% Brandon Fryslie
% With extensive modifications by:
% Vel (vel@latextemplates.com)
%
% License:
% CC BY-NC-SA 3.0 (http://creativecommons.org/licenses/by-nc-sa/3.0/)
%
% Authors:
% Sabbir Ahmed
%
%%%%%%%%%%%%%%%%%%%%%%%%%%%%%%%%%%%%%%%%%

\documentclass[paper=usletter, fontsize=12pt]{article}
%%%%%%%%%%%%%%%%%%%%%%%%%%%%%%%%%%%%%%%%%
% Contract Structural Definitions File Version 1.0 (December 8 2014)
%
% Created by: Vel (vel@latextemplates.com)
% 
% This file has been downloaded from: http://www.LaTeXTemplates.com
%
% License: CC BY-NC-SA 3.0 (http://creativecommons.org/licenses/by-nc-sa/3.0/)
%
%%%%%%%%%%%%%%%%%%%%%%%%%%%%%%%%%%%%%%%%%

\usepackage{geometry} % Required to modify the page layout
\usepackage{multicol}
\usepackage{amsmath}
\usepackage{amssymb}

\usepackage[pdftex]{graphicx}
\usepackage{wrapfig}
\usepackage[font=scriptsize, labelfont=bf]{caption}
\usepackage[utf8]{inputenc} % Required for including letters with accents
\usepackage[T1]{fontenc} % Use 8-bit encoding that has 256 glyphs

\usepackage{avant} % Use the Avantgarde font for headings
\usepackage{xparse}
\usepackage{xcolor}
\usepackage{listings}  % for code verbatim and console outputs

\setlength{\textwidth}{16cm} % Width of the text on the page
\setlength{\textheight}{23cm} % Height of the text on the page
\setlength{\oddsidemargin}{0cm} % Width of the margin - negative to move text left, positive to move it right
\setlength{\topmargin}{-1.25cm} % Reduce the top margin

\setlength{\parindent}{0mm} % Don't indent paragraphs
\setlength{\parskip}{2.5mm} % Whitespace between paragraphs
\renewcommand{\baselinestretch}{1.2}

\renewcommand\familydefault{\sfdefault}  % default font for entire document

\definecolor{green}{rgb}{0.18, 0.55, 0.34}

\graphicspath{ {figures/} }
\captionsetup[table]{skip=10pt}

\lstset{language=C, keywordstyle={\bfseries \color{black}}}

% defines algorithm counter for chapter-level
\newcounter{nalg}[section]

%defines appearance of the algorithm counter
\renewcommand{\thenalg}{\thesection .\arabic{nalg}}

% defines a new caption label as Algorithm x.y
\DeclareCaptionLabelFormat{algocaption}{Algorithm \thenalg}

%defines the algorithm listing environment
\lstnewenvironment{pseudocode}[1][] {
    \refstepcounter{nalg} %increments algorithm number

    \captionsetup{labelformat=algocaption,labelsep=colon}
    \lstset{
        mathescape=true,
        frame=tB,
        numbers=left,
        numberstyle=\tiny,
        basicstyle=\scriptsize,
        keywordstyle=\color{black}\bfseries\em,
        keywords={,input, output, return, datatype, function, in, if, else, foreach, while, begin, end, },
        xleftmargin=.04\textwidth,
        #1
    }
}{}
 % specifies the document layout and style
\allowdisplaybreaks
%------------------------------------------------------------------------------
% document info command
\newcommand{\documentinfo}[5]{
    \begin{centering}
        \parbox{2in}{
        \begin{spacing}{1}
            \begin{flushleft}
                \begin{tabular}{l l}
                    #1 \\
                    #2 \\
                    #3 \\
                \end{tabular}\\
                \rule{\textwidth}{1pt}
            \end{flushleft}
        \end{spacing}
        }
    \end{centering}
}

\newcommand{\Mod}[1]{\ (\mathrm{mod}\ #1)}

\begin{document}

    \documentinfo{Sabbir Ahmed}{\textbf{DATE:} \today}{\textbf{MATH 407:} HW 08}
    \vspace{-0.2in}

    \begin{itemize}

        \item[\textbf{3.2}]

        \begin{itemize}

            \item[\textbf{1}] In $GL_2(R)$, find the order of each of the
            following elements.
            \begin{enumerate}

                \item[\textbf{b}]
                \begin{equation*}
                    \left[
                        \begin{tabular}{cc}
                            0 & 1 \\
                            -1 & 0
                        \end{tabular}
                    \right]
                \end{equation*}
                \item[\textbf{Ans}]
                \begin{proof}[\unskip\nopunct]
                \end{proof}
                \vspace{0.2in}

                \item[\textbf{d}]
                \begin{equation*}
                    \left[
                        \begin{tabular}{cc}
                            -1 & 1 \\
                            0 & 1
                        \end{tabular}
                    \right]
                \end{equation*}
                \item[\textbf{Ans}]
                \begin{proof}[\unskip\nopunct]
                \end{proof}
                \vspace{0.2in}

            \end{enumerate}

            \item[\textbf{11}]
            Let $S$ be a set, and let $a$ be a fixed element of $S$. Show that
            $\{\sigma \in \text{Sym}(S) \mid \sigma(a)=a \}$ is a subgroup of
            Sym($S$).
            \item[\textbf{Ans}]
            \begin{proof}[\unskip\nopunct]
            \end{proof}
            \vspace{0.2in}

            \item[\textbf{12}] For each of the following groups, find all
            elements of finite order.
            \begin{enumerate}

                \item[\textbf{a}] $\mathbb{R}^{\times}$
                \item[\textbf{Ans}]
                \begin{proof}[\unskip\nopunct]
                \end{proof}
                \vspace{0.2in}

                \item[\textbf{d}] $\mathbb{C}^{\times}$
                \item[\textbf{Ans}]
                \begin{proof}[\unskip\nopunct]
                \end{proof}
                \vspace{0.2in}

            \end{enumerate}

            \item[\textbf{19}] Let $G$ be a group, and let $a \in G$. The set
            $C(a) = \{ x \in G \mid xa = ax \}$ of all elements of $G$ that
            commute with $a$ is called the \textbf{centralizer} of $a$.
            \begin{enumerate}

                \item[\textbf{a}] Show that $C(a)$ is a subgroup of $G$.
                \item[\textbf{Ans}]
                \begin{proof}[\unskip\nopunct]
                \end{proof}
                \vspace{0.2in}

                \item[\textbf{b}] Show that $\langle a \rangle \subseteq C(a)$.
                \item[\textbf{Ans}]
                \begin{proof}[\unskip\nopunct]
                \end{proof}
                \vspace{0.2in}

                \item[\textbf{c}] Compute $C(a)$ if $G=S_3$ and $a = (1,2,3)$.
                \item[\textbf{Ans}]
                \begin{proof}[\unskip\nopunct]
                \end{proof}
                \vspace{0.2in}

                \item[\textbf{d}] Compute $C(a)$ if $G=S_3$ and $a = (1,2)$.
                \item[\textbf{Ans}]
                \begin{proof}[\unskip\nopunct]
                \end{proof}
                \vspace{0.2in}

            \end{enumerate}

            \item[\textbf{20}] Compute the centralizer in $\text{GL}_2(\mathbb{R})$ of the matrix $\left[\begin{tabular}{cc}
                                        -1 & 1 \\
                                        0 & 1
                                    \end{tabular}\right]$
            \item[\textbf{Ans}]
            \begin{proof}[\unskip\nopunct]
            \end{proof}
            \vspace{0.2in}

            \item[\textbf{25}] Let $G$ be a finite group, let $n > 2$ be an
            integer, and let $S$ be the set of elements of $G$ that have order
            $n$. Show that $S$ has an even number of elements.
            \item[\textbf{Ans}]
            \begin{proof}[\unskip\nopunct]
            \end{proof}
            \vspace{0.2in}

        \end{itemize}

        \item[\textbf{3.3}]

        \begin{itemize}

            \item[\textbf{4}] Find the cyclic subgroup generated by $\left[\begin{tabular}{cc}
                2 & 1 \\
                0 & 2
            \end{tabular}\right]$ in $\text{GL}_2(\mathbb{Z}_3)$.
            \item[\textbf{Ans}]
            \begin{proof}[\unskip\nopunct]
            \end{proof}
            \vspace{0.2in}

            \item[\textbf{5}] Prove that if $G_1$ and $G_2$ are abelian groups,
            then the direct product $G_1 \times G_2$ is abelian.
            \item[\textbf{Ans}]
            \begin{proof}[\unskip\nopunct]
            \end{proof}
            \vspace{0.2in}

            \item[\textbf{8}] Let $G_1$ and $G_2$ be groups, with subgroups
            $H_1$ and $H_2$, respectively. Show that $\{(x_1,x_2) \mid x_1 \in
            H_1, x_2 \in H_2\}$ is a subgroup of the direct product $G_1 \times
            G_2$.
            \item[\textbf{Ans}]
            \begin{proof}[\unskip\nopunct]
            \end{proof}
            \vspace{0.2in}

            \item[\textbf{11}] Let $G_1$ and $G_2$ be groups, and let $G$ be
            the direct product $G_1 \times G_2$. Let $H=\{(x_1,x_2)\in G_1
            \times G_2 \mid x_2 = e\}$ and let $K=\{(x_1,x_2)\in G_1 \times G_2
            \mid x_1 = e\}$.

            \begin{enumerate}

                \item[\textbf{a}] Show that $H$ and $K$ are subgroups of $G$.
                \item[\textbf{Ans}]
                \begin{proof}[\unskip\nopunct]
                \end{proof}
                \vspace{0.2in}

                \item[\textbf{b}] Show that $HK=KH=G$.
                \item[\textbf{Ans}]
                \begin{proof}[\unskip\nopunct]
                \end{proof}
                \vspace{0.2in}

                \item[\textbf{c}] Show that $H \cap K = \{e,e\}$.
                \item[\textbf{Ans}]
                \begin{proof}[\unskip\nopunct]
                \end{proof}
                \vspace{0.2in}

            \end{enumerate}

            \item[\textbf{13}] Let $p, q$ be distinct prime numbers, and let
            $n=pq$. Show that $HK=\mathbb{Z}_{n}^{\times}$, for the subgroups
            $H=\{[x] \in \mathbb{Z}_{n}^{\times} \mid x \equiv 1 \Mod p\}$ and
            $K=\{[y] \in \mathbb{Z}_{n}^{\times} \mid y \equiv 1 \Mod q\}$ of
            $\mathbb{Z}_{n}^{\times}$.
            \item[\textbf{Ans}]
            \begin{proof}[\unskip\nopunct]
            \end{proof}
            \vspace{0.2in}

            \item[\textbf{16}] Let $G$ be a group of order 6, and suppose that
            $a, b \in G$ with $a$ of order 3 and $b$ of order 2. Show that
            either $G$ is cyclic or $ab \neq ba$.
            \item[\textbf{Ans}]
            \begin{proof}[\unskip\nopunct]
            \end{proof}
            \vspace{0.2in}

        \end{itemize}

    \end{itemize}

\end{document}
