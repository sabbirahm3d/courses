\section{Part 2}
    \subsection{Question}
    A random sample of 12 graduates of a secretarial school averaged 73.2 words per minute with a standard deviation of 7.9 words per minute on a typing test. What can we conclude, at the 0.05 level, regarding the claim that secretaries at this school average less than 75 words per minute on the typing test?

    \subsection{Answer}
    The null hypothesis, $H_{0}$, claims the school averaged greater or equal to 75, while the alternative hypothesis, $H_{a}$, says otherwise.

        \[ H_{0}: \mu < 75 \ vs \ H_{a}: \ \geq 75 \]

    Since the population standard deviation is unknown, and the sample was $n < 30$, the t-score was calculated as follows:

        \[ t=\frac{\overline{X}-\mu}{\sfrac{s}{\sqrt{n}}}, \ df=11 \]\\
    \[ =\frac{73.2-75.0}{\sfrac{7.9}{\sqrt{12}}} \]
    \[ =t=-0.7893, \ P(t)=-1.7959 \]

    The following snippet was used to generate the t-test and its probability:
\begin{lstlisting}
    X <- 73.2
    mu <- 75
    s <- 7.9
    n <- 12

    dumpComputation(X=X, mu=mu, sigma=s, n=n, 
        alpha=alpha, distType="t", twoSided=FALSE, "part2")
\end{lstlisting}

    The test statistic was computed to be:

        \begin{align*}
            \because t_{1-0.05,11}=t_{0.95,11}=1.7958 > -0.7893\\
        \end{align*}

    The p-value was computed with the following snippet:
\begin{lstlisting}
    pScore <- 1-pt(score, df=n-1)
    # 0.7767
\end{lstlisting}

    Since $t_{1-\alpha, df} > t$ and the very high p-value, there is strong evidence to not reject the null hypothesis.
