%%%%%%%%%%%%%%%%%%%%%%%%%%%%%%%%%%%%%%%%%
% Template
% LaTeX Template
% Version 1.0 (December 8 2014)
%
% This template has been downloaded from:
% http://www.LaTeXTemplates.com
%
% Original author:
% Brandon Fryslie
% With extensive modifications by:
% Vel (vel@latextemplates.com)
%
% License:
% CC BY-NC-SA 3.0 (http://creativecommons.org/licenses/by-nc-sa/3.0/)
%
% Authors:
% Sabbir Ahmed
%
%%%%%%%%%%%%%%%%%%%%%%%%%%%%%%%%%%%%%%%%%

\documentclass[paper=usletter, fontsize=12pt]{article}
%%%%%%%%%%%%%%%%%%%%%%%%%%%%%%%%%%%%%%%%%
% Contract Structural Definitions File Version 1.0 (December 8 2014)
%
% Created by: Vel (vel@latextemplates.com)
% 
% This file has been downloaded from: http://www.LaTeXTemplates.com
%
% License: CC BY-NC-SA 3.0 (http://creativecommons.org/licenses/by-nc-sa/3.0/)
%
%%%%%%%%%%%%%%%%%%%%%%%%%%%%%%%%%%%%%%%%%

\usepackage{geometry} % Required to modify the page layout
\usepackage{multicol}
\usepackage{amsmath}
\usepackage{amssymb}

\usepackage[pdftex]{graphicx}
\usepackage{wrapfig}
\usepackage[font=scriptsize, labelfont=bf]{caption}
\usepackage[utf8]{inputenc} % Required for including letters with accents
\usepackage[T1]{fontenc} % Use 8-bit encoding that has 256 glyphs

\usepackage{avant} % Use the Avantgarde font for headings
\usepackage{courier}
\usepackage{xparse}
\usepackage{xcolor}
\usepackage{listings}  % for code verbatim and console outputs

\setlength{\textwidth}{16cm} % Width of the text on the page
\setlength{\textheight}{23cm} % Height of the text on the page
\setlength{\oddsidemargin}{0cm} % Width of the margin - negative to move text left, positive to move it right
\setlength{\topmargin}{-1.25cm} % Reduce the top margin

\setlength{\parindent}{0mm} % Don't indent paragraphs
\setlength{\parskip}{2.5mm} % Whitespace between paragraphs
\renewcommand{\baselinestretch}{1.5}

\definecolor{green}{rgb}{0.18, 0.55, 0.34}

\graphicspath{ {figures/} }
\captionsetup[table]{skip=10pt}

\lstset{language=C, keywordstyle={\bfseries \color{black}}}

% defines algorithm counter for chapter-level
\newcounter{nalg}[section]

%defines appearance of the algorithm counter
\renewcommand{\thenalg}{\thesection .\arabic{nalg}}

% defines a new caption label as Algorithm x.y
\DeclareCaptionLabelFormat{algocaption}{Algorithm \thenalg}

% defines the algorithm listing environment
\lstnewenvironment{pseudocode}[1][] {
    \refstepcounter{nalg}  % increments algorithm number
    \captionsetup{font=normalsize, labelformat=algocaption, labelsep=colon}
    \lstset{
        breaklines=true,
        mathescape=true,
        numbers=left,
        numberstyle=\scriptsize,
        basicstyle=\footnotesize\ttfamily,
        keywordstyle=\color{black}\bfseries,
        keywords={input, output, return, parallel, function, for, to, in, if,
        else, foreach, while, and, or, new, print},
        xleftmargin=.04\textwidth,
        #1
    }
}{}

\renewcommand{\familydefault}{\sfdefault}  % default font for entire document
 % specifies the document layout and style

%------------------------------------------------------------------------------
% document info command
\newcommand{\documentinfo}[5]{
    \begin{centering}
        \parbox{2in}{
        \begin{spacing}{1}
            \begin{flushleft}
                \begin{tabular}{l l}
                    #1 \\
                    #2 \\
                    #3 \\
                \end{tabular}\\
                \rule{\textwidth}{1pt}
            \end{flushleft}
        \end{spacing}
        }
    \end{centering}
}

\begin{document}

    \documentinfo{Sabbir Ahmed}{\textbf{DATE:} \today}{\textbf{MATH 407} HW 03}
    \vspace{-0.2in}

    \begin{itemize}

        \item[\textbf{1.2}]

        \begin{itemize}

            \item[\textbf{7}] Let $m$ and $n$ be positive integers such that $m
            + n = 57$ and $[m, n] = 680$. Find $m$ and $n$.
            \item[\textbf{Ans}]
            \begin{proof}[\unskip\nopunct]
                $m + n = 57$ and $[m, n] = 680$\\
                Let $n$ be represented by $m$, such that,
                \begingroup
                \addtolength{\jot}{1em}
                \begin{align*}
                    m \cdot (57 - m) & = 680\\
                    \Rightarrow 0 & = m^2 - 57m + 680\\
                    & = \frac{57 \pm \sqrt{57^2-4\cdot 680}}{2}\\
                    & = \frac{57 \pm 23}{2}\\
                    m & = 40, 17
                \end{align*}
                \endgroup
                $\therefore m = 40$, $n = 17$ \qedhere
            \end{proof}
            \vspace{0.2in}

            \item[\textbf{10}] Show that $a\mathbb{Z} \cap b\mathbb{Z} =
            [a,b]\mathbb{Z}$.
            \item[\textbf{Ans}]
            \begin{proof}[\unskip\nopunct]
                Let $x \in [a,b]\mathbb{Z}$\\
                Since, $a \given [a,b]$ and $b \given [a,b]$,\\
                $a \given x$ and $b \given x$\\
                Therefore, $x \in a\mathbb{Z} \cap b\mathbb{Z}$\\

                Conversely,\\
                since $a \given [a,b]$, then $[a,b] \in a\mathbb{Z}$\\
                and $b \given [a,b]$, then $[a,b] \in b\mathbb{Z}$\\
                Then, $[a,b] \in (a \cap b)\mathbb{Z}$

                $\therefore a\mathbb{Z} \cap b\mathbb{Z} = [a,b]\mathbb{Z}$
                \qedhere
            \end{proof}
            \vspace{0.2in}

            \item[\textbf{16}] A positive integer $a$ is called a
            \textbf{square} if $a = n^2$ for some $n \in \integers$. Show that
            the integer $a > 1$ is an integer if and only if every exponent in
            its prime factorization is even.
            \item[\textbf{Ans}]
            \begin{proof}[\unskip\nopunct]
                Suppose $a = p_{1}^{r_{1}} \cdot p_{2}^{r_{2}} \cdot \ldots \cdot p_{k}^{r_{k}}$, where $r_{i}$ is even.\\
                Also, let $n = p_{1}^{r_{1}/2} \cdot p_{2}^{r_{2}/2} \cdot \ldots \cdot p_{k}^{r_{k}/2}$\\
                Then, $a = n^2$\\

                Conversely, suppose $a = n^2$.\\
                Let $n = p_{1}^{s_{1}} \cdot p_{2}^{s_{2}} \cdot \ldots \cdot p_{l}^{s_{l}}$\\
                Then, $a = p_{1}^{2s_{1}} \cdot p_{2}^{2s_{2}} \cdot \ldots \cdot p_{l}^{2s_{l}}$\\
                Therefore, all the primes of $a$ have even powers. \qedhere
            \end{proof}
            \vspace{0.2in}

            \item[\textbf{20}] A positive integer is called \textbf{square-free} if it is a product of distinct primes. Prove that every
            positive integer can be written uniquely as a product of a square
            and a square-free integer.
            \item[\textbf{Ans}]
            \begin{proof}[\unskip\nopunct]
                \texttt{\small NOTE: I needed help on this question}\\
                Let $x \in \mathbb{Z^{+}}$ not be a square or a square-free integer.\\
                Let
                \begin{equation*}
                    x = p_{1}^{r_1} \cdot p_{2}^{r_2} \cdot \ldots \cdot p_{k}^{r_k}
                \end{equation*}
                where $r_{i}$ for $1 \le i \le k$ can be either odd or even.\\
                However, they all cannot be even since $x$ is not a square integer.\\

                All the odd $r_i$'s can be represented as the sum of $r_i$ with all the even $r_i$'s.\\
                For example,\\
                if $r_i = 5$ (a prime integer),\\
                then $r_i = 1 + 4$\\

                If all the even $r_i$'s and all the exponents that are $1$ are grouped together,\\
                $x$ can be represented as:
                \begin{equation*}
                    x = p_{1}^{s_1} \cdot p_{2}^{s_2} \cdot \ldots \cdot p_{k}^{s_k} \cdot p_{k+1} \cdot p_{k+2} \cdot \ldots \cdot p_{k+r}
                \end{equation*}
                where all the $s_i$'s are even.\\
                This implies that $x$ is the product of a square and square free integer. \qedhere
            \end{proof}
            \vspace{0.2in}

        \end{itemize}

    \end{itemize}

\end{document}
