%%%%%%%%%%%%%%%%%%%%%%%%%%%%%%%%%%%%%%%%%
% Template
% LaTeX Template
% Version 1.0 (December 8 2014)
%
% This template has been downloaded from:
% http://www.LaTeXTemplates.com
%
% Original author:
% Brandon Fryslie
% With extensive modifications by:
% Vel (vel@latextemplates.com)
%
% License:
% CC BY-NC-SA 3.0 (http://creativecommons.org/licenses/by-nc-sa/3.0/)
%
% Authors:
% Sabbir Ahmed
%
%%%%%%%%%%%%%%%%%%%%%%%%%%%%%%%%%%%%%%%%%

\documentclass[paper=usletter, fontsize=12pt]{article}
%%%%%%%%%%%%%%%%%%%%%%%%%%%%%%%%%%%%%%%%%
% Contract Structural Definitions File Version 1.0 (December 8 2014)
%
% Created by: Vel (vel@latextemplates.com)
% 
% This file has been downloaded from: http://www.LaTeXTemplates.com
%
% License: CC BY-NC-SA 3.0 (http://creativecommons.org/licenses/by-nc-sa/3.0/)
%
%%%%%%%%%%%%%%%%%%%%%%%%%%%%%%%%%%%%%%%%%

\usepackage{geometry} % Required to modify the page layout
\usepackage{multicol}
\usepackage{amsmath}
\usepackage{amssymb}

\usepackage[pdftex]{graphicx}
\usepackage{wrapfig}
\usepackage[font=scriptsize, labelfont=bf]{caption}
\usepackage[utf8]{inputenc} % Required for including letters with accents
\usepackage[T1]{fontenc} % Use 8-bit encoding that has 256 glyphs

\usepackage{avant} % Use the Avantgarde font for headings
\usepackage{xparse}
\usepackage{xcolor}
\usepackage{listings}  % for code verbatim and console outputs

\setlength{\textwidth}{16cm} % Width of the text on the page
\setlength{\textheight}{23cm} % Height of the text on the page
\setlength{\oddsidemargin}{0cm} % Width of the margin - negative to move text left, positive to move it right
\setlength{\topmargin}{-1.25cm} % Reduce the top margin

\setlength{\parindent}{0mm} % Don't indent paragraphs
\setlength{\parskip}{2.5mm} % Whitespace between paragraphs
\renewcommand{\baselinestretch}{1.2}

\renewcommand\familydefault{\sfdefault}  % default font for entire document

\definecolor{green}{rgb}{0.18, 0.55, 0.34}

\graphicspath{ {figures/} }
\captionsetup[table]{skip=10pt}

\lstset{language=C, keywordstyle={\bfseries \color{black}}}

% defines algorithm counter for chapter-level
\newcounter{nalg}[section]

%defines appearance of the algorithm counter
\renewcommand{\thenalg}{\thesection .\arabic{nalg}}

% defines a new caption label as Algorithm x.y
\DeclareCaptionLabelFormat{algocaption}{Algorithm \thenalg}

%defines the algorithm listing environment
\lstnewenvironment{pseudocode}[1][] {
    \refstepcounter{nalg} %increments algorithm number

    \captionsetup{labelformat=algocaption,labelsep=colon}
    \lstset{
        mathescape=true,
        frame=tB,
        numbers=left,
        numberstyle=\tiny,
        basicstyle=\scriptsize,
        keywordstyle=\color{black}\bfseries\em,
        keywords={,input, output, return, datatype, function, in, if, else, foreach, while, begin, end, },
        xleftmargin=.04\textwidth,
        #1
    }
}{}
 % specifies the document layout and style

%------------------------------------------------------------------------------
% document info command
\newcommand{\documentinfo}[5]{
    \begin{centering}
        \parbox{2in}{
        \begin{spacing}{1}
            \begin{flushleft}
                \begin{tabular}{l l}
                    #1 \\
                    #2 \\
                    #3 \\
                \end{tabular}\\
                \rule{\textwidth}{1pt}
            \end{flushleft}
        \end{spacing}
        }
    \end{centering}
}

\begin{document}

    \documentinfo{Sabbir Ahmed}{\textbf{DATE:} \today}{\textbf{MATH 407:} HW 04}
    \vspace{-0.2in}

    \begin{itemize}

        \item[\textbf{1.3}]

        \begin{itemize}

            \item[\textbf{1}] Solve the following congruence

            \begin{itemize}

                \item[\textbf{d}] $19x \equiv 1 \text{ (mod  36)}$
                \item[\textbf{Ans}]
                \begin{proof}[\unskip\nopunct]
                \end{proof}
                \vspace{0.2in}

            \end{itemize}

            \item[\textbf{4}] Solve the following congruence: $20x \equiv 12
            \text{ (mod  72)}$
            \item[\textbf{Ans}]
            \begin{proof}[\unskip\nopunct]
            \end{proof}
            \vspace{0.2in}

            \item[\textbf{7}] The smallest positive solution of the congruence
            $ax \equiv 0 \text{ (mod  $n$)}$ is called the additive order of
            $a$ modulo $n$. Find the additive orders of each of the following
            elements, by solving the appropriate congruences.

            \begin{itemize}

                \item[\textbf{b}] 7 modulo 12
                \item[\textbf{Ans}]
                \begin{proof}[\unskip\nopunct]
                \end{proof}
                \vspace{0.2in}

                \item[\textbf{d}] 12 modulo 18
                \item[\textbf{Ans}]
                \begin{proof}[\unskip\nopunct]
                \end{proof}
                \vspace{0.2in}

            \end{itemize}

            \item[\textbf{14}] Find the units digit of $3^{29}+11^{12}+15$.\\
            \textit{Hint}: Choose an appropriate modulus $n$, and then reduce
            modulo $n$.
            \item[\textbf{Ans}]
            \begin{proof}[\unskip\nopunct]
            \end{proof}
            \vspace{0.2in}

            \item[\textbf{16}] Solve the following congruences by trial and
            error.
            \begin{itemize}

                \item[\textbf{a}] $x^3+2x+2 \equiv 0 \text{ (mod  5)}$
                \item[\textbf{Ans}]
                \begin{proof}[\unskip\nopunct]
                \end{proof}
                \vspace{0.2in}

            \end{itemize}

            \item[\textbf{20}] Solve the following system of congruences.
            \begin{align*}
                2x & \equiv 5 \text{ (mod  7)} & 3x & \equiv 4 \text{ (mod  8)}
            \end{align*}
            \item[\textbf{Ans}]
            \begin{proof}[\unskip\nopunct]
            \end{proof}
            \vspace{0.2in}

        \end{itemize}

        \item[\textbf{1.4}]

        \begin{itemize}

            \item[\textbf{2}] Make multiplication tables for the following
            sets.

            \begin{itemize}

                \item[\textbf{b}] $\integers_7$
                \item[\textbf{Ans}]
                \begin{proof}[\unskip\nopunct]
                \end{proof}
                \vspace{0.2in}

                \item[\textbf{c}] $\integers_8$
                \item[\textbf{Ans}]
                \begin{proof}[\unskip\nopunct]
                \end{proof}
                \vspace{0.2in}

            \end{itemize}

            \item[\textbf{6}] Let $m$ and $n$ be positive integers such that
            $m\mid n$. Show that for any integer $a$, the congruence class
            $[a]_m$ is the union of the congruence classes $[a]_n, [a+m]_n, [a+2m]_n, \ldots, [a+n-m]_n$
            \item[\textbf{Ans}]
            \begin{proof}[\unskip\nopunct]
            \end{proof}
            \vspace{0.2in}

            \item[\textbf{9}] Let $(a, n)=1$. The smallest positive integer $k$
            such that a $a^k \equiv 1 \text{ (mod  $n$)}$ is called the
            \textbf{multiplicative order} of $[a]$ in $\integers_n^\times$
            \begin{itemize}

                \item[\textbf{b}] Find the multiplicative orders of $[2]$ and
                $[5]$ in $\integers_{17}^\times$.
                \item[\textbf{Ans}]
                \begin{proof}[\unskip\nopunct]
                \end{proof}
                \vspace{0.2in}

            \end{itemize}

            \item[\textbf{10}] Let $(a, n)=1$. If $[a]$ has multiplicative
            order $k$ in $\integers_n^\times$, show that $k \mid \varphi(n)$.
            \item[\textbf{Ans}]
            \begin{proof}[\unskip\nopunct]
            \end{proof}
            \vspace{0.2in}

            \item[\textbf{13}] An element $[a]$ of is said to be
            \textbf{idempotent} if $[a]^2 [a]$.
            \begin{itemize}

                \item[\textbf{b}] Find all idempotent elements of
                $\integers_{10}^\times$ and $\integers_{30}^\times$.
                \item[\textbf{Ans}]
                \begin{proof}[\unskip\nopunct]
                \end{proof}
                \vspace{0.2in}

            \end{itemize}

            \item[\textbf{15}] If $n$ is not a prime power, show that
            $\integers_n$ has an idempotent element different from $[0]$ and
            $[1]$.\\ \textit{Hint}: Suppose that $n=bc$, with $(b, c)=1$. Solve
            the simultaneous congruences $x \equiv 1 \text{ (mod  $b$)}$ and $x
            \equiv 0 \text{ (mod  $c$)}$.
            \item[\textbf{Ans}]
            \begin{proof}[\unskip\nopunct]
            \end{proof}
            \vspace{0.2in}

            \item[\textbf{20}] Show that
            $\varphi(1)+\varphi(p)+\ldots+\varphi(p^\alpha)=\varphi^\alpha$ for
            any prime number $p$ and any positive integer $\alpha$.
            \item[\textbf{Ans}]
            \begin{proof}[\unskip\nopunct]
            \end{proof}
            \vspace{0.2in}

            \item[\textbf{26}] Let $p=2k+1$ be a prime number. Show that if $a$
            is an integer such that $p \nmid a$, then either $a^k \equiv 1
            \text{ (mod  $p$)}$ or $a^k \equiv -1 \text{ (mod  $p$)}$
            \item[\textbf{Ans}]
            \begin{proof}[\unskip\nopunct]
            \end{proof}
            \vspace{0.2in}

        \end{itemize}

    \end{itemize}

\end{document}
