%%%%%%%%%%%%%%%%%%%%%%%%%%%%%%%%%%%%%%%%%
% Template LaTeX Template Version 1.0 (December 8 2014)
%
% This template has been downloaded from: http://www.LaTeXTemplates.com
%
% Original author: Brandon Fryslie With extensive modifications by: Vel
% (vel@latextemplates.com)
%
% License: CC BY-NC-SA 3.0 (http://creativecommons.org/licenses/by-nc-sa/3.0/)
%
% Authors: Sabbir Ahmed
% 
%%%%%%%%%%%%%%%%%%%%%%%%%%%%%%%%%%%%%%%%%

\documentclass[10pt]{extarticle}
%%%%%%%%%%%%%%%%%%%%%%%%%%%%%%%%%%%%%%%%%
% Contract Structural Definitions File Version 1.0 (December 8 2014)
%
% Created by: Vel (vel@latextemplates.com)
% 
% This file has been downloaded from: http://www.LaTeXTemplates.com
%
% License: CC BY-NC-SA 3.0 (http://creativecommons.org/licenses/by-nc-sa/3.0/)
%
%%%%%%%%%%%%%%%%%%%%%%%%%%%%%%%%%%%%%%%%%

\usepackage{geometry} % Required to modify the page layout
\usepackage{multicol}
\usepackage{amsmath}
\usepackage{amssymb}

\usepackage[pdftex]{graphicx}
\usepackage{wrapfig}
\usepackage[font=scriptsize, labelfont=bf]{caption}
\usepackage[utf8]{inputenc} % Required for including letters with accents
\usepackage[T1]{fontenc} % Use 8-bit encoding that has 256 glyphs

\usepackage{avant} % Use the Avantgarde font for headings
\usepackage{courier}
\usepackage{xparse}
\usepackage{xcolor}
\usepackage{listings}  % for code verbatim and console outputs

\setlength{\textwidth}{16cm} % Width of the text on the page
\setlength{\textheight}{23cm} % Height of the text on the page
\setlength{\oddsidemargin}{0cm} % Width of the margin - negative to move text left, positive to move it right
\setlength{\topmargin}{-1.25cm} % Reduce the top margin

\setlength{\parindent}{0mm} % Don't indent paragraphs
\setlength{\parskip}{2.5mm} % Whitespace between paragraphs
\renewcommand{\baselinestretch}{1.5}

\definecolor{green}{rgb}{0.18, 0.55, 0.34}

\graphicspath{ {figures/} }
\captionsetup[table]{skip=10pt}

\lstset{language=C, keywordstyle={\bfseries \color{black}}}

% defines algorithm counter for chapter-level
\newcounter{nalg}[section]

%defines appearance of the algorithm counter
\renewcommand{\thenalg}{\thesection .\arabic{nalg}}

% defines a new caption label as Algorithm x.y
\DeclareCaptionLabelFormat{algocaption}{Algorithm \thenalg}

% defines the algorithm listing environment
\lstnewenvironment{pseudocode}[1][] {
    \refstepcounter{nalg}  % increments algorithm number
    \captionsetup{font=normalsize, labelformat=algocaption, labelsep=colon}
    \lstset{
        breaklines=true,
        mathescape=true,
        numbers=left,
        numberstyle=\scriptsize,
        basicstyle=\footnotesize\ttfamily,
        keywordstyle=\color{black}\bfseries,
        keywords={input, output, return, parallel, function, for, to, in, if,
        else, foreach, while, and, or, new, print},
        xleftmargin=.04\textwidth,
        #1
    }
}{}

\renewcommand{\familydefault}{\sfdefault}  % default font for entire document
 % specifies the document layout and style

% document info command
\newcommand{\documentinfo}[5]{
    \begin{centering}
        \parbox{2in}{
        \begin{spacing}{1}
            \begin{flushleft}
                \begin{tabular}{l l} \textbf{#1} \\ \textbf{#2} \\ #3 \\
                \end{tabular}\\
                \rule{\textwidth}{1pt}
            \end{flushleft}
        \end{spacing} }
    \end{centering} }

\begin{document}

    \documentinfo{Sabbir Ahmed}
    {CMSC 411 - HW 05}
    {\textbf{DATE:} \today}
    \vspace{-0.3in}

    \begin{enumerate}

        \item The table below lists parameters for different direct-mapped cache designs.

        \begin{table}[h]
            \centering
            \begin{tabular*}{200pt}{@{\extracolsep{\fill}} ccc}
                    & \textbf{Cache Data Size}  & \textbf{Cache Block Size} \\
                \hline
                i)  & 64 KB                     & 1 word                    \\
                ii) & 64 KB                     & 2 words                   \\
            \end{tabular*}
        \end{table}

        \begin{enumerate}

            \item \textbf{Question} Calculate the total number of bits required
            for the cache listed in the table, assuming a 32-bit address.

            \textbf{Answer}

            \item \textbf{Question} What is the total number of bits if the
            cache is organized as a 4-way associative with one word blocks?

            \textbf{Answer}

        \end{enumerate}

            \item For a pipeline with a perfect CPI=1 if no memory-access
            related stall, consider the following program and cache behaviors.

            \begin{table}[h]
                \centering
                \begin{tabular}{p{2cm}p{2cm}p{2cm}p{2cm}p{1.5cm}}
                    \textbf{Data Reads Per 1000 Instructions} & \textbf{Data
                    Writes Per 1000 Instructions} & \textbf{Instruction Cache
                    Miss Rate} & \textbf{Data Cache Miss Rate} & \textbf{Block
                    Size (Byte)} \\
                    \hline
                    200 & 160 & 0.20\% & 2\% & 8 \\
                \end{tabular}
            \end{table}

        \begin{enumerate}

            \item \textbf{Question} For a write-through, write-allocate cache
            with sufficiently large write buffer (i.e., no buffer caused
            stalls), what’s the minimum read and write bandwidths (measured by
            byte-per- cycle) needed to achieve a CPI of 2?

            \textbf{Answer} 

            \item \textbf{Question} For a write-back, write-allocate cache,
            assuming 30\% of replaced data cache blocks are dirty, what’s the
            minimal read and write bandwidths needed for a CPI of 2?

            \textbf{Answer}

        \end{enumerate}

        \item Using the sequences of 32-bit memory read
        references, given as word addresses in the following table:

        \begin{table}[h]
            \centering
            \begin{tabular*}{300pt}{@{\extracolsep{\fill}}
            |c|c|c|c|c|c|c|c|c|c|c|c|}
            \hline
            6 & 214 & 175 & 214 & 6 & 84 & 65 & 174 & 64 & 105 & 85 & 215 \\
            \hline
            \end{tabular*}
        \end{table}
        \newpage

        For each of these read accesses, identify the binary address, the tag,
        the index, and whether it experiences a hit or a miss, for each of the
        following cache configurations. Assume the cache is initially empty.

        \begin{enumerate}

            \item \textbf{Question} A direct-mapped cache with 16 one-word
            blocks.

            \textbf{Answer}


            \item \textbf{Question} A direct-mapped cache with two-word blocks
            and a total size of eight blocks.

            \textbf{Answer}


            \item \textbf{Question} A fully associative cache with two-word
            blocks and a total size of eight words. Use LRU replacement.

            \textbf{Answer}


            \item \textbf{Question} A 2-way set associative cache with one-word
            block size and total size of 8 words, while applying LRU
            replacement policy.

            \textbf{Answer}

        \end{enumerate}

    \end{enumerate}

\end{document}
