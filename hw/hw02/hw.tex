%
% Author: Sabbir Ahmed
%

\documentclass[paper=usletter, fontsize=12pt]{article}
%%%%%%%%%%%%%%%%%%%%%%%%%%%%%%%%%%%%%%%%%
% Contract Structural Definitions File Version 1.0 (December 8 2014)
%
% Created by: Vel (vel@latextemplates.com)
% 
% This file has been downloaded from: http://www.LaTeXTemplates.com
%
% License: CC BY-NC-SA 3.0 (http://creativecommons.org/licenses/by-nc-sa/3.0/)
%
%%%%%%%%%%%%%%%%%%%%%%%%%%%%%%%%%%%%%%%%%

\usepackage{geometry} % Required to modify the page layout
\usepackage{multicol}
\usepackage{amsmath}
\usepackage{amssymb}

\usepackage[pdftex]{graphicx}
\usepackage{wrapfig}
\usepackage[font=scriptsize, labelfont=bf]{caption}
\usepackage[utf8]{inputenc} % Required for including letters with accents
\usepackage[T1]{fontenc} % Use 8-bit encoding that has 256 glyphs

\usepackage{avant} % Use the Avantgarde font for headings
\usepackage{courier}
\usepackage{xparse}
\usepackage{xcolor}
\usepackage{listings}  % for code verbatim and console outputs

\setlength{\textwidth}{16cm} % Width of the text on the page
\setlength{\textheight}{23cm} % Height of the text on the page
\setlength{\oddsidemargin}{0cm} % Width of the margin - negative to move text left, positive to move it right
\setlength{\topmargin}{-1.25cm} % Reduce the top margin

\setlength{\parindent}{0mm} % Don't indent paragraphs
\setlength{\parskip}{2.5mm} % Whitespace between paragraphs
\renewcommand{\baselinestretch}{1.5}

\definecolor{green}{rgb}{0.18, 0.55, 0.34}

\graphicspath{ {figures/} }
\captionsetup[table]{skip=10pt}

\lstset{language=C, keywordstyle={\bfseries \color{black}}}

% defines algorithm counter for chapter-level
\newcounter{nalg}[section]

%defines appearance of the algorithm counter
\renewcommand{\thenalg}{\thesection .\arabic{nalg}}

% defines a new caption label as Algorithm x.y
\DeclareCaptionLabelFormat{algocaption}{Algorithm \thenalg}

% defines the algorithm listing environment
\lstnewenvironment{pseudocode}[1][] {
    \refstepcounter{nalg}  % increments algorithm number
    \captionsetup{font=normalsize, labelformat=algocaption, labelsep=colon}
    \lstset{
        breaklines=true,
        mathescape=true,
        numbers=left,
        numberstyle=\scriptsize,
        basicstyle=\footnotesize\ttfamily,
        keywordstyle=\color{black}\bfseries,
        keywords={input, output, return, parallel, function, for, to, in, if,
        else, foreach, while, and, or, new, print},
        xleftmargin=.04\textwidth,
        #1
    }
}{}

\renewcommand{\familydefault}{\sfdefault}  % default font for entire document
 % specifies the document layout and style

\begin{document}

    \documentinfo{\today}

    \begin{enumerate}

        \item Describe the functions of the Memory Management Unit in a modern
        computer system.\\
        A Memory Management Unit (MMU) is a hardware unit used for all memory
        references including translating virtual addresses to physical
        addresses. The mapping of the virtual addresses can be achieved by many
        different methods. One such scheme is utilizing relocation registers.
        The MMU maps the logical address dynamically by adding the value in the
        relocation register and sent to memory.

        \item What is a virtual address space? Why do we use virtual address
        spaces in modern operating systems?\\
        Virtual address space is the address range which a process typically
        works in. The range spans from low (0x0) to as high as the instruction
        set architecture and the maximum pointer size allows. It is used
        because of data security via process isolation. Each process is given a
        separate virtual address space so that they do not access or interfere
        each other.

        \item Compare and contrast the use of paging and segmentation for
        memory management in an OS.\\
        Paging is a form of memory management that eliminates the need for
        contiguous allocation of physical memory. It retrieves data from the
        secondary storage for use in the main memory. Segmentation stores all
        the data in segments along with their memory locations. Every segments
        are loaded into a contiguous block of available memory. Paging requires
        more memory overhead to maintain the translation structures.
        Segmentation requires two registers per segment, while paging requires
        only one entry per page.

        \item Consider a logical address space of 64 pages of 1024 words each,
        mapped onto a physical memory of 32 frames of 1024 words each. How many
        bits are required to represent a logical address (assuming word-based
        addressing). \\
        Since logical address space = (number of pages in logical address
        space) $\times$ (page size), with the number of bits required being
        $\log_2$ of the size\\
        Then,
        \begin{align*}
            \text{logical address space} & = 64 \times 1024\\
            & = 2^{6} \times 2^{10}\\
            & = 2^{16}\\
            & = \log_2(2^{16}) = 16 \text{ bits}
        \end{align*}

        \item With the same address space layout as the previous question, how
        many bits are required to represent a physical address (once again,
        assuming word-based addressing).\\
        Since physical address space = (number of frames in physical address
        space) $\times$ (frame size), with the number of bits required being
        $\log_2$ of the size\\
        Then,
        \begin{align*}
            \text{physical address space} & = 32 \times 1024\\
            & = 2^{5} \times 2^{10}\\
            & = 2^{15}\\
            & = \log_2(2^{15}) = 15 \text{ bits}
        \end{align*}

    \end{enumerate}

\end{document}
