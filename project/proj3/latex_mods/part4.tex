\section{Part 1}
    \subsection{Question}
    An oceanographer wants to test, on the basis of a random sample of size 35, whether the average depth of the ocean in a certain area is 72.4 fathoms. At the 0.05 level of significance, what will the oceanographer decide if she gets a sample mean of 73.2? Assume the population standard deviation is 2.1.

    \subsection{Answer}
    The null hypothesis, $H_{0}$, claims the mean depth of the ocean in a certain area is 72.4, while the alternative hypothesis, $H_{a}$, says otherwise.

        \[ H_{0}: \mu = 72.4 \ vs \ H_{a}: \mu \neq 72.4 \]

    Since the population mean and standard deviation was known with a sample size of $n > 30$, the Z-score was calculated as follows:

        \begin{equation*}
    Z=\frac{\overline{X}-\mu}{\sfrac{\sigma}{\sqrt{n}}}
    =\frac{73.2-72.4}{\sfrac{2.1}{\sqrt{35}}}=2.2537
    \end{equation*}\newline

    The following snippet was used to generate the Z-value and its probability:
\begin{lstlisting}
    X <- 73.2
    mu <- 72.4
    sigma <- 2.1
    n <- 35

    z <- (X - mu)/(sigma/sqrt(n))
    print(z)  # print the Z-score
    print(pnorm(z))  # print the probability
\end{lstlisting}

    The test statistic was computed to be:

    \begin{equation*}
        \pm Z_{0.05}=\pm 1.96 < 2.2537
    \end{equation*}

    The p-value was computed with the following snippet:
\begin{lstlisting}
    pScore <- 2 * (1 - (pnorm(score)))
    # 0.0242
\end{lstlisting}

    Since $Z_{\sfrac{\alpha}{2}} < Z$ and the p-score was under 0.05, the null hypothesis is rejected.
