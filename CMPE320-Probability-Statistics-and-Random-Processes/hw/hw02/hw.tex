%%%%%%%%%%%%%%%%%%%%%%%%%%%%%%%%%%%%%%%%%
% Template
% LaTeX Template
% Version 1.0 (December 8 2014)
%
% This template has been downloaded from:
% http://www.LaTeXTemplates.com
%
% Original author:
% Brandon Fryslie
% With extensive modifications by:
% Vel (vel@latextemplates.com)
%
% License:
% CC BY-NC-SA 3.0 (http://creativecommons.org/licenses/by-nc-sa/3.0/)
%
% Authors:
% Sabbir Ahmed
%
%%%%%%%%%%%%%%%%%%%%%%%%%%%%%%%%%%%%%%%%%

\documentclass[paper=usletter, fontsize=12pt]{article}
%%%%%%%%%%%%%%%%%%%%%%%%%%%%%%%%%%%%%%%%%
% Contract Structural Definitions File Version 1.0 (December 8 2014)
%
% Created by: Vel (vel@latextemplates.com)
% 
% This file has been downloaded from: http://www.LaTeXTemplates.com
%
% License: CC BY-NC-SA 3.0 (http://creativecommons.org/licenses/by-nc-sa/3.0/)
%
%%%%%%%%%%%%%%%%%%%%%%%%%%%%%%%%%%%%%%%%%

\usepackage{geometry} % Required to modify the page layout
\usepackage{multicol}
\usepackage{amsmath}
\usepackage{amssymb}

\usepackage[pdftex]{graphicx}
\usepackage{wrapfig}
\usepackage[font=scriptsize, labelfont=bf]{caption}
\usepackage[utf8]{inputenc} % Required for including letters with accents
\usepackage[T1]{fontenc} % Use 8-bit encoding that has 256 glyphs

\usepackage{avant} % Use the Avantgarde font for headings
\usepackage{xparse}
\usepackage{xcolor}
\usepackage{listings}  % for code verbatim and console outputs

\setlength{\textwidth}{16cm} % Width of the text on the page
\setlength{\textheight}{23cm} % Height of the text on the page
\setlength{\oddsidemargin}{0cm} % Width of the margin - negative to move text left, positive to move it right
\setlength{\topmargin}{-1.25cm} % Reduce the top margin

\setlength{\parindent}{0mm} % Don't indent paragraphs
\setlength{\parskip}{2.5mm} % Whitespace between paragraphs
\renewcommand{\baselinestretch}{1.2}

\renewcommand\familydefault{\sfdefault}  % default font for entire document

\definecolor{green}{rgb}{0.18, 0.55, 0.34}

\graphicspath{ {figures/} }
\captionsetup[table]{skip=10pt}

\lstset{language=C, keywordstyle={\bfseries \color{black}}}

% defines algorithm counter for chapter-level
\newcounter{nalg}[section]

%defines appearance of the algorithm counter
\renewcommand{\thenalg}{\thesection .\arabic{nalg}}

% defines a new caption label as Algorithm x.y
\DeclareCaptionLabelFormat{algocaption}{Algorithm \thenalg}

%defines the algorithm listing environment
\lstnewenvironment{pseudocode}[1][] {
    \refstepcounter{nalg} %increments algorithm number

    \captionsetup{labelformat=algocaption,labelsep=colon}
    \lstset{
        mathescape=true,
        frame=tB,
        numbers=left,
        numberstyle=\tiny,
        basicstyle=\scriptsize,
        keywordstyle=\color{black}\bfseries\em,
        keywords={,input, output, return, datatype, function, in, if, else, foreach, while, begin, end, },
        xleftmargin=.04\textwidth,
        #1
    }
}{}
 % specifies the document layout and style
\allowdisplaybreaks

%------------------------------------------------------------------------------
% document info command
\newcommand{\documentinfo}[5]{
    \begin{centering}
        \parbox{2in}{
        \begin{spacing}{1}
            \begin{flushleft}
                \begin{tabular}{l l}
                    #1 \\
                    #2 \\
                    #3 \\
                \end{tabular}\\
                \rule{\textwidth}{1pt}
            \end{flushleft}
        \end{spacing}
        }
    \end{centering}
}

\newcommand{\ans}{\textbf{Answer} \ }

\begin{document}

    \documentinfo{Sabbir Ahmed}{\textbf{DATE:} \today}{\textbf{CMPE 320} HW 02}
    \vspace{-0.2in}

    \begin{enumerate}

        % 1
        \item The following problem was given to 60 students and doctors at the
        famous Hevardi Medical School (HMS): Assume there exists a test to
        detect a disease, say $D$, whose prevalence is 0.001, that is, the
        probability, $P[D]$, that a person picked at random is suffering from
        $D$, is 0.001 .The test has a false positive rate of 0.005 and a
        correct detection rate of 1. The correct detection rate is the
        probability that if you have $D$, the test will say that you have $D$.
        Given that you test positive for $D$, what is the probability that you
        actually have it? Many of the HMS experts answered 0.95 and the average
        answer was 0.56. Show that your knowledge of probability is greater
        than that of the HMS experts by getting the right answer of 0.17.
        \item[\textbf{Ans}]
        \begin{proof}[\unskip\nopunct]
            Given: \\
            $P(D) = 0.001$, $P(D^C) = 1 - P(D) = 0.999$ \\
            $P(\text{positive} \given D) = 1$, $P(\text{positive} \given D^C) =
            0.005$
            Therefore, $P(D \given \text{positive})$:
            \begin{align*}
                P(D \given \text{positive}) & = \frac{P(\text{positive} \given D)P(D)}{P(\text{positive} \given D)P(D) + P(\text{positive} \given D^C)P(D^C)} \\
                & = \frac{(1)(0.001)}{(1)(0.001) + (0.005)(0.999)} \\
                & = 0.1668 \\
                & \approx 0.17 \qedhere
            \end{align*}
        \end{proof}
        \vspace{0.2in}

        % 2
        \item In the ternary communication channel shown below, a 3 is sent
        three times more frequently than a 1, and a 2 is sent two times more
        frequently than a 1. A 1 is observed; what is the conditional
        probability that a 1 was sent?
        \item[\textbf{Ans}]
        \begin{proof}[\unskip\nopunct]
            Let $P(\text{1}) = x$, so $P(\text{2}) = 2x$ and $P(\text{3}) =
            3x$. \\
            $P(\text{1}) + P(\text{2}) + P(\text{3}) = 1$ \\
            $x + 2x + 3x = 1 \Rightarrow x = 1/6$ \\
            Therefore, $P(\text{1}) = x = 1/6$ \\
            $P(\text{2}) = 2x = 1/3$ \\
            $P(\text{3}) = 3x = 1/2$ \\
            Therefore:
            \begin{align*}
                & P(\text{1 sent} \given \text{1 obs}) = \frac{P(\text{1
                sent and 1 obs})}{P(\text{1 obs})} \\
                & = \frac{P(\text{1 obs} \given \text{1 sent})P(\text{1
                sent})}{P(\text{1 obs} \given \text{1 sent})P(\text{1
                sent}) + P(\text{1 obs} \given \text{2 sent})P(\text{2
                sent}) + P(\text{1 obs} \given \text{3 sent})P(\text{3
                sent})} \\
                & = \frac{(1 - \alpha) \cdot \frac{1}{6}}{(1 - \alpha) \cdot \frac{1}{6} + \frac{\beta}{2} \cdot \frac{1}{3} + \frac{\gamma}{2} \cdot \frac{1}{2}} \\
                & = \frac{\frac{(1 - \alpha)}{6}}{\frac{(1 - \alpha)}{6} + \frac{\beta}{6} + \frac{\gamma}{4}} \\
                & = \frac{2(1 - \alpha)}{- 2\alpha + 2\beta + 3\gamma + 2} \qedhere
            \end{align*}
        \end{proof}
        \vspace{0.2in}

        % 3
        \item A large class in probability theory is taking a multiple-choice
        test. For a particular question on the test, the fraction of examinees
        who know the answer is $p$; $1-p$ is the fraction that will guess. The
        probability of answering a question correctly is unity for an examinee
        who knows the answer and $1/m$ for a guessee; $m$ is the number of
        multiple-choice alternatives. Compute the probability that an examinee
        knew the answer to a question given that he or she has correctly
        answered it.
        \item[\textbf{Ans}]
        \begin{proof}[\unskip\nopunct]
            Let $K$ represent the examinee knowing the correct answer, and $A$
            represent the examinee answering the correct answer. \\
            Therefore, $P(K \cap A) = 1$ since the examinee answered correctly
            if they knew the correct answer. $P(A)$ consists of the total examinees who know the correct answer and those who guessed with the $1/m$ probability.
            \begin{align*}
                P(K \given A) & = \frac{P(K \cap A)}{P(A)} \\
                & = \frac{1}{p + (1-p)\frac{1}{m}} \\
                & = \frac{1}{p + \frac{1}{m}-\frac{p}{m}} \\
                & = \frac{m}{mp -p + 1} \qedhere
            \end{align*}
        \end{proof}
        \vspace{0.2in}

        % 4
        \item Assume there are three machines A, B, and C in a semiconductor
        manufacturing facility that makes chips. They manufacture,
        respectively, 25, 35, and 40 percent of the total semiconductor chips
        there. Of their outputs, respectively, 5, 4, and 2 percent of the chips
        are defective. A chip is drawn randomly from the combined output of the
        three machines and is found defective. What is the probability that
        this defective chip is manufactured by machine A? by machine B? by
        machine C?
        \item[\textbf{Ans}]
        \begin{proof}[\unskip\nopunct]
            Given: $P(A) = 0.25, P(B) = 0.35, P(C) = 0.40$. \\
            Let $D$ represent the event of finding a defective chip.\\
            Therefore $P(D \given A) = 0.05, P(D \given B) = 0.04, P(D \given C) = 0.02$.
            \begin{align*}
                P(A \given D) & = \frac{P(D \given A)P(A)}{P(D \given A)P(A) +
                P(D \given B)P(B) + P(D \given C)P(C)} \\
                & = \frac{(0.05)(0.25)}{(0.05)(0.25) + (0.04)(0.35) +
                (0.02)(0.40)} \\
                & = 0.3623
            \end{align*}
            \begin{align*}
                P(B \given D) & = \frac{P(D \given B)P(B)}{P(D \given A)P(A) +
                P(D \given B)P(B) + P(D \given C)P(C)} \\
                & = \frac{(0.04)(0.35)}{(0.05)(0.25) + (0.04)(0.35) +
                (0.02)(0.40)} \\
                & = 0.4058
            \end{align*}
            \begin{align*}
                P(C \given D) & = \frac{P(D \given C)P(C)}{P(D \given A)P(A) +
                P(D \given B)P(B) + P(D \given C)P(C)} \\
                & = \frac{(0.02)(0.40)}{(0.05)(0.25) + (0.04)(0.35) +
                (0.02)(0.40)} \\
                & = 0.2319 \qedhere
            \end{align*}
        \end{proof}
        \vspace{0.2in}

        % 5
        \item  A card is randomly selected from a standard deck of 52 cards.
        Let $A$ be the event of selecting an ace and let $B$ be the event of
        selecting a red card. There are 4 aces and 26 red cards in the normal
        deck. Are $A$ and $B$ independent?
        \item[\textbf{Ans}]
        \begin{proof}[\unskip\nopunct]
            Given: $P(A) = 4/52$, $P(B) = 26/52$. \\
            If $P(A \cap B) = P(A) \cdot P(B)$, then $A$ and $B$ are
            independent. \\
            Since $A \cap B$ is equivalent to selecting red aces, $A \cap B =
            2$. \\
            Therefore,
            \begin{align*}
                P(A \cap B) & = \frac{2}{52} \\
                & = \frac{1}{26} \qedhere
            \end{align*}
        \end{proof}
        \vspace{0.2in}

        % 6
        \item A fair die is tossed three times. Given that a 2 appears on the
        first toss, what is the probability of obtaining the sum 7 on the three
        tosses?
        \item[\textbf{Ans}]
        \begin{proof}[\unskip\nopunct]
            Since the first toss resulted in a 2, the other tosses have to
            total to $7 - 2 = 5$ for the sum of the three tosses to be 7. \\
            Therefore the combinations: $(1,4), (4, 1), (2,3), (3,2)$ are the
            only ones that would work. \\ $\therefore P(\text{sum = 7}) = 4/(3
            \cdot 12) = 4/36 = 1/9$ \qedhere
        \end{proof}
        \vspace{0.2in}

        % 7
        \item An Internet access provider (IAP) owns two servers. Each server
        has a 50\% chance of being "down" independently of the other.
        Fortunately, only one server is necessary to allow the IAP to provide
        service to its customers, i.e., only one server is needed to keep the
        IAP's system up. Suppose a customer tries to access the Internet on
        four different occasions, which are sufficiently spaced apart in time,
        so that we may assume that the states of the system corresponding to
        these four occasions are independent. What is the probability that the
        customer will only be able to access the Internet on 3 out of the 4
        occasions?
        \item[\textbf{Ans}]
        \begin{proof}[\unskip\nopunct]
            Since $P(\text{Internet access}) = 1 - P(\text{no Internet
            access})$ \\
            $\Rightarrow 1 - P(\text{both servers not working})) = 1 -
            (0.5)(0.5) = 0.75$ \\
            Therefore, $P(\text{any one server not working})$:
            \begin{align*}
                P(\text{any one server not working}) & = \binom{4}{3} \cdot P(\text{no Internet access}) \cdot P(\text{Internet access}) \\
                & = \binom{4}{3} \bigg(\frac{3}{4}\bigg)^3
                \bigg(\frac{1}{4}\bigg)^1 \qedhere
            \end{align*}
        \end{proof}
        \vspace{0.2in}

        % 8
        \item A peculiar six-sided die has uneven faces. In particular, the
        faces showing 1 or 6 are 1 $\times$ 1.5 inches, the faces showing 2 or
        5 are 1 $\times$ 0.4 inches, the faces showing 3 or 4 are 0.4 $\times$
        1.5 inches. Assume that the probability of a particular face coming up
        is proportional to its area. We independently roll the die twice. What
        is the probability that we get doubles?
        \item[\textbf{Ans}]
        \begin{proof}[\unskip\nopunct]
            Area of faces showing 1 or 6: $(1 \times 1.5) \text{\ in}^2 = 1.5
            \text{\ in}^2$ \\
            Area of faces showing 2 or 5: $(1 \times 0.4) \text{\ in}^2 = 0.4
            \text{\ in}^2$ \\
            Area of faces showing 3 or 4: $(0.4 \times 1.5) \text{\ in}^2 = 0.6
            \text{\ in}^2$ \\
            Total area: $2(1.5 + 0.4 + 0.6) = 5 \text{\ in}^2$. \\
            \begin{align*}
                P(\text{1 or 6}) & = \frac{1.5}{5} = 0.3 \\
                P(\text{2 or 5}) & = \frac{0.4}{5} = 0.08 \\
                P(\text{1 or 6}) & = \frac{0.6}{5} = 0.12
            \end{align*}
            \begin{align*}
                P(\text{doubles}) & = P(\text{1})P(\text{1}) + P(\text{2})P(\text{2}) + P(\text{3})P(\text{3}) + P(\text{4})P(\text{4}) + P(\text{5})P(\text{5}) + P(\text{6})P(\text{6}) \\
                & = 2(0.3^2) + 2(0.08^2) + 2(0.12^2) \\
                & = 0.2216
            \end{align*}
            $\therefore P(\text{doubles}) = 0.2216$ \qedhere
        \end{proof}
        \vspace{0.2in}

    \end{enumerate}

\end{document}
