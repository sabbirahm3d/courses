%%%%%%%%%%%%%%%%%%%%%%%%%%%%%%%%%%%%%%%%%
% Template
% LaTeX Template
% Version 1.0 (December 8 2014)
%
% This template has been downloaded from:
% http://www.LaTeXTemplates.com
%
% Original author:
% Brandon Fryslie
% With extensive modifications by:
% Vel (vel@latextemplates.com)
%
% License:
% CC BY-NC-SA 3.0 (http://creativecommons.org/licenses/by-nc-sa/3.0/)
%
% Authors:
% Sabbir Ahmed
%
%%%%%%%%%%%%%%%%%%%%%%%%%%%%%%%%%%%%%%%%%

\documentclass[paper=usletter, fontsize=12pt]{article}
%%%%%%%%%%%%%%%%%%%%%%%%%%%%%%%%%%%%%%%%%
% Contract Structural Definitions File Version 1.0 (December 8 2014)
%
% Created by: Vel (vel@latextemplates.com)
% 
% This file has been downloaded from: http://www.LaTeXTemplates.com
%
% License: CC BY-NC-SA 3.0 (http://creativecommons.org/licenses/by-nc-sa/3.0/)
%
%%%%%%%%%%%%%%%%%%%%%%%%%%%%%%%%%%%%%%%%%

\usepackage{geometry} % Required to modify the page layout
\usepackage{multicol}
\usepackage{amsmath}
\usepackage{amssymb}

\usepackage[pdftex]{graphicx}
\usepackage{wrapfig}
\usepackage[font=scriptsize, labelfont=bf]{caption}
\usepackage[utf8]{inputenc} % Required for including letters with accents
\usepackage[T1]{fontenc} % Use 8-bit encoding that has 256 glyphs

\usepackage{avant} % Use the Avantgarde font for headings
\usepackage{courier}
\usepackage{xparse}
\usepackage{xcolor}
\usepackage{listings}  % for code verbatim and console outputs

\setlength{\textwidth}{16cm} % Width of the text on the page
\setlength{\textheight}{23cm} % Height of the text on the page
\setlength{\oddsidemargin}{0cm} % Width of the margin - negative to move text left, positive to move it right
\setlength{\topmargin}{-1.25cm} % Reduce the top margin

\setlength{\parindent}{0mm} % Don't indent paragraphs
\setlength{\parskip}{2.5mm} % Whitespace between paragraphs
\renewcommand{\baselinestretch}{1.5}

\definecolor{green}{rgb}{0.18, 0.55, 0.34}

\graphicspath{ {figures/} }
\captionsetup[table]{skip=10pt}

\lstset{language=C, keywordstyle={\bfseries \color{black}}}

% defines algorithm counter for chapter-level
\newcounter{nalg}[section]

%defines appearance of the algorithm counter
\renewcommand{\thenalg}{\thesection .\arabic{nalg}}

% defines a new caption label as Algorithm x.y
\DeclareCaptionLabelFormat{algocaption}{Algorithm \thenalg}

% defines the algorithm listing environment
\lstnewenvironment{pseudocode}[1][] {
    \refstepcounter{nalg}  % increments algorithm number
    \captionsetup{font=normalsize, labelformat=algocaption, labelsep=colon}
    \lstset{
        breaklines=true,
        mathescape=true,
        numbers=left,
        numberstyle=\scriptsize,
        basicstyle=\footnotesize\ttfamily,
        keywordstyle=\color{black}\bfseries,
        keywords={input, output, return, parallel, function, for, to, in, if,
        else, foreach, while, and, or, new, print},
        xleftmargin=.04\textwidth,
        #1
    }
}{}

\renewcommand{\familydefault}{\sfdefault}  % default font for entire document
 % specifies the document layout and style
\allowdisplaybreaks

%------------------------------------------------------------------------------
% document info command
\newcommand{\documentinfo}[5]{
    \begin{centering}
        \parbox{2in}{
        \begin{spacing}{1}
            \begin{flushleft}
                \begin{tabular}{l l}
                    #1 \\
                    #2 \\
                    #3 \\
                \end{tabular}\\
                \rule{\textwidth}{1pt}
            \end{flushleft}
        \end{spacing}
        }
    \end{centering}
}

\begin{document}

    \documentinfo{Sabbir Ahmed}{\textbf{DATE:} \today}{\textbf{CMPE 320:} HW 05}
    \vspace{-0.2in}

    \begin{enumerate}[label=\textbf{\arabic*}.]

        % 1
        \item
        A stock market trader buys 100 shares of stock A and 200 shares of
        stock B. Let $X$ and $Y$ be the price changes of A and B, respectively,
        over a certain time period. And assume that the joint PMF of $X$ and
        $Y$ is uniform over the set of integers $x$ and $y$ satisfying
        \begin{align*}
            -2 \le x \le 4 && -1 \le y-x \le 1.
        \end{align*}
        \begin{enumerate}[label=(\alph*)]

            \item Find the marginal PMFs and the means of $X$ and $Y$.
            \begin{proof}[\unskip\nopunct]
            \end{proof}
            \vspace{0.2in}

            \item Find the mean of the trader's profit.
            \begin{proof}[\unskip\nopunct]
            \end{proof}
            \vspace{0.2in}

        \end{enumerate}

        % 2
        \item
        The MIT football team wins any one game with probability $p$. and loses
        it with probability $1 - p$. Its performance in each game is
        independent of its performance in other games. Let $L_1$ be the number
        of losses before its first win, and let $L_2$ be the number of losses
        after its first win and before its second win. Find the joint PMF of
        $L_1$ and $L_2$.
        \begin{proof}[\unskip\nopunct]
        \end{proof}
        \vspace{0.2in}

        % 3
        \item
        A class of $n$ students takes a test in which each student gets an A
        with probability $p$, a B with probability $q$, and a grade below B
        with probability $1 - p - q$, independently of any other student. If
        $X$ and $Y$ are the numbers of students that get an A and a B,
        respectively. calculate the joint PMF $p_{x,y}$.

        % 4
        \item
        Your probability class has 250 undergraduate students and 50 graduate
        students. The probability of an undergraduate (or graduate) student
        getting an A is 1/3 (or 1/2, respectively). Let $X$ be the number of
        students that get an A in your class.
        \begin{enumerate}[label=(\alph*)]

            \item Calculate $E[X]$ by first finding the PMF of $X$
            \begin{proof}[\unskip\nopunct]
            \end{proof}
            \vspace{0.2in}

            \item Calculate $E[X]$ by viewing $X$ as a sum of random variables,
            whose mean is easily calculated.
            \begin{proof}[\unskip\nopunct]
            \end{proof}
            \vspace{0.2in}

        \end{enumerate}

        % 5
        \item
        A scalper is considering buying tickets for a particular game. The
        price of the tickets is \$75, and the scalper will sell them at \$150.
        However, if she can't sell them at \$150, she won't sell them at all.
        Given that the demand for tickets is a binomial random variable with
        parameters $n=10$ and $p=1/2$, how many tickets should she buy in order
        to maximize her expected profit?

        % 6
        \item
        Suppose that $X$ and $Y$ are independent discrete random variables with
        the same geometric PMF:
        \begin{align*}
            p_X(k)=p_Y(k)=p(1-p)^{k-1}&& k=1,2,\ldots,
        \end{align*}
        where $p$ is a scalar with $0,p<1$. Show that for any integer $n \ge
        2$, the conditional PMF
        \begin{equation*}
            P(X=k \mid X+Y=n)
        \end{equation*}
        is uniform.

        % 7
        \item
        Consider four independent rolls of a 6-sides die. Let $X$ be the number
        of 1s and let $Y$ be the number of 2s obtained. What is the joint PMF
        of $X$ and $Y$?

        % 8
        \item
        Alvin shops for probability books for $K$ hours, where $K$ is a random
        variable that is equally likely to be 1, 2, 3, or 4. The number of
        books $N$ that he buys is random and depends on how long he shops
        according to the conditional PMF
        \begin{align*}
            p_{N \mid K}(n \mid k) = \frac{1}{k}, && \text{ for } n=1,\ldots,k
        \end{align*}
        \begin{enumerate}[label=(\alph*)]

            \item Find the joint PMF of $K$ and $N$
            \begin{proof}[\unskip\nopunct]
            \end{proof}
            \vspace{0.2in}

            \item Find the marginal PMF of $N$
            \begin{proof}[\unskip\nopunct]
            \end{proof}
            \vspace{0.2in}

            \item Find the conditional PMF of $K$ given that $N=2$
            \begin{proof}[\unskip\nopunct]
            \end{proof}
            \vspace{0.2in}

            \item Find the conditional mean and variance of $K$, given that he
            bought at least 2 but no more than 3 books.
            \begin{proof}[\unskip\nopunct]
            \end{proof}
            \vspace{0.2in}

            \item The cost of each book is a random variable with mean \$30.
            What is the expected value of his total expenditure? \textit{Hint:}
            Condition on the events $\{N=1\},\ldots, \{N=4\}$, and use the
            total expectation theorem.
            \begin{proof}[\unskip\nopunct]
            \end{proof}
            \vspace{0.2in}

        \end{enumerate}

    \end{enumerate}

\end{document}
