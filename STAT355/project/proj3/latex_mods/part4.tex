\section{Part 4}
    \subsection{Question}
    Suppose the lengths in millimeters of metal fibers produced by a certain process have a normal distributions for which the mean was 5.2 and the variance 0.8. After a system upgrade of the process, the engineer wants to test whether the mean length and the variance have changed or not. He took a sample of 15 fibers and measured. The sample mean was 5.4 and the sample variance was 1.0. Based on these measurements, state $H_{0}$ and $H_{a}$ and conduct a test to verify the engineer's concern with $\alpha$ = 0.05.

    \subsection{Answer}
    The null hypothesis, $H_{0}$, claims the mean of the lengths of the metal fibers after the upgrade is still 5.2, while the alternative hypothesis, $H_{a}$, says otherwise.

        \[ H_{0}: \mu = 5.2 \ vs \ H_{a}: \mu \neq 5.2 \]

    Since the population mean and standard deviation are known with a sample size of $n < 30$, the Z-score was calculated as follows:

        \begin{equation*}
    t=\frac{\overline{X}-\mu}{\sfrac{\sigma}{\sqrt{n}}}
    =\frac{5.4-5.2}{\sfrac{0.89}{\sqrt{15}}}=0.8660
    \end{equation*}\newline

    The following snippet was used to generate the t-test and its probability:
\begin{lstlisting}
    X <- 5.4
    mu <- 5.2
    sigma <- sqrt(0.8)
    n <- 15

    dumpComputation(X=X, mu=mu, sigma=sigma, n=n, 
        alpha=alpha, distType="t", twoSided=TRUE, "part4")
\end{lstlisting}

    The test statistic was computed to be:

    \begin{equation*}
        t_{0.025, 14}=-2.14 < 0.8660 < t_{1-0.025, 14}=2.14
    \end{equation*}

    The p-value was computed with the following snippet:
\begin{lstlisting}
    pScore <- 2 * (1 - (pt(score, df=n-1)))
    # 0.4011
\end{lstlisting}

    Since $t_{\sfrac{\alpha}{2}} < t < t_{1-\sfrac{\alpha}{2}}$ and the p-score was well over 0.05, there is strong evidence to not reject the null hypothesis.\n

    The null hypothesis for the variance claims the population variance is 0.8, while the alternative hypothesis proposes otherwise.

        \[ H_{0}: \sigma^{2} = 0.8 \ vs \ H_{a}: \sigma^{2} \neq 0.8 \]

    As a separate test, the chi-squared hypothesis test is used:

    \begin{equation*}
        T=(n-1)\frac{s^{2}}{\sigma^{2}}
        =(14)\frac{1}{0.64}=21.8750
    \end{equation*}

    The test statistic was computed to be:

    \begin{equation*}
        {\chi}^{2}_{0.025, 14}=5.63<21.8750
    \end{equation*}

    The p-value for the Chi-squared test was computed with the following snippet:

\begin{lstlisting}
    pScore <- 2 * (1 - (pchisq(score, df=n-1)))
    # 0.1624
\end{lstlisting}

    The high p-value also suggests sufficient evidence to not reject the null hypothesis based on the variance testing.