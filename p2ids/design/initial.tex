%
% CMSC 421: Project 2 Initial Design Document
% Template layout and preamble defined in structure.tex
%
% Author: Sabbir Ahmed
%

\documentclass[paper=usletter, fontsize=12pt]{article}
%%%%%%%%%%%%%%%%%%%%%%%%%%%%%%%%%%%%%%%%%
% Contract Structural Definitions File Version 1.0 (December 8 2014)
%
% Created by: Vel (vel@latextemplates.com)
% 
% This file has been downloaded from: http://www.LaTeXTemplates.com
%
% License: CC BY-NC-SA 3.0 (http://creativecommons.org/licenses/by-nc-sa/3.0/)
%
%%%%%%%%%%%%%%%%%%%%%%%%%%%%%%%%%%%%%%%%%

\usepackage{geometry} % Required to modify the page layout
\usepackage{multicol}
\usepackage{amsmath}
\usepackage{amssymb}

\usepackage[pdftex]{graphicx}
\usepackage{wrapfig}
\usepackage[font=scriptsize, labelfont=bf]{caption}
\usepackage[utf8]{inputenc} % Required for including letters with accents
\usepackage[T1]{fontenc} % Use 8-bit encoding that has 256 glyphs

\usepackage{avant} % Use the Avantgarde font for headings
\usepackage{xparse}
\usepackage{xcolor}
\usepackage{listings}  % for code verbatim and console outputs

\setlength{\textwidth}{16cm} % Width of the text on the page
\setlength{\textheight}{23cm} % Height of the text on the page
\setlength{\oddsidemargin}{0cm} % Width of the margin - negative to move text left, positive to move it right
\setlength{\topmargin}{-1.25cm} % Reduce the top margin

\setlength{\parindent}{0mm} % Don't indent paragraphs
\setlength{\parskip}{2.5mm} % Whitespace between paragraphs
\renewcommand{\baselinestretch}{1.2}

\renewcommand\familydefault{\sfdefault}  % default font for entire document

\definecolor{green}{rgb}{0.18, 0.55, 0.34}

\graphicspath{ {figures/} }
\captionsetup[table]{skip=10pt}

\lstset{language=C, keywordstyle={\bfseries \color{black}}}

% defines algorithm counter for chapter-level
\newcounter{nalg}[section]

%defines appearance of the algorithm counter
\renewcommand{\thenalg}{\thesection .\arabic{nalg}}

% defines a new caption label as Algorithm x.y
\DeclareCaptionLabelFormat{algocaption}{Algorithm \thenalg}

%defines the algorithm listing environment
\lstnewenvironment{pseudocode}[1][] {
    \refstepcounter{nalg} %increments algorithm number

    \captionsetup{labelformat=algocaption,labelsep=colon}
    \lstset{
        mathescape=true,
        frame=tB,
        numbers=left,
        numberstyle=\tiny,
        basicstyle=\scriptsize,
        keywordstyle=\color{black}\bfseries\em,
        keywords={,input, output, return, datatype, function, in, if, else, foreach, while, begin, end, },
        xleftmargin=.04\textwidth,
        #1
    }
}{}
 % specifies the document layout and style

\begin{document}

    \documentinfo{April 15, 2018}{Initial}

    % import commmon sections on Introduction
    %
% CMSC 421: Project 2 Introduction
% Template layout and preamble defined in structure.tex
%
% Author: Sabbir Ahmed
%

\section{Introduction} This project implements a new version of the Linux
kernel that adds functionality to support a simple intrusion detection
system (IDS). This system will operate by logging the system calls made by
a process in the kernel, while analysis and intrusion detection will be
done in user space. This assignment is designed to teach a simple method of
intrusion detection, as well as to reinforce the idea of how user space and
kernel space interact through the use of system calls.

An intrusion detection system is a computer program that attempts to
identify (and thwart) attacks that might be performed on the system by
attackers. There are several time-tested approaches to the development of
an IDS. The project will keep track of the system calls made by a monitored
process and check for abnormalities in the sequences of system calls made.
When an attacker breaks into a process, they will need to make system calls
in order to attempt to access the resources of the system that are under
attack. As the system calls that the attacker will perform will likely be
different than those performed by a process that is not under attack, it
follows that by monitoring both healthy and broken processes, it is
possible to develop a scheme to identify those that might be under attack
for further action to be taken.

\section{Objective} The project will compare sequences of system calls made
by a monitored process to known good sequences.

\subsection{Kernel Space Requirements} The kernel-space program will
instrument the system call dispatcher of the Linux kernel with code that
logs each time a system call is made. Built-in system calls such as
\texttt{ptrace} are not allowed to trace the usage of system calls to
generate the logs for the project.

\subsection{User Space Requirements} The analysis of the logs will be
handled by the user-space program, which may be implemented in any
supported programming language. The user-space process should construct a
bit array for each process under monitoring showing which system calls have
been run in a window of the last $k$ system calls. If a particular system
call is made in the window, the bit for that system call will be set to 1.
The bit arrays will then be measured for their hamming distance with the
example of a healthy system call sequence for a process.


    \section{Design Approach} Development in the preliminary stages began with
    research into the topics involved in the project. Understanding the
    objectives and scope of the project to identify the requirements was the
    initial step taken for the project.

    The next step in development involved outlining the list of tasks necessary
    to complete the project within the milestone deadlines. After considering
    several approaches, it was decided that the user-space program will be
    completed and tested for its functionality first, concurrently while
    conducting research on the Linux system call dispatcher.

    \subsection{Kernel Space} Initially, the logging of the system calls were
    perceived as a non-trivial but moderately difficult task that would not
    require much time committed. After preliminary research, it was evident
    that without the built-in system calls such as \texttt{ptrace}, logging
    system calls would require much greater effort.

    The kernel-space program will trace the system calls and dump them on a log
    consisting of the \texttt{pid}, the system call numbers, and their
    corresponding timestamps.

    Due to the extensive research required to build the kernel-level logging
    system, the development of its counterpart user-space program would be
    completed first. Research on plausible methods to achieve the logging of
    the system calls would resume simultaneously.

    Two approaches are currently being considered to influence the flow of the
    research. Regardless of the approach chosen to develop the functionality in
    the kernel space, extensive research on the system call dispatcher is
    required.

    \subsubsection{Trivially Insert Breakpoints to the System Call
    Implementations} A potential but not very likely approach would involve
    fully understanding the implementation of all the system calls that would
    be used in the logging system. Once their implementation is understood,
    breakpoints would be added to suitable areas to pause the system call while
    it indicates to the log of its activity. This approach was being considered
    seriously, until it was apparent that it would not be feasible with the
    number of system calls involved. This would also increase complications
    when attempting to bind system calls. Adding breakpoints may involve
    modifying numerous assembly files, such as inserting \texttt{jmp}
    instructions to the writing system calls to write to the log.

    Even though this approach appears very unlikely, it is not being completely
    ruled out because of the potential uses of parts of its ideology, such as
    using the system calls table to identify implementations of each system
    calls.

    \subsubsection{Reverse Engineer and Create A Stripped-Down Version of
    \texttt{ptrace}} The more probable approach involves fully understanding
    \texttt{<linux/ptrace.h>} and simulate it for the project. The built-in
    system call offers several functionalities that may not be required for the
    logging system. Creating a stripped-down version of the system call would
    most likely allow the system call logging functionality for the IDS.

    \subsection{User Space} The user-space program will be implemented in
    Python 2 without any third-party libraries. Time-permitting, the
    implementation will be translated to C to trade off the lines of codes for
    speed.

    The program will continually parse the log file shared with the kernel-
    space program while properly utilizing mutex locks to avoid any
    synchronization issues. The user-space program will begin parsing the logs
    only after it has been closed by the kernel-space program. Once the lines
    of system calls for a process are properly parsed, they will be converted
    to their binary vector representations. The binary vectors will then be
    analyzed for their hamming distances, and the results will be logged in a
    separate file.

    Additional Python 2 and shell scripts will be added to assist in properly
    formatting the outputs and managing the process as it runs in the
    background.

\end{document}
