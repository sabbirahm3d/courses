%%%%%%%%%%%%%%%%%%%%%%%%%%%%%%%%%%%%%%%%%%%%%%%%%%%%%%%%%%%%%%%%%%%%%%%%%%%%%%
% Report LaTeX Template Version 1.0 (December 8 2014)
%
% This template has been downloaded from: http://www.LaTeXTemplates.com
%
% Original author: Brandon Fryslie With extensive modifications by: Vel
% (vel@latextemplates.com)
%
% License: CC BY-NC-SA 3.0 (http://creativecommons.org/licenses/by-nc-sa/3.0/)
%
%%%%%%%%%%%%%%%%%%%%%%%%%%%%%%%%%%%%%%%%%%%%%%%%%%%%%%%%%%%%%%%%%%%%%%%%%%%%%%

\documentclass[usletter, 12pt]{article}
%%%%%%%%%%%%%%%%%%%%%%%%%%%%%%%%%%%%%%%%%
% Contract Structural Definitions File Version 1.0 (December 8 2014)
%
% Created by: Vel (vel@latextemplates.com)
% 
% This file has been downloaded from: http://www.LaTeXTemplates.com
%
% License: CC BY-NC-SA 3.0 (http://creativecommons.org/licenses/by-nc-sa/3.0/)
%
%%%%%%%%%%%%%%%%%%%%%%%%%%%%%%%%%%%%%%%%%

\usepackage{geometry} % Required to modify the page layout
\usepackage{multicol}
\usepackage{amsmath}
\usepackage{amssymb}

\usepackage[pdftex]{graphicx}
\usepackage{wrapfig}
\usepackage[font=scriptsize, labelfont=bf]{caption}
\usepackage[utf8]{inputenc} % Required for including letters with accents
\usepackage[T1]{fontenc} % Use 8-bit encoding that has 256 glyphs

\usepackage{avant} % Use the Avantgarde font for headings
\usepackage{courier}
\usepackage{xparse}
\usepackage{xcolor}
\usepackage{listings}  % for code verbatim and console outputs

\setlength{\textwidth}{16cm} % Width of the text on the page
\setlength{\textheight}{23cm} % Height of the text on the page
\setlength{\oddsidemargin}{0cm} % Width of the margin - negative to move text left, positive to move it right
\setlength{\topmargin}{-1.25cm} % Reduce the top margin

\setlength{\parindent}{0mm} % Don't indent paragraphs
\setlength{\parskip}{2.5mm} % Whitespace between paragraphs
\renewcommand{\baselinestretch}{1.5}

\definecolor{green}{rgb}{0.18, 0.55, 0.34}

\graphicspath{ {figures/} }
\captionsetup[table]{skip=10pt}

\lstset{language=C, keywordstyle={\bfseries \color{black}}}

% defines algorithm counter for chapter-level
\newcounter{nalg}[section]

%defines appearance of the algorithm counter
\renewcommand{\thenalg}{\thesection .\arabic{nalg}}

% defines a new caption label as Algorithm x.y
\DeclareCaptionLabelFormat{algocaption}{Algorithm \thenalg}

% defines the algorithm listing environment
\lstnewenvironment{pseudocode}[1][] {
    \refstepcounter{nalg}  % increments algorithm number
    \captionsetup{font=normalsize, labelformat=algocaption, labelsep=colon}
    \lstset{
        breaklines=true,
        mathescape=true,
        numbers=left,
        numberstyle=\scriptsize,
        basicstyle=\footnotesize\ttfamily,
        keywordstyle=\color{black}\bfseries,
        keywords={input, output, return, parallel, function, for, to, in, if,
        else, foreach, while, and, or, new, print},
        xleftmargin=.04\textwidth,
        #1
    }
}{}

\renewcommand{\familydefault}{\sfdefault}  % default font for entire document
 % Input the structure.tex file which specifies the document layout and style

\newcommand{\project}{Homework 4: \\ Follow The Light}
\newcommand{\Sabbir}{Sabbir Ahmed}

\begin{document}

    \begin{titlepage}

        \vspace*{\fill} % Add whitespace above to center the title page content
        \begin{center}

            {\LARGE \project~Report}\\ [1.5cm]

            Submitted: \today
            
            \vspace*{\fill}

            \Sabbir

        \end{center}
        \vspace*{\fill} % Add whitespace below to center the title page content

    \end{titlepage}

    \section{Description} This project utilized a mechanical servo mounted with
    a Light-Sensitive resistor to mechanically respond to a moving light. \\~\\
    \noindent The servo's movement range is about 180 degrees and is controlled
    with a PWM signal. The frequency should be around 125 Hz and the pulse
    width should correspond to approximately 6\% to 30\% duty cycle. The servo
    moves the arm in the range according to the width of the pulse. \\~\\
    \noindent The system is primary defined by two behaviors FULLSWEEP,
    LOCALSWEEP and two modes, FOLLOW THE LIGHT and AVOID THE LIGHT. \\~\\
    \noindent \textbf{Modes} \\
    In one mode, Follow The Light, the objective is to find the angle of most
    illumination. The other mode, Avoid The Light, the objective is to find the
    angle of least illumination. The mode is set via a user menu using the
    onboard joystick and LCD. The mode should be displayed as \texttt{FTL}
    (follow the light) or \texttt{ATL} (avoid the light). Up and down button
    presses select the mode of operation. A right button press should initiate
    a full sweep. \\~\\
    \noindent \textbf{Behaviors} \\
    In each mode, there are two behaviors: \texttt{FULLSWEEP} and
    \texttt{LOCALSWEEP}. The \texttt{FULLSWEEP} behavior is automatically
    followed by the \texttt{LOCALSWEEP} behavior. While in the
    \texttt{FULLSWEEP} behavior, the system should not respond to any button
    presses. \\

    \noindent \textbf{FULLSURVEY} A full sweep of the range should be performed
    starting at 0$^{\circ}$ and sweep to 180$^{\circ}$ in about 20$^{\circ}$
    increments. An illumination measurement should be taken at each step. The
    optimum angle (most light or least light depending on the mode) is decided
    and called the initial primary angle and is displayed on the left 3
    character positions on the LCD as a value in the range [0,180]. The next
    behavior should then commence.

    \noindent \textbf{LOCALSWEEP} Starting at the primary angle, the servo should continuously move in the cycle given below, finding
    the optimum of the three angles.

        \[ \text{primary angle} \rightarrow (\text{primary angle} - 10^\circ)
        \rightarrow (\text{primary angle} + 10^\circ) \]

    \noindent After each cycle, a new primary angle should be decided and
    displayed. If the primary angle is at the limit of the servo range, the
    local search pattern will not involve one of the +10$^{\circ}$ or
    -10$^{\circ}$ measurements. \\~\\
    \noindent This document serves as the final report for the project. The
    report details the implementation of the project using C with \texttt{avr-gcc} and its development in a full Linux environment and the Atmel
    development environment.

    \section{Implementation} The project was developed with a bottom-up
    design. Functions with more frequent usage and higher priority were
    developed first, and later pieced together to contribute to higher level
    functionalities. Because of reasons later detailed in the Troubleshooting
    section, the entire source code had to be contained in a single
    \texttt{main.c} script.

    \subsection{Light Sensor} The light sensor connected to PORTB utilized the
    ADC functionalities. The ADC value was used to determine the brightness of
    the light source, where more light corresponded to a smaller value whereas
    less light corresponded to a larger value. The value was used to control
    the direction of rotation of the servo mounted. \\~\\

    \subsection{Memory Management} The implementation emphasized the limited
    memory on the chip. Usage of strings were minimized, and menu strings were
    stored as constant \codeword{PROGMEM} variables for reuse. The program
    memory was also utilized to communicate with the LCD with
    \codeword{lcd_puts_P}.

    \section{Usage} The project depends on user inputs from the Butterfly
    joystick. The program sits in a loop until the \texttt{RIGHT} joystick
    button is pressed. The user may press the \texttt{UP} or \texttt{DOWN}
    joystick buttons to toggle between the \texttt{FTL} and \texttt{ATL} modes.
    After a mode is selected, the servo begins sweeping with its corresponding
    configuration. The program terminates 10 seconds after a
    \texttt{LOCALSWEEP} followed by a \texttt{FULLSWEEP} are executed.

    \section{Testing and Troubleshooting} Extensive usage of hardware was
    required for debugging the functionalities of the program. 

        \subsection{Light Sensor}


        \subsection{JTAG Cable}

        \subsection{Servo}

    \section{Code} The C scripts used for the implementation has been attached
    alongside the report.

        \subsection{main.c} The entire source code of the project.

\end{document}
