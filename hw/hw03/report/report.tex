%%%%%%%%%%%%%%%%%%%%%%%%%%%%%%%%%%%%%%%%%
% Report LaTeX Template Version 1.0 (December 8 2014)
%
% This template has been downloaded from: http://www.LaTeXTemplates.com
%
% Original author: Brandon Fryslie With extensive modifications by: Vel
% (vel@latextemplates.com)
%
% License: CC BY-NC-SA 3.0 (http://creativecommons.org/licenses/by-nc-sa/3.0/)
%
%%%%%%%%%%%%%%%%%%%%%%%%%%%%%%%%%%%%%%%%%

\documentclass[usletter, 12pt]{article}
%%%%%%%%%%%%%%%%%%%%%%%%%%%%%%%%%%%%%%%%%
% Contract Structural Definitions File Version 1.0 (December 8 2014)
%
% Created by: Vel (vel@latextemplates.com)
% 
% This file has been downloaded from: http://www.LaTeXTemplates.com
%
% License: CC BY-NC-SA 3.0 (http://creativecommons.org/licenses/by-nc-sa/3.0/)
%
%%%%%%%%%%%%%%%%%%%%%%%%%%%%%%%%%%%%%%%%%

\usepackage{geometry} % Required to modify the page layout
\usepackage{multicol}
\usepackage{amsmath}
\usepackage{amssymb}

\usepackage[pdftex]{graphicx}
\usepackage{wrapfig}
\usepackage[font=scriptsize, labelfont=bf]{caption}
\usepackage[utf8]{inputenc} % Required for including letters with accents
\usepackage[T1]{fontenc} % Use 8-bit encoding that has 256 glyphs

\usepackage{avant} % Use the Avantgarde font for headings
\usepackage{courier}
\usepackage{xparse}
\usepackage{xcolor}
\usepackage{listings}  % for code verbatim and console outputs

\setlength{\textwidth}{16cm} % Width of the text on the page
\setlength{\textheight}{23cm} % Height of the text on the page
\setlength{\oddsidemargin}{0cm} % Width of the margin - negative to move text left, positive to move it right
\setlength{\topmargin}{-1.25cm} % Reduce the top margin

\setlength{\parindent}{0mm} % Don't indent paragraphs
\setlength{\parskip}{2.5mm} % Whitespace between paragraphs
\renewcommand{\baselinestretch}{1.5}

\definecolor{green}{rgb}{0.18, 0.55, 0.34}

\graphicspath{ {figures/} }
\captionsetup[table]{skip=10pt}

\lstset{language=C, keywordstyle={\bfseries \color{black}}}

% defines algorithm counter for chapter-level
\newcounter{nalg}[section]

%defines appearance of the algorithm counter
\renewcommand{\thenalg}{\thesection .\arabic{nalg}}

% defines a new caption label as Algorithm x.y
\DeclareCaptionLabelFormat{algocaption}{Algorithm \thenalg}

% defines the algorithm listing environment
\lstnewenvironment{pseudocode}[1][] {
    \refstepcounter{nalg}  % increments algorithm number
    \captionsetup{font=normalsize, labelformat=algocaption, labelsep=colon}
    \lstset{
        breaklines=true,
        mathescape=true,
        numbers=left,
        numberstyle=\scriptsize,
        basicstyle=\footnotesize\ttfamily,
        keywordstyle=\color{black}\bfseries,
        keywords={input, output, return, parallel, function, for, to, in, if,
        else, foreach, while, and, or, new, print},
        xleftmargin=.04\textwidth,
        #1
    }
}{}

\renewcommand{\familydefault}{\sfdefault}  % default font for entire document
 % Input the structure.tex file which specifies the document layout and style

\newcommand{\project}{Homework 3: \\ Grade Database Management}
\newcommand{\Sabbir}{Sabbir Ahmed}

\begin{document}

    \begin{titlepage}

        \vspace*{\fill} % Add whitespace above to center the title page content
        \begin{center}

            {\LARGE \project~Report}\\ [1.5cm]

            Submitted: \today
            
            \vspace*{\fill}

            \Sabbir

        \end{center}
        \vspace*{\fill} % Add whitespace below to center the title page content

    \end{titlepage}

    \section{Description} This project simulates a simple student and grade
    database management system using linked lists. \\~\\
    \noindent The program will read in a text file of students and their
    grades. Each student will have their own node in a linked list containing
    the student's name, a list of their grades and their grade average. A user
    will be able to search for a student's name via the command line. The user
    will also be able search a grade and to list out all students with that
    grade. Along with being able to search for a student or grade, the user
    will also be able to add and remove students from the database. \\~\\
    \noindent This document serves as the final report for the project. The
    report details the implementation of the project using C in a full Linux
    environment. No AVR libraries or environments were used in the development.

    \section{Implementation} The project was developed with a bottom-up
    design. Functions with more frequent usage and higher priority were
    developed first, and later pieced together to contribute to higher level
    functionalities.

        \subsection{database} The nodes and linked list containers making up
        the database were developed first. Basic functionality of the linked
        lists, such as adding and removing nodes were developed in the initial
        stages, followed by an iterator wrapper to help print out the contents
        of the linked list. \\~\\
        \noindent The concepts of using iterators and doubly-linked
        lists were influenced from \texttt{Clibs} \cite{clibs}. No code was
        borrowed from their implementation. Using a doubly-linked list allowed
        efficiency in the \codeword{db_remove()} method. Linking the previous
        node of the current node queued for removal could be referenced in
        constant time; whereas a singly-linked would have to iterate through
        the entire linked list to reach the previous node of the current node.
        \\~\\
        \noindent The sorting functionality was implemented using a simple
        bubble sort algorithm modified from the snippet provided by
        \texttt{GeeksforGeeks} \cite{gfg}.

        \subsection{list\_of\_grades} The nodes and linked list containers
        making up the \texttt{grade\_t} structures were developed later.
        \texttt{grade\_t} was implemented as a struct unionizing the two
        \texttt{quiz\_t} and \texttt{test\_t} structs.

        \subsection{Memory Management} \texttt{valgrind} was used to detect and
        track memeory leaks of the program. An essential rule of thumb was
        developed, where each instances of \texttt{mallocs} were followed by
        \texttt{free} statements after usage. All memory leaks have been
        accounted for in the current version of the program.

    \section{Usage}
    The project depends on user inputs from the terminal. Once executed, the
    Main Menu is displayed, prompting the user for a choice of 6 options:
    \texttt{Print Database}, \texttt{Search by Name}, \texttt{Search by Grade},
    \texttt{Add Students}, \texttt{Remove Students} and \texttt{Exit}. All
    these options, excluding \texttt{Print Database} and \texttt{Exit}, runs on
    independent loops until the user decides to exit.

        \subsection{Print Database} The \texttt{Print Database} option displays
        all the students in the database, along with their assignment and final
        course grades. The database is initialized if an input file with valid
        student information is provided. If no input file is provided, the
        database is initialized as an empty linked list.

        \subsection{Search by Name} The \texttt{Search by Name} option allows
        the user to search for a student and their grade information with their
        last name. The user is prompted for the student's last name, and their
        input is compared case-insensitively with the last names of all the
        students in the database.

        \subsection{Search by Grade} The \texttt{Search by Grade} option allows
        the user to search for a student and their grade information with their
        letter grade. The user is prompted for the their desired letter grade
        (A, B, C, D, or F), and their input is compared with the final grades
        of all the students in the database. The program will print out every
        instances of student final grades matching the user input.

        \subsection{Add Students} The \texttt{Add Students} option allows the
        user to add new students to the database. The user is prompted for the
        new student's name, and then given the option to select the type of the
        new grade. The types include quizzes and tests. Once a grade type has
        been established, the user gets prompted for the quiz / test name,
        grade and weight. This process runs in a loop until the user is
        satisfied with the student information.

        \subsection{Remove Students} The \texttt{Remove Students} option allows
        users to remove students from the database using their last names. The
        user is prompted for the student's last name, and their input is
        compared case-insensitively with the last names of all the students in
        the database. Once a match has been established, the node is freed from
        memory and the student is removed from the database.

        \subsection{Exit} The Exit option breaks out of the loop and
        immediately terminates the program.

        \subsection{Invalid Choices} The program will continue displaying the
        Main Menu until a valid choice is inputted by the user.

    \section{Testing and Troubleshooting} Managing memory leaks proved to be
    much more challenging than the actual implementation of the project.
    \texttt{<strings.h>} functions consisted of several constraints regarding
    the size of input string buffers, locations of the NULL terminator, etc.
    Once all the memory leak was taken care of, the code was frozen and
    documented. In the final stages, multiple test cases were considered in
    validating the functionality of the program such as duplicate last names
    being properly sorted, invalid letter grades as inputs, etc.

    \section{Code} The C scripts used for the implementation has been attached
    alongside the report.

        \subsection{main.c} Driver file for Homework 3: Grade Database
        Management.

        \subsection{menu.c} Contains the implemention of the functions used to
        provide high level interfaces to the data structures in the project.

        \subsection{database.c} Contains the implementation of student\_t nodes
        and database\_t linked lists constructors, methods and its
        corresponding iterators.

        \subsection{list\_of\_grades.c} Contains the implementation of grade\_t
        nodes and list\_of\_grades\_t linked lists constructors, methods and
        its corresponding iterators.

        \subsection{Makefile} A Makefile has been provided with the project
        with the following recipes.
        \begin{itemize}

            \item \codeword{build}: compile the program with GCC

            \item \codeword{run INPUT=path/to/file.txt}: run the GCC-compiled
            executable. If the optional \texttt{INPUT} arugment is not used,
            the database is initialized empty.

            \item \codeword{val INPUT=path/to/file.txt}: run \texttt{valgrind}
            on the executable. If the optional \texttt{INPUT} arugment is not
            used, the database is initialized empty.

            \item \codeword{clean}: remove temporary and executable files 

        \end{itemize}

    \begin{thebibliography}{9}
        \bibitem{clibs} Clibs Authors. ``Package manager for the C programming
        langage.'' \textit{clib(1)}. Accessed \today,
        \textit{http://clibs.org/}.

        \bibitem{gfg} GeeksforGeeks. ``C Program for Bubble Sort on Linked
        List.'' \textit{GeeksforGeeks}. Accessed \today,
        \textit{http://www.geeksforgeeks.org/c-program-bubble-sort-linked-
        list/}.
    \end{thebibliography}

\end{document}
