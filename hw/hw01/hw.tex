%%%%%%%%%%%%%%%%%%%%%%%%%%%%%%%%%%%%%%%%%
% Template
% LaTeX Template
% Version 1.0 (December 8 2014)
%
% This template has been downloaded from:
% http://www.LaTeXTemplates.com
%
% Original author:
% Brandon Fryslie
% With extensive modifications by:
% Vel (vel@latextemplates.com)
%
% License:
% CC BY-NC-SA 3.0 (http://creativecommons.org/licenses/by-nc-sa/3.0/)
%
% Authors:
% Sabbir Ahmed
%
%%%%%%%%%%%%%%%%%%%%%%%%%%%%%%%%%%%%%%%%%

\documentclass[paper=usletter, fontsize=12pt]{extarticle}
%%%%%%%%%%%%%%%%%%%%%%%%%%%%%%%%%%%%%%%%%
% Contract Structural Definitions File Version 1.0 (December 8 2014)
%
% Created by: Vel (vel@latextemplates.com)
% 
% This file has been downloaded from: http://www.LaTeXTemplates.com
%
% License: CC BY-NC-SA 3.0 (http://creativecommons.org/licenses/by-nc-sa/3.0/)
%
%%%%%%%%%%%%%%%%%%%%%%%%%%%%%%%%%%%%%%%%%

\usepackage{geometry} % Required to modify the page layout
\usepackage{multicol}
\usepackage{amsmath}
\usepackage{amssymb}

\usepackage[pdftex]{graphicx}
\usepackage{wrapfig}
\usepackage[font=scriptsize, labelfont=bf]{caption}
\usepackage[utf8]{inputenc} % Required for including letters with accents
\usepackage[T1]{fontenc} % Use 8-bit encoding that has 256 glyphs

\usepackage{avant} % Use the Avantgarde font for headings
\usepackage{courier}
\usepackage{xparse}
\usepackage{xcolor}
\usepackage{listings}  % for code verbatim and console outputs

\setlength{\textwidth}{16cm} % Width of the text on the page
\setlength{\textheight}{23cm} % Height of the text on the page
\setlength{\oddsidemargin}{0cm} % Width of the margin - negative to move text left, positive to move it right
\setlength{\topmargin}{-1.25cm} % Reduce the top margin

\setlength{\parindent}{0mm} % Don't indent paragraphs
\setlength{\parskip}{2.5mm} % Whitespace between paragraphs
\renewcommand{\baselinestretch}{1.5}

\definecolor{green}{rgb}{0.18, 0.55, 0.34}

\graphicspath{ {figures/} }
\captionsetup[table]{skip=10pt}

\lstset{language=C, keywordstyle={\bfseries \color{black}}}

% defines algorithm counter for chapter-level
\newcounter{nalg}[section]

%defines appearance of the algorithm counter
\renewcommand{\thenalg}{\thesection .\arabic{nalg}}

% defines a new caption label as Algorithm x.y
\DeclareCaptionLabelFormat{algocaption}{Algorithm \thenalg}

% defines the algorithm listing environment
\lstnewenvironment{pseudocode}[1][] {
    \refstepcounter{nalg}  % increments algorithm number
    \captionsetup{font=normalsize, labelformat=algocaption, labelsep=colon}
    \lstset{
        breaklines=true,
        mathescape=true,
        numbers=left,
        numberstyle=\scriptsize,
        basicstyle=\footnotesize\ttfamily,
        keywordstyle=\color{black}\bfseries,
        keywords={input, output, return, parallel, function, for, to, in, if,
        else, foreach, while, and, or, new, print},
        xleftmargin=.04\textwidth,
        #1
    }
}{}

\renewcommand{\familydefault}{\sfdefault}  % default font for entire document
 % specifies the document layout and style

%------------------------------------------------------------------------------
% document info command
\newcommand{\documentinfo}[5]{
    \begin{centering}
        \parbox{2in}{
        \begin{spacing}{1}
            \begin{flushleft}
                \begin{tabular}{l l}
                    #1 \\
                    #2 \\
                    #3 \\
                \end{tabular}\\
                \rule{\textwidth}{1pt}
            \end{flushleft}
        \end{spacing}
        }
    \end{centering}
}

\begin{document}

    \documentinfo{Sabbir Ahmed}{\textbf{DATE:} \today}{\textbf{CMSC 421:} HW 01}
    \vspace{-0.2in}

    \begin{enumerate}[label=\textbf{\arabic*}]

        \item Compare and contrast microkernel and monolithic kernel-based
        operating systems. Name one kernel that follows each of these models.
        \item[\textbf{Ans}]
        A microkernel implements user and kernel services in different address
        spaces, whereas a monolithic kernel system uses the same address space.
        This separation makes microkernels easily extendable with new services
        being added to its user space. The monolithic kernel requires the
        entire kernel to be rebuilt after adding new services. Since only the
        kernel services are located in its address spaces, the microkernels are
        comparatively smaller than a monolithic kernel. Also, if a service
        fails in a microkernel, the operating system remains unaffected.
        Monolithic kernels, however, execute faster as they use system calls to
        communicate between application and hardware. Microkernels rely on the
        relatively slower message passing to communicate. Performance of
        microkernels are also affected by increased system-function overhead.
        Monolithic kernels also require lesser code in their design than
        microkernels.
        \vspace{0.2in}

        \item What is the purpose of a system call? Give at least 3 concrete
        examples of potential system calls in an OS and explain why each would
        be a system call.
        \item[\textbf{Ans}]
        System calls provide an interface to the services made available by an
        operating system.
        \vspace{0.2in}

        \item Describe what a context switch is. How does context switching
        differ between processes and between threads within a single process?
        \item[\textbf{Ans}]
        When an interrupt occurs, the system needs to save the current context
        of the process running on the CPU so that it can restore that context
        when its processing is done. Context switching is the task of switching
        the CPU to another process. It requires performing a state save of the
        current process and a state restore of a different process.
        \vspace{0.2in}

        \item Name two methods of IPC (interprocess communication) and discuss
        the pros and cons of each.
        \item[\textbf{Ans}]
        The two fundamental models of interprocess communication are shared
        memory and message passing. Message passing is useful for exchanging
        smaller amounts of data, because no conflicts need be avoided. It is
        also easier to implement in a distributed system than shared memory.
        Shared memory can be faster than message passing, since message-passing
        systems are typically implemented using system calls. It however
        suffers from cache coherency issues, which arise because shared data
        migrate among the several caches.
        \vspace{0.2in}

        \item What is the difference between a program being executed by a
        single thread and one that is being executed by multiple threads?
        Compare and contrast task parallelism and data parallelism as it
        relates to multithreaded programs.
        \item[\textbf{Ans}]
        \vspace{0.2in}

    \end{enumerate}

\end{document}
