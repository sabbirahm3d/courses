%%%%%%%%%%%%%%%%%%%%%%%%%%%%%%%%%%%%%%%%%
%
% Sabbir Ahmed
%
%%%%%%%%%%%%%%%%%%%%%%%%%%%%%%%%%%%%%%%%%

\documentclass[paper=usletter, fontsize=12pt]{article}
%%%%%%%%%%%%%%%%%%%%%%%%%%%%%%%%%%%%%%%%%
% Contract Structural Definitions File Version 1.0 (December 8 2014)
%
% Created by: Vel (vel@latextemplates.com)
% 
% This file has been downloaded from: http://www.LaTeXTemplates.com
%
% License: CC BY-NC-SA 3.0 (http://creativecommons.org/licenses/by-nc-sa/3.0/)
%
%%%%%%%%%%%%%%%%%%%%%%%%%%%%%%%%%%%%%%%%%

\usepackage{geometry} % Required to modify the page layout
\usepackage{multicol}
\usepackage{amsmath}
\usepackage{amssymb}

\usepackage[pdftex]{graphicx}
\usepackage{wrapfig}
\usepackage[font=scriptsize, labelfont=bf]{caption}
\usepackage[utf8]{inputenc} % Required for including letters with accents
\usepackage[T1]{fontenc} % Use 8-bit encoding that has 256 glyphs

\usepackage{avant} % Use the Avantgarde font for headings
\usepackage{courier}
\usepackage{xparse}
\usepackage{xcolor}
\usepackage{listings}  % for code verbatim and console outputs

\setlength{\textwidth}{16cm} % Width of the text on the page
\setlength{\textheight}{23cm} % Height of the text on the page
\setlength{\oddsidemargin}{0cm} % Width of the margin - negative to move text left, positive to move it right
\setlength{\topmargin}{-1.25cm} % Reduce the top margin

\setlength{\parindent}{0mm} % Don't indent paragraphs
\setlength{\parskip}{2.5mm} % Whitespace between paragraphs
\renewcommand{\baselinestretch}{1.5}

\definecolor{green}{rgb}{0.18, 0.55, 0.34}

\graphicspath{ {figures/} }
\captionsetup[table]{skip=10pt}

\lstset{language=C, keywordstyle={\bfseries \color{black}}}

% defines algorithm counter for chapter-level
\newcounter{nalg}[section]

%defines appearance of the algorithm counter
\renewcommand{\thenalg}{\thesection .\arabic{nalg}}

% defines a new caption label as Algorithm x.y
\DeclareCaptionLabelFormat{algocaption}{Algorithm \thenalg}

% defines the algorithm listing environment
\lstnewenvironment{pseudocode}[1][] {
    \refstepcounter{nalg}  % increments algorithm number
    \captionsetup{font=normalsize, labelformat=algocaption, labelsep=colon}
    \lstset{
        breaklines=true,
        mathescape=true,
        numbers=left,
        numberstyle=\scriptsize,
        basicstyle=\footnotesize\ttfamily,
        keywordstyle=\color{black}\bfseries,
        keywords={input, output, return, parallel, function, for, to, in, if,
        else, foreach, while, and, or, new, print},
        xleftmargin=.04\textwidth,
        #1
    }
}{}

\renewcommand{\familydefault}{\sfdefault}  % default font for entire document
 % specifies the document layout and style

\begin{document}

    \documentinfo{\today}{08}

    \begin{enumerate}

        % 1
        \item Let $X$ have a uniform distribution in the unit interval $[0,1]$,
        and let $Y$ have an exponential distribution with parameter $\nu=2$.
        Assume that $X$ and $Y$ are independent. Let $Z=X+Y$.
        \begin{enumerate}

            \item Find $P(Y \ge X)$.
            \begin{proof}

                Since $X$ and $Y$ are independent,
                $f_{X,Y}(x,y)=f_X(x)g_Y(y)$\\
                Therefore, the joint PDF,
                \begin{equation*}
                    f_{X,Y}(x,y)=
                    \begin{cases}
                        2e^{-2y}, & \text{ if } 0 \le x \le 1, \ y \ge 0\\
                        0, & \text{otherwise}
                    \end{cases}
                \end{equation*}
                Therefore,
                \salign{1}
                \begin{align*}
                    P(Y \ge X) & = 1 - P(X \le Y) \\
                    & = \iint_{y \ge x}f_{X,Y}(x,y)\diff{x}\diff{y}\\
                    & = 1-\int_{0}^{1}\int_{0}^{x}2e^{-2y}\diff{x}\diff{y}\\
                    & = 1-\int_{0}^{1}1-e^{-2x}\diff{x}\\
                    & = \frac{1}{2}-\frac{e^{-2}}{2} \qedhere
                \end{align*}
                \endgroup

            \end{proof}

            \item Find the conditional PDF of $Z$ given that $Y=y$.
            \begin{proof}

                \begin{align*}
                    f_{Z \mid Y=y}(z) & = f_{X+Y \mid Y=y}(x+y)\\
                    & = \begin{cases}
                        1, & \text{ if } y \le z \le 1+y\\
                        0, & \text{ otherwise},
                    \end{cases} \qedhere
                \end{align*}

            \end{proof}

            \item Find the conditional PDF of $Y$ given that $Z=3$.
            \begin{proof}

                With the laws of conditional probability,
                \begin{equation*}
                    f_{Y\mid 3}(y \mid 3) = \frac{f_{Y,Z}(y,3)}{f_Z(3)} = \frac{f_{Z \mid Y=y}(3 \mid y)f_Y(y)}{f_Z(3)}
                \end{equation*}
                And,
                \salign{1}
                \begin{align*}
                    F_Z(3) & = \int_{0}^{1}\int_{0}^{z-x}f_{X,Y}(x,y)\diff{x}\diff{y}\\
                    & = \int_{0}^{1}\int_{0}^{z-x}2e^{-2y}\diff{x}\diff{y}\\
                    & = 1 - \frac{e^{-2(3)+2}}{2} + \frac{e^{-2(3)}}{2}\\
                    & = e^{-4}-e^{-6}
                \end{align*}
                \endgroup
                Therefore,
                \salign{1}
                \begin{equation*}
                    f_{Y\mid 3}(y \mid 3) =
                    \begin{cases}
                        \frac{2e^{-2y}}{e^{-4}-e^{-6}}, & \text{ if } 2 \le y \le 3\\
                        0, & \text{ otherwise},
                    \end{cases} \qedhere
                \end{equation*}
                \endgroup

            \end{proof}

        \end{enumerate}

        % 2
        \item Let $P$, a random variable which is uniformly distributed between
        0 and 1. On any given day, a particular machine is functional with
        probability $P$. Furthermore, given the value of $P$, the status of the
        machine on different days is independent
        \begin{enumerate}

            \item Find the probability that the machine is functional on a
            particular day.
            \begin{proof}

                Let $W$ represent the event that the machine is functional\\
                Then,
                \salign{0.5}
                \begin{align*}
                    P(W) & = \int_{0}^{1}P(W \mid X=x)f_X(x)\diff{x}\\
                    & = \int_{0}^{1}x\diff{x}\\
                    & = \frac{1}{2} \qedhere
                \end{align*}
                \endgroup

            \end{proof}

            \item We are told that the machine was functional on $m$ out of the
            last $n$ days. Find the conditional PDF of $P$. You may use the
            identity
            \begin{equation*}
                \int_{0}^{1}p^k(1-p)^{n-k}dp=\frac{k!(n-k)!}{(n+1)!}
            \end{equation*}
            \begin{proof}

                \salign{1}
                \begin{align*}
                    P(W_m) & = \int_{0}^{1}P(W_m \mid X=x)f_X(x)\diff{x}\\
                    & = \int_{0}^{1}\binom{n}{m}x^m(1-x)^{n-m}f_X(x)\diff{x}\\
                    & = \binom{n}{m}\frac{m!(n-m)!}{(n+1)!}
                \end{align*}
                \endgroup

                Therefore, using Bayes rule,
                \salign{1}
                \begin{align*}
                    f_{X \mid W_m}(x) & = \frac{P(W_m \mid X = x)f_X(x)}{P(W_m)} \\
                    & = \frac{x^m(1-x)^{n-m}}{\frac{m!(n-m)!}{(n+1)!}}, \ 0 \le q \le 1, \ n \ge m \qedhere
                \end{align*}
                \endgroup

            \end{proof}

            \item Find the conditional probability that the machine is
            functional today given that it was functional on $m$ out of the
            last $n$ days.
            \begin{proof}

                \salign{1}
                \begin{align*}
                    P(W_m) & = \int_{0}^{1}P(W_m \mid X=x)f_X(x)\diff{x}\\
                    & = \int_{0}^{1}\binom{n}{m}x^m(1-x)^{n-m}f_X(x)\diff{x}\\
                    & = \binom{n}{m}\frac{m!(n-m)!}{(n+1)!}  \qedhere
                \end{align*}
                \endgroup

            \end{proof}

        \end{enumerate}

        % 3
        \item Let $B \triangleq \{a < X \le b\}$. Derive a general expression
        for $E[X \mid B]$ if $X$ is a continuous RV. Let $X : N(0,1)$ with
        $B=\{-1 < X \le 2\}$. Compute $E[X \mid B]$.
        \begin{proof}
        \end{proof}

        % 4
        \item A particular model of an HDTV is manufactured in three different
        plants, say, $A$, $B$ and $C$, of the same company. Because the workers
        at $A$, $B$ and $C$ are not equally experienced, the quality of the
        units differs from plant to plant. The pdf's of the time-to-failure
        $X$, in years, are
        \salign{0.5}
        \begin{align*}
            f_X(x) &= \frac{1}{5}\exp(-x/5)u(x) \text{ for } A\\
            f_X(x) &= \frac{1}{6.5}\exp(-x/6.5)u(x) \text{ for } B\\
            f_X(x) &= \frac{1}{10}\exp(-x/10)u(x) \text{ for } C,
        \end{align*}
        \endgroup
        where $u(x)$ is the unit step. Plant $A$ produces three times as many
        units as $B$, which produces twice as many as $C$. The TVs are all sent
        to a central warehouse, intermingled, and shipped to retail stores all
        around the country. What is the expected lifetime of a unit purchased
        at random?
        \begin{proof}

            The expectations of the exponential distributions, $1/\lambda$,
            \begin{align*}
                E[A] & = 5\\
                E[B] & = 6.5\\
                E[C] & = 10
            \end{align*}
            Given the ratio of the units is $6:2:1$,
            \begin{align*}
                P(A) & = \frac{6}{9}\\
                P(B) & = \frac{2}{9}\\
                P(C) & = \frac{1}{9}
            \end{align*}
            Therefore, the expected lifetime of a unit purchased at random,
            \salign{0.5}
            \begin{align*}
                E & = \frac{5 \times 6}{9} + \frac{6.5 \times 2}{9} + \frac{10 \times 1}{9}\\
                & = \frac{53}{6} \qedhere
            \end{align*}
            \endgroup

        \end{proof}

        % 5
        \item The coordinate $X$ and $Y$ of a point are independent zero mean
        normal random variables with common variances $\sigma^2$. Given that
        the point is at a distance of at least $c$ from the origin, find the
        conditional joint PDF of $X$ and $Y$.
        \begin{proof}
        \end{proof}

        % 6
        \item Alexei is vacationing in Monte Carol. The amount $X$ (in dollars)
        he takes to the casino each evening is a random variable with a PDF of
        the form
        \begin{equation*}
            f_X(x)=
            \begin{cases}
                ax, & \text{ if } 0 \le x \le 40,\\
                0, & \text{otherwise}
            \end{cases}
        \end{equation*}
        At the end of each night, the amount $Y$ that he has when leaving the
        casino is uniformly distributed between zero and twice the amount that
        he came with.
        \begin{enumerate}

            \item Determine the joint PDF $f_{X,Y}(x,y)$.
            \begin{proof}

                Since
                \begin{align*}
                    \int_{0}^{40}ax\diff{x} & = 1\\
                    a\frac{40^2}{2} & = 1\\
                    \implies a & = \frac{1}{800}
                \end{align*}

            \end{proof}

            \item What is the probability that on a given night Alexei makes a
            positive profit at the casino?
            \begin{proof}
            \end{proof}

            \item Find the PDF of Alexei's profit $Y-X$ on a particular night,
            and also determine its expected value.
            \begin{proof}
            \end{proof}

        \end{enumerate}

        % 7
        \item Consider a communication channel corrupted by noise. Let $X$ be
        the value of the transmitted signal and $Y$ be the value of the
        received signal. Assume that the conditional density of $Y$ given
        $\{X=x\}$ is Gaussian, that is,
        \begin{equation*}
            f_{Y \mid X}(y \mid x) = \frac{1}{\sqrt{2\pi\sigma^2}}\exp\bigg(\frac{-(y-x)^2}{2\sigma^2}\bigg),
        \end{equation*}
        and $X$ is uniformly distributed on $[-1,1]$. What is the conditional
        pdf of $X$ given $Y$, that is, $f_{X \mid Y}(x \mid y)$
        \begin{proof}

            Given,
            \begin{equation*}
                f_{X}(x) =
                \begin{cases}
                    \frac{1}{2}, & \text{ if } -1 \le x \le 1\\
                    0, & \text{ otherwise},
                \end{cases}
            \end{equation*}

            And,
            \salign{1}
            \begin{align*}
                f_Y(y) & = \int_{\infty}f_{X,Y}(x,y)\diff{x}\\
                & = \int_{\infty}f_{Y \mid X}(y \mid x)f_X(x)\diff{x}\\
                & = \int_{-1}^{1}\frac{1}{2}\frac{1}{\sqrt{2\pi\sigma^2}}\exp\Big(\frac{-(y-x)^2}{2\sigma^2}\Big)\diff{x}\\
                & = \frac{1}{2}\bigg(\phi\Big(\frac{y-1}{\sigma}\Big)-\phi\Big(\frac{y+1}{\sigma}\Big)\bigg)
            \end{align*}
            \endgroup

            Therefore,
            \salign{1}
            \begin{align*}
                f_{X \mid Y}(x \mid y) & = \frac{f_{Y \mid X}(y \mid x)f_X(x)}{f_Y(y)}\\
                & = \frac{\frac{1}{\sqrt{2\pi\sigma^2}}\exp\Big(\frac{-(y-x)^2}{2\sigma^2}\Big)f_X(x)}{\phi\Big(\frac{y-1}{\sigma}\Big)-\phi\Big(\frac{y+1}{\sigma}\Big)} \qedhere
            \end{align*}
            \endgroup

        \end{proof}

        % 8
        \item A U.S. defense radar scans the skies for unidentified flying
        objects (UFOs). Let $M$ be the event that a UFO is present and $M^C$
        the venemt that a UFO is absent. Let $f_{X \mid M}(x \mid
        M)=\frac{1}{\sqrt{2\pi}}\exp(-0.5[x-r]^2)$ be the conditional pdf of
        the radar return signal $X$ when a UFO is actually there, and let $f_{X
        \mid M^C}(x \mid M)=\frac{1}{\sqrt{2\pi}}\exp(-0.5[x]^2)$ be the
        conditional pdf of the radar return signal $X$ when there is no UFO. To
        be specific, let $r=1$ and let the \textit{alert level} be $x_A=0.5$.
        Let $A$ denote the event of an alert, that is, $\{X>x_A\}$. Compute
        $P[A\mid M]$, $P[A^C\mid M]$, $P[A\mid M^C]$, $P[A^C\mid M^C]$.\\
        Assume that $P[M]-10^{-3}$. Compute $P[A\mid M]$, $P[A^C\mid M]$,
        $P[A\mid M^C]$, $P[A^C\mid M^C]$.\\
        Repeat for $P[M]=10^{-6}$
        \begin{proof}
        \end{proof}

    \end{enumerate}

\end{document}
