%
% CMSC 421: Project 2 Introduction
% Template layout and preamble defined in structure.tex
%
% Author: Sabbir Ahmed
%

\section{Introduction} This project implements a new version of the Linux
kernel that adds functionality to support a simple intrusion detection
system (IDS). This system will operate by logging the system calls made by
a process in the kernel, while analysis and intrusion detection will be
done in user space. This assignment is designed to teach a simple method of
intrusion detection, as well as to reinforce the idea of how user space and
kernel space interact through the use of system calls.

An intrusion detection system is a computer program that attempts to
identify (and thwart) attacks that might be performed on the system by
attackers. There are several time-tested approaches to the development of
an IDS. The project will keep track of the system calls made by a monitored
process and check for abnormalities in the sequences of system calls made.
When an attacker breaks into a process, they will need to make system calls
in order to attempt to access the resources of the system that are under
attack. As the system calls that the attacker will perform will likely be
different than those performed by a process that is not under attack, it
follows that by monitoring both healthy and broken processes, it is
possible to develop a scheme to identify those that might be under attack
for further action to be taken.

\section{Objective} The project will compare sequences of system calls made
by a monitored process to known good sequences.

\subsection{Kernel Space Requirements} The kernel-space program will
instrument the system call dispatcher of the Linux kernel with code that
logs each time a system call is made. Built-in system calls such as
\texttt{ptrace} are not allowed to trace the usage of system calls to
generate the logs for the project.

\subsection{User Space Requirements} The analysis of the logs will be
handled by the user-space program, which may be implemented in any
supported programming language. The user-space process should construct a
bit array for each process under monitoring showing which system calls have
been run in a window of the last $k$ system calls. If a particular system
call is made in the window, the bit for that system call will be set to 1.
The bit arrays will then be measured for their hamming distance with the
example of a healthy system call sequence for a process.
