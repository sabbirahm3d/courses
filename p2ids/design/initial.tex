%
% CMSC 421: Project 2 Initial Design Document
% Template layout and preamble defined in structure.tex
%
% Author: Sabbir Ahmed
%

\documentclass[paper=usletter, fontsize=12pt]{article}
%%%%%%%%%%%%%%%%%%%%%%%%%%%%%%%%%%%%%%%%%
% Contract Structural Definitions File Version 1.0 (December 8 2014)
%
% Created by: Vel (vel@latextemplates.com)
% 
% This file has been downloaded from: http://www.LaTeXTemplates.com
%
% License: CC BY-NC-SA 3.0 (http://creativecommons.org/licenses/by-nc-sa/3.0/)
%
%%%%%%%%%%%%%%%%%%%%%%%%%%%%%%%%%%%%%%%%%

\usepackage{geometry} % Required to modify the page layout
\usepackage{multicol}
\usepackage{amsmath}
\usepackage{amssymb}

\usepackage[pdftex]{graphicx}
\usepackage{wrapfig}
\usepackage[font=scriptsize, labelfont=bf]{caption}
\usepackage[utf8]{inputenc} % Required for including letters with accents
\usepackage[T1]{fontenc} % Use 8-bit encoding that has 256 glyphs

\usepackage{avant} % Use the Avantgarde font for headings
\usepackage{courier}
\usepackage{xparse}
\usepackage{xcolor}
\usepackage{listings}  % for code verbatim and console outputs

\setlength{\textwidth}{16cm} % Width of the text on the page
\setlength{\textheight}{23cm} % Height of the text on the page
\setlength{\oddsidemargin}{0cm} % Width of the margin - negative to move text left, positive to move it right
\setlength{\topmargin}{-1.25cm} % Reduce the top margin

\setlength{\parindent}{0mm} % Don't indent paragraphs
\setlength{\parskip}{2.5mm} % Whitespace between paragraphs
\renewcommand{\baselinestretch}{1.5}

\definecolor{green}{rgb}{0.18, 0.55, 0.34}

\graphicspath{ {figures/} }
\captionsetup[table]{skip=10pt}

\lstset{language=C, keywordstyle={\bfseries \color{black}}}

% defines algorithm counter for chapter-level
\newcounter{nalg}[section]

%defines appearance of the algorithm counter
\renewcommand{\thenalg}{\thesection .\arabic{nalg}}

% defines a new caption label as Algorithm x.y
\DeclareCaptionLabelFormat{algocaption}{Algorithm \thenalg}

% defines the algorithm listing environment
\lstnewenvironment{pseudocode}[1][] {
    \refstepcounter{nalg}  % increments algorithm number
    \captionsetup{font=normalsize, labelformat=algocaption, labelsep=colon}
    \lstset{
        breaklines=true,
        mathescape=true,
        numbers=left,
        numberstyle=\scriptsize,
        basicstyle=\footnotesize\ttfamily,
        keywordstyle=\color{black}\bfseries,
        keywords={input, output, return, parallel, function, for, to, in, if,
        else, foreach, while, and, or, new, print},
        xleftmargin=.04\textwidth,
        #1
    }
}{}

\renewcommand{\familydefault}{\sfdefault}  % default font for entire document
 % specifies the document layout and style

\begin{document}

    \documentinfo{April 15, 2018}{Initial}

    % import commmon sections on Introduction
    \documentclass[11pt]{article}
\usepackage{setspace}
\usepackage{enumitem}
\usepackage{subcaption}
\usepackage[letterpaper, margin=1in]{geometry}
\usepackage{graphicx}
\setcounter{secnumdepth}{2}
\linespread{1.3}

% -----------------------------------------------------------
% Margin setup

\setlength{\evensidemargin}{-0.25in}
\setlength{\headheight}{0in}
\setlength{\headsep}{0in}
\setlength{\oddsidemargin}{-0.25in}
\setlength{\paperheight}{11in}
\setlength{\paperwidth}{8.5in}
\setlength{\tabcolsep}{0in}
\setlength{\textheight}{9.5in}
\setlength{\textwidth}{7in}
\setlength{\topmargin}{-0.3in}
\setlength{\topskip}{0in}
\setlength{\voffset}{0.1in}

% -----------------------------------------------------------
% Custom commands

% header command
\newcommand{\header}[5]{
    \begin{centering}
        \parbox{6.8in}{
        \begin{flushright}
        \begin{spacing}{.8}{
        \fontfamily{cmss}{\large{\textbf{#1}}\\}}
        \small{
            #2\\
            #3\\
            #4\\
            #5}\\
        \end{spacing}
        \end{flushright}
        \vspace{-7.5mm}
        }\\
        \rule{\textwidth}{0.5pt}\\
        \vspace{-4mm}
    \end{centering}
}

% document info command
\newcommand{\documentinfo}[5]{
    \begin{centering}
        \parbox{6.8in}{
        \begin{spacing}{1}
            \begin{flushleft}
                \begin{tabular}{l l}
                    \fontfamily{cmss}{\textbf{DOC \#: }} & #1 \\
                    \fontfamily{cmss}{\textbf{DATE: }} & #2 \\
                    \fontfamily{cmss}{\textbf{TO: }} & #3 \\
                    \fontfamily{cmss}{\textbf{FROM: }} & #4 \\
                    \fontfamily{cmss}{\textbf{SUBJECT: }} & #5 \\
                \end{tabular}\\
                \rule{\textwidth}{1pt}
            \end{flushleft}
        \end{spacing}
        }
    \end{centering}
}


\begin{document}

    \header{CMPE349}{Precision Landing System Project}{MLS Ground Receiver}{1000 Hilltop Cir}{Catonsville, MD 21250}
    \documentinfo{TSR04}{\today}{Dr. LaBerge}{Team M: Sabbir Ahmed, Shawn Bastani, Abdul-Shahid Wali, Brian Weber}{System Design Document}

    \section{Introduction}
    This System Design Document was created to outline the system design for the Microwave Landing System Ground-Based Receiver (MGR). The MGR is responsible for gathering and processing data from digital signals received from the ACME front-end to be transfers them on to the Executive Monitor (XMON). The technical designs and requirements elaborated on this document will comply with the higher level infrastructure of the Microwave Landing System standards.

        \subsection{Purpose}
        The purpose of this System Design Document is to provide a detailed description of the operational concepts, system external interfaces and the processing resources of the system and the design constraints on them.

        \subsection{System Overview}
        ???

        \subsection{Design Constraints}
        The output from the MGR to the XMON must consist of the elevation angle data, and any warning flags determined by signal degradation due to interference.

        \subsection{Reference Documents}
        The <LOOK UP DOCUMENTS> were used to detail the specifications in the MGR and to cross reference the individual configuration items through the System Design Document.


\end{document}


    \section{Design Approach} Development in the preliminary stages began with
    research into the topics involved in the project. Understanding the
    objectives and scope of the project to identify the requirements was the
    initial step taken for the project.

    The next step in development involved outlining the list of tasks necessary
    to complete the project within the milestone deadlines. After considering
    several approaches, it was decided that the user-space program will be
    completed and tested for its functionality first, concurrently while
    conducting research on the Linux system call dispatcher.

    \subsection{Kernel Space} The kernel-space program will trace the system
    calls and dump them on a log consisting of the \texttt{pid}, the system
    call numbers, and the corresponding timestamps of their execution. The path
    of the log file will be shared with the user-space program. The program
    will provide the user the ability to initiate and terminate the logging of
    the system calls. The system call will include a parameter $N$ to indicate
    the maximum number of system calls per log. This parameter will ensure the
    logging system writes at most $N$ system calls in the log file, close the
    file, then signal the user-space program to begin analysis. Once the signal
    is sent, the logging system will resume logging to a new file.

    Initially, the logging of the system calls were perceived as a non-trivial
    but moderately difficult task that would not require much time committed.
    After preliminary research, it was evident that without the built-in system
    calls such as \texttt{ptrace}, logging system calls would require much
    greater effort.

    Due to the extensive research required to build the kernel-level logging
    system, the development of its counterpart user-space program would be
    completed first. Research on plausible methods to achieve the logging of
    the system calls would resume simultaneously.

    Two approaches are currently being considered to influence the flow of the
    research. Regardless of the approach chosen to develop the functionality in
    the kernel space, extensive research on the system call dispatcher is
    required.

    \subsubsection{Trivially Insert Breakpoints to the System Call
    Implementations} A potential but not very likely approach would involve
    fully understanding the implementation of all the system calls that would
    be used in the logging system. Once their implementation is understood,
    breakpoints would be added to suitable areas to pause the system call while
    it indicates to the log of its activity. This approach was being considered
    seriously, until it was apparent that it would not be feasible with the
    number of system calls involved. This would also increase complications
    when attempting to bind system calls. Adding breakpoints may involve
    modifying numerous assembly files, such as inserting \texttt{jmp}
    instructions to the writing system calls to write to the log.

    Even though this approach appears very unlikely, it is not being completely
    ruled out because of the potential uses of parts of its ideology, such as
    using the system calls table to identify implementations of each system
    calls.

    \subsubsection{Reverse Engineer and Create A Stripped-Down Version of
    \texttt{ptrace}} The more probable approach involves fully understanding
    \texttt{<linux/ptrace.h>} and simulate it for the project. The built-in
    system call offers several functionalities that may not be required for the
    logging system. Creating a stripped-down version of the system call would
    most likely allow the system call logging functionality for the IDS.

    \subsection{User Space} The user-space program will be implemented in
    Python 2 without any third-party libraries. Time-permitting, the
    implementation will be translated to C to trade off the lines of codes for
    processing speed.

    The program will continually parse the log file shared with the kernel-
    space program while properly utilizing mutex locks to avoid any
    synchronization issues. The user-space program will begin parsing the logs
    only after it has been closed by the kernel-space program. Once the lines
    of system calls for a process are properly parsed, they will be converted
    to their binary vector representations. The binary vectors will then be
    analyzed for their hamming distances, and the results will be logged in a
    separate file.

    Additional Python 2 and shell scripts will be added to assist in properly
    formatting the outputs and managing the process as it runs in the
    background.

\end{document}
