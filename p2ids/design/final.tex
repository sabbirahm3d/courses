%
% CMSC 421: Project 2 Final Design Document
% Template layout and preamble defined in structure.tex
%
% Author: Sabbir Ahmed
%

\documentclass[paper=usletter, fontsize=12pt]{article}
%%%%%%%%%%%%%%%%%%%%%%%%%%%%%%%%%%%%%%%%%
% Contract Structural Definitions File Version 1.0 (December 8 2014)
%
% Created by: Vel (vel@latextemplates.com)
% 
% This file has been downloaded from: http://www.LaTeXTemplates.com
%
% License: CC BY-NC-SA 3.0 (http://creativecommons.org/licenses/by-nc-sa/3.0/)
%
%%%%%%%%%%%%%%%%%%%%%%%%%%%%%%%%%%%%%%%%%

\usepackage{geometry} % Required to modify the page layout
\usepackage{multicol}
\usepackage{amsmath}
\usepackage{amssymb}

\usepackage[pdftex]{graphicx}
\usepackage{wrapfig}
\usepackage[font=scriptsize, labelfont=bf]{caption}
\usepackage[utf8]{inputenc} % Required for including letters with accents
\usepackage[T1]{fontenc} % Use 8-bit encoding that has 256 glyphs

\usepackage{avant} % Use the Avantgarde font for headings
\usepackage{courier}
\usepackage{xparse}
\usepackage{xcolor}
\usepackage{listings}  % for code verbatim and console outputs

\setlength{\textwidth}{16cm} % Width of the text on the page
\setlength{\textheight}{23cm} % Height of the text on the page
\setlength{\oddsidemargin}{0cm} % Width of the margin - negative to move text left, positive to move it right
\setlength{\topmargin}{-1.25cm} % Reduce the top margin

\setlength{\parindent}{0mm} % Don't indent paragraphs
\setlength{\parskip}{2.5mm} % Whitespace between paragraphs
\renewcommand{\baselinestretch}{1.5}

\definecolor{green}{rgb}{0.18, 0.55, 0.34}

\graphicspath{ {figures/} }
\captionsetup[table]{skip=10pt}

\lstset{language=C, keywordstyle={\bfseries \color{black}}}

% defines algorithm counter for chapter-level
\newcounter{nalg}[section]

%defines appearance of the algorithm counter
\renewcommand{\thenalg}{\thesection .\arabic{nalg}}

% defines a new caption label as Algorithm x.y
\DeclareCaptionLabelFormat{algocaption}{Algorithm \thenalg}

% defines the algorithm listing environment
\lstnewenvironment{pseudocode}[1][] {
    \refstepcounter{nalg}  % increments algorithm number
    \captionsetup{font=normalsize, labelformat=algocaption, labelsep=colon}
    \lstset{
        breaklines=true,
        mathescape=true,
        numbers=left,
        numberstyle=\scriptsize,
        basicstyle=\footnotesize\ttfamily,
        keywordstyle=\color{black}\bfseries,
        keywords={input, output, return, parallel, function, for, to, in, if,
        else, foreach, while, and, or, new, print},
        xleftmargin=.04\textwidth,
        #1
    }
}{}

\renewcommand{\familydefault}{\sfdefault}  % default font for entire document
 % specifies the document layout and style
\begin{document}

    \documentinfo{May 13, 2018}{Final}

    % import commmon sections on Introduction
    \documentclass[11pt]{article}
\usepackage{setspace}
\usepackage{enumitem}
\usepackage{subcaption}
\usepackage[letterpaper, margin=1in]{geometry}
\usepackage{graphicx}
\setcounter{secnumdepth}{2}
\linespread{1.3}

% -----------------------------------------------------------
% Margin setup

\setlength{\evensidemargin}{-0.25in}
\setlength{\headheight}{0in}
\setlength{\headsep}{0in}
\setlength{\oddsidemargin}{-0.25in}
\setlength{\paperheight}{11in}
\setlength{\paperwidth}{8.5in}
\setlength{\tabcolsep}{0in}
\setlength{\textheight}{9.5in}
\setlength{\textwidth}{7in}
\setlength{\topmargin}{-0.3in}
\setlength{\topskip}{0in}
\setlength{\voffset}{0.1in}

% -----------------------------------------------------------
% Custom commands

% header command
\newcommand{\header}[5]{
    \begin{centering}
        \parbox{6.8in}{
        \begin{flushright}
        \begin{spacing}{.8}{
        \fontfamily{cmss}{\large{\textbf{#1}}\\}}
        \small{
            #2\\
            #3\\
            #4\\
            #5}\\
        \end{spacing}
        \end{flushright}
        \vspace{-7.5mm}
        }\\
        \rule{\textwidth}{0.5pt}\\
        \vspace{-4mm}
    \end{centering}
}

% document info command
\newcommand{\documentinfo}[5]{
    \begin{centering}
        \parbox{6.8in}{
        \begin{spacing}{1}
            \begin{flushleft}
                \begin{tabular}{l l}
                    \fontfamily{cmss}{\textbf{DOC \#: }} & #1 \\
                    \fontfamily{cmss}{\textbf{DATE: }} & #2 \\
                    \fontfamily{cmss}{\textbf{TO: }} & #3 \\
                    \fontfamily{cmss}{\textbf{FROM: }} & #4 \\
                    \fontfamily{cmss}{\textbf{SUBJECT: }} & #5 \\
                \end{tabular}\\
                \rule{\textwidth}{1pt}
            \end{flushleft}
        \end{spacing}
        }
    \end{centering}
}


\begin{document}

    \header{CMPE349}{Precision Landing System Project}{MLS Ground Receiver}{1000 Hilltop Cir}{Catonsville, MD 21250}
    \documentinfo{TSR04}{\today}{Dr. LaBerge}{Team M: Sabbir Ahmed, Shawn Bastani, Abdul-Shahid Wali, Brian Weber}{System Design Document}

    \section{Introduction}
    This System Design Document was created to outline the system design for the Microwave Landing System Ground-Based Receiver (MGR). The MGR is responsible for gathering and processing data from digital signals received from the ACME front-end to be transfers them on to the Executive Monitor (XMON). The technical designs and requirements elaborated on this document will comply with the higher level infrastructure of the Microwave Landing System standards.

        \subsection{Purpose}
        The purpose of this System Design Document is to provide a detailed description of the operational concepts, system external interfaces and the processing resources of the system and the design constraints on them.

        \subsection{System Overview}
        ???

        \subsection{Design Constraints}
        The output from the MGR to the XMON must consist of the elevation angle data, and any warning flags determined by signal degradation due to interference.

        \subsection{Reference Documents}
        The <LOOK UP DOCUMENTS> were used to detail the specifications in the MGR and to cross reference the individual configuration items through the System Design Document.


\end{document}


    \section{Design Approach} The final design of the project adopted a much
    simpler approach than what was previously assumed. The methods to achieve
    the requirements became apparent after extensive research.

    \subsection{Kernel Space} The kernel-space program will trace the system
    calls and dump them on logs consisting of the \texttt{pid} and the system
    call numbers of each processes spawned by a program. The log was intended
    to also contain the corresponding timestamps of the execution of the system
    calls, but the kernel-space did not appear to have nanosecond resolution of
    their timestamps. Initially, the kernel-space program intended to create
    the log files in $N$ chunks specified by the user to pass on to the user-
    space program for analysis. This parameter will be removed as an option to
    the user and instead be converted to a constant. The constant $N$ number of
    system calls per log files would provide a smooth communication method with
    the user-space program.

    Initially, \texttt{ptrace} appeared to be the only method of tracing system
    calls. Two different approaches were proposed in the initial design. The
    first approach suggested inserting breakpoints in the implementations for
    each system calls in the kernel. The second and more probable approach
    suggested a stripped down version of \texttt{ptrace} that behaved
    identically. After extensive research, a third approach was considered that
    merged aspects of both of the prior approaches. The conditional jump
    instruction, \texttt{jnz}, was updated to an unconditional jump,
    \texttt{jmp}, inside the system call entry point section in
    \texttt{entry\_64.S}. This update forces all the system calls to save their
    system call number to registers before disabling the interrupt request. The
    system call number is then accessible in the register \texttt{ORIG\_RAX} in
    \texttt{common.c} which can be provided as a macro to external files.

    The kernel-space program will require the user to provide the programs to
    be traced. The program will fork its own copy of the process, and gather
    the process IDs and the system call numbers. The \texttt{proc} file system
    will be utilized to provide a convenient interface between the kernels-
    space program and writing log files accessible to the user space. The
    common directory shared by the entire IDS, \texttt{idsdata}, will be used
    to store the log files.

    \subsection{User Space} The user-space program was developed before the
    kernel-space program. The program was tested using dummy data simulating
    log files created by the kernel-space program. An object oriented approach
    was taken to develop the interface with the user and the kernel-space
    logger. The user would be required to initialize the user-space program
    after the kernel-space program has been initiated to run concurrently. The
    user-space program would wait for the logger to generate a chunk of the
    system call logs, and then automatically label the first sequences of
    system calls as ``healthy'' sequences. The healthy sequences get stored in
    a separate file so that it may be read back into memory for analysis of the
    specified program at a later time.

    The user-space program is written in Python 2. The user-space program
    executes a C wrapper for the kernel-space system call. The program provides
    feature options via the command line to the user. The options allow the
    users to utilize the the detection system for the list of programs, window
    sizes or other analysis parameters as needed.

    \section{Implementation} Not all the requirements of the project were
    completed. The implementation for the user-space program mostly followed
    the design approach initially planned. Portions of the program were
    required to be modified due to the incomplete kernel-space program it
    accompanied. Development for the kernel-space program involved programming
    in user-space for simulation in the same programming language (C). Once
    sufficient progress was made in the user-space version, the source code
    would be copied over and translated into the kernel-space program with
    kernel libraries. This method proved unsuccessful, since the translation
    process did not have enough time as the development neared the deadline.

    The only requirement successfully implemented in the kernel-space program
    was creating a reference to the current system call in the architecture's
    register. The system call is labeled \texttt{sys\_ids\_log}. The kernel
    compiles and packages successfully using the kernel-space program.

    The other requirements that were not successfully implemented in the
    kernel-space were forking processes from the programs listed by the user to
    trace, and writing to special filesystems using \texttt{proc} instead of
    normal filesystems. These requirements were implemented in the C wrapper
    user-space program which invoked the \texttt{sys\_ids\_log} system call.

    The user-space program, labeled \texttt{ids.py}, therefore had to modify
    its design approach to compensate for the missing functionalities in the
    kernel-space. The initial design approach suggested both the kernel and
    user-space programs utilize the log files concurrently with mutexes to
    prevent any reader-writer issues. The current version of the
    \texttt{ids.py} design invokes the \texttt{sys\_ids\_log} system call
    before analyzing the sequences, removing the need for synchronization
    locks. The current version of \texttt{ids.py} also serves as the convenient
    high-level interface to the system call from the user.

    \subsection{Usage} To use the IDS, the user-space program, \texttt{ids.py},
    provides various options for tracing processes. The current version of
    \texttt{ids.py} does not support tracing processes by their process
    identifier (PID). The user must provide the processes to be traced as
    command line arguments. Additional optional features are provided to the
    user:
    \begin{itemize}

        \item The window size for the sequences of system calls to be analyzed
        for their hamming distances is also provided as an option to the user.
        The default value for the parameter is $k=5$

        \item The minimum threshold for the hamming distance at which the IDS
        will mark a sequence as ``unhealthy'' is set to the default value of
        $t=3$. The feature is provided to the user via \texttt{ids.py}

        \item The user also has the option to trace their processes from any
        working directory, given the absolute path $l$ provided is valid

    \end{itemize}

    The command line arguments of \texttt{ids.py} are as follows:
    \begin{arguments}
    python2 ids.py -p=<PROC1> -p=<PROC2> ... -p=<PROCN> [-k=K] [-t=T] [-l=PATH]
    \end{arguments}

    An example usage of the IDS with simple parameters are as follows:
    \begin{arguments}
    python2 ids.py -p="ls -l" -p="pwd" -k=5 -t=4
    \end{arguments}

\end{document}
