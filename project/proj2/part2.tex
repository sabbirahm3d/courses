\subsection{Output}

    The first sample mean and standard deviation were computed:

    \[ E(\overline{X}) = 1.20000, \ \sigma_{\overline{X}} = 0.29155 \]

    All the samples were then used to find the sample mean and standard
    deviation. The theoretical values were also computed based on the
    relationships:

    \[ \mu = np \]
    \[ E(\overline{X}) = np \]
    \[ \sigma = \sqrt{np(1-p)} \]
    \[ \sigma_{\overline{X}} = \sqrt{\frac{np(1-p)}{N}} \]

    \begin{table}[h]
        \centering
        \begin{tabular*}{200pt}{@{\extracolsep{\fill}} c c c}

        & \textbf{Actual} & \textbf{Theoretical} \\
        \hline
        $\mu$ & 1.50000  & 1.50000 \\
        E($\overline{X}$) & 1.51100 & 1.50000 \\
        $\sigma$ & 1.12916 & 1.12916 \\
        $\sigma$\textsubscript{$\overline{X}$} & 0.29739 & 0.29155 \\

        \end{tabular*}
    \end{table}
