\documentclass[11pt]{article}
\usepackage{setspace}
\usepackage{enumitem}
\usepackage{subcaption}
\usepackage[letterpaper, margin=1in]{geometry}
\usepackage{graphicx}
\setcounter{secnumdepth}{2}
\linespread{1.3}

% -----------------------------------------------------------
% Margin setup

\setlength{\evensidemargin}{-0.25in}
\setlength{\headheight}{0in}
\setlength{\headsep}{0in}
\setlength{\oddsidemargin}{-0.25in}
\setlength{\paperheight}{11in}
\setlength{\paperwidth}{8.5in}
\setlength{\tabcolsep}{0in}
\setlength{\textheight}{9.5in}
\setlength{\textwidth}{7in}
\setlength{\topmargin}{-0.3in}
\setlength{\topskip}{0in}
\setlength{\voffset}{0.1in}

% -----------------------------------------------------------
% Custom commands

% header command
\newcommand{\header}[5]{
	\begin{centering}
		\parbox{6.8in}{
		\begin{flushright}
		\begin{spacing}{.8}{
		\fontfamily{cmss}{\large{\textbf{#1}}\\}}
		\small{
			#2\\
			#3\\
			#4\\
			#5}\\
		\end{spacing}
		\end{flushright}
		\vspace{-7.5mm}
		}\\
		\rule{\textwidth}{0.5pt}\\
		\vspace{-4mm}
	\end{centering}
}

% document info command
\newcommand{\documentinfo}[5]{
	\begin{centering}
		\parbox{6.8in}{
		\begin{spacing}{1}
			\begin{flushleft}
				\begin{tabular}{l l}
					\fontfamily{cmss}{\textbf{DOC \#: }} & #1 \\
					\fontfamily{cmss}{\textbf{DATE: }} & #2 \\
					\fontfamily{cmss}{\textbf{TO: }} & #3 \\
					\fontfamily{cmss}{\textbf{FROM: }} & #4 \\
					\fontfamily{cmss}{\textbf{SUBJECT: }} & #5 \\
				\end{tabular}\\
				\rule{\textwidth}{1pt}
			\end{flushleft}
		\end{spacing}
		}
	\end{centering}
}

% -----------------------------------------------------------
% Technote Start

\begin{document}

\header{CMPE349}{Precision Landing System Project}{MLS Ground Receiver}{1000 Hilltop Cir}{Catonsville, MD 21250}

\documentinfo{TSR02}{March 10, 2017}{Dr. LaBerge}{Group M}{MGR Original Specs Draft}

	\section{Introduction}
	The Microwave Landing System Ground-Based Receiver (MGR) is responsible for gathering and processing data from digital signals received from the ACME front-end and to be passed on to the Executive Monitor. For the smooth transfer of both the preprocessed and processed data, several specifications are required to be met. This document highlights and elaborates on those requirements.

	\section{XMON Message Format}
	The signals that the MGR provides to the executive monitor are given in section 2.2 of DO-177. The format for the data words for input to the executive monitor is given in section 2.3-5.3 of FAA-E-2721/12.\\
	The MGR shall provide at least one of the following angle output signals to the executive monitor: visual, electrical guidance output B, electrical guidance output A, and basic electrical output. The accuracy of the guidance provided by the MGR shall be in terms of path following error (PFE) and control monitor noise (CMN), which are elaborated in Appendix E of DO-177. The guidance provided by the MGR shall not contain a PFE or, with a 95\% probability, a CMN that exceeds the tolerances listed in table 2-1 of DO-177.\\
	The MLS Ground receiver shall provide the elevation angle data, and any warning flags determined by signal degradation due to interference to the executive monitor in the following formats (described in sections 2.3-5.3 of FAA-E2721/12):
		\begin{enumerate}

			\item Function preamble

			\item Morse code identification

			\item Basic Data word

			\item Auxiliary Data word

		\end{enumerate}

	\section{Warning and Error Communication to the Executive Monitor}
	If for any reason the MGR is not able to send the required data to the executive monitor, a proper warning flag shall be raised to the executive monitor informing the executive monitor of the error. Specifics on for what reasons and within what time period pilots should be informed of warnings relevant to the MGR are specified in sections 2.2.2.1.2 - 2.2.2.1.4 of RTCA DO-177. In order for the executive monitor to issue these warnings to the pilot in the required time period, the MGR shall meet the additional requirements outlined in this section. The intention of these warnings is to give the executive monitor as much detail as possible as quickly as possible about errors, so that it may interpret this information and decide when to inform the pilot.

		\subsection{Distinguishability}
			\begin{enumerate}

				\item At the minimum, separate warning flags shall be raised to distinguish hardware failure, signal loss, the varying degrees of signal degradation explained in point \ref{OCI} of section \ref{signal warning}, and the signal degradation described in point \ref{alternative} of section \ref{signal warning}.

				\item A separate warning flag should be raised for each condition listed in this section in order to communicate as much information as possible to make troubleshooting easier.

			\end{enumerate}

		\subsection{Hardware Failure}

			\begin{enumerate}

				\item If the MGR stops receiving a signal altogether from the ACME front-end, a hardware warning flag shall be raised to the executive monitor within [$100\mu s$].   

				\item If the ACME front-end informs the MGR of a hardware error, the warning flag shall be propagated to the executive monitor within [$100\mu s$].

				\item If some part of the MGR fails, and the MGR is successfully able to catch the failure, the MGR should raise a hardware warning flag to the executive monitor within [$100\mu s$].

				\item The origin of the hardware error should be communicated to the executive monitor.

			\end{enumerate}

		\subsection{Signal Loss/OCI (Out of Coverage Indication)} \label{signal warning}

			\begin{enumerate}

				\item If a signal which is not carrying positional data is received from the ACME front-end for the duration of [1] signal frame, a signal loss warning flag shall be raised to the executive monitor within [$100\mu s$] until a signal frame with valid positional data is received.

				\item \label{OCI} The MGR shall communicate to the executive monitor, and distinguish between, when $55\%$, $75\%$, or $100\%$ of the OCI signal exceeds the primary signal by 1dB or more \textbf{for each signal frame} until a signal frame with a valid OCI signal is received.

				\item \label{alternative} If an alternate signal exceeds the primary signal by 1dB, a signal degradation warning flag shall be raised \textbf{for each signal frame} until a signal frame with valid a alternate signal is received.

				\item If in band interference outside the bounds outlined in section 2.2.1.2.2 of RTSC DO-177 is detected, the MBR shall output a signal degradation warning to the executive monitor until a signal frame which is compliant with 2.2.1.2.2 is received.

			\end{enumerate}

		\subsection{Warning Flag Lowering}

			\begin{enumerate}

				\item A warning flag shall be lowered only when the condition causing the flag stops.

				\item The warning flag shall be lowered within [$100\mu s$] of when the condition causing the flag stops.

			\end{enumerate}

	\section{EMC Compliance}
	The MGR is a life-critical system. In order to avoid failure or malfunction the MGR shall operate normally in the presence of electromagnetic interference (EMI). The ability of a device to withstand EMI is called electromagnetic compatibility (EMC). The EMC of a device is determined by its ability to filter and shield. A filter removes unwanted frequencies in order to enhance desirable signals. Shielding is a metal or conductive surface that is placed on the device that acts as a barrier to electromagnetic fields. There are national and international standards that electronic appliances must comply with. In the USA standards are published by the Federal Communications Commission (FCC) in the Electronic Code of Federal Regulations (Title 47 Telecommunication). Furthermore the Radio Technical Commission for Aeronautics (RTCA) and Federal Aviation Administration (FAA) place additional requirements on avionics electronics. RTCA DO-160, Environmental Conditions and Test Procedures for Airborne Equipment, requires that aircraft hardware is tested to verify and validate EMC. FAA report DOT/FAA/CT 86/40, Aircraft electromagnetic compatibility, is a guideline document that deals with EMC for commercial aviation. In order to meet the requirements placed on a life-critical system, the MGR must comply with the guidelines presented in DOT/FAA/CT 86/40.

		\subsection{Verification and Validation}
		The MGR equipment shall undergo full-compliance EMC testing at a certified supplier. Conducted testing and Radiated testing shall occur and will involve:
			\begin{itemize}
				
				\item Radiated emissions tests	
				
				\item Conducted emissions tests	
				
				\item Radiated immunity tests	
				
				\item Conducted immunity tests	
				
				\item Electrostatic discharge tests
				
				\item Fast transient/burst tests

				\item Surge/Voltage Spike tests
				
				\item Magnetic Effect tests
				
				\item Power Input tests
				
				\item Audio Frequency Conducted Susceptibility tests
				
				\item Induced Signal Susceptibility tests
				
				\item Radio frequency susceptibility tests

			\end{itemize}

		\subsection{Certification}
		In order to ensure prompt approval by regulatory authorities EMC certification objectives will be developed by a system qualification engineering team. Throughout the design of the MGR equipment audits shall take place to ensure compliance with the EMC certification objectives.

	\section{Environmental Factors}
	In addition to the EMC compliance, the equipment throughout the system shall operate normally in non-ideal climate conditions.

		\subsection{Category II}
		The International Civil Aviation Organization (ICAO) classifies precision landing system approaches based on their decision height, minimum runway visual range and minimum guidance altitude. The categories are named Category I through III, with the dependency on the system increasing and the error window decreasing as the name increments. A Category II landing system is required in Mordant because of its foggy and low-visibility climate. The system shall comply within the minimum parameters for the error window as specified by Category II systems in Table 1:

		\renewcommand{\arraystretch}{1.5}
		\setlength{\tabcolsep}{18pt}
		\begin{table}[h!]
			\centering
			\begin{tabular}{|p{5cm}|p{5cm}|p{5cm}|}
				 \hline
				 \multicolumn{3}{|c|}{Table 1: Category II Specifications} \\
				 \hline
				 Decision Height & Runway Visual Range & Guidance Altitude\\
				 \hline
				 100 feet & $\frac{1}{2}$ nautical miles & 50 feet\\
				 \hline
			\end{tabular}
		\end{table}

		\subsection{Earthquakes}
		The equipment shall be functional during the course of an earthquake, or other violent and destructive tremors.


\end{document}