%%%%%%%%%%%%%%%%%%%%%%%%%%%%%%%%%%%%%%%%%%%%%%%%%%%%%%%%%%%%%%%%%%%%%%%%%%%%%%
% Report LaTeX Template Version 1.0 (December 8 2014)
%
% This template has been downloaded from: http://www.LaTeXTemplates.com
%
% Original author: Brandon Fryslie With extensive modifications by: Vel
% (vel@latextemplates.com)
%
% License: CC BY-NC-SA 3.0 (http://creativecommons.org/licenses/by-nc-sa/3.0/)
%
%%%%%%%%%%%%%%%%%%%%%%%%%%%%%%%%%%%%%%%%%%%%%%%%%%%%%%%%%%%%%%%%%%%%%%%%%%%%%%

\documentclass[usletter, 12pt]{article}
%%%%%%%%%%%%%%%%%%%%%%%%%%%%%%%%%%%%%%%%%
% Contract Structural Definitions File Version 1.0 (December 8 2014)
%
% Created by: Vel (vel@latextemplates.com)
% 
% This file has been downloaded from: http://www.LaTeXTemplates.com
%
% License: CC BY-NC-SA 3.0 (http://creativecommons.org/licenses/by-nc-sa/3.0/)
%
%%%%%%%%%%%%%%%%%%%%%%%%%%%%%%%%%%%%%%%%%

\usepackage{geometry} % Required to modify the page layout
\usepackage{multicol}
\usepackage{amsmath}
\usepackage{amssymb}

\usepackage[pdftex]{graphicx}
\usepackage{wrapfig}
\usepackage[font=scriptsize, labelfont=bf]{caption}
\usepackage[utf8]{inputenc} % Required for including letters with accents
\usepackage[T1]{fontenc} % Use 8-bit encoding that has 256 glyphs

\usepackage{avant} % Use the Avantgarde font for headings
\usepackage{courier}
\usepackage{xparse}
\usepackage{xcolor}
\usepackage{listings}  % for code verbatim and console outputs

\setlength{\textwidth}{16cm} % Width of the text on the page
\setlength{\textheight}{23cm} % Height of the text on the page
\setlength{\oddsidemargin}{0cm} % Width of the margin - negative to move text left, positive to move it right
\setlength{\topmargin}{-1.25cm} % Reduce the top margin

\setlength{\parindent}{0mm} % Don't indent paragraphs
\setlength{\parskip}{2.5mm} % Whitespace between paragraphs
\renewcommand{\baselinestretch}{1.5}

\definecolor{green}{rgb}{0.18, 0.55, 0.34}

\graphicspath{ {figures/} }
\captionsetup[table]{skip=10pt}

\lstset{language=C, keywordstyle={\bfseries \color{black}}}

% defines algorithm counter for chapter-level
\newcounter{nalg}[section]

%defines appearance of the algorithm counter
\renewcommand{\thenalg}{\thesection .\arabic{nalg}}

% defines a new caption label as Algorithm x.y
\DeclareCaptionLabelFormat{algocaption}{Algorithm \thenalg}

% defines the algorithm listing environment
\lstnewenvironment{pseudocode}[1][] {
    \refstepcounter{nalg}  % increments algorithm number
    \captionsetup{font=normalsize, labelformat=algocaption, labelsep=colon}
    \lstset{
        breaklines=true,
        mathescape=true,
        numbers=left,
        numberstyle=\scriptsize,
        basicstyle=\footnotesize\ttfamily,
        keywordstyle=\color{black}\bfseries,
        keywords={input, output, return, parallel, function, for, to, in, if,
        else, foreach, while, and, or, new, print},
        xleftmargin=.04\textwidth,
        #1
    }
}{}

\renewcommand{\familydefault}{\sfdefault}  % default font for entire document


\newcommand{\project}{Homework 5: \\ 777 For 7s}
\newcommand{\Sabbir}{Sabbir Ahmed}

\begin{document}

    \begin{titlepage}

        \vspace*{\fill} % Add whitespace above to center the title page content
        \begin{center}

            {\LARGE \project~Report}\\ [1.5cm]

            Submitted: \today
            
            \vspace*{\fill}

            \Sabbir

        \end{center}
        \vspace*{\fill} % Add whitespace below to center the title page content

    \end{titlepage}

    \section{Description} This project required implementation of a slot machine simulation on the AVR Dragon. \\~\\
    \noindent To lure the user into playing the game, they are greeted by a
    welcoming “HELLO” screen on the LCD until the push button is pressed. \\~\\
    \noindent The player, who starts with 10 credits/earnings, will be playing
    a slot machine game where only A through F are represented. The slot
    machine has only three slots, so only words of three characters long can be
    made. The player will place a bet with a minimum of 1 and maximum of 9 to
    see if they can form a word. The player cannot bet more than their total
    earnings. \\~\\
    \noindent Once a betting amount is decided, the player will press the
    pushbutton to start spinning the first character, and a second pushbutton
    press should slow the spinning down until it stops. The next pushbutton
    press will activate the second slot, then slow the second slot until it
    stops, similar to the first, and the same process will follow for the last
    slot. If a word is formed that matches one in the dictionary stored in the
    device’s memory, then multiply the bet by a predetermined value depending
    on the word that was spelled and add it to their total earnings. \\~\\
    \noindent The table below describes the various tiers of words and their
    multiplier value:

    \begin{table}[h]
        \centering
        \begin{tabular}{@{\extracolsep{\fill}} cl}

        \textbf{Multiplier} & \textbf{Words} \\
        \hline
        1 & BAC, CAF, DEF, FAB, FAE, FAC  \\
        2 & ADC, CAD, DAE, DEB, FEB \\
        3 & CAB, FAD, FDA \\
        4 & ACE, BAD \\
        5 & BED, DAB, FED \\
        \end{tabular}
    \end{table}

    \noindent If a word with a multiplier of 4 or 5 is spelled, then flash the
    LED on and off for 2 seconds to signify some kind of jackpot. \\~\\
    \noindent If no word is formed then subtract the bet from their earnings.
    \\~\\
    \noindent If the player has 0 earnings remaining, then sound the buzzer to
    signify that they have lost and have no more earnings to spend. A “LOSE”
    screen is displayed and reset is the only accepted input at this point.
    \\~\\
    \noindent A reset may occur at any point of the game and should bring the
    state of the device back to the initial “HELLO” screen. \\~\\
    \section{Implementation} The project was developed with a bottom-up
    design. Functions with more frequent usage and higher priority were
    developed first, and later pieced together to contribute to higher level
    functionalities. Because of reasons later detailed in the Troubleshooting
    section, the entire source code had to be contained in a single
    \texttt{main.c} script.

    \subsection{Memory Management} The implementation emphasized the limited
    memory on the chip. Usage of strings were minimized, and menu strings were
    stored as constant \codeword{PROGMEM} variables for reuse. The program
    memory was also utilized to communicate with the LCD with
    \codeword{lcd_puts_P}.

    \section{Usage} The project depends on user inputs from the Butterfly
    joystick. The program sits in a loop until the \texttt{RIGHT} joystick
    button is pressed. The user may press the \texttt{UP} or \texttt{DOWN}
    joystick buttons to toggle between the \texttt{FTL} and \texttt{ATL} modes.
    After a mode is selected, the servo begins sweeping with its corresponding
    configuration. The program terminates 10 seconds after a
    \texttt{LOCALSWEEP} followed by a \texttt{FULLSWEEP} are executed.

    \section{Testing and Troubleshooting} Debugging the functionalities of the
    program required extensive usage of hardware. This attribute resulted in
    numerous reprogramming of the AVR Dragon. Small changes, especially during
    experimentation of values of signals sent to external devices such as the
    servo, slowed down the development process. \\~\\
    \noindent The development process took place on a Linux machine since it
    was much more user-friendly and convenient that Atmel Studio. \texttt{avr-
    gcc} was used to compile the source code, and \texttt{avrdude} provided the
    AVR libraries. Once substantial progress was determined in the development,
    the source code would be pushed to the version control system to transfer
    to a ready-for-upload Windows machine to program the chip. Although this
    process proved to be a faster approach for development, it required all the
    changes to be contained on a single file for frequent reprogramming. \\~\\
    \noindent Most of the issues arose from the external devices.

        \subsection{JTAG Cable Issues} The JTAG cable from the kit does not
        work anymore. Temporary replacements were obtained from peers, which
        slowed down development further.

    \section{Code} The C scripts used for the implementation has been attached
    alongside the report.

        \subsection{main.c} The entire source code of the project.

\end{document}
