%%%%%%%%%%%%%%%%%%%%%%%%%%%%%%%%%%%%%%%%%
% Template
% LaTeX Template
% Version 1.0 (December 8 2014)
%
% This template has been downloaded from:
% http://www.LaTeXTemplates.com
%
% Original author:
% Brandon Fryslie
% With extensive modifications by:
% Vel (vel@latextemplates.com)
%
% License:
% CC BY-NC-SA 3.0 (http://creativecommons.org/licenses/by-nc-sa/3.0/)
%
% Authors:
% Sabbir Ahmed
%
%%%%%%%%%%%%%%%%%%%%%%%%%%%%%%%%%%%%%%%%%

\documentclass[paper=usletter, fontsize=12pt]{article}
%%%%%%%%%%%%%%%%%%%%%%%%%%%%%%%%%%%%%%%%%
% Contract Structural Definitions File Version 1.0 (December 8 2014)
%
% Created by: Vel (vel@latextemplates.com)
% 
% This file has been downloaded from: http://www.LaTeXTemplates.com
%
% License: CC BY-NC-SA 3.0 (http://creativecommons.org/licenses/by-nc-sa/3.0/)
%
%%%%%%%%%%%%%%%%%%%%%%%%%%%%%%%%%%%%%%%%%

\usepackage{geometry} % Required to modify the page layout
\usepackage{multicol}
\usepackage{amsmath}
\usepackage{amssymb}

\usepackage[pdftex]{graphicx}
\usepackage{wrapfig}
\usepackage[font=scriptsize, labelfont=bf]{caption}
\usepackage[utf8]{inputenc} % Required for including letters with accents
\usepackage[T1]{fontenc} % Use 8-bit encoding that has 256 glyphs

\usepackage{avant} % Use the Avantgarde font for headings
\usepackage{courier}
\usepackage{xparse}
\usepackage{xcolor}
\usepackage{listings}  % for code verbatim and console outputs

\setlength{\textwidth}{16cm} % Width of the text on the page
\setlength{\textheight}{23cm} % Height of the text on the page
\setlength{\oddsidemargin}{0cm} % Width of the margin - negative to move text left, positive to move it right
\setlength{\topmargin}{-1.25cm} % Reduce the top margin

\setlength{\parindent}{0mm} % Don't indent paragraphs
\setlength{\parskip}{2.5mm} % Whitespace between paragraphs
\renewcommand{\baselinestretch}{1.5}

\definecolor{green}{rgb}{0.18, 0.55, 0.34}

\graphicspath{ {figures/} }
\captionsetup[table]{skip=10pt}

\lstset{language=C, keywordstyle={\bfseries \color{black}}}

% defines algorithm counter for chapter-level
\newcounter{nalg}[section]

%defines appearance of the algorithm counter
\renewcommand{\thenalg}{\thesection .\arabic{nalg}}

% defines a new caption label as Algorithm x.y
\DeclareCaptionLabelFormat{algocaption}{Algorithm \thenalg}

% defines the algorithm listing environment
\lstnewenvironment{pseudocode}[1][] {
    \refstepcounter{nalg}  % increments algorithm number
    \captionsetup{font=normalsize, labelformat=algocaption, labelsep=colon}
    \lstset{
        breaklines=true,
        mathescape=true,
        numbers=left,
        numberstyle=\scriptsize,
        basicstyle=\footnotesize\ttfamily,
        keywordstyle=\color{black}\bfseries,
        keywords={input, output, return, parallel, function, for, to, in, if,
        else, foreach, while, and, or, new, print},
        xleftmargin=.04\textwidth,
        #1
    }
}{}

\renewcommand{\familydefault}{\sfdefault}  % default font for entire document
 % specifies the document layout and style
\include{tikz}
\usetikzlibrary{arrows.meta}

\begin{document}

    \documentinfo{\today}{08}
    \vspace{-0.2in}

    \begin{itemize}

        \item[\textbf{3.4}]
        \begin{itemize}

            \item[\textbf{4}] Show that $\mathbb{Z}_{5}^{\times}$ is not
            isomorphic to $\mathbb{Z}_{8}^{\times}$ by showing that the first
            group has an element of order 4 but the second group does not
            \begin{cproof}

                The elements in each of the groups
                \begin{align*}
                    \{[1], [2], [3], [4]\} & \in \mathbb{Z}_{5}^{\times},\ o(\mathbb{Z}_{5}^{\times}) = 4\\
                    \{[1], [3], [5], [7]\} & \in \mathbb{Z}_{8}^{\times},\ o(\mathbb{Z}_{8}^{\times}) = 4
                \end{align*}

                In $\mathbb{Z}_{5}^{\times}$
                \begin{align*}
                    [2]^4 &= [1], \ o([2]) = 4\\
                    [3]^4 &= [1], \ o([3]) = 4\\
                    [4]^2 &= [1], \ o([4]) = 2
                \end{align*}
                Therefore, $\mathbb{Z}_{5}^{\times}$ is a cyclic group with generators $[2]$ and $[3]$\\
                In $\mathbb{Z}_{8}^{\times}$
                \begin{align*}
                    [3]^2 &= [1], \ o([3]) = 2\\
                    [5]^2 &= [1], \ o([5]) = 2\\
                    [7]^2 &= [1], \ o([7]) = 2
                \end{align*}
                No elements in $\mathbb{Z}_{8}^{\times}$ is of the same order as its group order\\
                which implies $\mathbb{Z}_{8}^{\times}$ is non-cyclic\\
                Therefore, $\mathbb{Z}_{5}^{\times}$ is not isomorphic to
                $\mathbb{Z}_{8}^{\times}$ since the first group is cyclic
                unlike the latter \qedhere

            \end{cproof}

            \item[\textbf{7}] Let $G$ be a group. Show that the group $(G, *)$
            defined in Exercise 3 of Section 3.1 is isomorphic to $G$.
            \begin{cproof}

                Given $(G, *)$ is a group where $a*b=b\cdot a$\\
                Let $\phi: (G, *) \rightarrow (G, \cdot)$ as
                \begin{align*}
                    \phi(a) & = \phi(e * a)\\
                    & = a * e\\
                    & = a, \ \forall \ a \in (G, *)
                \end{align*}

                We need to show $\phi(a*b)=\phi(a)\cdot\phi(b)$
                \begin{align*}
                    \phi(a*b) & = b \cdot a\\
                    & = b*e \cdot a*a\\
                    & = \phi(b)\cdot\phi(a) \qedhere
                \end{align*}

            \end{cproof}

            \item[\textbf{11}] Let $G$ be the set of all matrices in
            $GL_2(\mathbb{Z}_3)$ of the form $\left[\begin{tabular}{cc}
                    m & b \\
                    0 & 1
                \end{tabular}\right]$. That is, $m, b \in \mathbb{Z}_3$ and
            $m\neq [0]_3$. Show that $G$ is a subgroup of
            $GL_2(\mathbb{Z}_3)$ that is isomorphic to $S_3$.
            \begin{cproof}

                Given
                \begin{equation*}
                    G = \left\{
                        \left[\begin{tabular}{cc}
                            1 & 0 \\
                            0 & 1
                        \end{tabular}\right],
                        \left[\begin{tabular}{cc}
                            1 & 1 \\
                            0 & 1
                        \end{tabular}\right],
                        \left[\begin{tabular}{cc}
                            1 & 2 \\
                            0 & 1
                        \end{tabular}\right],
                        \left[\begin{tabular}{cc}
                            2 & 0 \\
                            0 & 1
                        \end{tabular}\right],
                        \left[\begin{tabular}{cc}
                            2 & 1 \\
                            0 & 1
                        \end{tabular}\right],
                        \left[\begin{tabular}{cc}
                            2 & 2 \\
                            0 & 1
                        \end{tabular}\right]
                    \right\}
                \end{equation*}
                The non-empty, finite set $G$ is a subgroup if $xy^{-1} \in G$, $\forall \ x,y \in G$\\

                Let $\left[\begin{tabular}{cc}
                            m & b \\
                            0 & 1
                    \end{tabular}\right],\left[\begin{tabular}{cc}
                            n & a \\
                            0 & 1
                    \end{tabular}\right] \in G$, where $m,n \neq [0]_3$
                Then
                \begin{align*}
                    \left[\begin{tabular}{cc}
                            m & b \\
                            0 & 1
                    \end{tabular}\right]\left[\begin{tabular}{cc}
                            n & a \\
                            0 & 1
                    \end{tabular}\right] & = \left[\begin{tabular}{cc}
                            mn & b+am \\
                            0 & 1
                    \end{tabular}\right]
                \end{align*}
                Since $m,n \neq [0]_3$, then $mn \neq [0]_3$\\

                Therefore $\left[\begin{tabular}{cc}
                            mn & b+am \\
                            0 & 1
                    \end{tabular}\right] \in G$, and $G$ is a subgroup of
                    $GL_2(\mathbb{Z}_3)$\\

                Also, if $a=\left[\begin{tabular}{cc}
                            1 & 1 \\
                            0 & 1
                    \end{tabular}\right], b=\left[\begin{tabular}{cc}
                            2 & 1 \\
                            0 & 1
                    \end{tabular}\right]$, $a,b \in G$,\\
                then
                \salign{1}
                \begin{align*}
                    a^3 & = \left(\left[\begin{tabular}{cc}
                            1 & 1 \\
                            0 & 1
                    \end{tabular}\right]\right)^3\\
                    & = \left[\begin{tabular}{cc}
                            1 & 0 \\
                            0 & 1
                    \end{tabular}\right]\\
                    b^2 & = \left(\left[\begin{tabular}{cc}
                            2 & 1 \\
                            0 & 1
                    \end{tabular}\right]\right)^2\\
                    & = \left[\begin{tabular}{cc}
                            1 & 0 \\
                            0 & 1
                    \end{tabular}\right]\\
                    a^2b & = \left(\left[\begin{tabular}{cc}
                            1 & 1 \\
                            0 & 1
                    \end{tabular}\right]\right)^2\left[\begin{tabular}{cc}
                            1 & 2 \\
                            0 & 1
                    \end{tabular}\right]\\
                    & = \left[\begin{tabular}{cc}
                            1 & 2 \\
                            0 & 1
                    \end{tabular}\right]\left[\begin{tabular}{cc}
                            1 & 1 \\
                            0 & 1
                    \end{tabular}\right]\\
                    & = ba
                \end{align*}
                \endgroup
                Therefore, $G$ is similar to $S_3=\{e,a,a^2,b,ab,a^2b\}$,\\
                where $a^3=e$, $b^2=e$, $ba=a^2b$\\

                Thus, let $\phi: G \rightarrow S_3$ as
                \salign{1}
                \begin{align*}
                    \phi\left(\left[\begin{tabular}{cc}
                            1 & 1 \\
                            0 & 1
                    \end{tabular}\right]\right) & = (1,2,3)\\
                    \phi\left(\left[\begin{tabular}{cc}
                            2 & 1 \\
                            0 & 1
                    \end{tabular}\right]\right) & = (1,2)
                \end{align*}
                \endgroup
                Then
                \salign{1}
                \begin{align*}
                    \phi\left(\left(\left[\begin{tabular}{cc}
                            1 & 1 \\
                            0 & 1
                    \end{tabular}\right]\right)^i\left(\left[\begin{tabular}{cc}
                            2 & 1 \\
                            0 & 1
                    \end{tabular}\right]\right)^i\right) & = (1,2,3)^i(1,2)^i, \ i = 0,1,2, \ j = 0,1
                \end{align*}
                \endgroup

                Which is both one-to-one and onto \qedhere

            \end{cproof}

            \item[\textbf{14}] Let $G=\{x \in \mathbb{R} \mid x > 0 \text{ and
            } x \neq 1\}$, and define $*$ on $G$ by $a * b = a^{\ln{b}}$. Show
            that $G$ is isomorphic to the multiplicative group
            $\mathbb{R}^{\times}$. (See Exercise 9 of Section 3.1.)
            \begin{cproof}

                Assume $\phi: G \rightarrow \mathbb{R}^{\times}$ is one-to-one
                and onto\\
                Let $y \neq 0 \in \mathbb{R}^{\times}$, such that $e^y>0 \in G$
                \begin{align*}
                    \phi(e^y) & = \ln{e^y}\\
                    & = y
                \end{align*}

            \end{cproof}

            \item[\textbf{17}] Let $\phi: G_1 \rightarrow G_2$ be a group
            isomorphism. Prove that if $H$ is a subgroup of $G_1$, then
            $\phi(H)=\{y\in G_2 \mid y = \phi(h) \text{ for some } h \in H\}$
            is a subgroup of $G_2$.
            \begin{cproof}

                Since $\phi:G_1 \rightarrow G_2$ is a group isomorphism,
                $\phi(e_1)=e_2$\\
                Since $H$ is a subgroup,
                \begin{align*}
                    e_1 & \in H\\
                    \Rightarrow e_2 & \in \phi(H)
                \end{align*}

                A non-empty set $G$ is a subgroup if $xy^{-1}\in G$, $\forall \
                x,y\in G$\\
                Let $x,y \in \phi(H)$\\
                Then, there exists $h_1,h_2 \in H$, such that
                \begin{align*}
                    \phi(h_1) & = x\\
                    \phi(h_2) & = y
                \end{align*}
                Also, since $\phi$ is homomorphic,
                \begin{align*}
                    \phi(h_2^{-1}) & = (\phi(h_2))^{-1}\\
                    & = y^{-1}\\
                    \phi(h_1h_2^{-1}) & = \phi(h_1)\phi(h_2^{-1})\\
                    & = xy^{-1}
                \end{align*}

                Since $H$ is a subgroup, $h_1h_2^{-1}\in H$, $\forall \ h_1,h_2
                \in H$\\
                Therefore,
                \begin{align*}
                    \phi(h_1h_2^{-1}) & = xy^{-1}\\
                    & \in \phi(H)
                \end{align*}

                That is, $\phi(h_1h_2^{-1}) \in \phi(H)$, $\forall \ x,y \in
                \phi(H)$ \qedhere

            \end{cproof}

            \item[\textbf{24}] Let $G = \mathbb{R} - \{-1\}$. Define $*$ on $G$
            by $a*b = a+b+ab$. Show that $G$ is isomorphic to the
            multiplicative group $\mathbb{R}^{\times}$. (See Exercise 13 of
            Section 3.1.) \\ \textit{Hint}: Remember that an isomorphism maps
            identity to identity. Use this fact to help find the necessary
            mapping.
            \begin{cproof}

                Let $\phi: G \rightarrow \mathbb{R}^{\times}$ as
                $\phi(a)=1+a$\\

                Let $a=b$\\
                Then, $1+a=1+b$\\
                Therefore, $\phi(a)=\phi(b)$ and $\phi$ is well defined\\

                Let $\phi(a)=\phi(b)$\\
                Then $1+a=1+b$, which implies $a=b$\\
                Therefore, $\phi$ is one-to-one\\

                Let $x \in \mathbb{R}^{\times}$\\
                Therefore, $x \neq 0$ and $\exists{y = x-1} \in G$\\
                Since $\phi(x-1)=1+x-1=x$, $\phi$ is also onto\\

                To show $\phi(a*b)=\phi(a)\phi(b)$, consider
                \begin{align*}
                    \phi(a*b) & = 1+(a*b)\\
                    & = 1 + a + b + ab\\
                    & = (1+a)(1+b)\\
                    & = \phi(a)\phi(b)
                \end{align*}
                Therefore, $G \cong \mathbb{R}^{\times}$ \qedhere

            \end{cproof}

            \item[\textbf{26}] Let $G_1$ and $G_2$ be groups. A function from
            $G$ into $G_2$ that preserves products but is not necessarily a
            one-to-one correspondence will be called a group homomorphism, from
            the Greek word \textit{homos} meaning same. Show that $\phi:
            \text{GL}_2(\mathbb{R}) \rightarrow \mathbb{R}^{\times}$ defined by
            $\phi(A)=\det(A)$ for all matrices $A \in \text{GL}_2(\mathbb{R})$
            is a group homomorphism.
            \begin{cproof}

                Consider $\phi(A)=\det(A)$\\
                Since $\text{GL}_2(\mathbb{R})$ is a field, it is also abelian,
                and therefore
                \begin{equation*}
                    \det(AB) = \det(A)\det(B)
                \end{equation*}
                Thus,
                \begin{align*}
                    \phi(AB) & = \det(AB) \\
                    & = \det(A)\det(B)\\
                    & = \phi(A)\phi(B) \qedhere
                \end{align*}


            \end{cproof}

        \end{itemize}

        \item[\textbf{3.5}]

        \begin{itemize}

            \item[\textbf{2}] Let $G$ be a group and let $a \in G$ be an
            element of order 30. List the powers of $a$ that have order 2,
            order 3 or order 5.
            \begin{cproof}

                \begin{align*}
                    (a^{15})^{2} & = e
                \end{align*}
                \begin{align*}
                    (a^{10})^{3} & = e \\
                    (a^{20})^{3} & = e
                \end{align*}
                \begin{align*}
                    (a^{6})^{5} & = e \\
                    (a^{12})^{5} & = e \\
                    (a^{18})^{5} & = e \\
                    (a^{24})^{5} & = e
                \end{align*}

                Therefore,\\
                the powers of $a$ of order 2 is $a^{15}$\\
                the powers of $a$ of order 3 are $a^{10},a^{20}$\\
                the powers of $a$ of order 5 are $a^{6},a^{12},a^{18},a^{24}$ \qedhere

            \end{cproof}

            \item[\textbf{3}] Give the subgroup diagrams of the following
            groups.
            \begin{enumerate}

                \item[\textbf{a}] $\mathbb{Z}_{24}$
                \begin{cproof}

                    The generators of $\mathbb{Z}_{24}$ are $\langle1\rangle,
                    \langle2\rangle, \langle3\rangle, \langle4\rangle,
                    \langle6\rangle, \langle8\rangle,
                    \langle12\rangle, \langle0\rangle$
                    \begin{align*}
                        \langle1\rangle & = \mathbb{Z}_{24}\\
                        \langle2\rangle & = \{2,4,6,8,10,12,14,16,18,20,22,0\}\\
                        \langle3\rangle & = \{3,6,9,12,15,18,21,0\}\\
                        \langle4\rangle & = \{4,8,12,16,20,0\}\\
                        \langle6\rangle & = \{6,12,18,0\}\\
                        \langle8\rangle & = \{8,16,0\}\\
                        \langle12\rangle & = \{12,0\}\\
                        \langle0\rangle & = \{0\}
                    \end{align*}

                    \begin{figure}[h]
                        \centering
                        \caption{Subgroup Diagram of $\mathbb{Z}_{24}$}
                        \begin{tikzpicture}[->,scale=1]
                            \begin{scope}[every node/.style={circle}]
                                \node (1) at (2,0) {$\mathbb{Z}_{24}$};
                                \node (2) at (0,2) {$\langle2\rangle$};
                                \node (3) at (4,2) {$\langle3\rangle$};
                                \node (4) at (-2,4) {$\langle4\rangle$};
                                \node (6) at (2,4) {$\langle6\rangle$};
                                \node (8) at (-4,6) {$\langle8\rangle$};
                                \node (12) at (0,6) {$\langle12\rangle$};
                                \node (0) at (-2,8) {$\langle0\rangle$};
                            \end{scope}
                            \begin{scope}
                                \path [-] (1) edge (2);
                                \path [-] (2) edge (3);
                                \path [-] (3) edge (1);
                                \path [-] (2) edge (4);
                                \path [-] (3) edge (6);
                                \path [-] (4) edge (6);
                                \path [-] (4) edge (8);
                                \path [-] (6) edge (12);
                                \path [-] (8) edge (12);
                                \path [-] (8) edge (0);
                                \path [-] (12) edge (0);
                            \end{scope}
                        \end{tikzpicture}
                    \end{figure}

                \end{cproof}
                \newpage

                \item[\textbf{b}] $\mathbb{Z}_{36}$
                \begin{cproof}

                    The generators of $\mathbb{Z}_{36}$ are $\langle1\rangle,
                    \langle2\rangle, \langle3\rangle, \langle6\rangle,
                    \langle9\rangle, \langle12\rangle, \langle18\rangle,
                    \langle0\rangle$
                    \begin{align*}
                        \langle1\rangle & = \mathbb{Z}_{36}\\
                        \langle2\rangle & = \{2,4,6,8,10,12,14,16,18,20,22,24,26,28,30,32,34,0\}\\
                        \langle3\rangle & = \{3,6,9,12,15,18,21,24,27,30,33,0\}\\
                        \langle4\rangle & = \{4,8,12,16,20,24,28,32,0\}\\
                        \langle6\rangle & = \{6,12,18,24,30,0\}\\
                        \langle9\rangle & = \{9,18,27,0\}\\
                        \langle12\rangle & = \{12,24,0\}\\
                        \langle18\rangle & = \{18,0\}\\
                        \langle0\rangle & = \{0\}
                    \end{align*}

                    \begin{figure}[h]
                        \centering
                        \caption{Subgroup Diagram of $\mathbb{Z}_{36}$}
                        \begin{tikzpicture}[->,scale=1]
                            \begin{scope}[every node/.style={circle}]
                                \node (1) at (2,0) {$\mathbb{Z}_{36}$};
                                \node (2) at (0,2) {$\langle2\rangle$};
                                \node (3) at (4,2) {$\langle3\rangle$};
                                \node (4) at (-2,4) {$\langle4\rangle$};
                                \node (6) at (2,4) {$\langle6\rangle$};
                                \node (9) at (6,4) {$\langle9\rangle$};
                                \node (12) at (0,6) {$\langle12\rangle$};
                                \node (18) at (4,6) {$\langle18\rangle$};
                                \node (0) at (2,8) {$\langle0\rangle$};
                            \end{scope}
                            \begin{scope}
                                \path [-] (1) edge (2);
                                \path [-] (3) edge (1);
                                \path [-] (2) edge (4);
                                \path [-] (2) edge (6);
                                \path [-] (3) edge (6);
                                \path [-] (3) edge (9);
                                \path [-] (4) edge (12);
                                \path [-] (9) edge (18);
                                \path [-] (6) edge (12);
                                \path [-] (6) edge (18);
                                \path [-] (12) edge (0);
                                \path [-] (18) edge (0);
                            \end{scope}
                        \end{tikzpicture}
                    \end{figure}

                \end{cproof}
                \newpage

            \end{enumerate}

            \item[\textbf{10}] Find all cyclic subgroups of $\mathbb{Z}_{6}
            \times \mathbb{Z}_{3}$
            \begin{cproof}

                All the cyclic subgroups by checking the multiples of all
                elements in the group
                \begin{align*}
                    \langle(0, 0)\rangle & = \{(0, 0)\} \\
                    \langle(0, 1)\rangle & = \{(0, 0),(0, 1),(0, 2)\} \\
                    & = \langle(0, 2)\rangle \\
                    \langle(1, 0)\rangle & = \{(0, 0),(1, 0),(2, 0),(3, 0),(4, 0),(5, 0)\} \\
                    & = \langle(5, 0)\rangle \\
                    \langle(1, 1)\rangle & = \{(0, 0),(1, 1),(2, 2),(3, 0),(4, 1),(5, 2)\} \\
                    & = \langle(5, 2)\rangle \\
                    \langle(1, 2)\rangle & = \{(0, 0),(1, 2),(2, 1),(3, 0),(4, 2),(5, 1)\} \\
                    & = \langle(5, 1)\rangle \\
                    \langle(2, 0)\rangle & = \{(0, 0),(2, 0),(4, 0)\} \\
                    & = \langle(4, 0)\rangle \\
                    \langle(2, 1)\rangle & = \{(0, 0),(2, 1),(4, 2)\} \\
                    & = \langle(4, 2)\rangle \\
                    \langle(2, 2)\rangle & = \{(0, 0),(2, 2),(4, 1)\} \\
                    & = \langle(4, 1)\rangle \\
                    \langle(3, 0)\rangle & = \{(0, 0),(3, 0)\} \\
                    \langle(3, 1)\rangle & = \{(0, 0),(3, 1),(0, 2),(3, 0),(0, 1),(3, 2)\} \\
                    & = \langle(3, 2)\rangle \qedhere
                \end{align*}

            \end{cproof}

            \item[\textbf{12}] Let $a,b$ be positive integers, and let
            $d=\gcd(a,b)$ and $m=\text{lcm}(a,b)$. Use Proposition 3.5.5 to
            prove that $\mathbb{Z}_{a}\times \mathbb{Z}_{b} \cong
            \mathbb{Z}_{d} \times \mathbb{Z}_{m}$
            \begin{cproof}
            \end{cproof}

            \item[\textbf{13}] Show that in a finite cyclic group of order $n$,
            the equation $x^m=e$ has exactly $m$ solutions, for each positive
            integer $m$ that is a divisor of $n$.
            \begin{cproof}
            \end{cproof}

            \item[\textbf{17}] Let $G$ be the set of all $3\times 3$ matrices
            of the form $\left[\begin{tabular}{ccc}
                    1 & a & b \\
                    0 & 1 & c \\
                    0 & 0 & 1
                \end{tabular}\right]$.

            \begin{itemize}

                \item[\textbf{a}] Show that if $a,b,c\in \mathbb{Z}_3$, the $G$
                is a group with exponent 3.
                \begin{cproof}

                    Consider
                    \salign{1}
                    \begin{align*}
                        \left(\left[\begin{tabular}{ccc}
                            1 & a & b \\
                            0 & 1 & c \\
                            0 & 0 & 1
                        \end{tabular}\right]\right)^2 & =
                        \left[\begin{tabular}{ccc}
                            1 & a & b \\
                            0 & 1 & c \\
                            0 & 0 & 1
                        \end{tabular}\right]
                        \left[\begin{tabular}{ccc}
                            1 & a & b \\
                            0 & 1 & c \\
                            0 & 0 & 1
                        \end{tabular}\right] \\
                        & = \left[\begin{tabular}{ccc}
                            1 & a+a & b+ac+b \\
                            0 & 1 & c+c \\
                            0 & 0 & 1
                        \end{tabular}\right]\\
                        \left(\left[\begin{tabular}{ccc}
                            1 & a & b \\
                            0 & 1 & c \\
                            0 & 0 & 1
                        \end{tabular}\right]\right)^3 & =
                        \left[\begin{tabular}{ccc}
                            1 & a & b \\
                            0 & 1 & c \\
                            0 & 0 & 1
                        \end{tabular}\right]
                        \left[\begin{tabular}{ccc}
                            1 & a+a & b+ac+b \\
                            0 & 1 & c+c \\
                            0 & 0 & 1
                        \end{tabular}\right] \\
                        & = \left[\begin{tabular}{ccc}
                            1 & 3a & 3b+3ac \\
                            0 & 1 & 3c \\
                            0 & 0 & 1
                        \end{tabular}\right]
                    \end{align*}
                    \endgroup

                    Since $G$ has an exponent of 3,
                    \salign{1}
                    \begin{align*}
                        \left[\begin{tabular}{ccc}
                            1 & 3a & 3b+3ac \\
                            0 & 1 & 3c \\
                            0 & 0 & 1
                        \end{tabular}\right] & =
                        \left[\begin{tabular}{ccc}
                            1 & 0 & 0 \\
                            0 & 1 & 0 \\
                            0 & 0 & 1
                        \end{tabular}\right] \qedhere
                    \end{align*}
                    \endgroup

                \end{cproof}

                \item[\textbf{b}] Show that if $a,b,c\in \mathbb{Z}_2$, the $G$
                is a group with exponent 4.
                \begin{cproof}

                    Consider
                    \salign{1}
                    \begin{align*}
                        \left(\left[\begin{tabular}{ccc}
                            1 & a & b \\
                            0 & 1 & c \\
                            0 & 0 & 1
                        \end{tabular}\right]\right)^2 & =
                        \left[\begin{tabular}{ccc}
                            1 & a & b \\
                            0 & 1 & c \\
                            0 & 0 & 1
                        \end{tabular}\right]
                        \left[\begin{tabular}{ccc}
                            1 & a & b \\
                            0 & 1 & c \\
                            0 & 0 & 1
                        \end{tabular}\right] \\
                        & = \left[\begin{tabular}{ccc}
                            1 & a+a & b+ac+b \\
                            0 & 1 & c+c \\
                            0 & 0 & 1
                        \end{tabular}\right]\\
                        & = \left[\begin{tabular}{ccc}
                            1 & 0 & ac \\
                            0 & 1 & 0 \\
                            0 & 0 & 1
                        \end{tabular}\right]\\
                        \left(\left[\begin{tabular}{ccc}
                            1 & 0 & ac \\
                            0 & 1 & 0 \\
                            0 & 0 & 1
                        \end{tabular}\right]\right)^2 &=
                        \left[\begin{tabular}{ccc}
                            1 & 0 & ac \\
                            0 & 1 & 0 \\
                            0 & 0 & 1
                        \end{tabular}\right]
                        \left[\begin{tabular}{ccc}
                            1 & 0 & ac \\
                            0 & 1 & 0 \\
                            0 & 0 & 1
                        \end{tabular}\right]\\
                        & = \left[\begin{tabular}{ccc}
                            1 & 0 & ac+ac \\
                            0 & 1 & 0 \\
                            0 & 0 & 1
                        \end{tabular}\right]\\
                        & = \left[\begin{tabular}{ccc}
                            1 & 0 & 0 \\
                            0 & 1 & 0 \\
                            0 & 0 & 1
                        \end{tabular}\right] \qedhere
                    \end{align*}
                    \endgroup

                \end{cproof}

            \end{itemize}

            \item[\textbf{19}] Let $n=2^k$ for $k>2$. Prove that
            $\mathbb{Z}_{n}^{\times}$ is not cyclic. \\ \textit{Hint}: Show
            that $\pm 1$ satisfy the equation $x^2=1$, and that this is
            impossible in any cyclic group.
            \begin{cproof}

                Let $x=\frac{n}{2}+1$. Then
                \salign{1}
                \begin{align*}
                    x & = \bigg(\frac{n}{2}+1\bigg)^2\\
                    & = \bigg(\frac{2^k}{2}+1\bigg)^2\\
                    & = (2^{k-1}+1)^2\\
                    & = 2^{2k-2}+1+2^k\\
                    & = 1+2^k(2^k+1)
                \end{align*}
                \endgroup
                Therefore, $x^2 - 1 \equiv 0 \Mod{2^k}$, or $x^2=1$\\

                Now let $x=\frac{n}{2}-1$. Then
                \salign{1}
                \begin{align*}
                    x & = \bigg(\frac{n}{2}-1\bigg)^2\\
                    & = \bigg(\frac{2^k}{2}-1\bigg)^2\\
                    & = (2^{k-1}-1)^2\\
                    & = 2^{2k-2}+1-2^k\\
                    & = 1+2^k(2^k-1)
                \end{align*}
                \endgroup
                Therefore, $x^2 - 1 \equiv 0 \Mod{2^k}$, or $x^2=1$\\

                Therefore, the solutions to $x^2=1$ are $\pm 1,
                \frac{n}{2}\pm1$\\
                Therefore, the order of $\mathbb{Z}_{n}^{\times}$ are even,
                which is not possible in a cyclic group\\
                Therefore, $\mathbb{Z}_{n}^{\times}$ is not cyclic (by
                contradiction) \qedhere

            \end{cproof}

        \end{itemize}

    \end{itemize}

\end{document}
