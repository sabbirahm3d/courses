%%%%%%%%%%%%%%%%%%%%%%%%%%%%%%%%%%%%%%%%%
% Template
% LaTeX Template
% Version 1.0 (December 8 2014)
%
% This template has been downloaded from:
% http://www.LaTeXTemplates.com
%
% Original author:
% Brandon Fryslie
% With extensive modifications by:
% Vel (vel@latextemplates.com)
%
% License:
% CC BY-NC-SA 3.0 (http://creativecommons.org/licenses/by-nc-sa/3.0/)
%
% Authors:
% Sabbir Ahmed
%
%%%%%%%%%%%%%%%%%%%%%%%%%%%%%%%%%%%%%%%%%

\documentclass[paper=usletter, fontsize=12pt]{article}
%%%%%%%%%%%%%%%%%%%%%%%%%%%%%%%%%%%%%%%%%
% Contract Structural Definitions File Version 1.0 (December 8 2014)
%
% Created by: Vel (vel@latextemplates.com)
% 
% This file has been downloaded from: http://www.LaTeXTemplates.com
%
% License: CC BY-NC-SA 3.0 (http://creativecommons.org/licenses/by-nc-sa/3.0/)
%
%%%%%%%%%%%%%%%%%%%%%%%%%%%%%%%%%%%%%%%%%

\usepackage{geometry} % Required to modify the page layout
\usepackage{multicol}
\usepackage{amsmath}
\usepackage{amssymb}

\usepackage[pdftex]{graphicx}
\usepackage{wrapfig}
\usepackage[font=scriptsize, labelfont=bf]{caption}
\usepackage[utf8]{inputenc} % Required for including letters with accents
\usepackage[T1]{fontenc} % Use 8-bit encoding that has 256 glyphs

\usepackage{avant} % Use the Avantgarde font for headings
\usepackage{xparse}
\usepackage{xcolor}
\usepackage{listings}  % for code verbatim and console outputs

\setlength{\textwidth}{16cm} % Width of the text on the page
\setlength{\textheight}{23cm} % Height of the text on the page
\setlength{\oddsidemargin}{0cm} % Width of the margin - negative to move text left, positive to move it right
\setlength{\topmargin}{-1.25cm} % Reduce the top margin

\setlength{\parindent}{0mm} % Don't indent paragraphs
\setlength{\parskip}{2.5mm} % Whitespace between paragraphs
\renewcommand{\baselinestretch}{1.2}

\renewcommand\familydefault{\sfdefault}  % default font for entire document

\definecolor{green}{rgb}{0.18, 0.55, 0.34}

\graphicspath{ {figures/} }
\captionsetup[table]{skip=10pt}

\lstset{language=C, keywordstyle={\bfseries \color{black}}}

% defines algorithm counter for chapter-level
\newcounter{nalg}[section]

%defines appearance of the algorithm counter
\renewcommand{\thenalg}{\thesection .\arabic{nalg}}

% defines a new caption label as Algorithm x.y
\DeclareCaptionLabelFormat{algocaption}{Algorithm \thenalg}

%defines the algorithm listing environment
\lstnewenvironment{pseudocode}[1][] {
    \refstepcounter{nalg} %increments algorithm number

    \captionsetup{labelformat=algocaption,labelsep=colon}
    \lstset{
        mathescape=true,
        frame=tB,
        numbers=left,
        numberstyle=\tiny,
        basicstyle=\scriptsize,
        keywordstyle=\color{black}\bfseries\em,
        keywords={,input, output, return, datatype, function, in, if, else, foreach, while, begin, end, },
        xleftmargin=.04\textwidth,
        #1
    }
}{}
 % specifies the document layout and style
\allowdisplaybreaks

%------------------------------------------------------------------------------
% document info command
\newcommand{\documentinfo}[5]{
    \begin{centering}
        \parbox{2in}{
        \begin{spacing}{1}
            \begin{flushleft}
                \begin{tabular}{l l}
                    #1 \\
                    #2 \\
                    #3 \\
                \end{tabular}\\
                \rule{\textwidth}{1pt}
            \end{flushleft}
        \end{spacing}
        }
    \end{centering}
}

\newcommand{\ans}{\textbf{Answer} \ }

\begin{document}

    \documentinfo{Sabbir Ahmed}{\textbf{DATE:} February 26, 2018}{\textbf{CMPE 320} HW 03}
    \vspace{-0.2in}

    \begin{enumerate}

        % 1
        \item
        \begin{proof}[\unskip\nopunct]
            With 3 $n$-sided rolls, there are $n^3$ possibilities. \\
            The probability that either of the pair of persons roll the same
            face of the die is therefore $n/n^3=1/n^2$. \\
            Therefore
            \begin{align*}
                P(A_{12}) = P(A_{13}) = P(A_{23}) & = \frac{n}{n^3} \\
                & = \frac{1}{n^2}
            \end{align*}
            But if both the events $A_{12}$ and $A_{13}$ takes place, that is
            both persons $1$ and $2$ and persons $1$ and $3$ roll the same
            face, then that yields $A_{23}$. \\
            That is, if both persons $1$ and $2$ and persons $1$ and $3$ roll
            the same face, then that implies persons $1$ and $3$ rolled the
            same face.\\
            But the outcome of person $3$'s roll is not dependent on the other
            persons.\\
            That is, pairwise $A_{12}$ and $A_{13}$, $A_{12}$ and $A_{23}$, and
            $A_{13}$ and $A_{23}$ are independent.\\
            But if considered individually, they are dependent. \qedhere
        \end{proof}
        \vspace{0.2in}

        % 2
        \item
        \begin{proof}[\unskip\nopunct]
        \end{proof}
        \vspace{0.2in}

        % 3
        \item
        \begin{proof}[\unskip\nopunct]
        \end{proof}
        \vspace{0.2in}

        % 4
        \item
        \begin{proof}[\unskip\nopunct]
        \end{proof}
        \vspace{0.2in}

        % 5
        \item
        \begin{proof}[\unskip\nopunct]
        \end{proof}
        \vspace{0.2in}

        % 6
        \item
        \begin{proof}[\unskip\nopunct]
        \end{proof}
        \vspace{0.2in}

        % 7
        \item
        \begin{proof}[\unskip\nopunct]
        \end{proof}
        \vspace{0.2in}

        % 8
        \item
        \begin{proof}[\unskip\nopunct]
        \end{proof}
        \vspace{0.2in}

        % 9
        \item
        \begin{proof}[\unskip\nopunct]
        \end{proof}
        \vspace{0.2in}

        % 10
        \item
        The permutations of a word is given by:
        \begin{equation*}
            \frac{(\text{length of word})!}{(\text{repetitions of A})!(\text{repetitions of B})!\ldots (\text{repetitions of Z})!}
        \end{equation*}
        \begin{enumerate}

            % 10a
            \item
            \begin{proof}[\unskip\nopunct]
                Since there are no repeating characters, the permutation is
                simply:
                \begin{align*}
                    \text{permutations} & = length(\text{children})! \\
                    & = 8! \\
                    & = 40320 \qedhere
                \end{align*}
            \end{proof}
            \vspace{0.2in}

            % 10b
            \item
            \begin{proof}[\unskip\nopunct]
                Since the characters $o$ repeats 2 times, $k$ repeats 2 times,
                and $e$ repeats 3 times:
                \begin{align*}
                    \text{permutations} & = \frac{length(\text{bookkeeper})!}{(\text{repetitions of o})!(\text{repetitions of k})!(\text{repetitions of e})!} \\
                    & = \frac{10!}{2!2!3!} \\
                    & = 151200 \qedhere
                \end{align*}
            \end{proof}
            \vspace{0.2in}

        \end{enumerate}

    \end{enumerate}

\end{document}
