%%%%%%%%%%%%%%%%%%%%%%%%%%%%%%%%%%%%%%%%%
% Template
% LaTeX Template
% Version 1.0 (December 8 2014)
%
% This template has been downloaded from:
% http://www.LaTeXTemplates.com
%
% Original author:
% Brandon Fryslie
% With extensive modifications by:
% Vel (vel@latextemplates.com)
%
% License:
% CC BY-NC-SA 3.0 (http://creativecommons.org/licenses/by-nc-sa/3.0/)
%
% Authors:
% Sabbir Ahmed
%
%%%%%%%%%%%%%%%%%%%%%%%%%%%%%%%%%%%%%%%%%

\documentclass[paper=usletter, fontsize=12pt]{article}
%%%%%%%%%%%%%%%%%%%%%%%%%%%%%%%%%%%%%%%%%
% Contract Structural Definitions File Version 1.0 (December 8 2014)
%
% Created by: Vel (vel@latextemplates.com)
% 
% This file has been downloaded from: http://www.LaTeXTemplates.com
%
% License: CC BY-NC-SA 3.0 (http://creativecommons.org/licenses/by-nc-sa/3.0/)
%
%%%%%%%%%%%%%%%%%%%%%%%%%%%%%%%%%%%%%%%%%

\usepackage{geometry} % Required to modify the page layout
\usepackage{multicol}
\usepackage{amsmath}
\usepackage{amssymb}

\usepackage[pdftex]{graphicx}
\usepackage{wrapfig}
\usepackage[font=scriptsize, labelfont=bf]{caption}
\usepackage[utf8]{inputenc} % Required for including letters with accents
\usepackage[T1]{fontenc} % Use 8-bit encoding that has 256 glyphs

\usepackage{avant} % Use the Avantgarde font for headings
\usepackage{courier}
\usepackage{xparse}
\usepackage{xcolor}
\usepackage{listings}  % for code verbatim and console outputs

\setlength{\textwidth}{16cm} % Width of the text on the page
\setlength{\textheight}{23cm} % Height of the text on the page
\setlength{\oddsidemargin}{0cm} % Width of the margin - negative to move text left, positive to move it right
\setlength{\topmargin}{-1.25cm} % Reduce the top margin

\setlength{\parindent}{0mm} % Don't indent paragraphs
\setlength{\parskip}{2.5mm} % Whitespace between paragraphs
\renewcommand{\baselinestretch}{1.5}

\definecolor{green}{rgb}{0.18, 0.55, 0.34}

\graphicspath{ {figures/} }
\captionsetup[table]{skip=10pt}

\lstset{language=C, keywordstyle={\bfseries \color{black}}}

% defines algorithm counter for chapter-level
\newcounter{nalg}[section]

%defines appearance of the algorithm counter
\renewcommand{\thenalg}{\thesection .\arabic{nalg}}

% defines a new caption label as Algorithm x.y
\DeclareCaptionLabelFormat{algocaption}{Algorithm \thenalg}

% defines the algorithm listing environment
\lstnewenvironment{pseudocode}[1][] {
    \refstepcounter{nalg}  % increments algorithm number
    \captionsetup{font=normalsize, labelformat=algocaption, labelsep=colon}
    \lstset{
        breaklines=true,
        mathescape=true,
        numbers=left,
        numberstyle=\scriptsize,
        basicstyle=\footnotesize\ttfamily,
        keywordstyle=\color{black}\bfseries,
        keywords={input, output, return, parallel, function, for, to, in, if,
        else, foreach, while, and, or, new, print},
        xleftmargin=.04\textwidth,
        #1
    }
}{}

\renewcommand{\familydefault}{\sfdefault}  % default font for entire document
 % specifies the document layout and style

%----------------------------------------------------------------------------------------

% document info command
\newcommand{\documentinfo}[5]{
    \begin{centering}
        \parbox{2in}{
        \begin{spacing}{1}
            \begin{flushleft}
                \begin{tabular}{l l}
                    #1 \\
                    #2 \\
                    #3 \\
                \end{tabular}\\
                \rule{\textwidth}{1pt}
            \end{flushleft}
        \end{spacing}
        }
    \end{centering}
}

\begin{document}


    \documentinfo{Sabbir Ahmed}{\textbf{DATE:} \today}{\textbf{MATH 407} HW 02}
    \vspace{-0.10in}

    \begin{itemize}

        \item[\textbf{A.1}] Let A, B, C be subsets of a given set S. Prove the
        following statements.

            \begin{itemize}

                \item[\textbf{10}] $(A - B) \cup (B - A) = (A \cup B) - (A \cap
                B)$

                \item[\textbf{11}] $(A \cup B) \times C = (A \times C) \cup (B
                \times C)$

            \end{itemize}

        \item[\textbf{A.4}]

        \begin{itemize}

            \item[\textbf{9}] Let $a_1, \ldots, a_n$ be positive real numbers,
            $G_n$ = $\sqrt[n]{a_{1}a_{1}\ldots a_{n}}$, and $A_n$ =
            $\frac{1}{n}\sum_{i=1}^{n}a_i$. Then $G_n$ is called the gemoteric
            mean and $A_n$ is called the arithmetic mean. We wish to show that
            $G_n \le A_n$.

            \begin{enumerate}

                \item Show that $G_2 \le A_2$.

                \item Show that $G_{2^n} \le A_{2^n}$ by using induction on
                $n$.

                \item Show that $G_n \le A_n$.

            \end{enumerate}

            \textit{Hint:} Let $m$ be such that $2^m \ge n$, and set $a_{n+1}$
            = $a_{n+2}$ = $a_{2^m}$ = $A_n$ and apply part (2).

            \item[\textbf{10}] Let $a$ and $b$ be real numbers. Prove the
            binomial theorem , which states that

                \[ (a+b)^n = \sum_{i=0}^{n} \binom{n}{i}a^{i}b^{n-i} \text{\ \
                where \ } \binom{n}{i}=\frac{n!}{i!(n-i)!} \]

            and $n!$ = $n(n-1)\ldots 2 \cdot 1$ for $n \ge 1$ and $0! = 1$.

            \textit{Hint:} $\binom{m+1}{k}=\binom{m}{k}+\binom{m}{k-1}$.

            \item[\textbf{11}] Find a formula for the derivative of the product
            of $n$ functions, and give a detailed proof by induction (assuming
            the product rule for the derivative of two functions).

            \item[\textbf{12}] Find a formula for the $n$th derivative of the
            product of two functions, and give a detailed proof by induction.

        \end{itemize}

    \end{itemize}

\end{document}
