%%%%%%%%%%%%%%%%%%%%%%%%%%%%%%%%%%%%%%%%%
% Template
% LaTeX Template
% Version 1.0 (December 8 2014)
%
% This template has been downloaded from:
% http://www.LaTeXTemplates.com
%
% Original author:
% Brandon Fryslie
% With extensive modifications by:
% Vel (vel@latextemplates.com)
%
% License:
% CC BY-NC-SA 3.0 (http://creativecommons.org/licenses/by-nc-sa/3.0/)
%
% Authors:
% Sabbir Ahmed
%
%%%%%%%%%%%%%%%%%%%%%%%%%%%%%%%%%%%%%%%%%

\documentclass[paper=usletter, fontsize=12pt]{article}
%%%%%%%%%%%%%%%%%%%%%%%%%%%%%%%%%%%%%%%%%
% Contract Structural Definitions File Version 1.0 (December 8 2014)
%
% Created by: Vel (vel@latextemplates.com)
% 
% This file has been downloaded from: http://www.LaTeXTemplates.com
%
% License: CC BY-NC-SA 3.0 (http://creativecommons.org/licenses/by-nc-sa/3.0/)
%
%%%%%%%%%%%%%%%%%%%%%%%%%%%%%%%%%%%%%%%%%

\usepackage{geometry} % Required to modify the page layout
\usepackage{multicol}
\usepackage{amsmath}
\usepackage{amssymb}

\usepackage[pdftex]{graphicx}
\usepackage{wrapfig}
\usepackage[font=scriptsize, labelfont=bf]{caption}
\usepackage[utf8]{inputenc} % Required for including letters with accents
\usepackage[T1]{fontenc} % Use 8-bit encoding that has 256 glyphs

\usepackage{avant} % Use the Avantgarde font for headings
\usepackage{xparse}
\usepackage{xcolor}
\usepackage{listings}  % for code verbatim and console outputs

\setlength{\textwidth}{16cm} % Width of the text on the page
\setlength{\textheight}{23cm} % Height of the text on the page
\setlength{\oddsidemargin}{0cm} % Width of the margin - negative to move text left, positive to move it right
\setlength{\topmargin}{-1.25cm} % Reduce the top margin

\setlength{\parindent}{0mm} % Don't indent paragraphs
\setlength{\parskip}{2.5mm} % Whitespace between paragraphs
\renewcommand{\baselinestretch}{1.2}

\renewcommand\familydefault{\sfdefault}  % default font for entire document

\definecolor{green}{rgb}{0.18, 0.55, 0.34}

\graphicspath{ {figures/} }
\captionsetup[table]{skip=10pt}

\lstset{language=C, keywordstyle={\bfseries \color{black}}}

% defines algorithm counter for chapter-level
\newcounter{nalg}[section]

%defines appearance of the algorithm counter
\renewcommand{\thenalg}{\thesection .\arabic{nalg}}

% defines a new caption label as Algorithm x.y
\DeclareCaptionLabelFormat{algocaption}{Algorithm \thenalg}

%defines the algorithm listing environment
\lstnewenvironment{pseudocode}[1][] {
    \refstepcounter{nalg} %increments algorithm number

    \captionsetup{labelformat=algocaption,labelsep=colon}
    \lstset{
        mathescape=true,
        frame=tB,
        numbers=left,
        numberstyle=\tiny,
        basicstyle=\scriptsize,
        keywordstyle=\color{black}\bfseries\em,
        keywords={,input, output, return, datatype, function, in, if, else, foreach, while, begin, end, },
        xleftmargin=.04\textwidth,
        #1
    }
}{}
 % specifies the document layout and style

%------------------------------------------------------------------------------
% document info command
\newcommand{\documentinfo}[5]{
    \begin{centering}
        \parbox{2in}{
        \begin{spacing}{1}
            \begin{flushleft}
                \begin{tabular}{l l}
                    #1 \\
                    #2 \\
                    #3 \\
                \end{tabular}\\
                \rule{\textwidth}{1pt}
            \end{flushleft}
        \end{spacing}
        }
    \end{centering}
}

\newcommand{\Mod}[1]{\ (\mathrm{mod}\ #1)}

\begin{document}

    \documentinfo{Sabbir Ahmed}{\textbf{DATE:} \today}{\textbf{MATH 407:} HW 06}
    \vspace{-0.2in}

    \begin{itemize}

        \item[\textbf{2.3}]

        \begin{itemize}

            \item[\textbf{1}] Consider the following permutations in $S_7$
            \begin{align*}
                \sigma & = \left(
                    \begin{tabular}{ccccccc}
                        1 & 2 & 3 & 4 & 5 & 6 & 7 \\
                        3 & 2 & 5 & 4 & 6 & 1 & 7
                    \end{tabular}
                \right) &
                \tau & = \left(
                    \begin{tabular}{ccccccc}
                        1 & 2 & 3 & 4 & 5 & 6 & 7 \\
                        2 & 1 & 5 & 7 & 4 & 6 & 3
                    \end{tabular}
                \right)
            \end{align*}
            Compute the following products:
            \begin{itemize}

                \item[\textbf{b}] $\tau\sigma$
                \item[\textbf{Ans}]
                \begin{proof}[\unskip\nopunct]
                \end{proof}
                \vspace{0.2in}

                \item[\textbf{f}] $\tau^{-1}\sigma\tau$
                \item[\textbf{Ans}]
                \begin{proof}[\unskip\nopunct]
                \end{proof}
                \vspace{0.2in}

            \end{itemize}

            \item[\textbf{3}] Write $\left(
                \begin{tabular}{cccccccccc}
                    1 & 2 & 3 & 4 & 5 & 6 & 7 & 8 & 9 & 10 \\
                    3 & 4 & 10 & 5 & 7 & 8 & 2 & 6 & 9 & 1
                \end{tabular}
            \right)$ as a product of disjoint cycles and as a product of
            transpositions. Construct its associated diagram, find its inverse,
            and find its order.
            \item[\textbf{Ans}]
            \begin{proof}[\unskip\nopunct]
            \end{proof}
            \vspace{0.2in}

            \item[\textbf{5}] Let $3 \le m \le n$. Calculate $\sigma\tau^{-1}$
            for the cycles $\sigma = (1,2,\ldots,m-1)$ and
            $\tau=(1,2,\ldots,m-1,m)$ in $S_n$.
            \item[\textbf{Ans}]
            \begin{proof}[\unskip\nopunct]
            \end{proof}
            \vspace{0.2in}

            \item[\textbf{11}] Prove that in $S_n$, with $n \ge 3$ , any even
            permutation is a product of cycles of length three.\\
            \textit{Hint}: $(a,b)(b,c)=(a,b,c)$ and
            $(a,b)(c,d)=(a,b,c)(b,c,d)$.
            \item[\textbf{Ans}]
            \begin{proof}[\unskip\nopunct]
            \end{proof}
            \vspace{0.2in}

            \item[\textbf{15}] For $\alpha,\beta \in  S_n$, let
            $\alpha\sim\beta$ if there exists $\sigma \in S_n$ such that
            $\sigma\alpha\sigma^{-1}=\beta$. Show that $\sim$ is an equivalence
            relation on $S_n$.
            \item[\textbf{Ans}]
            \begin{proof}[\unskip\nopunct]
            \end{proof}
            \vspace{0.2in}

            \item[\textbf{16}] View $S_3$ as a subset of $S_5$, in the obvious
            way. For $\sigma, \tau \in S_5$, define $\sigma \sim \tau$ if
            $\sigma\tau^{-1} \in S_3$.
            \begin{itemize}

                \item[\textbf{a}] Show that $\sim$ is an equivalence relation
                on $S_5$.
                \item[\textbf{Ans}]
                \begin{proof}[\unskip\nopunct]
                \end{proof}
                \vspace{0.2in}

                \item[\textbf{b}] Find the equivalence class of (4, 5).
                \item[\textbf{Ans}]
                \begin{proof}[\unskip\nopunct]
                \end{proof}
                \vspace{0.2in}

                \item[\textbf{c}] Find the equivalence class of (1, 2, 3, 4, 5).
                \item[\textbf{Ans}]
                \begin{proof}[\unskip\nopunct]
                \end{proof}
                \vspace{0.2in}

                \item[\textbf{d}] Determine the total number of equivalence
                classes.
                \item[\textbf{Ans}]
                \begin{proof}[\unskip\nopunct]
                \end{proof}
                \vspace{0.2in}

            \end{itemize}

        \end{itemize}

    \end{itemize}

\end{document}
