%%%%%%%%%%%%%%%%%%%%%%%%%%%%%%%%%%%%%%%%%
% Template
% LaTeX Template
% Version 1.0 (December 8 2014)
%
% This template has been downloaded from:
% http://www.LaTeXTemplates.com
%
% Original author:
% Brandon Fryslie
% With extensive modifications by:
% Vel (vel@latextemplates.com)
%
% License:
% CC BY-NC-SA 3.0 (http://creativecommons.org/licenses/by-nc-sa/3.0/)
%
% Authors:
% Sabbir Ahmed
%
%%%%%%%%%%%%%%%%%%%%%%%%%%%%%%%%%%%%%%%%%

\documentclass[paper=usletter, fontsize=12pt]{article}
%%%%%%%%%%%%%%%%%%%%%%%%%%%%%%%%%%%%%%%%%
% Contract Structural Definitions File Version 1.0 (December 8 2014)
%
% Created by: Vel (vel@latextemplates.com)
% 
% This file has been downloaded from: http://www.LaTeXTemplates.com
%
% License: CC BY-NC-SA 3.0 (http://creativecommons.org/licenses/by-nc-sa/3.0/)
%
%%%%%%%%%%%%%%%%%%%%%%%%%%%%%%%%%%%%%%%%%

\usepackage{geometry} % Required to modify the page layout
\usepackage{multicol}
\usepackage{amsmath}
\usepackage{amssymb}

\usepackage[pdftex]{graphicx}
\usepackage{wrapfig}
\usepackage[font=scriptsize, labelfont=bf]{caption}
\usepackage[utf8]{inputenc} % Required for including letters with accents
\usepackage[T1]{fontenc} % Use 8-bit encoding that has 256 glyphs

\usepackage{avant} % Use the Avantgarde font for headings
\usepackage{courier}
\usepackage{xparse}
\usepackage{xcolor}
\usepackage{listings}  % for code verbatim and console outputs

\setlength{\textwidth}{16cm} % Width of the text on the page
\setlength{\textheight}{23cm} % Height of the text on the page
\setlength{\oddsidemargin}{0cm} % Width of the margin - negative to move text left, positive to move it right
\setlength{\topmargin}{-1.25cm} % Reduce the top margin

\setlength{\parindent}{0mm} % Don't indent paragraphs
\setlength{\parskip}{2.5mm} % Whitespace between paragraphs
\renewcommand{\baselinestretch}{1.5}

\definecolor{green}{rgb}{0.18, 0.55, 0.34}

\graphicspath{ {figures/} }
\captionsetup[table]{skip=10pt}

\lstset{language=C, keywordstyle={\bfseries \color{black}}}

% defines algorithm counter for chapter-level
\newcounter{nalg}[section]

%defines appearance of the algorithm counter
\renewcommand{\thenalg}{\thesection .\arabic{nalg}}

% defines a new caption label as Algorithm x.y
\DeclareCaptionLabelFormat{algocaption}{Algorithm \thenalg}

% defines the algorithm listing environment
\lstnewenvironment{pseudocode}[1][] {
    \refstepcounter{nalg}  % increments algorithm number
    \captionsetup{font=normalsize, labelformat=algocaption, labelsep=colon}
    \lstset{
        breaklines=true,
        mathescape=true,
        numbers=left,
        numberstyle=\scriptsize,
        basicstyle=\footnotesize\ttfamily,
        keywordstyle=\color{black}\bfseries,
        keywords={input, output, return, parallel, function, for, to, in, if,
        else, foreach, while, and, or, new, print},
        xleftmargin=.04\textwidth,
        #1
    }
}{}

\renewcommand{\familydefault}{\sfdefault}  % default font for entire document
 % specifies the document layout and style
\allowdisplaybreaks
%------------------------------------------------------------------------------
% document info command
\newcommand{\documentinfo}[5]{
    \begin{centering}
        \parbox{2in}{
        \begin{spacing}{1}
            \begin{flushleft}
                \begin{tabular}{l l}
                    #1 \\
                    #2 \\
                    #3 \\
                \end{tabular}\\
                \rule{\textwidth}{1pt}
            \end{flushleft}
        \end{spacing}
        }
    \end{centering}
}

\newcommand{\Mod}[1]{\ (\mathrm{mod}\ #1)}

\begin{document}

    \documentinfo{Sabbir Ahmed}{\textbf{DATE:} \today}{\textbf{MATH 407:} HW 08}
    \vspace{-0.2in}

    \begin{itemize}

        \item[\textbf{3.2}]
        \begin{itemize}

            \item[\textbf{1}] In $GL_2(R)$, find the order of each of the
            following elements.
            \begin{enumerate}

                \item[\textbf{b}]
                \begin{equation*}
                    \left[
                        \begin{tabular}{cc}
                            0 & 1 \\
                            -1 & 0
                        \end{tabular}
                    \right]
                \end{equation*}
                \item[\textbf{Ans}]
                \begin{proof}[\unskip\nopunct]
                    \begingroup
                    \addtolength{\jot}{1em}
                    \begin{align*}
                        \left(\left[
                            \begin{tabular}{cc}
                                0 & 1 \\
                                -1 & 0
                            \end{tabular}
                        \right]\right)^2 & =
                        \left[
                            \begin{tabular}{cc}
                                0 & 1 \\
                                -1 & 0
                            \end{tabular}
                        \right]
                        \left[
                            \begin{tabular}{cc}
                                0 & 1 \\
                                -1 & 0
                            \end{tabular}
                        \right] =
                        \left[
                            \begin{tabular}{cc}
                                -1 & 0 \\
                                0 & -1
                            \end{tabular}
                        \right] \\
                        \left(\left[
                            \begin{tabular}{cc}
                                0 & 1 \\
                                -1 & 0
                            \end{tabular}
                        \right]\right)^3 & =
                        \left[
                            \begin{tabular}{cc}
                                -1 & 0 \\
                                0 & -1
                            \end{tabular}
                        \right]
                        \left[
                            \begin{tabular}{cc}
                                0 & 1 \\
                                -1 & 0
                            \end{tabular}
                        \right] =
                        \left[
                            \begin{tabular}{cc}
                                0 & -1 \\
                                1 & 0
                            \end{tabular}
                        \right] \\
                        \left(\left[
                            \begin{tabular}{cc}
                                0 & 1 \\
                                -1 & 0
                            \end{tabular}
                        \right]\right)^4 & =
                        \left[
                            \begin{tabular}{cc}
                                0 & -1 \\
                                1 & 0
                            \end{tabular}
                        \right]
                        \left[
                            \begin{tabular}{cc}
                                0 & 1 \\
                                -1 & 0
                            \end{tabular}
                        \right] =
                        \left[
                            \begin{tabular}{cc}
                                1 & 0 \\
                                0 & 1
                            \end{tabular}
                        \right]
                    \end{align*}
                    \endgroup

                    Therefore,
                    \begin{equation*}
                    o\left(\left[
                            \begin{tabular}{cc}
                                0 & 1 \\
                                -1 & 0
                            \end{tabular}
                        \right]\right)=4 \qedhere
                    \end{equation*}

                \end{proof}
                \vspace{0.2in}

                \item[\textbf{d}]
                \begin{equation*}
                    \left[
                        \begin{tabular}{cc}
                            -1 & 1 \\
                            0 & 1
                        \end{tabular}
                    \right]
                \end{equation*}
                \item[\textbf{Ans}]
                \begin{proof}[\unskip\nopunct]

                    \begingroup
                    \addtolength{\jot}{1em}
                    \begin{align*}
                        \left(\left[
                            \begin{tabular}{cc}
                                -1 & 1 \\
                                0 & 1
                            \end{tabular}
                        \right]\right)^2 & =
                        \left[
                            \begin{tabular}{cc}
                                -1 & 1 \\
                                0 & 1
                            \end{tabular}
                        \right]
                        \left[
                            \begin{tabular}{cc}
                                -1 & 1 \\
                                0 & 1
                            \end{tabular}
                        \right] =
                        \left[
                            \begin{tabular}{cc}
                                1 & 0 \\
                                0 & 1
                            \end{tabular}
                        \right]
                    \end{align*}
                    \endgroup

                    Therefore,
                    \begin{equation*}
                    o\left(\left[
                            \begin{tabular}{cc}
                                -1 & 1 \\
                                0 & 1
                            \end{tabular}
                        \right]\right)=2 \qedhere
                    \end{equation*}

                \end{proof}
                \vspace{0.2in}

            \end{enumerate}

            \item[\textbf{11}]
            Let $S$ be a set, and let $a$ be a fixed element of $S$. Show that
            $\{\sigma \in \text{Sym}(S) \mid \sigma(a)=a \}$ is a subgroup of
            Sym($S$).
            \item[\textbf{Ans}]
            \begin{proof}[\unskip\nopunct]
                Let $H=\{\sigma \in \text{Sym}(S) \mid \sigma(a)=a \}$\\
                Since $\sigma(a)$ is the identity function, $H$ is a non-empty subset\\
                For $H$ to be subgroup, $\sigma\tau^{-1} \in H, \ \forall \sigma,\tau \in H$\\
                Let $\sigma, \tau \in H$\\
                Then,
                \begin{align*}
                    \sigma(a) & = a\\
                    \tau(a) & = a\\
                    \tau^{-1}(a) & = a
                \end{align*}

                Therefore,
                \begin{align*}
                    (\sigma \circ \tau^{-1})(a) & = \sigma(\tau^{-1}(a))\\
                    & = \sigma(a)\\
                    & = a
                \end{align*}
                Therefore, $\sigma \circ \tau^{-1} \in H$, and $H$ is a
                subgroup of $\text{Sym}(S)$ \qedhere

            \end{proof}
            \vspace{0.2in}

            \item[\textbf{12}] For each of the following groups, find all
            elements of finite order.
            \begin{enumerate}

                \item[\textbf{a}] $\mathbb{R}^{\times}$
                \item[\textbf{Ans}]
                \begin{proof}[\unskip\nopunct]
                    We want elements $r \in \mathbb{R}^{\times}$ such that $r^n
                    =e=1$ for $n \in \mathbb{Z}^{+}$
                    \begin{align*}
                        r^n & = 1 \\
                        r & = \pm 1, \text{with } n = 2
                    \end{align*}
                    Therefore, $\{-1, 1\} \in \mathbb{R}^{\times}$ \qedhere
                \end{proof}
                \vspace{0.2in}

                \item[\textbf{b}] $\mathbb{C}^{\times}$
                \item[\textbf{Ans}]
                \begin{proof}[\unskip\nopunct]
                    We want elements $c \in \mathbb{C}^{\times}$ such that $c^n
                    =e=1$ for $n \in \mathbb{Z}^{+}$
                    \begin{align*}
                        c^n & = 1 \\
                        c & = \pm 1, \text{with } n = 2 \\
                        c & = \pm i, \text{with } n = 4
                    \end{align*}
                    Therefore, $\{-1, 1, -i, i\} \in \mathbb{C}^{\times}$ \qedhere
                \end{proof}
                \vspace{0.2in}

            \end{enumerate}

            \item[\textbf{19}] Let $G$ be a group, and let $a \in G$. The set
            $C(a) = \{ x \in G \mid xa = ax \}$ of all elements of $G$ that
            commute with $a$ is called the \textbf{centralizer} of $a$.
            \begin{enumerate}

                \item[\textbf{a}] Show that $C(a)$ is a subgroup of $G$.
                \item[\textbf{Ans}]
                \begin{proof}[\unskip\nopunct]
                    Since $C(a) = \{ x \in G \mid xa = ax \}$,\\
                    $ea = ab, e \in C(a)$\\
                    Therefore, $C(a)$ is a non-empty set\\
                    For $C(a)$ to be a subgroup, $xy^{-1} \in C(a), \ \forall x,y \in C(a)$\\
                    Let $x,y, \in C(a)$,\\
                    Then,
                    \begin{align*}
                        xa & = ax \\
                        ya & = ay
                    \end{align*}

                    And,
                    \begin{align*}
                        ya & = ay \\
                        (y^{-1}y)ay^{-1} & = y^{-1}a(yy^{-1}) \\
                        ay^{-1} & = y^{-1}a
                    \end{align*}

                    Therefore,
                    \begin{align*}
                        xa & = ax \\
                        \Rightarrow (xy^{-1})a & = x(y^{-1}a) \\
                        & = x(ay^{-1}) \\
                        & = (xa)y^{-1} \\
                        & = (ax)y^{-1} \\
                        & = a(xy^{-1})
                    \end{align*}

                    Therefore, $xy^{-1} \in C(a)$ and $C(a)$ is a subgroup of
                    $G$ \qedhere

                \end{proof}
                \vspace{0.2in}

                \item[\textbf{b}] Show that $\langle a \rangle \subseteq C(a)$.
                \item[\textbf{Ans}]
                \begin{proof}[\unskip\nopunct]
                    For $\langle a \rangle$ to be a generator of $C(a)$,\\
                    \begin{equation*}
                        \langle a \rangle = \{x \in G \mid x = a^n \text{ for some } n \in \mathbb{Z}\}
                    \end{equation*}

                    Let $x = a^n \in \langle a \rangle$\\
                    Then,
                    \begin{align*}
                        xa & = a^na \\
                        & = a^{n+1} \\
                        & = a^{1+n} \\
                        & = aa^n \\
                        & = ax
                    \end{align*}

                    Therefore, $x \in \langle a \rangle \subseteq C(a)$
                    \qedhere
                \end{proof}
                \vspace{0.2in}

                \item[\textbf{c}] Compute $C(a)$ if $G=S_3$ and $a = (1,2,3)$.
                \item[\textbf{Ans}]
                \begin{proof}[\unskip\nopunct]

                    \begin{align*}
                        (1,2,3)(1) & = (1,2,3)\\
                        & = (1,2,3)\\
                        (1,2,3)(1,3,2) & = (1)\\
                        (1,3,2)(1,2,3) & = (1)
                    \end{align*}

                    Therefore, $C(1,2,3)=\{(1),(1,2,3),(1,3,2)\}$ \qedhere

                \end{proof}
                \vspace{0.2in}

                \item[\textbf{d}] Compute $C(a)$ if $G=S_3$ and $a = (1,2)$.
                \item[\textbf{Ans}]
                \begin{proof}[\unskip\nopunct]

                    \begin{align*}
                        (1,2)(1) & = (1,2)\\
                        (1)(1,2) & = (1,2)\\
                        (1,2)(1,2) & = (1)
                    \end{align*}

                    Therefore, $C(1,2)=\{(1),(1,2)\}$ \qedhere

                \end{proof}
                \vspace{0.2in}

            \end{enumerate}

            \item[\textbf{20}] Compute the centralizer in $\text{GL}_2(\mathbb{R})$ of the matrix $\left[\begin{tabular}{cc}
                                        -1 & 1 \\
                                        0 & 1
                                    \end{tabular}\right]$
            \item[\textbf{Ans}]
            \begin{proof}[\unskip\nopunct]

                Let $a=\left[\begin{tabular}{cc}
                                -1 & 1 \\
                                0 & 1
                            \end{tabular}\right]$ and the centralizer $C(a) = \{ x \in G \mid xa = ax \}$\\
                Let $x=\left[\begin{tabular}{cc}
                                a & b \\
                                c & d
                            \end{tabular}\right]$, so,\\
                \begin{align*}
                    C(a) \Rightarrow
                    \left[
                        \begin{tabular}{cc}
                            1 & 1 \\
                            0 & 1
                        \end{tabular}
                    \right]
                    \left[
                        \begin{tabular}{cc}
                            a & b \\
                            c & d
                        \end{tabular}
                        \right] & =
                    \left[
                        \begin{tabular}{cc}
                            a & b \\
                            c & d
                        \end{tabular}
                    \right]
                    \left[
                        \begin{tabular}{cc}
                            1 & 1 \\
                            0 & 1
                        \end{tabular}
                    \right] \\
                    \Rightarrow \left[
                        \begin{tabular}{cc}
                            a+c & b+d \\
                            c & d
                        \end{tabular}
                    \right] & = \left[
                        \begin{tabular}{cc}
                            a & a+b \\
                            c & c+d
                        \end{tabular}
                    \right]
                \end{align*}
                Therefore,
                \begin{align*}
                    a+c & = a\\
                    \Rightarrow c & = 0\\
                    b+d & = a + b\\
                    \Rightarrow d & = a
                \end{align*}
                And
                \begin{align*}
                    \left[
                        \begin{tabular}{cc}
                            a & b \\
                            c & d
                        \end{tabular}
                    \right] & \rightarrow \left[
                        \begin{tabular}{cc}
                            a & b \\
                            0 & a
                        \end{tabular}
                    \right]
                \end{align*}

                Therefore,
                \begin{equation*}
                    C\left(\left[
                        \begin{tabular}{cc}
                            1 & 1 \\
                            0 & 1
                        \end{tabular}
                    \right]\right) = \left\{\left[
                        \begin{tabular}{cc}
                            a & b \\
                            0 & a
                        \end{tabular}
                    \right] \;\middle|\; a,b \in \mathbb{R} \right\} \qedhere
                \end{equation*}

            \end{proof}
            \vspace{0.2in}

            \item[\textbf{25}] Let $G$ be a finite group, let $n > 2$ be an
            integer, and let $S$ be the set of elements of $G$ that have order
            $n$. Show that $S$ has an even number of elements.
            \item[\textbf{Ans}]
            \begin{proof}[\unskip\nopunct]

                Let $a \in S$, so $o(a)=n>2$\\
                Then, $a^n=e$\\
                Consider the element,
                \begin{align*}
                    a^{-1} & \in S\\
                    (a^{-1})^{n} & = (a^{n})^{-1}\\
                    & = e
                \end{align*}
                Therefore, $o(a^{-1})=n$ and if $a \in S$ then $a^{-1} \in S$\\
                Since the elements in $S$ can be paired with its inverse,\\
                $S$ has an even number of elements \qedhere

            \end{proof}
            \vspace{0.2in}

        \end{itemize}

        \item[\textbf{3.3}]
        \begin{itemize}

            \item[\textbf{4}] Find the cyclic subgroup generated by $\left[\begin{tabular}{cc}
                2 & 1 \\
                0 & 2
            \end{tabular}\right]$ in $\text{GL}_2(\mathbb{Z}_3)$.
            \item[\textbf{Ans}]
            \begin{proof}[\unskip\nopunct]

                \begingroup
                \addtolength{\jot}{1em}
                \begin{align*}
                    \left(\left[
                        \begin{tabular}{cc}
                            2 & 1 \\
                            0 & 2
                        \end{tabular}
                    \right]\right)^2 & =
                    \left[
                        \begin{tabular}{cc}
                            2 & 1 \\
                            0 & 2
                        \end{tabular}
                    \right]
                    \left[
                        \begin{tabular}{cc}
                            2 & 1 \\
                            0 & 2
                        \end{tabular}
                    \right] =
                    \left[
                        \begin{tabular}{cc}
                            1 & 1 \\
                            0 & 1
                        \end{tabular}
                    \right] \\
                    \left(\left[
                        \begin{tabular}{cc}
                            2 & 1 \\
                            0 & 2
                        \end{tabular}
                    \right]\right)^3 & =
                    \left[
                        \begin{tabular}{cc}
                            2 & 1 \\
                            0 & 2
                        \end{tabular}
                    \right]
                    \left[
                        \begin{tabular}{cc}
                            1 & 1 \\
                            0 & 1
                        \end{tabular}
                    \right] =
                    \left[
                        \begin{tabular}{cc}
                            2 & 0 \\
                            0 & 2
                        \end{tabular}
                    \right] \\
                    \left(\left[
                        \begin{tabular}{cc}
                            2 & 1 \\
                            0 & 2
                        \end{tabular}
                    \right]\right)^4 & =
                    \left[
                        \begin{tabular}{cc}
                            2 & 1 \\
                            0 & 2
                        \end{tabular}
                    \right]
                    \left[
                        \begin{tabular}{cc}
                            2 & 0 \\
                            0 & 2
                        \end{tabular}
                    \right] =
                    \left[
                        \begin{tabular}{cc}
                            1 & 2 \\
                            0 & 1
                        \end{tabular}
                    \right] \\
                    \left(\left[
                        \begin{tabular}{cc}
                            2 & 1 \\
                            0 & 2
                        \end{tabular}
                    \right]\right)^5 & =
                    \left[
                        \begin{tabular}{cc}
                            2 & 1 \\
                            0 & 2
                        \end{tabular}
                    \right]
                    \left[
                        \begin{tabular}{cc}
                            1 & 2 \\
                            0 & 1
                        \end{tabular}
                    \right] =
                    \left[
                        \begin{tabular}{cc}
                            2 & 2 \\
                            0 & 2
                        \end{tabular}
                    \right] \\
                    \left(\left[
                        \begin{tabular}{cc}
                            2 & 1 \\
                            0 & 2
                        \end{tabular}
                    \right]\right)^6 & =
                    \left[
                        \begin{tabular}{cc}
                            2 & 1 \\
                            0 & 2
                        \end{tabular}
                    \right]
                    \left[
                        \begin{tabular}{cc}
                            2 & 2 \\
                            0 & 2
                        \end{tabular}
                    \right] =
                    \left[
                        \begin{tabular}{cc}
                            1 & 0 \\
                            0 & 1
                        \end{tabular}
                    \right]
                \end{align*}
                \endgroup

                Therefore, the cyclic groups generated are
                \begingroup
                \addtolength{\jot}{1em}
                \begin{equation*}
                    \left\{
                        \left[
                            \begin{tabular}{cc}
                                2 & 1 \\
                                0 & 2
                            \end{tabular}
                        \right],
                        \left[
                            \begin{tabular}{cc}
                                1 & 1 \\
                                0 & 1
                            \end{tabular}
                        \right],
                        \left[
                            \begin{tabular}{cc}
                                2 & 0 \\
                                0 & 2
                            \end{tabular}
                        \right],
                        \left[
                            \begin{tabular}{cc}
                                1 & 2 \\
                                0 & 1
                            \end{tabular}
                        \right],
                        \left[
                            \begin{tabular}{cc}
                                2 & 2 \\
                                0 & 2
                            \end{tabular}
                        \right],
                        \left[
                            \begin{tabular}{cc}
                                1 & 0 \\
                                0 & 1
                            \end{tabular}
                        \right]
                    \right\} \qedhere
                \end{equation*}
                \endgroup

            \end{proof}
            \vspace{0.2in}

            \item[\textbf{5}] Prove that if $G_1$ and $G_2$ are abelian groups,
            then the direct product $G_1 \times G_2$ is abelian.
            \item[\textbf{Ans}]
            \begin{proof}[\unskip\nopunct]

                Let $x_1,x_2 \in G_1$, so $x_1x_2=x_2x_1$\\
                and $y_1,y_2 \in G_2$ so $y_1y_2=y_2y_1$\\
                Let $(x_1,y_1),(x_2,y_2) \in G_1G_2$\\
                Consider
                \begin{align*}
                    (x_1,y_1)\times(x_2,y_2) & = (x_1x_2, y_1y_2)\\
                    & = (x_2x_1 ,y_2y_1)\\
                    & = (x_2,y_2)\times(x_1,y_1)\\
                    \Rightarrow G_1 \times G_2 & = G_2 \times G_1
                \end{align*}
                Therefore, $G_1\times G_2$ is commutative and therefore abeilian \qedhere

            \end{proof}
            \vspace{0.2in}

            \item[\textbf{8}] Let $G_1$ and $G_2$ be groups, with subgroups
            $H_1$ and $H_2$, respectively. Show that $\{(x_1,x_2) \mid x_1 \in
            H_1, x_2 \in H_2\}$ is a subgroup of the direct product $G_1 \times
            G_2$.
            \item[\textbf{Ans}]
            \begin{proof}[\unskip\nopunct]

                Since $H_1$ and $H_2$ are subgroups of $G_1$ and $G_2$ respectively,
                \begin{align*}
                    e_1 \in & H_1\\
                    e_2 \in & H_2\\
                    (e_1,e_2) \in & \{(x_1,x_2) \mid x_1 \in H_1, x_2 \in H_2\}
                \end{align*}

                Let $K$ be a non-empty subset\\
                For $K$ to be a subgroup, $xy^{-1} \in K, \ \forall x,y \in K$\\
                Let $(x_1,x_2),(y_1,y_2) \in \{(x_1,x_2) \mid x_1 \in H_1, x_2 \in H_2\}$\\
                Then $x_1,y_1 \in H_1$, $x_2,y_2 \in H_2$\\
                Therefore,
                \begin{align*}
                    (x_1,x_2)(y_1,y_2)^{-1} & = (x_1,x_2)(y_1^{-1},y_2^{-1})\\
                    & = (x_1y_1^{-1},x_2y_2^{-1})
                \end{align*}
                And since $H_1$ and $H_2$ are subgroups,\\
                if $x_1,y_1 \in H_1$, $x_2,y_2 \in H_2$,\\
                then $x_1y_1^{-1} \in H_1$, $x_2y_2^{-1} \in H_2$\\
                Therefore, $(x_1,x_2)(y_1,y_2)^{-1} \in \{(x_1,x_2) \mid x_1 \in H_1, x_2 \in H_2\}$ \qedhere

            \end{proof}
            \vspace{0.2in}

            \item[\textbf{11}] Let $G_1$ and $G_2$ be groups, and let $G$ be
            the direct product $G_1 \times G_2$. Let $H=\{(x_1,x_2)\in G_1
            \times G_2 \mid x_2 = e\}$ and let $K=\{(x_1,x_2)\in G_1 \times G_2
            \mid x_1 = e\}$.

            \begin{enumerate}

                \item[\textbf{a}] Show that $H$ and $K$ are subgroups of $G$.
                \item[\textbf{Ans}]
                \begin{proof}[\unskip\nopunct]

                    Consider $H=\{(x_1,x_2)\in G_1 \times G_2 \mid x_2 = e\}$\\
                    Since $e \in G_1$ and $e \in G_2$\\
                    $(e,e) \in\{(x_1,x_2)\in G_1 \times G_2 \mid x_2 = e\}$\\
                    A non-empty set $L$ is a subgroup if $xy^{-1} \in L, \ \forall x,y \in L$\\
                    Let $(x_1,x_2), (y_1,y_2) \in H$ so,\\
                    $x_1,y_1 \in G_1$ and $x_2,y_2 \in G_2$, where $x_2=y_2=e$\\
                    Therefore,
                    \begin{align*}
                        (x_1,x_2)(y_1,y_2)^{-1} & = (x_1,x_2)(y_1^{-1},y_2^{-1})\\
                        & = (x_1y_1^{-1},x_2y_2^{-1})
                    \end{align*}
                    And since $G_1$ and $G_2$ are groups,\\
                    then $x_1y_1^{-1} \in G_1$, $x_2y_2^{-1} \in G_2$\\
                    Since $x_2=y_2=e$
                    \begin{align*}
                        (x_1,x_2)(y_1,y_2)^{-1} & = (x_1,x_2)(y_1^{-1},y_2^{-1})\\
                        & = (x_1y_1^{-1},e)
                    \end{align*}
                    Therefore, $H=\{(x_1,x_2)\in G_1 \times G_2 \mid x_2 = e\}$
                    is a subgroup of $G_1 \times G_2$,\\
                    and similarly for $K=\{(x_1,x_2)\in G_1 \times G_2 \mid x_1 = e\}$ \qedhere

                \end{proof}
                \vspace{0.2in}

                \item[\textbf{b}] Show that $HK=KH=G$.
                \item[\textbf{Ans}]
                \begin{proof}[\unskip\nopunct]

                    Given
                    \begin{align*}
                        H &=\{(x_1,x_2)\in G_1 \times G_2 \mid x_2 = e\}\\
                        K &=\{(x_1,x_2)\in G_1 \times G_2 \mid x_1 = e\}\\
                        HK &= \{(x_1,y_1)(x_2,y_2) \mid (x_1,y_1)\in H, (x_2,y_2)\in K\}
                    \end{align*}
                    Since $(x_1,y_1)\in H, (x_2,y_2)\in K$, $y_1=x_2=e$\\
                    Therefore
                    \begin{align*}
                        HK &= \{(x_1,y_1)(x_2,y_2) \mid (x_1,y_1)\in H, (x_2,y_2)\in K\}\\
                        & = \{(x_1,e)(e,y_2) \mid (x_1,e)\in G_1, (e,y_2)\in G_2\}\\
                        & = \{(x_1,y_2) \mid x_1 \in G_1, y_2\in G_2\}\\
                        & = G
                    \end{align*}
                    Therefore, $HK=G$, and similarly for $KH=G$\\
                    $HK=KH=G$ \qedhere

                \end{proof}
                \vspace{0.2in}

                \item[\textbf{c}] Show that $H \cap K = \{e,e\}$.
                \item[\textbf{Ans}]
                \begin{proof}[\unskip\nopunct]

                    Let $(x_1,x_2) \in H \cap K$\\
                    Then, $(x_1,x_2) \in H$, $(x_1,x_2) \in K$\\
                    If $(x_1,x_2) \in H$, then $x_2=e$\\
                    And if $(x_1,x_2) \in K$, then $x_1=e$\\
                    Therefore, $(x_1,x_2)=(e,e)$ \qedhere

                \end{proof}
                \vspace{0.2in}

            \end{enumerate}

            \item[\textbf{13}] Let $p, q$ be distinct prime numbers, and let
            $n=pq$. Show that $HK=\mathbb{Z}_{n}^{\times}$, for the subgroups
            $H=\{[x] \in \mathbb{Z}_{n}^{\times} \mid x \equiv 1 \Mod p\}$ and
            $K=\{[y] \in \mathbb{Z}_{n}^{\times} \mid y \equiv 1 \Mod q\}$ of
            $\mathbb{Z}_{n}^{\times}$.\\ \textit{Hint}: You can either use a
            counting argument to show that $HK$ has $\varphi(n)$ elements, or
            use the Chinese Remainder Theorem to show that the sets are the
            same.
            \item[\textbf{Ans}]
            \begin{proof}[\unskip\nopunct]

                Given, $HK$ is a subgroup of $\mathbb{Z}_{n}^{\times}$,\\
                If $\forall a,b \in HK$
                \begin{align*}
                    a \equiv 1 \Mod {pq} \\
                    b \equiv 1 \Mod {pq} \\
                    ab \equiv 1 \Mod {pq}
                \end{align*}
                If $x \in H \cap K$
                \begin{align*}
                    x \equiv 1 \Mod p \\
                    x \equiv 1 \Mod q
                \end{align*}
                And by the Chinese Remainder Theorem, $x=1$\\
                Therefore,
                \begin{align*}
                    |HK| &=\varphi(n) \\
                    & = |\mathbb{Z}_{n}^{\times}|\\
                    \Rightarrow HK & = \mathbb{Z}_{n}^{\times} \qedhere
                \end{align*}

            \end{proof}
            \vspace{0.2in}

            \item[\textbf{16}] Let $G$ be a group of order 6, and suppose that
            $a, b \in G$ with $a$ of order 3 and $b$ of order 2. Show that
            either $G$ is cyclic or $ab \neq ba$.
            \item[\textbf{Ans}]
            \begin{proof}[\unskip\nopunct]

                If $ab \neq ba$, then $G$ is non-abelian\\
                If $ab = ba$, then\\
                \begin{align*}
                    (ab)^6 & = a^6b^6\\
                    & = (a^3)^2(b^2)^3\\
                    & \Rightarrow e \ (\because o(a) = 3, o(b) = 2)
                \end{align*}
                Since $o(ab)=6$, $G$ is cyclic generated by $ab$\\
                Therefore, $G$ is cyclic or $ab \neq ba$ \qedhere

            \end{proof}
            \vspace{0.2in}

        \end{itemize}

    \end{itemize}

\end{document}
