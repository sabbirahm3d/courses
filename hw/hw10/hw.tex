%%%%%%%%%%%%%%%%%%%%%%%%%%%%%%%%%%%%%%%%%
% Template
% LaTeX Template
% Version 1.0 (December 8 2014)
%
% This template has been downloaded from:
% http://www.LaTeXTemplates.com
%
% Original author:
% Brandon Fryslie
% With extensive modifications by:
% Vel (vel@latextemplates.com)
%
% License:
% CC BY-NC-SA 3.0 (http://creativecommons.org/licenses/by-nc-sa/3.0/)
%
% Authors:
% Sabbir Ahmed
%
%%%%%%%%%%%%%%%%%%%%%%%%%%%%%%%%%%%%%%%%%

\documentclass[paper=usletter, fontsize=12pt]{article}
%%%%%%%%%%%%%%%%%%%%%%%%%%%%%%%%%%%%%%%%%
% Contract Structural Definitions File Version 1.0 (December 8 2014)
%
% Created by: Vel (vel@latextemplates.com)
% 
% This file has been downloaded from: http://www.LaTeXTemplates.com
%
% License: CC BY-NC-SA 3.0 (http://creativecommons.org/licenses/by-nc-sa/3.0/)
%
%%%%%%%%%%%%%%%%%%%%%%%%%%%%%%%%%%%%%%%%%

\usepackage{geometry} % Required to modify the page layout
\usepackage{multicol}
\usepackage{amsmath}
\usepackage{amssymb}

\usepackage[pdftex]{graphicx}
\usepackage{wrapfig}
\usepackage[font=scriptsize, labelfont=bf]{caption}
\usepackage[utf8]{inputenc} % Required for including letters with accents
\usepackage[T1]{fontenc} % Use 8-bit encoding that has 256 glyphs

\usepackage{avant} % Use the Avantgarde font for headings
\usepackage{courier}
\usepackage{xparse}
\usepackage{xcolor}
\usepackage{listings}  % for code verbatim and console outputs

\setlength{\textwidth}{16cm} % Width of the text on the page
\setlength{\textheight}{23cm} % Height of the text on the page
\setlength{\oddsidemargin}{0cm} % Width of the margin - negative to move text left, positive to move it right
\setlength{\topmargin}{-1.25cm} % Reduce the top margin

\setlength{\parindent}{0mm} % Don't indent paragraphs
\setlength{\parskip}{2.5mm} % Whitespace between paragraphs
\renewcommand{\baselinestretch}{1.5}

\definecolor{green}{rgb}{0.18, 0.55, 0.34}

\graphicspath{ {figures/} }
\captionsetup[table]{skip=10pt}

\lstset{language=C, keywordstyle={\bfseries \color{black}}}

% defines algorithm counter for chapter-level
\newcounter{nalg}[section]

%defines appearance of the algorithm counter
\renewcommand{\thenalg}{\thesection .\arabic{nalg}}

% defines a new caption label as Algorithm x.y
\DeclareCaptionLabelFormat{algocaption}{Algorithm \thenalg}

% defines the algorithm listing environment
\lstnewenvironment{pseudocode}[1][] {
    \refstepcounter{nalg}  % increments algorithm number
    \captionsetup{font=normalsize, labelformat=algocaption, labelsep=colon}
    \lstset{
        breaklines=true,
        mathescape=true,
        numbers=left,
        numberstyle=\scriptsize,
        basicstyle=\footnotesize\ttfamily,
        keywordstyle=\color{black}\bfseries,
        keywords={input, output, return, parallel, function, for, to, in, if,
        else, foreach, while, and, or, new, print},
        xleftmargin=.04\textwidth,
        #1
    }
}{}

\renewcommand{\familydefault}{\sfdefault}  % default font for entire document
 % specifies the document layout and style

\begin{document}

    \documentinfo{\today}{10}

    \begin{itemize}

        \item[\textbf{3.6}]
        \begin{itemize}

            \item[\textbf{5}] Show that no proper subgroup of $S_4$ contains
            both $(1, 2, 3, 4)$ and $(1, 2)$.
            \begin{proof}
            \end{proof}

            \item[\textbf{9}] A rigid motion of a cube can be thought of either
            as a permutation of its eight vertices or as a permutation of its
            six sides. Find a rigid motion of the cube that has order 3, and
            express the permutation that represents it in both ways, as a
            permutation on eight elements and as a permutation on six elements.
            \begin{proof}
            \end{proof}

            \item[\textbf{10}] Show that the following matrices form a subgroup
            of $GL_2(C)$ isomorphic to $D_4$:
            \begin{equation*}
                \pm\left[\begin{tabular}{cc}
                            1 & 0 \\
                            0 & 1
                \end{tabular}\right],
                \pm\left[\begin{tabular}{cc}
                            i & 0 \\
                            0 & -i
                \end{tabular}\right],
                \pm\left[\begin{tabular}{cc}
                            0 & 1 \\
                            1 & 0
                \end{tabular}\right],
                \pm\left[\begin{tabular}{cc}
                            0 & i \\
                            -i & 0
                \end{tabular}\right]
            \end{equation*}
            \begin{proof}
            \end{proof}

            \item[\textbf{15}]
            \begin{enumerate}[label=\textbf{(\alph*)}]

                \item Show that $A_4=\{\sigma \in S_4 \mid \sigma = \tau^2
                \text{ for some } \tau \in S_4\}$
                \begin{proof}
                \end{proof}

                \item Show that $A_5=\{\sigma \in S_5 \mid \sigma = \tau^2
                \text{ for some } \tau \in S_5\}$
                \begin{proof}
                \end{proof}

                \item Show that $A_6=\{\sigma \in S_6 \mid \sigma = \tau^2
                \text{ for some } \tau \in S_6\}$
                \begin{proof}
                \end{proof}

                \item What can you say about $A_n$ if $n>6$?
                \begin{proof}
                \end{proof}

            \end{enumerate}

            \item[\textbf{17}] For any elements $\sigma, \tau \in S_n$, show
            that $\sigma\tau\sigma^{-1}\tau^{-1}\in A_n$.
            \begin{proof}
            \end{proof}

            \item[\textbf{21}] Find the center of the dihedral group $D_n$. \\
            \textit{Hint:} Consider two cases, depending on whether $n$ is odd
            or even.
            \begin{proof}
            \end{proof}

            \item[\textbf{24}] Show that the product of two transpositions is
            one of (i) the identity; (ii) a 3-cycle; (iii) a product of two
            (nondisjoint) 3-cycles. Deduce that every element of $A_n$ can be
            written as a product of 3-cycles.
            \begin{proof}
            \end{proof}

        \end{itemize}

    \end{itemize}

\end{document}
