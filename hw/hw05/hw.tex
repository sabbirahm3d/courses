%%%%%%%%%%%%%%%%%%%%%%%%%%%%%%%%%%%%%%%%%
% Template
% LaTeX Template
% Version 1.0 (December 8 2014)
%
% This template has been downloaded from:
% http://www.LaTeXTemplates.com
%
% Original author:
% Brandon Fryslie
% With extensive modifications by:
% Vel (vel@latextemplates.com)
%
% License:
% CC BY-NC-SA 3.0 (http://creativecommons.org/licenses/by-nc-sa/3.0/)
%
% Authors:
% Sabbir Ahmed
%
%%%%%%%%%%%%%%%%%%%%%%%%%%%%%%%%%%%%%%%%%

\documentclass[paper=usletter, fontsize=12pt]{article}
%%%%%%%%%%%%%%%%%%%%%%%%%%%%%%%%%%%%%%%%%
% Contract Structural Definitions File Version 1.0 (December 8 2014)
%
% Created by: Vel (vel@latextemplates.com)
% 
% This file has been downloaded from: http://www.LaTeXTemplates.com
%
% License: CC BY-NC-SA 3.0 (http://creativecommons.org/licenses/by-nc-sa/3.0/)
%
%%%%%%%%%%%%%%%%%%%%%%%%%%%%%%%%%%%%%%%%%

\usepackage{geometry} % Required to modify the page layout
\usepackage{multicol}
\usepackage{amsmath}
\usepackage{amssymb}

\usepackage[pdftex]{graphicx}
\usepackage{wrapfig}
\usepackage[font=scriptsize, labelfont=bf]{caption}
\usepackage[utf8]{inputenc} % Required for including letters with accents
\usepackage[T1]{fontenc} % Use 8-bit encoding that has 256 glyphs

\usepackage{avant} % Use the Avantgarde font for headings
\usepackage{courier}
\usepackage{xparse}
\usepackage{xcolor}
\usepackage{listings}  % for code verbatim and console outputs

\setlength{\textwidth}{16cm} % Width of the text on the page
\setlength{\textheight}{23cm} % Height of the text on the page
\setlength{\oddsidemargin}{0cm} % Width of the margin - negative to move text left, positive to move it right
\setlength{\topmargin}{-1.25cm} % Reduce the top margin

\setlength{\parindent}{0mm} % Don't indent paragraphs
\setlength{\parskip}{2.5mm} % Whitespace between paragraphs
\renewcommand{\baselinestretch}{1.5}

\definecolor{green}{rgb}{0.18, 0.55, 0.34}

\graphicspath{ {figures/} }
\captionsetup[table]{skip=10pt}

\lstset{language=C, keywordstyle={\bfseries \color{black}}}

% defines algorithm counter for chapter-level
\newcounter{nalg}[section]

%defines appearance of the algorithm counter
\renewcommand{\thenalg}{\thesection .\arabic{nalg}}

% defines a new caption label as Algorithm x.y
\DeclareCaptionLabelFormat{algocaption}{Algorithm \thenalg}

% defines the algorithm listing environment
\lstnewenvironment{pseudocode}[1][] {
    \refstepcounter{nalg}  % increments algorithm number
    \captionsetup{font=normalsize, labelformat=algocaption, labelsep=colon}
    \lstset{
        breaklines=true,
        mathescape=true,
        numbers=left,
        numberstyle=\scriptsize,
        basicstyle=\footnotesize\ttfamily,
        keywordstyle=\color{black}\bfseries,
        keywords={input, output, return, parallel, function, for, to, in, if,
        else, foreach, while, and, or, new, print},
        xleftmargin=.04\textwidth,
        #1
    }
}{}

\renewcommand{\familydefault}{\sfdefault}  % default font for entire document
 % specifies the document layout and style

%------------------------------------------------------------------------------
% document info command
\newcommand{\documentinfo}[5]{
    \begin{centering}
        \parbox{2in}{
        \begin{spacing}{1}
            \begin{flushleft}
                \begin{tabular}{l l}
                    #1 \\
                    #2 \\
                    #3 \\
                \end{tabular}\\
                \rule{\textwidth}{1pt}
            \end{flushleft}
        \end{spacing}
        }
    \end{centering}
}

\newcommand{\Mod}[1]{\ (\mathrm{mod}\ #1)}

\begin{document}

    \documentinfo{Sabbir Ahmed}{\textbf{DATE:} \today}{\textbf{MATH 407:} HW 04}
    \vspace{-0.2in}

    \begin{itemize}

        \item[\textbf{1.3}]

        \begin{itemize}

            \item[\textbf{1}] Solve the following congruence

            \begin{itemize}

                \item[\textbf{d}] $19x \equiv 1 \Mod{36}$
                \item[\textbf{Ans}]
                \begin{proof}[\unskip\nopunct]
                    \begin{align*}
                        19x & \equiv 1 \Mod{36} \\
                        19x & = 1 + 36n \text{, for } n \in \integers\\
                        \Rightarrow 1 & = 19x - 36n\\
                        1 & = 19(19) - 36(10)
                    \end{align*}
                    Therefore, $x \equiv 19 \Mod{36}$
                \end{proof}
                \vspace{0.2in}

            \end{itemize}

            \item[\textbf{4}] Solve the following congruence: $20x \equiv 12
            \Mod{72}$
            \item[\textbf{Ans}]
            \begin{proof}[\unskip\nopunct]
                Since $(20,72)=4$, there exists 4 solutions.
                \begin{align*}
                    20x & \equiv 12 \Mod{72} \\
                    20x & = 12 + 72n \text{, for } n \in \integers\\
                    \Rightarrow 5x & = 3 + 18n\\
                    5x & \equiv 3 \Mod{18}
                \end{align*}
                Then, $x \equiv 15 \Mod{18} \Rightarrow 18 \mid (5x-3)$\\
                Therefore,
                \begin{align*}
                    x & \equiv 15 \Mod{18}\\
                    x & \equiv 33 \Mod{18}\\
                    x & \equiv 51 \Mod{18}\\
                    x & \equiv 69 \Mod{18} \qedhere
                \end{align*}
            \end{proof}
            \vspace{0.2in}

            \item[\textbf{7}] The smallest positive solution of the congruence
            $ax \equiv 0 \Mod{n}$ is called the additive order of
            $a$ modulo $n$. Find the additive orders of each of the following
            elements, by solving the appropriate congruences.

            \begin{itemize}

                \item[\textbf{b}] 7 modulo 12
                \item[\textbf{Ans}]
                \begin{proof}[\unskip\nopunct]
                    The smallest positive solution: $7x \equiv 0 \Mod{12}$\\
                    That is, the smallest positive integer $x$ such that $12
                    \mid 7x \Rightarrow x=4$\\ Therefore, the additive order of
                    7 modulo 12 is $x = 12$ \qedhere
                \end{proof}
                \vspace{0.2in}

                \item[\textbf{d}] 12 modulo 18
                \item[\textbf{Ans}]
                \begin{proof}[\unskip\nopunct]
                    The smallest positive solution: $12x \equiv 0 \Mod{18}$\\
                    That is, the smallest positive integer $x$ such that $18
                    \mid 12x \Rightarrow x=3$\\ Therefore, the additive order
                    of 12 modulo 18 is $x = 3$ \qedhere
                \end{proof}
                \vspace{0.2in}

            \end{itemize}

            \item[\textbf{14}] Find the units digit of $3^{29}+11^{12}+15$.\\
            \textit{Hint}: Choose an appropriate modulus $n$, and then reduce
            modulo $n$.
            \item[\textbf{Ans}]
            \begin{proof}[\unskip\nopunct]
                Since $3^4=81$ with a units digit of $1$,\\
                then $3^{29}=(3^{4})^{7} \cdot 3$ with a units digit of $3$\\

                Since $11^2=121$ with a units digit of $1$,\\
                then $11^{12}=(11^{2})^{6}$ with a units digit of $1$\\

                Therefore, the units digit of $3^{29}+11^{12}+15$ is: $1 + 3 + 5 = 9$ \qedhere
            \end{proof}
            \vspace{0.2in}

            \item[\textbf{16}] Solve the following congruences by trial and
            error.
            \begin{itemize}

                \item[\textbf{a}] $x^3+2x+2 \equiv 0 \Mod{5}$
                \item[\textbf{Ans}]
                \begin{proof}[\unskip\nopunct]
                    By trial and error
                    \begin{align*}
                        x=1 &\Rightarrow 5 \mid (1)^3+2(1)+2=5\\
                        x=2 &\Rightarrow 5 \nmid (2)^3+2(2)+2=14\\
                        x=3 &\Rightarrow 5 \mid (3)^3+2(3)+2=35\\
                        x=4 &\Rightarrow 5 \nmid (4)^3+2(4)+2=74
                    \end{align*}
                    Therefore,\\
                    $x \equiv 1 \Mod{5}$ and $x \equiv 3 \Mod{5}$ \qedhere
                \end{proof}
                \vspace{0.2in}

            \end{itemize}

            \item[\textbf{20}] Solve the following system of congruences.
            \begin{align*}
                2x & \equiv 5 \Mod{7} & 3x & \equiv 4 \Mod{8}
            \end{align*}
            \item[\textbf{Ans}]
            \begin{proof}[\unskip\nopunct]

                Simplifying the congruences first,\\
                $2x \equiv 5 \Mod{7}$
                \begin{align*}
                    2x & \equiv 5 \Mod{7} \\
                    2v & \equiv 1 \Mod{7} \\
                    2v & = 1 - 7n \text{, for } n \in \integers\\
                    \Rightarrow 1 & = 2v + 7n\\
                    1 & = 2(4) + 7(-1)\\
                    \Rightarrow x & \equiv 4v \Mod{7}
                \end{align*}
                Therefore,
                \begin{align*}
                    2x & \equiv 4 \cdot 5 \Mod{7}\\
                    x & \equiv 6 \Mod{7}
                \end{align*}
                And $3x \equiv 4 \Mod{8}$
                \begin{align*}
                    3x & \equiv 4 \Mod{8} \\
                    3v & \equiv 1 \Mod{8} \\
                    3v & = 1 - 8n \text{, for } n \in \integers\\
                    \Rightarrow 1 & = 3v + 8n\\
                    1 & = 3(3) + 8(-1)\\
                    \Rightarrow x & \equiv 3v \Mod{8}
                \end{align*}
                Therefore,
                \begin{align*}
                    3x & \equiv 3 \cdot 4 \Mod{8}\\
                    x & \equiv 4 \Mod{8}
                \end{align*}
                Now the system can be solved using the Chinese Remainder Theorem:
                \begin{align*}
                    x & \equiv 6 \Mod{7} & x & \equiv 4 \Mod{8}
                \end{align*}
                Since $(n_1,n_2)=(7,8)=1$, let $u_1 = 7k_1$ and $u_2 = 8k_2$\\
                Then
                \begin{align*}
                    u_1+u_2=1 & \Rightarrow 7k_1 + 8k_2=1\\
                    1 & = 7(-1) + 8(1)
                \end{align*}
                Thus
                \begin{align*}
                    u_1 = 7(-1) & = -7 \equiv 1 \Mod{8}\\
                    u_1 = 7(-1) & = -7 \equiv 0 \Mod{7}
                \end{align*}
                And
                \begin{align*}
                    u_2 = 8(1) & = 8 \equiv 0 \Mod{8}\\
                    u_2 = 8(1) & = 8 \equiv 1 \Mod{7}
                \end{align*}
                Therefore,
                \begin{align*}
                    x & = 6u_1 + 4u_2\\
                    & = 6(-7) + 4(8)\\
                    & = -10\\
                \end{align*}
                Therefore, the general solution with the smallest nonnegative integer is
                \begin{align*}
                    x & \equiv -10 \Mod{n_1n_2} \\
                    x & \equiv -10 \Mod{56} \\
                    x & \equiv 46 \Mod{56} \qedhere
                \end{align*}
            \end{proof}

        \end{itemize}

        \item[\textbf{1.4}]

        \begin{itemize}

            \item[\textbf{2}] Make multiplication tables for the following
            sets.

            \begin{proof}[\unskip\nopunct]

                \begin{table}
                    \centering
                    \caption{\textbf{b:} Multiplication table of $\integers_7$}
                    \renewcommand{\arraystretch}{2.5}
                    \begin{tabular}{|c|c|c|c|c|c|c|c|}
\hline
$\bm{\times}$ & $\bm{[0]}$ & $\bm{[1]}$ & $\bm{[2]}$ & $\bm{[3]}$ & $\bm{[4]}$ & $\bm{[5]}$ & $\bm{[6]}$ \\
\hline
$\bm{[0]}$ & $[0]$ & $[0]$ & $[0]$ & $[0]$ & $[0]$ & $[0]$ & $[0]$ \\
\hline
$\bm{[1]}$ & $[0]$ & $[1]$ & $[2]$ & $[3]$ & $[4]$ & $[5]$ & $[6]$ \\
\hline
$\bm{[2]}$ & $[0]$ & $[2]$ & $[4]$ & $[6]$ & $[1]$ & $[3]$ & $[5]$ \\
\hline
$\bm{[3]}$ & $[0]$ & $[3]$ & $[6]$ & $[2]$ & $[5]$ & $[1]$ & $[4]$ \\
\hline
$\bm{[4]}$ & $[0]$ & $[4]$ & $[1]$ & $[5]$ & $[2]$ & $[6]$ & $[3]$ \\
\hline
$\bm{[5]}$ & $[0]$ & $[5]$ & $[3]$ & $[1]$ & $[6]$ & $[4]$ & $[2]$ \\
\hline
$\bm{[6]}$ & $[0]$ & $[6]$ & $[5]$ & $[4]$ & $[3]$ & $[2]$ & $[1]$ \\
\hline
                    \end{tabular}
                \end{table}

            \end{proof}

            \begin{proof}[\unskip\nopunct]

                \begin{table}
                    \centering
                    \caption{\textbf{c:} Multiplication table of $\integers_8$}
                    \renewcommand{\arraystretch}{2.5}
                    \begin{tabular}{|c|c|c|c|c|c|c|c|c|}
\hline
$\bm{\times}$ & $\bm{[0]}$ & $\bm{[1]}$ & $\bm{[2]}$ & $\bm{[3]}$ & $\bm{[4]}$ & $\bm{[5]}$ & $\bm{[6]}$ & $\bm{[7]}$ \\
\hline
$\bm{[0]}$ & $[0]$ & $[0]$ & $[0]$ & $[0]$ & $[0]$ & $[0]$ & $[0]$ & $[0]$ \\
\hline
$\bm{[1]}$ & $[0]$ & $[1]$ & $[2]$ & $[3]$ & $[4]$ & $[5]$ & $[6]$ & $[7]$ \\
\hline
$\bm{[2]}$ & $[0]$ & $[2]$ & $[4]$ & $[6]$ & $[0]$ & $[2]$ & $[4]$ & $[6]$ \\
\hline
$\bm{[3]}$ & $[0]$ & $[3]$ & $[6]$ & $[1]$ & $[4]$ & $[7]$ & $[2]$ & $[5]$ \\
\hline
$\bm{[4]}$ & $[0]$ & $[4]$ & $[0]$ & $[4]$ & $[0]$ & $[4]$ & $[0]$ & $[4]$ \\
\hline
$\bm{[5]}$ & $[0]$ & $[5]$ & $[2]$ & $[7]$ & $[4]$ & $[1]$ & $[6]$ & $[3]$ \\
\hline
$\bm{[6]}$ & $[0]$ & $[6]$ & $[4]$ & $[2]$ & $[0]$ & $[6]$ & $[4]$ & $[2]$ \\
\hline
$\bm{[7]}$ & $[0]$ & $[7]$ & $[5]$ & $[4]$ & $[3]$ & $[2]$ & $[1]$ & $[1]$ \\
\hline
                    \end{tabular}
                \end{table}

            \end{proof}
            \newpage

            \item[\textbf{6}] Let $m$ and $n$ be positive integers such that
            $m\mid n$. Show that for any integer $a$, the congruence class
            $[a]_m$ is the union of the congruence classes $[a]_n, [a+m]_n, [a+2m]_n, \ldots, [a+n-m]_n$
            \item[\textbf{Ans}]
            \begin{proof}[\unskip\nopunct]
            \end{proof}
            \vspace{0.2in}

            \item[\textbf{9}] Let $(a, n)=1$. The smallest positive integer $k$
            such that $a^k \equiv 1 \Mod{n}$ is called the
            \textbf{multiplicative order} of $[a]$ in $\integers_n^\times$
            \begin{itemize}

                \item[\textbf{b}] Find the multiplicative orders of $[2]$ and
                $[5]$ in $\integers_{17}^\times$.
                \item[\textbf{Ans}]
                \begin{proof}[\unskip\nopunct]

                    Show $2^k \equiv 1 \Mod{17}$, for $k \in \integers$\\
                    Then, $2^k = 1 + 17n$, for $n \in \integers$\\
                    Then, $n = (2^k-1)/17$\\
                    Therefore, for $n$ to be an integer, $k = 8$.\\

                    Similarly, show $5^k \equiv 1 \Mod{17}$, for $k \in \integers$\\
                    Then, $5^k = 1 + 17n$, for $n \in \integers$\\
                    Then, $n = (5^k-1)/17$\\
                    Therefore, for $n$ to be an integer, $k = 16$.\\

                    Therefore, the multiplicative order of $[2]$ and $[5]$ in $\integers_{17}^\times$ is $k=8$ \qedhere

                \end{proof}
                \vspace{0.2in}

            \end{itemize}

            \item[\textbf{10}] Let $(a, n)=1$. If $[a]$ has multiplicative
            order $k$ in $\integers_n^\times$, show that $k \mid \varphi(n)$.
            \item[\textbf{Ans}]
            \begin{proof}[\unskip\nopunct]
                By Euler's theorem, if $(a, n)=1$ then $a^{\varphi(n)} \equiv 1
                \Mod{n}$\\
                Also, if $k$ is the multiplicative order of $[a]$, \\
                then $k$ is the smallest positive interger such that $a^k \equiv 1 \Mod{n}$\\
                Therefore, there exists an $m \in \integers$ such that
                \begin{equation*}
                    a^{mk} = a^{\varphi(n)} \equiv 1 \Mod{n}
                \end{equation*}
                Then $mk = \varphi(n)$\\
                That is, $k \mid \varphi(n)$ \qedhere
            \end{proof}
            \vspace{0.2in}

            \item[\textbf{13}] An element $[a]$ of is said to be
            \textbf{idempotent} if $[a]^2=[a]$.
            \begin{itemize}

                \item[\textbf{b}] Find all idempotent elements of
                $\integers_{10}^\times$ and $\integers_{30}^\times$.
                \item[\textbf{Ans}]
                \begin{proof}[\unskip\nopunct]
                    For $\integers_{10}^\times$:
                    \begin{align*}
                        [0]^2 & = [0]\\
                        [1]^2 & = [1]\\
                        [5]^2 & = [5]\\
                        [6]^2 & = [6]
                    \end{align*}

                    For $\integers_{30}^\times$:
                    \begin{align*}
                        [0]^2 & = [0]\\
                        [1]^2 & = [1]\\
                        [6]^2 & = [6]\\
                        [10]^2 & = [10] \qedhere
                    \end{align*}

                \end{proof}
                \vspace{0.2in}

            \end{itemize}

            \item[\textbf{15}] If $n$ is not a prime power, show that
            $\integers_n$ has an idempotent element different from $[0]$ and
            $[1]$.\\ \textit{Hint}: Suppose that $n=bc$, with $(b, c)=1$. Solve
            the simultaneous congruences $x \equiv 1 \Mod{b}$ and $x
            \equiv 0 \Mod{c}$.
            \item[\textbf{Ans}]
            \begin{proof}[\unskip\nopunct]
            \end{proof}
            \vspace{0.2in}

            \item[\textbf{20}] Show that
            $\varphi(1)+\varphi(p)+\ldots+\varphi(p^\alpha)=\varphi^\alpha$ for
            any prime number $p$ and any positive integer $\alpha$.
            \item[\textbf{Ans}]
            \begin{proof}[\unskip\nopunct]
            \end{proof}
            \vspace{0.2in}

            \item[\textbf{26}] Let $p=2k+1$ be a prime number. Show that if $a$
            is an integer such that $p \nmid a$, then either $a^k \equiv 1
            \Mod{p}$ or $a^k \equiv -1 \Mod{p}$
            \item[\textbf{Ans}]
            \begin{proof}[\unskip\nopunct]
            \end{proof}
            \vspace{0.2in}

        \end{itemize}

    \end{itemize}

\end{document}
