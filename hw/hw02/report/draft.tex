%%%%%%%%%%%%%%%%%%%%%%%%%%%%%%%%%%%%%%%%%
% Contract
% LaTeX Template
% Version 1.0 (December 8 2014)
%
% This template has been downloaded from:
% http://www.LaTeXTemplates.com
%
% Original author:
% Brandon Fryslie
% With extensive modifications by:
% Vel (vel@latextemplates.com)
%
% License:
% CC BY-NC-SA 3.0 (http://creativecommons.org/licenses/by-nc-sa/3.0/)
%
% Note:
% If you are using Apple OS X, go into structure.tex and uncomment the font
% specifications for OS X and comment out the default specifications - this will
% drastically increase how good the document looks. You will now need to
% compile with XeLaTeX.
%
%%%%%%%%%%%%%%%%%%%%%%%%%%%%%%%%%%%%%%%%%

\documentclass[usletter, 12pt]{article}
%%%%%%%%%%%%%%%%%%%%%%%%%%%%%%%%%%%%%%%%%
% Contract Structural Definitions File Version 1.0 (December 8 2014)
%
% Created by: Vel (vel@latextemplates.com)
% 
% This file has been downloaded from: http://www.LaTeXTemplates.com
%
% License: CC BY-NC-SA 3.0 (http://creativecommons.org/licenses/by-nc-sa/3.0/)
%
%%%%%%%%%%%%%%%%%%%%%%%%%%%%%%%%%%%%%%%%%

\usepackage{geometry} % Required to modify the page layout
\usepackage{multicol}
\usepackage{amsmath}
\usepackage{amssymb}

\usepackage[pdftex]{graphicx}
\usepackage{wrapfig}
\usepackage[font=scriptsize, labelfont=bf]{caption}
\usepackage[utf8]{inputenc} % Required for including letters with accents
\usepackage[T1]{fontenc} % Use 8-bit encoding that has 256 glyphs

\usepackage{avant} % Use the Avantgarde font for headings
\usepackage{courier}
\usepackage{xparse}
\usepackage{xcolor}
\usepackage{listings}  % for code verbatim and console outputs

\setlength{\textwidth}{16cm} % Width of the text on the page
\setlength{\textheight}{23cm} % Height of the text on the page
\setlength{\oddsidemargin}{0cm} % Width of the margin - negative to move text left, positive to move it right
\setlength{\topmargin}{-1.25cm} % Reduce the top margin

\setlength{\parindent}{0mm} % Don't indent paragraphs
\setlength{\parskip}{2.5mm} % Whitespace between paragraphs
\renewcommand{\baselinestretch}{1.5}

\definecolor{green}{rgb}{0.18, 0.55, 0.34}

\graphicspath{ {figures/} }
\captionsetup[table]{skip=10pt}

\lstset{language=C, keywordstyle={\bfseries \color{black}}}

% defines algorithm counter for chapter-level
\newcounter{nalg}[section]

%defines appearance of the algorithm counter
\renewcommand{\thenalg}{\thesection .\arabic{nalg}}

% defines a new caption label as Algorithm x.y
\DeclareCaptionLabelFormat{algocaption}{Algorithm \thenalg}

% defines the algorithm listing environment
\lstnewenvironment{pseudocode}[1][] {
    \refstepcounter{nalg}  % increments algorithm number
    \captionsetup{font=normalsize, labelformat=algocaption, labelsep=colon}
    \lstset{
        breaklines=true,
        mathescape=true,
        numbers=left,
        numberstyle=\scriptsize,
        basicstyle=\footnotesize\ttfamily,
        keywordstyle=\color{black}\bfseries,
        keywords={input, output, return, parallel, function, for, to, in, if,
        else, foreach, while, and, or, new, print},
        xleftmargin=.04\textwidth,
        #1
    }
}{}

\renewcommand{\familydefault}{\sfdefault}  % default font for entire document
 % Input the structure.tex file which specifies the document layout and style

%----------------------------------------------------------------------------------------
%   DYNAMIC CONTRACT INFORMATION
%----------------------------------------------------------------------------------------

% Member's information
\newcommand{\project}{Homework 2: Song Composer}
\newcommand{\Sabbir}{Sabbir Ahmed}

%----------------------------------------------------------------------------------------

\begin{document}

    %----------------------------------------------------------------------------------------
    %   TITLE PAGE
    %----------------------------------------------------------------------------------------

    \begin{titlepage}

        \vspace*{\fill} % Add whitespace above to center the title page content
        \begin{center}

            {\LARGE \project~Report Draft}\\ [1.5cm]

            Submitted: \today
            
            \vspace*{\fill}

            \Sabbir

        \end{center}
        \vspace*{\fill} % Add whitespace below to center the title page content

    \end{titlepage}

    \section{Description}

    This project utilizes the AVR Butterfly through the terminal to allow a user to enter songs, along with a title and author, and play them back. \\

    \noindent This document serves as the first draft of the project report. The submission for the first week only requires the functions in \codeword{music.c} to be implemented along with an example for a proof of accuracy. The current version of the project implements functional menu printing and option selecting features. The functions called from the menu include printing out the list of songs, and creating and playing hard-coded songs. Portions of the code have been commented out to exclude declarations from \codeword{<avr/io>} and other AVR-GCC libraries. The current version of the code is configured to build and run on GCC for debugging, until the final version is ready.

    \section{Library Functions}

    The \codeword{music.c} file contains the functions required for converting the user inputs to notes.

        \subsection{User Inputs to Notes}

        Notes comprise of an ASCII character, (A-G, and R) and a duration of the note (0-31). They are converted from user inputted strings to \codeword{uint8_t} by left-shifting the integer value mapped to the ASCII character by 5 bits and adding the duration. \\
        Unpacking the notes is possible by right-shifting by 5 bits to retrieve the ASCII character or by bitwise and-ing with 0x1F (0001 1111) to retrieve the duration.

    \section{Code}

        The C scripts used for the implementation has been attached alongside the report.

        \subsection{music.h}
        Header file that contains all the constants and the declarations for the functions in the program.

        \subsection{music.c}
        Implements the functionality for handling packing and playing songs.

\end{document}
