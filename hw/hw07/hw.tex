%%%%%%%%%%%%%%%%%%%%%%%%%%%%%%%%%%%%%%%%%
% Template
% LaTeX Template
% Version 1.0 (December 8 2014)
%
% This template has been downloaded from:
% http://www.LaTeXTemplates.com
%
% Original author:
% Brandon Fryslie
% With extensive modifications by:
% Vel (vel@latextemplates.com)
%
% License:
% CC BY-NC-SA 3.0 (http://creativecommons.org/licenses/by-nc-sa/3.0/)
%
% Authors:
% Sabbir Ahmed
%
%%%%%%%%%%%%%%%%%%%%%%%%%%%%%%%%%%%%%%%%%

\documentclass[paper=usletter, fontsize=12pt]{article}
%%%%%%%%%%%%%%%%%%%%%%%%%%%%%%%%%%%%%%%%%
% Contract Structural Definitions File Version 1.0 (December 8 2014)
%
% Created by: Vel (vel@latextemplates.com)
% 
% This file has been downloaded from: http://www.LaTeXTemplates.com
%
% License: CC BY-NC-SA 3.0 (http://creativecommons.org/licenses/by-nc-sa/3.0/)
%
%%%%%%%%%%%%%%%%%%%%%%%%%%%%%%%%%%%%%%%%%

\usepackage{geometry} % Required to modify the page layout
\usepackage{multicol}
\usepackage{amsmath}
\usepackage{amssymb}

\usepackage[pdftex]{graphicx}
\usepackage{wrapfig}
\usepackage[font=scriptsize, labelfont=bf]{caption}
\usepackage[utf8]{inputenc} % Required for including letters with accents
\usepackage[T1]{fontenc} % Use 8-bit encoding that has 256 glyphs

\usepackage{avant} % Use the Avantgarde font for headings
\usepackage{xparse}
\usepackage{xcolor}
\usepackage{listings}  % for code verbatim and console outputs

\setlength{\textwidth}{16cm} % Width of the text on the page
\setlength{\textheight}{23cm} % Height of the text on the page
\setlength{\oddsidemargin}{0cm} % Width of the margin - negative to move text left, positive to move it right
\setlength{\topmargin}{-1.25cm} % Reduce the top margin

\setlength{\parindent}{0mm} % Don't indent paragraphs
\setlength{\parskip}{2.5mm} % Whitespace between paragraphs
\renewcommand{\baselinestretch}{1.2}

\renewcommand\familydefault{\sfdefault}  % default font for entire document

\definecolor{green}{rgb}{0.18, 0.55, 0.34}

\graphicspath{ {figures/} }
\captionsetup[table]{skip=10pt}

\lstset{language=C, keywordstyle={\bfseries \color{black}}}

% defines algorithm counter for chapter-level
\newcounter{nalg}[section]

%defines appearance of the algorithm counter
\renewcommand{\thenalg}{\thesection .\arabic{nalg}}

% defines a new caption label as Algorithm x.y
\DeclareCaptionLabelFormat{algocaption}{Algorithm \thenalg}

%defines the algorithm listing environment
\lstnewenvironment{pseudocode}[1][] {
    \refstepcounter{nalg} %increments algorithm number

    \captionsetup{labelformat=algocaption,labelsep=colon}
    \lstset{
        mathescape=true,
        frame=tB,
        numbers=left,
        numberstyle=\tiny,
        basicstyle=\scriptsize,
        keywordstyle=\color{black}\bfseries\em,
        keywords={,input, output, return, datatype, function, in, if, else, foreach, while, begin, end, },
        xleftmargin=.04\textwidth,
        #1
    }
}{}
 % specifies the document layout and style

\begin{document}

    \documentinfo{\today}{07}

    \begin{enumerate}

        % 1
        \item A radar tends to overestimate the distance of an aircraft, and
        the error is a normal random variable with a mean of 50 meters and a
        standard deviation 100 meters. What is the probability that the
        measured distance will be smaller than the true distance?
        \begin{cproof}

            \salign{1}
            \begin{align*}
                P(X<0) & = P\bigg(Y<\frac{x-\mu}{\sigma} \bigg)\\
                & = P\bigg(Y<\frac{0-50}{100} \bigg)\\
                & = P(Y<-0.5)\\
                & = \Phi(-0.5)\\
                & = 1-\Phi(0.5)\\
                & = 0.3085 \qedhere
            \end{align*}
            \endgroup

        \end{cproof}

        % 2
        \item Let $X$ be normal with mean 1 and variance 4. Let $Y=2X+3$.
        \begin{enumerate}

            % a
            \item Calculate the PDF of $Y$.
            \begin{cproof}

                \begin{align*}
                    E[Y] & = E[2X+3] \\
                    & = 2E[X]+3\\
                    & = 2(1)+3\\
                    & = 5
                \end{align*}

                \begin{align*}
                    var(Y) & = var(2X+3) \\
                    & = (2)^2var(X) \\
                    & = (2)^2(4)\\
                    & = 16
                \end{align*}
                The PDF of the normal random variable is
                \salign{1}
                \begin{equation*}
                    f_{Y}(y) = \begin{cases}
                        \frac{1}{\sqrt{2\pi}}e^{\frac{-(x-5)^2}{32}}, & \text{ if } \infty < y \le \infty,\\
                        0, & \text{ otherwise, }
                    \end{cases} \qedhere
                \end{equation*}
                \endgroup

            \end{cproof}

            % b
            \item Find $P(Y\ge 0)$.
            \begin{cproof}

                \salign{1}
                \begin{align*}
                    P(Y\ge 0) & = P\bigg(Z\ge \frac{x-\mu}{\sigma} \bigg)\\
                    & = P\bigg(Z\ge \frac{0-5}{\sqrt{16}} \bigg)\\
                    & = P(Z\ge -1.25)\\
                    & = \Phi(-1.25)\\
                    & = 1-\Phi(1.25)\\
                    & = 0.1056 \qedhere
                \end{align*}
                \endgroup

            \end{cproof}

        \end{enumerate}

        % 3
        \item A signal of amplitude $s = 2$ is transmitted from a satellite but
        is corrupted by noise, and the received signal is $X = s+W$, where $W$
        is noise. When the weather is good, $W$ is normal with zero mean and
        variance 1. When the weather is bad, $W$ is normal with zero mean and
        variance 4. In the absence of any weather information:
        \begin{enumerate}

            % a
            \item Calculate the PDF of $X$.
            \begin{cproof}
            \end{cproof}

            % b
            \item Calculate the probability that $X$ is between 1 and 3.
            \begin{cproof}
            \end{cproof}

        \end{enumerate}

        % 4
        \item Oscar uses his high-speed modem to connect to the internet. The
        modem transmits zeros and ones by sending signals $-1$ and $+1$,
        respectively. We assume that any given bit has probability $p$ of being
        a zero. The network cable introduces additive zero-mean Gaussian noise
        with variance $\sigma^2$ (so, the receiver at the other end receives a
        signal which is the sum of the transmitted signal and the channel
        noise). The value of the noise is assumed to be independent of the
        encoded signal value.
        \begin{enumerate}

            % a
            \item Let $a$ be a constant between $-1$ and $1$.
            The receiver at the other end decides that the signal $-1$
            (respectively, $+1$) was transmitted if the value it receives is
            less (respectively, more) than $a$. Find a formula for the
            probability of making an error.
            \begin{cproof}
            \end{cproof}

            % b
            \item Find a numerical answer for the question of part (a) assuming
            that $p=2/5$, $a=1/2$ and $\sigma^2=1/4$.
            \begin{cproof}
            \end{cproof}

        \end{enumerate}

        % 5
        \item An old modem can take anywhere from 0 to 30 seconds to establish
        a connection, with all times between 0 and 30 being equally likely.
        \begin{enumerate}

            % a
            \item What is the probability that if you use this modem you will
            have to wait more than 15 seconds to connect?
            \begin{cproof}

                For the uniformly distributed normal variable
                \salign{1}
                \begin{align*}
                    \int_{15}^{30} \frac{1}{30}dx & = \frac{1}{30}x\big\vert_{15}^{30}\\
                    & = 0.5 \qedhere
                \end{align*}
                \endgroup

            \end{cproof}

            % b
            \item Given that you have already waited 10 seconds, what is the
            probability of having to wait at least 10 more seconds?
            \begin{cproof}

                \salign{1}
                \begin{align*}
                    \int_{20}^{30} \frac{1}{20}dx & = \frac{1}{20}x\big\vert_{10}^{20}\\
                    & = 0.5 \qedhere
                \end{align*}
                \endgroup

            \end{cproof}

        \end{enumerate}

        % 6
        \item Consider a random variable $X$ with PDF
        \begin{equation*}
            f_X(x) = \begin{cases}
                2x/3, & \text{ if } 1 < x \le 2,\\
                0, & \text{ otherwise, }
            \end{cases}
        \end{equation*}
        and let $A$ be the event $\{X \ge 1.5\}$. Calculate $E[X]$, $P(A)$ and
        $E[X \mid A]$.
        \begin{cproof}
        \end{cproof}

        % 7
        \item Dine, the cook, has good days and bad days with equal frequency.
        On a good day, the time (in hours) it takes Dino to cook a souffle is
        described by the PDF
        \begin{equation*}
            f_G(g) = \begin{cases}
                2, & \text{ if } 1/2 < g \le 1,\\
                0, & \text{ otherwise, }
            \end{cases}
        \end{equation*}
        but on a bad day, the time it takes is described by the PDF
        \begin{equation*}
            f_B(b) = \begin{cases}
                1, & \text{ if } 1/2 < b \le 3/2,\\
                0, & \text{ otherwise, }
            \end{cases}
        \end{equation*}
        Find the conditional probability that today was a bad day, given that
        it took Dine less than three quarters of an hour to cook a souffle.
        \begin{cproof}
        \end{cproof}

        % 8
        \item One of the two wheels of fortune, $A$ and $B$, is selected by the
        toss of a fair coin, and the wheel chosen is spun once to determine the
        value of a random variable $X$. If wheel $A$ is selected, the PDF of
        $X$ is
        \begin{equation*}
            f_{X\mid A}(x\mid A) = \begin{cases}
                1, & \text{ if } 0 < x \le 1,\\
                0, & \text{ otherwise, }
            \end{cases}
        \end{equation*}
        If wheel $B$ is selected, the PDF of $X$ is
        \begin{equation*}
            f_{X\mid B}(x\mid B) = \begin{cases}
                3, & \text{ if } 0 < x \le 1/3,\\
                0, & \text{ otherwise, }
            \end{cases}
        \end{equation*}
        If we are told that the value of $X$ was less than 1/4, what is the
        conditional probability that wheel $A$ was the one selected.
        \begin{cproof}
        \end{cproof}

    \end{enumerate}

\end{document}
