%%%%%%%%%%%%%%%%%%%%%%%%%%%%%%%%%%%%%%%%%
% Template
% LaTeX Template
% Version 1.0 (December 8 2014)
%
% This template has been downloaded from:
% http://www.LaTeXTemplates.com
%
% Original author:
% Brandon Fryslie
% With extensive modifications by:
% Vel (vel@latextemplates.com)
%
% License:
% CC BY-NC-SA 3.0 (http://creativecommons.org/licenses/by-nc-sa/3.0/)
%
% Authors:
% Sabbir Ahmed
%
%%%%%%%%%%%%%%%%%%%%%%%%%%%%%%%%%%%%%%%%%

\documentclass[paper=usletter, fontsize=12pt]{article}
%%%%%%%%%%%%%%%%%%%%%%%%%%%%%%%%%%%%%%%%%
% Contract Structural Definitions File Version 1.0 (December 8 2014)
%
% Created by: Vel (vel@latextemplates.com)
% 
% This file has been downloaded from: http://www.LaTeXTemplates.com
%
% License: CC BY-NC-SA 3.0 (http://creativecommons.org/licenses/by-nc-sa/3.0/)
%
%%%%%%%%%%%%%%%%%%%%%%%%%%%%%%%%%%%%%%%%%

\usepackage{geometry} % Required to modify the page layout
\usepackage{multicol}
\usepackage{amsmath}
\usepackage{amssymb}

\usepackage[pdftex]{graphicx}
\usepackage{wrapfig}
\usepackage[font=scriptsize, labelfont=bf]{caption}
\usepackage[utf8]{inputenc} % Required for including letters with accents
\usepackage[T1]{fontenc} % Use 8-bit encoding that has 256 glyphs

\usepackage{avant} % Use the Avantgarde font for headings
\usepackage{courier}
\usepackage{xparse}
\usepackage{xcolor}
\usepackage{listings}  % for code verbatim and console outputs

\setlength{\textwidth}{16cm} % Width of the text on the page
\setlength{\textheight}{23cm} % Height of the text on the page
\setlength{\oddsidemargin}{0cm} % Width of the margin - negative to move text left, positive to move it right
\setlength{\topmargin}{-1.25cm} % Reduce the top margin

\setlength{\parindent}{0mm} % Don't indent paragraphs
\setlength{\parskip}{2.5mm} % Whitespace between paragraphs
\renewcommand{\baselinestretch}{1.5}

\definecolor{green}{rgb}{0.18, 0.55, 0.34}

\graphicspath{ {figures/} }
\captionsetup[table]{skip=10pt}

\lstset{language=C, keywordstyle={\bfseries \color{black}}}

% defines algorithm counter for chapter-level
\newcounter{nalg}[section]

%defines appearance of the algorithm counter
\renewcommand{\thenalg}{\thesection .\arabic{nalg}}

% defines a new caption label as Algorithm x.y
\DeclareCaptionLabelFormat{algocaption}{Algorithm \thenalg}

% defines the algorithm listing environment
\lstnewenvironment{pseudocode}[1][] {
    \refstepcounter{nalg}  % increments algorithm number
    \captionsetup{font=normalsize, labelformat=algocaption, labelsep=colon}
    \lstset{
        breaklines=true,
        mathescape=true,
        numbers=left,
        numberstyle=\scriptsize,
        basicstyle=\footnotesize\ttfamily,
        keywordstyle=\color{black}\bfseries,
        keywords={input, output, return, parallel, function, for, to, in, if,
        else, foreach, while, and, or, new, print},
        xleftmargin=.04\textwidth,
        #1
    }
}{}

\renewcommand{\familydefault}{\sfdefault}  % default font for entire document
 % specifies the document layout and style
\allowdisplaybreaks
%------------------------------------------------------------------------------
% document info command
\newcommand{\documentinfo}[5]{
    \begin{centering}
        \parbox{2in}{
        \begin{spacing}{1}
            \begin{flushleft}
                \begin{tabular}{l l}
                    #1 \\
                    #2 \\
                    #3 \\
                \end{tabular}\\
                \rule{\textwidth}{1pt}
            \end{flushleft}
        \end{spacing}
        }
    \end{centering}
}

\begin{document}

    \documentinfo{Sabbir Ahmed}{\textbf{DATE:} \today}{\textbf{MATH 407} HW 03}
    \vspace{-0.2in}

    \begin{itemize}

        \item[\textbf{1.1}]

        \begin{itemize}

            \item[\textbf{4}] Use the Euclidean algorithm to find the
            following greatest common divisors.

            \begin{itemize}

                \item[\textbf{a}] $(6643, 2873)$
                \item[\textbf{Ans}]
                \begin{proof}[\unskip\nopunct]
                    \begin{align*}
                        6643 & = 2873 \cdot 2 + 897, r_1 = 897 \\
                        2873 & = 897 \cdot 3 + 182, r_2 = 182 \\
                        897 & = 182 \cdot 4 + 169, r_3 = 169 \\
                        182 & = 169 \cdot 1 + 13, r_4 = 13 \\
                        169 & = 13 \cdot 13 + 0, r_5 = 0
                    \end{align*}
                    $\therefore (6643, 2873) = 13$ \qedhere
                \end{proof}
                \vspace{0.2in}

                \item[\textbf{c}] $(26460, 12600)$
                \item[\textbf{Ans}]
                \begin{proof}[\unskip\nopunct]
                    \begin{align*}
                        26460 & = 12600 \cdot 2 + 1260, r_1 = 1260 \\
                        12600 & = 1260 \cdot 10 + 0, r_2 = 0
                    \end{align*}
                    $\therefore (26460, 12600) = 1260$ \qedhere
                \end{proof}
                \vspace{0.2in}

                \item[\textbf{e}] $(12091, 8439)$
                \item[\textbf{Ans}]
                \begin{proof}[\unskip\nopunct]
                    \begin{align*}
                        12091 & = 8439 \cdot 1 + 3652, r_1 = 3652 \\
                        8439 & = 3652 \cdot 2 + 1135, r_2 = 1135 \\
                        3652 & = 1135 \cdot 3 + 247, r_3 = 247 \\
                        1135 & = 247 \cdot 4 + 147, r_4 = 147 \\
                        247 & = 147 \cdot 1 + 100, r_5 = 100 \\
                        147 & = 100 \cdot 1 + 47, r_6 = 47 \\
                        100 & = 47 \cdot 2 + 6, r_7 = 6 \\
                        47 & = 6 \cdot 6 + 5, r_8 = 5 \\
                        6 & = 5 \cdot 1 + 1, r_9 = 1 \\
                        5 & = 1 \cdot 5 + 0, r_{10} = 0
                    \end{align*}
                    $\therefore (12091, 8439) = 1$ \qedhere
                \end{proof}
                \vspace{0.2in}

            \end{itemize}

            \item[\textbf{6}] For each part of Exercise 4, find integers $m$
            and $n$ such that $(a, b)$ is expressed in the form $ma + nb$.

            \begin{itemize}

                \item[\textbf{a}] $(6643, 2873)$
                \item[\textbf{Ans}]
                \begin{proof}[\unskip\nopunct]
                    $\because (6643, 2873) = 13$ \\
                    $\Rightarrow 13 = n \cdot 6643 + m \cdot 2873$
                    \begin{align*}
                        r_1 = 897 & = 1 \cdot 6643 - 2 \cdot 2873 \\
                        r_2 = 182 & = 1 \cdot 2873 - 3 \cdot 897 \\
                        & =  1 \cdot 2873 - 3 \cdot (1 \cdot 6643 - 2 \cdot 2873) \\
                        & =  1 \cdot 2873 - 3 \cdot 6643 + 6 \cdot 2873 \\
                        & =  7 \cdot 2873 - 3 \cdot 6643 \\
                        r_3 = 169 & = 1 \cdot 897 - 4 \cdot 182 \\
                        & = 1 \cdot (1 \cdot 6643 - 2 \cdot 2873) - 4 \cdot (7
                        \cdot 2873 - 3 \cdot 6643) \\
                        & = 1 \cdot 6643 - 2 \cdot 2873 - 28 \cdot 2873 + 12 \cdot 6643 \\
                        & = 13 \cdot 6643 - 30 \cdot 2873 \\
                        r_4 = 13 & = 1 \cdot 182 - 1 \cdot 169 \\
                        & = 1 \cdot (7 \cdot 2873 - 3 \cdot 6643) - 1 \cdot (13 \cdot 6643 - 30 \cdot 2873) \\
                        & = 7 \cdot 2873 - 3 \cdot 6643 - 13 \cdot 6643 + 30 \cdot 2873 \\
                        & = 37 \cdot 2873 - 16 \cdot 6643
                    \end{align*}
                    $\therefore 13 = 37 \cdot 2873 - 16 \cdot 6643$, with $n = -16, m = 37$ \qedhere
                \end{proof}
                \vspace{0.2in}

                \item[\textbf{c}] $(26460, 12600)$
                \item[\textbf{Ans}]
                \begin{proof}[\unskip\nopunct]
                    $\because (26460, 12600) = 1260$ \\
                    $\Rightarrow 1260 = n \cdot 26460 + m \cdot 12600$
                    \begin{align*}
                        r_1 = 1260 & = 1 \cdot 26460 - 2 \cdot 12600
                    \end{align*}
                    $\therefore 1260 = 1 \cdot 26460 - 2 \cdot 12600$, with $n
                    = 1, m = -2$
                \end{proof}
                \vspace{0.2in}

                \item[\textbf{e}] $(12091, 8439)$
                \item[\textbf{Ans}]
                \begin{proof}[\unskip\nopunct]
                    $\because (12091, 8439) = 1$ \\
                    $\Rightarrow 1 = n \cdot 12091 + m \cdot 8439$
                    \begin{align*}
                        % r_1 = 3652
                        r_1 = 3652 & = 1 \cdot 12091 - 1 \cdot 8439 \\
                        % r_2 = 1135
                        r_2 = 1135 & = 1 \cdot 8439 - 2 \cdot 3652 \\
                        & = 1 \cdot 8439 - 2 \cdot (1 \cdot 12091 - 1 \cdot 8439) \\
                        & = 1 \cdot 8439 - 2 \cdot 12091 + 2 \cdot 8439 \\
                        & = 3 \cdot 8439 - 2 \cdot 12091 \\
                        % r_3 = 247
                        r_3 = 247 & = 1 \cdot 3652 - 3 \cdot 1135 \\
                        & = 1 \cdot (1 \cdot 12091 - 1 \cdot 8439) - 3 \cdot (3 \cdot 8439 - 2 \cdot 12091) \\
                        & = 1 \cdot 12091 - 1 \cdot 8439 - 9 \cdot 8439 + 6 \cdot 12091 \\
                        & = 7 \cdot 12091 - 10 \cdot 8439 \\
                        % r_4 = 147
                        r_4 = 147 & = 1 \cdot 1135 - 4 \cdot 247 \\
                        & = 1 \cdot (3 \cdot 8439 - 2 \cdot 12091) - 4 \cdot (7 \cdot 12091 - 10 \cdot 8439) \\
                        & = 3 \cdot 8439 - 2 \cdot 12091 - 28 \cdot 12091 + 40 \cdot 8439 \\
                        & = 43 \cdot 8439 - 30 \cdot 12091 \\
                        % r_5 = 100
                        r_5 = 100 & = 1 \cdot 247 - 1 \cdot 147 \\
                        & = 1 \cdot (7 \cdot 12091 - 10 \cdot 8439) - 1 \cdot (43 \cdot 8439 - 30 \cdot 12091) \\
                        & = 7 \cdot 12091 - 10 \cdot 8439 - 43 \cdot 8439 + 30 \cdot 12091 \\
                        & = 37 \cdot 12091 - 53 \cdot 8439 \\
                        % r_6 = 47
                        r_6 = 47 & = 1 \cdot 147 - 1 \cdot 100 \\
                        & = 1 \cdot (43 \cdot 8439 - 30 \cdot 12091) - 1 \cdot (37 \cdot 12091 - 53 \cdot 8439) \\
                        & = 43 \cdot 8439 - 30 \cdot 12091 - 37 \cdot 12091 + 53 \cdot 8439 \\
                        & = 96 \cdot 8439 - 67 \cdot 12091 \\
                        % r_7 = 6
                        r_7 = 6 & = 1 \cdot 100 - 2 \cdot 47 \\
                        & = 1 \cdot (37 \cdot 12091 - 53 \cdot 8439) - 2 \cdot (96 \cdot 8439 - 67 \cdot 12091) \\
                        & = 37 \cdot 12091 - 53 \cdot 8439 - 192 \cdot 8439 + 134 \cdot 12091 \\
                        & = 171 \cdot 12091 - 245 \cdot 8439 \\
                        % r_8 = 5
                        r_8 = 5 & = 1 \cdot 47 - 7 \cdot 6 \\
                        & = 1 \cdot (96 \cdot 8439 - 67 \cdot 12091) - 7 \cdot (171 \cdot 12091 - 245 \cdot 8439) \\
                        & = 96 \cdot 8439 - 67 \cdot 12091 - 1197 \cdot 12091 + 1715 \cdot 8439 \\
                        & = 1811 \cdot 8439 - 1264 \cdot 12091 \\
                        % r_9 = 1
                        r_9 = 1 & = 1 \cdot 6 - 1 \cdot 5 \\
                        & = 1 \cdot (171 \cdot 12091 - 245 \cdot 8439) - 1 \cdot (1811 \cdot 8439 - 1264 \cdot 12091) \\
                        & = 171 \cdot 12091 - 245 \cdot 8439 - 1811 \cdot 8439 + 1264 \cdot 12091 \\
                        & = 1435 \cdot 12091 - 2056 \cdot 8439
                    \end{align*}
                    $\therefore 1 = 1435 \cdot 12091 - 2056 \cdot 8439$, with
                    $n = 1435, m = -2056$
                \end{proof}
                \vspace{0.2in}

            \end{itemize}

            \item[\textbf{7}] Let $a, b, c$ be integers. Give a proof for these
            facts about divisors:

            \begin{itemize}

                \item[\textbf{a}] If $b \given a$, then $b \given ac$.
                \item[\textbf{Ans}]
                \begin{proof}[\unskip\nopunct]
                    Let $a = mb$, $m \in \integers$. \\
                    Multiplying both sides by $c$: \\
                    $a \cdot c = mb \cdot c$ \\
                    $a \cdot c = mc \cdot b$ (commutative law of multiplication) \\
                    Let $n = mb$, $n \in \integers$. \\
                    $a \cdot c = n \cdot c$ \\
                    $\therefore b \given ac$ if $b \given a$ \qedhere
                \end{proof}
                \vspace{0.2in}

                \item[\textbf{b}] If $b \given a$ and $c \given b$, then $c
                \given a$.
                \item[\textbf{Ans}]
                \begin{proof}[\unskip\nopunct]
                    Let $a = m \cdot b$ and $b = n \cdot c$ for $m, n \in
                    \integers$ \\
                    $\because a = m \cdot b$, $b = \frac{a}{m}$.
                    \begin{align*}
                        \therefore \frac{a}{m} & = n \cdot c \\
                        \Rightarrow a & = mn \cdot c
                    \end{align*}
                    $\therefore c \given a$ \qedhere
                \end{proof}
                \vspace{0.2in}

                \item[\textbf{c}] If $c \given a$ and $c \given b$, then $c
                \given (ma + nb)$ for any integers $m, n$.
                \item[\textbf{Ans}]
                \begin{proof}[\unskip\nopunct]
                    Since $c \given a$ and $c \given b$, they can be expressed
                    as \\ $a = m \cdot c$ and $b = n \cdot c$ for $m, n \in
                    \integers$. \\
                    Then:
                    \begin{align*}
                        ma + nb & = m(mc) + n(nc) \\
                        & = m^2c + n^2c \\
                        & = (m^2 + n^2)c
                    \end{align*}
                    Thus $c \given (m^2 + n^2)$ for some $(m^2 + n^2)
                    \in \integers$. \\
                    $\therefore c \given (ma + nb)$ \qedhere
                \end{proof}
                \vspace{0.2in}

            \end{itemize}

            \item[\textbf{11}] Show that if $a > 0$, then $(ab, ac) = a(b,
            c)$
            \item[\textbf{Ans}]
            \begin{proof}[\unskip\nopunct]
                Let $d = (b,c)$, so $d \given b$ and $d \given c$. \\
                $\therefore$ $b = m \cdot d$, $c = n \cdot d$, $m, n \in
                \integers$.
                Then $ab = m \cdot ad$ and $ac = n \cdot ad$. \\
                Thus $ad \given ab$ and $ad \given ac$ \\
                $\therefore a(b, c) \Rightarrow (ab, ac)$ \\
                Conversely, \\
                Let $x \given ab$ and $x \given ac$. \\
                $\therefore$ $ab = k \cdot x$ and $ac = l \cdot x$, for
                some $k, l \in \integers$. \\
                Since $d = (b,c)$, $d = mb + nc$ for some $m, n \in
                \integers$. \\
                Then:
                \begin{align*}
                    ad & = a \cdot mb + a \cdot nc \\
                    & = x \cdot km + x \cdot ln \\
                    & = x(km, ln)
                \end{align*}
                Thus, $x \given ad$ \\
                $\therefore (ab, ac) = a(b,c)$ if $a > 0$. \qedhere
            \end{proof}
            \vspace{0.2in}

            \item[\textbf{14}] For what positive integers $n$ is it true
            that $(n , n + 2) = 2$? Prove your claim.
            \item[\textbf{Ans}]
            \begin{proof}[\unskip\nopunct]
                Assume $n$ is even, such that $(n, 2) = 2$. \\
                Let $d$ be a divisor of $n$ and $n + 2$. \\
                So $d \given n$ and $d \given (n+2)$. \\
                Since $(n,2) = 2$, then $(n + 2, 2) = 2$. Therefore, 2 is a
                divisor of both $n$ and $n+2$. \\
                Since $d \given n$ and $d \given (n+2)$, then $d \given (|n -
                (n + 2)|) \Rightarrow d \given 2$. \\
                Therefore, $d$ must be $1$ or $2$. \\
                $\therefore$ $n$ can be any positive even integer. \qedhere
            \end{proof}
            \vspace{0.2in}

            \item[\textbf{17}] Let $a, b, n$ be integers with $n > 1$.
            Suppose that $a = nq_1 + r_1$ with $0 \le r_1 < n$ and $b =
            nq_2 + r_2$ with $0 \le r_2 < n$. Prove that $n \given (a - b)$
            if and only if $r_1 = r_2$.
            \item[\textbf{Ans}]
            \begin{proof}[\unskip\nopunct]
                Suppose $r_1 \le r_2$ \\
                If $n \given (a - b)$, then $a - b = nq_3$ for $q_3 \in
                \integers$. \\
                Therefore:
                \begin{align*}
                    a - b & = nq_3 \\
                    \Rightarrow a - b + b & = nq_3 + b \\
                    \Rightarrow a & = nq_3 + b
                \end{align*}
                Since $b = nq_2 + r_2$:
                \begin{align*}
                    a & = nq_3 + nq_2 + r_2 \\
                    & = n(q_3 + q_2) + r_2
                \end{align*}
                Since $a = nq_1 + r_1$:
                \begin{align*}
                    nq_1 + r_1 & = n(q_3 + q_2) + r_2 \\
                    nq_1 - n(q_3 + q_2) & = r_2 - r_1 \\
                    n(q_1 - q_2 - q_3) & = r_2 - r_1
                \end{align*}
                Thus, $n \given (r_2 - r_1)$, $0 \le r_2 - r_1 < r_2 < n$.
                \\
                Therefore, $r_2 - r_1 = 0$, $\Rightarrow r_2 = r_1$. \\
                Conversely, suppose $n \given (a - b)$ if $r_1 = r_2$. \\
                Therefore, $a - b = n(q_1 - q_2) + (r_1 - r_2)$. \\
                $\therefore n \given (a-b)$ \qedhere
            \end{proof}
            \vspace{0.2in}

            \item[\textbf{19}] Let $a, b, q, n$ be integers such that $b
            \ne 0$ and $a = bq + r$. Prove that $(a, b) = (b, r)$ by
            showing that $(b, r)$ satisfies the definition of the greatest
            common divisor of $a$ and $b$.
            \item[\textbf{Ans}]
            \begin{proof}[\unskip\nopunct]
                Suppose $(a,b)$, then $(a,b) \given a$ and $(a,b) \given b$. \\
                Since $a = bq + r$,\\
                $\Rightarrow (a,b) \given a - bq = (a,b) \given r$ \\
                Therefore $(a,b) \given b$ and $(a,b) \given r$ \\
                $\Rightarrow (a,b) \given (b, r)$ \\
                Conversely, suppose $(b,r)$, then $(b,r) \given b$ and $(b,r) \given r$. \\
                Since $a = bq + r$,\\
                $\Rightarrow (b,r) \given bq + r = (b,r) \given a$ \\
                Therefore $(b,r) \given a$ and $(b,r) \given b$ \\
                $\Rightarrow (b,r) \given (a, b)$ \\
                $\therefore (a, b) = (b, r)$ \qedhere
            \end{proof}
            \vspace{0.2in}

        \end{itemize}

    \end{itemize}

\end{document}
