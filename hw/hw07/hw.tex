%%%%%%%%%%%%%%%%%%%%%%%%%%%%%%%%%%%%%%%%%
% Template
% LaTeX Template
% Version 1.0 (December 8 2014)
%
% This template has been downloaded from:
% http://www.LaTeXTemplates.com
%
% Original author:
% Brandon Fryslie
% With extensive modifications by:
% Vel (vel@latextemplates.com)
%
% License:
% CC BY-NC-SA 3.0 (http://creativecommons.org/licenses/by-nc-sa/3.0/)
%
% Authors:
% Sabbir Ahmed
%
%%%%%%%%%%%%%%%%%%%%%%%%%%%%%%%%%%%%%%%%%

\documentclass[paper=usletter, fontsize=12pt]{article}
%%%%%%%%%%%%%%%%%%%%%%%%%%%%%%%%%%%%%%%%%
% Contract Structural Definitions File Version 1.0 (December 8 2014)
%
% Created by: Vel (vel@latextemplates.com)
% 
% This file has been downloaded from: http://www.LaTeXTemplates.com
%
% License: CC BY-NC-SA 3.0 (http://creativecommons.org/licenses/by-nc-sa/3.0/)
%
%%%%%%%%%%%%%%%%%%%%%%%%%%%%%%%%%%%%%%%%%

\usepackage{geometry} % Required to modify the page layout
\usepackage{multicol}
\usepackage{amsmath}
\usepackage{amssymb}

\usepackage[pdftex]{graphicx}
\usepackage{wrapfig}
\usepackage[font=scriptsize, labelfont=bf]{caption}
\usepackage[utf8]{inputenc} % Required for including letters with accents
\usepackage[T1]{fontenc} % Use 8-bit encoding that has 256 glyphs

\usepackage{avant} % Use the Avantgarde font for headings
\usepackage{courier}
\usepackage{xparse}
\usepackage{xcolor}
\usepackage{listings}  % for code verbatim and console outputs

\setlength{\textwidth}{16cm} % Width of the text on the page
\setlength{\textheight}{23cm} % Height of the text on the page
\setlength{\oddsidemargin}{0cm} % Width of the margin - negative to move text left, positive to move it right
\setlength{\topmargin}{-1.25cm} % Reduce the top margin

\setlength{\parindent}{0mm} % Don't indent paragraphs
\setlength{\parskip}{2.5mm} % Whitespace between paragraphs
\renewcommand{\baselinestretch}{1.5}

\definecolor{green}{rgb}{0.18, 0.55, 0.34}

\graphicspath{ {figures/} }
\captionsetup[table]{skip=10pt}

\lstset{language=C, keywordstyle={\bfseries \color{black}}}

% defines algorithm counter for chapter-level
\newcounter{nalg}[section]

%defines appearance of the algorithm counter
\renewcommand{\thenalg}{\thesection .\arabic{nalg}}

% defines a new caption label as Algorithm x.y
\DeclareCaptionLabelFormat{algocaption}{Algorithm \thenalg}

% defines the algorithm listing environment
\lstnewenvironment{pseudocode}[1][] {
    \refstepcounter{nalg}  % increments algorithm number
    \captionsetup{font=normalsize, labelformat=algocaption, labelsep=colon}
    \lstset{
        breaklines=true,
        mathescape=true,
        numbers=left,
        numberstyle=\scriptsize,
        basicstyle=\footnotesize\ttfamily,
        keywordstyle=\color{black}\bfseries,
        keywords={input, output, return, parallel, function, for, to, in, if,
        else, foreach, while, and, or, new, print},
        xleftmargin=.04\textwidth,
        #1
    }
}{}

\renewcommand{\familydefault}{\sfdefault}  % default font for entire document
 % specifies the document layout and style
\allowdisplaybreaks
%------------------------------------------------------------------------------
% document info command
\newcommand{\documentinfo}[5]{
    \begin{centering}
        \parbox{2in}{
        \begin{spacing}{1}
            \begin{flushleft}
                \begin{tabular}{l l}
                    #1 \\
                    #2 \\
                    #3 \\
                \end{tabular}\\
                \rule{\textwidth}{1pt}
            \end{flushleft}
        \end{spacing}
        }
    \end{centering}
}

\include{tikz}
\usetikzlibrary{arrows.meta}
\newcommand{\Mod}[1]{\ (\mathrm{mod}\ #1)}

\begin{document}

    \documentinfo{Sabbir Ahmed}{\textbf{DATE:} \today}{\textbf{MATH 407:} HW 06}
    \vspace{-0.2in}

    \begin{itemize}

        \item[\textbf{2.3}]

        \begin{itemize}

            \item[\textbf{1}] Consider the following permutations in $S_7$
            \begin{align*}
                \sigma & = \left(
                    \begin{tabular}{ccccccc}
                        1 & 2 & 3 & 4 & 5 & 6 & 7 \\
                        3 & 2 & 5 & 4 & 6 & 1 & 7
                    \end{tabular}
                \right) &
                \tau & = \left(
                    \begin{tabular}{ccccccc}
                        1 & 2 & 3 & 4 & 5 & 6 & 7 \\
                        2 & 1 & 5 & 7 & 4 & 6 & 3
                    \end{tabular}
                \right)
            \end{align*}
            Compute the following products:
            \begin{itemize}

                \item[\textbf{b}] $\tau\sigma$
                \item[\textbf{Ans}]
                \begin{proof}[\unskip\nopunct]
                    \begingroup
                    \addtolength{\jot}{1em}
                    \begin{align*}
                        \tau\sigma &= \left(
                            \begin{tabular}{ccccccc}
                                1 & 2 & 3 & 4 & 5 & 6 & 7 \\
                                3 & 2 & 5 & 4 & 6 & 1 & 7
                            \end{tabular}
                        \right)\left(
                            \begin{tabular}{ccccccc}
                                1 & 2 & 3 & 4 & 5 & 6 & 7 \\
                                2 & 1 & 5 & 7 & 4 & 6 & 3
                            \end{tabular}
                        \right) \\
                        &= \left(
                            \begin{tabular}{ccccccc}
                                1 & 2 & 3 & 4 & 5 & 6 & 7 \\
                                2 & 3 & 6 & 7 & 4 & 1 & 5
                            \end{tabular}
                        \right) \qedhere
                    \end{align*}
                    \endgroup
                \end{proof}
                \vspace{0.2in}

                \item[\textbf{f}] $\tau^{-1}\sigma\tau$
                \item[\textbf{Ans}]
                \begin{proof}[\unskip\nopunct]
                    \begingroup
                    \addtolength{\jot}{1em}
                    \begin{align*}
                        \tau^{-1} &= \left(
                            \begin{tabular}{ccccccc}
                                2 & 1 & 5 & 7 & 4 & 6 & 3 \\
                                1 & 2 & 3 & 4 & 5 & 6 & 7
                            \end{tabular}
                        \right)\\
                        &= \left(
                            \begin{tabular}{ccccccc}
                                2 & 3 & 6 & 7 & 4 & 1 & 5 \\
                                1 & 7 & 6 & 4 & 5 & 2 & 3
                            \end{tabular}
                        \right)\\
                        \tau^{-1}\sigma &= \left(
                            \begin{tabular}{ccccccc}
                                2 & 3 & 6 & 7 & 4 & 1 & 5 \\
                                1 & 7 & 6 & 4 & 5 & 2 & 3
                            \end{tabular}
                        \right)\left(
                            \begin{tabular}{ccccccc}
                                1 & 2 & 3 & 4 & 5 & 6 & 7 \\
                                3 & 2 & 5 & 4 & 6 & 1 & 7
                            \end{tabular}
                        \right) \\
                        & = \left(
                            \begin{tabular}{ccccccc}
                                2 & 3 & 6 & 7 & 4 & 1 & 5 \\
                                3 & 7 & 1 & 4 & 6 & 2 & 5
                            \end{tabular}
                        \right) \\
                        \tau^{-1}\sigma\tau & = \left(
                            \begin{tabular}{ccccccc}
                                2 & 3 & 6 & 7 & 4 & 1 & 5 \\
                                3 & 7 & 1 & 4 & 6 & 2 & 5
                            \end{tabular}
                        \right)\left(
                            \begin{tabular}{ccccccc}
                                1 & 2 & 3 & 4 & 5 & 6 & 7 \\
                                3 & 2 & 5 & 4 & 6 & 1 & 7
                            \end{tabular}
                        \right) \\
                        & = \left(
                            \begin{tabular}{ccccccc}
                                2 & 3 & 6 & 7 & 4 & 1 & 5 \\
                                5 & 7 & 3 & 4 & 1 & 2 & 6
                            \end{tabular}
                        \right) \qedhere
                    \end{align*}
                    \endgroup
                \end{proof}
                \vspace{0.2in}

            \end{itemize}

            \item[\textbf{3}] Write $\left(
                \begin{tabular}{cccccccccc}
                    1 & 2 & 3 & 4 & 5 & 6 & 7 & 8 & 9 & 10 \\
                    3 & 4 & 10 & 5 & 7 & 8 & 2 & 6 & 9 & 1
                \end{tabular}
            \right)$ as a product of disjoint cycles and as a product of
            transpositions. Construct its associated diagram, find its inverse,
            and find its order.
            \item[\textbf{Ans}]
            \begin{proof}[\unskip\nopunct]

                The product of disjoint cycles:
                \begin{equation*}
                    \{(1,3,10)(2,4,5,7)(6,8)\}
                \end{equation*}
                The product of transpositions:
                \begin{equation*}
                    \{(1,3,10)(2,4,5,7)(6,8)\} = \{(1,3)(3,10)(2,4)(4,5)(5,7)(6,8)\}
                \end{equation*}
                Reconstructing the permutation based on the product of transpositions:
                \begingroup
                \addtolength{\jot}{1em}
                \begin{align*}
                    \sigma &= \left(
                        \begin{tabular}{cccccccccc}
                            1 & 2 & 3 & 4 & 5 & 6 & 7 & 8 & 9 & 10 \\
                            3 & 4 & 10 & 5 & 7 & 8 & 2 & 6 & 9 & 1
                        \end{tabular}
                    \right) \\
                    & = \left(
                        \begin{tabular}{cccccccccc}
                            1 & 3 & 10 & 2 & 4 & 5 & 7 & 6 & 8 & 9 \\
                            3 & 10 & 1 & 4 & 5 & 7 & 2 & 8 & 6 & 9
                        \end{tabular}
                    \right)
                \end{align*}
                \endgroup

                Constructing the associated diagrams
                \begin{figure}[!htbp]
                    \centering
                    \begin{tikzpicture}[->,scale=1]
                        \begin{scope}[every node/.style={circle}]
                            \node (1) at (0,0) {1};
                            \node (2) at (0,3) {3};
                            \node (3) at (3,3) {10};
                        \end{scope}
                        \begin{scope}
                            \path [->] (1) edge (2);
                            \path [->] (2) edge (3);
                            \path [->] (3) edge (1);
                        \end{scope}
                    \end{tikzpicture}
                    \begin{tikzpicture}[->,scale=1]
                        \begin{scope}[every node/.style={circle}]
                            \node (1) at (0,0) {2};
                            \node (2) at (0,3) {4};
                            \node (3) at (3,3) {5};
                            \node (4) at (3,0) {7};
                        \end{scope}
                        \begin{scope}
                            \path [->] (1) edge (2);
                            \path [->] (2) edge (3);
                            \path [->] (3) edge (4);
                            \path [->] (4) edge (1);
                        \end{scope}
                    \end{tikzpicture}
                    \begin{tikzpicture}[->,scale=1]
                        \begin{scope}[every node/.style={circle}]
                            \node (1) at (0,0) {6};
                            \node (2) at (0,3) {8};
                        \end{scope}
                        \begin{scope}
                            \path [->] (1) edge[bend left=60] (2);
                            \path [->] (2) edge[bend left=60] (1);
                        \end{scope}
                    \end{tikzpicture}
                    \begin{tikzpicture}[->,scale=1]
                        \begin{scope}[every node/.style={circle}]
                            \node (1) at (0,0) {9};
                        \end{scope}
                        \begin{scope}[every loop/.style={min distance=10mm,in=0,out=100,looseness=15}]
                            \path [->] (1) edge[loop above] (1);
                        \end{scope}
                    \end{tikzpicture}
                \end{figure}

                The inverse of the permutation:
                \begingroup
                \addtolength{\jot}{1em}
                \begin{align*}
                    \sigma^{-1} &= \left(
                        \begin{tabular}{cccccccccc}
                            3 & 4 & 10 & 5 & 7 & 8 & 2 & 6 & 9 & 1 \\
                            1 & 2 & 3 & 4 & 5 & 6 & 7 & 8 & 9 & 10
                        \end{tabular}
                    \right) \\
                    & = \left(
                        \begin{tabular}{cccccccccc}
                            1 & 2 & 3 & 4 & 5 & 6 & 7 & 8 & 9 & 10 \\
                            10 & 7 & 1 & 2 & 4 & 8 & 5 & 6 & 9 & 3
                        \end{tabular}
                    \right)
                \end{align*}
                \endgroup

                Since
                \begin{align*}
                    o(\sigma) & = lcm(\text{length(cycles)})\\
                    & = lcm(\{2,4,3\})\\
                    & = 12 \qedhere
                \end{align*}

            \end{proof}
            \vspace{0.2in}

            \item[\textbf{5}] Let $3 \le m \le n$. Calculate $\sigma\tau^{-1}$
            for the cycles $\sigma = (1,2,\ldots,m-1)$ and
            $\tau=(1,2,\ldots,m-1,m)$ in $S_n$.
            \item[\textbf{Ans}]
            \begin{proof}[\unskip\nopunct]
                Given
                \begingroup
                \addtolength{\jot}{1em}
                \begin{align*}
                    \tau &= \left(
                        \begin{tabular}{ccccccc}
                            1 & 2 & 3 & \ldots & m-2 & m-1 & m \\
                            2 & 3 & 4 & \ldots & m-1 & m & 1
                        \end{tabular}
                    \right) \\
                    \tau^{-1} & = \left(
                        \begin{tabular}{ccccccc}
                            2 & 3 & 4 & \ldots & m-1 & m & 1 \\
                            1 & 2 & 3 & \ldots & m-2 & m-1 & m
                        \end{tabular}
                    \right)\\
                    &= \left(
                        \begin{tabular}{ccccccc}
                            1 & 2 & 3 & 4 & \ldots & m-1 & m \\
                            m & 1 & 2 & 3 & \ldots & m-2 & m-1
                        \end{tabular}
                    \right)
                \end{align*}
                Therefore the product
                \begin{align*}
                    \sigma\tau^{-1} & = \left(
                        \begin{tabular}{cccccc}
                            1 & 2 & 3 & \ldots & m-2 & m-1 \\
                            2 & 3 & 4 & \ldots & m-1 & 1
                        \end{tabular}
                    \right)\left(
                        \begin{tabular}{cccccc}
                            1 & 2 & 3 & \ldots & m-1 & m \\
                            m & 1 & 2 & \ldots & m-2 & m-1
                        \end{tabular}
                    \right)\\
                    & = \left(
                        \begin{tabular}{cccccc}
                            1 & 2 & 3 & \ldots & m-1 & m \\
                            m & 2 & 3 & \ldots & m-1 & 1
                        \end{tabular}
                    \right)\\
                    & = (1,m) \qedhere
                \end{align*}
                \endgroup
            \end{proof}
            \vspace{0.2in}

            \item[\textbf{11}] Prove that in $S_n$, with $n \ge 3$ , any even
            permutation is a product of cycles of length three.\\
            \textit{Hint}: $(a,b)(b,c)=(a,b,c)$ and
            $(a,b)(c,d)=(a,b,c)(b,c,d)$.
            \item[\textbf{Ans}]
            \begin{proof}[\unskip\nopunct]
                To show any even permutation in $S_n$, with $n \ge 3$ is a product of cycles of length three, consider the case where the pair of transpositions are disjoint:
                \begin{equation*}
                    (a,b)(c,d) = (a,b,c)(b,c,d)
                \end{equation*}
                The product yields cycles of length three.\\
                The other case would be the transpositions consisting of repeating elements, such as
                \begin{equation*}
                    (a,b)(b,c)=(a,b,c)
                \end{equation*}
                where the product yields cycles of length three as well. \qedhere
            \end{proof}
            \vspace{0.2in}

            \item[\textbf{15}] For $\alpha, \beta \in S_n$, let
            $\alpha\sim\beta$ if there exists $\sigma \in S_n$ such that
            $\sigma\alpha\sigma^{-1}=\beta$. Show that $\sim$ is an equivalence
            relation on $S_n$.
            \item[\textbf{Ans}]
            \begin{proof}[\unskip\nopunct]
                To prove \textbf{reflexivity}, let $\alpha = 1_S$\\
                Then,
                \begin{align*}
                    \sigma\alpha\sigma^{-1} & = \sigma\sigma^{-1}\cdot 1_S\\
                    & = 1 \cdot 1_S\\
                    & = \alpha
                \end{align*}
                for any $\alpha \in S_n$\\
                Therefore, $\alpha \sim \alpha$\\

                To prove \textbf{symmetry}, let $\alpha \sim \beta$\\
                Then, there exists $\sigma \in S_n$ such that $\sigma\alpha\sigma^{-1} = \beta$\\
                Thus
                \begin{align*}
                    \sigma\alpha\sigma^{-1} &= \beta \\
                    \Rightarrow \sigma^{-1}\sigma\alpha\sigma^{-1} &= \sigma^{-1}\beta\\
                    \alpha\sigma^{-1} &= \sigma^{-1}\beta\\
                    \Rightarrow \alpha\sigma^{-1}\sigma &= \sigma^{-1}\beta\sigma\\
                    \alpha &= \sigma^{-1}\beta\sigma\\
                    \Rightarrow \alpha &= \sigma^{-1}\beta(\sigma^{-1})^{-1}
                \end{align*}
                Or $\sigma^{-1}\beta(\sigma^{-1})^{-1} = \alpha$\\
                Implying $\beta \sim \alpha$\\

                The prove \textbf{transitivity}, let $\alpha \sim \beta$ and
                $\beta \sim \gamma$\\
                Then, there exists $\sigma_1, \sigma_2 \in S_n$ such that
                $\sigma_1\alpha\sigma_1^{-1} = \beta$ and
                $\sigma_2\beta\sigma_2^{-1} = \gamma$\\
                Then $\sigma_2\sigma_1\alpha\sigma_1^{-1}\sigma_2^{-1} =
                \gamma$,\\
                Or $(\sigma_2\sigma_1)\alpha(\sigma_1\sigma_2)^{-1} =
                \gamma$\\
                Thus, $\alpha \sim \gamma$\\
                Therefore, $\sim$ is an equivalence relation on $S_n$ \qedhere

            \end{proof}
            \vspace{0.2in}

            \item[\textbf{16}] View $S_3$ as a subset of $S_5$, in the obvious
            way. For $\sigma, \tau \in S_5$, define $\sigma \sim \tau$ if
            $\sigma\tau^{-1} \in S_3$.
            \begin{itemize}

                \item[\textbf{a}] Show that $\sim$ is an equivalence relation
                on $S_5$.
                \item[\textbf{Ans}]
                \begin{proof}[\unskip\nopunct]
                    To prove \textbf{reflexive}, let $\sigma \in S_3$\\
                    Then,
                    \begin{align*}
                        \sigma\sigma^{-1}=1_{S_3} \in S_3
                    \end{align*}
                    Thus, $\sigma \sim \sigma$\\

                    To prove \textbf{symmetric}, let $\sigma, \tau \in S_3$\\
                    Then
                    \begin{align*}
                        \sigma \sim \tau & \in S_3\\
                        \Rightarrow \sigma\tau^{-1} & \in S_3\\
                        \Rightarrow (\sigma\tau^{-1})^{-1} & \in S_3\\
                        \Rightarrow \sigma^{-1}\tau & \in S_3
                    \end{align*}
                    Thus, $\tau \sim \sigma$\\

                    To prove \textbf{transitivity}, let $\sigma, \tau, \upsilon
                    \in S_3$.\\
                    Then, $\sigma \sim \tau \Rightarrow \sigma\tau^{-1} \in
                    S_3$\\
                    and $\tau \sim \upsilon \Rightarrow \tau\upsilon^{-1} \in
                    S_3$\\
                    Thus,
                    \begin{align*}
                        (\sigma\tau^{-1})(\tau\upsilon^{-1}) & \in S_3\\
                        \sigma(\tau^{-1}\tau)\upsilon^{-1} & \in S_3\\
                        \sigma\upsilon^{-1} & \in S_3\\
                        \Rightarrow \sigma \sim \upsilon & \in S_3
                    \end{align*}
                    Therefore, $\sim$ is an equivalence relation on $S_5$
                    \qedhere
                \end{proof}
                \vspace{0.2in}

                \item[\textbf{b}] Find the equivalence class of (4, 5).
                \item[\textbf{Ans}]
                \begin{proof}[\unskip\nopunct]
                    Since $(4,5) \in S_3$, $\Rightarrow (4,5)(5,4) \in S_3$ and
                    $(4,5) \sim (4,5)$\\
                    Similarly,\\
                    \begin{equation*}
                        (4,5)(1,2,3)(5,4), (4,5)(1,3,2)(5,4) \in S_3
                    \end{equation*}
                    and
                    \begin{equation*}
                        (4,5)(1,2)(5,4), (4,5)(1,3)(5,4), (4,5)(2,3)(5,4) \in S_3
                    \end{equation*}
                    Therefore,
                    \begin{equation*}
                        [(4,5)] = \{(4,5), (1,2,3)(4,5), (1,3,2)(4,5), (1,2)(4,5), (1,3)(4,5), (2,3)(4,5)\} \qedhere
                    \end{equation*}
                \end{proof}
                \vspace{0.2in}

                \item[\textbf{c}] Find the equivalence class of (1, 2, 3, 4, 5).
                \item[\textbf{Ans}]
                \begin{proof}[\unskip\nopunct]
                    Since
                    \begin{align*}
                        (1,2,3,4,5) & = (1,2)(1,3)(1,4)(1,5)\\
                        \text{and } (1,2)(1,3)(1,4)(4,1)(5,1) & = (1,2)(1,3)\\
                        & \in S_3
                    \end{align*}
                    Then $(1,2,3,4,5) \sim (1,4)(1,5)$\\
                    Similarly,
                    \begin{align*}
                        \{(1,2)(1,3)(1,4)(4,1)(5,1)(1,3,2), & \\
                        (1,2)(1,3)(1,4)(4,1)(5,1)(1,2), & \\
                        (1,2)(1,3)(1,4)(4,1)(5,1)(1,3), & \\
                        (1,2)(1,3)(1,4)(4,1)(5,1)(2,3), & \\
                        (1,2)(1,3)(1,4)(4,1)(5,1)(1,2,3)\} & \in S_3
                    \end{align*}
                    Therefore,
                    \begin{multline*}
                        [(1,2,3,4,5)] = \{(1,4)(1,5)(1,3,2),
                        (1,4)(1,5)(1,2),\\(1,4)(1,5)(1,3), (1,4)(1,5)(2,3),
                        (1,4)(1,5)(1,2,3)\} \qedhere
                    \end{multline*}
                \end{proof}
                \vspace{0.2in}

                \item[\textbf{d}] Determine the total number of equivalence
                classes.
                \item[\textbf{Ans}]
                \begin{proof}[\unskip\nopunct]
                    $S_3$ contains $3!=6$ elements, and $S_5$ contains $5!=120$
                    elements\\ Therefore, the number of equivalence classes are
                    $120/6=20$ \qedhere
                \end{proof}
                \vspace{0.2in}

            \end{itemize}

        \end{itemize}

    \end{itemize}

\end{document}
