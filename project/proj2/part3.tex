\subsection{Output}

    The first sample mean and standard deviation were computed:

    \[ E(\overline{X}) = 1.492, \ \sigma_{\overline{X}} = 0.011 \]

    All the samples were then used to find the sample mean and standard
    deviation. The theoretical values were also computed based on the
    relationships:

    \[ \mu = np \]
    \[ E(\overline{X}) = np \]
    \[ \sigma = np(1-p) \]
    \[ \sigma_{\overline{X}} = \frac{np(1-p)}{\sqrt{n}} \]


    \begin{table}[h]
        \centering
        \begin{tabular*}{200pt}{@{\extracolsep{\fill}} c c c}

        & \textbf{Actual} & \textbf{Theoretical} \\
        \hline
        $\mu$ & 1.500  & 1.500 \\
        E($\overline{X}$) & 1.502 & 1.500 \\
        $\sigma$ & 1.275 & 1.275 \\
        $\sigma$\textsubscript{$\overline{X}$} & 0.101 & 0.116 \\

        \end{tabular*}
    \end{table}
