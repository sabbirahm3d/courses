%%%%%%%%%%%%%%%%%%%%%%%%%%%%%%%%%%%%%%%%%
% Template
% LaTeX Template
% Version 1.0 (December 8 2014)
%
% This template has been downloaded from:
% http://www.LaTeXTemplates.com
%
% Original author:
% Brandon Fryslie
% With extensive modifications by:
% Vel (vel@latextemplates.com)
%
% License:
% CC BY-NC-SA 3.0 (http://creativecommons.org/licenses/by-nc-sa/3.0/)
%
% Authors:
% Sabbir Ahmed
%
%%%%%%%%%%%%%%%%%%%%%%%%%%%%%%%%%%%%%%%%%

\documentclass[paper=usletter, fontsize=12pt]{article}
%%%%%%%%%%%%%%%%%%%%%%%%%%%%%%%%%%%%%%%%%
% Contract Structural Definitions File Version 1.0 (December 8 2014)
%
% Created by: Vel (vel@latextemplates.com)
% 
% This file has been downloaded from: http://www.LaTeXTemplates.com
%
% License: CC BY-NC-SA 3.0 (http://creativecommons.org/licenses/by-nc-sa/3.0/)
%
%%%%%%%%%%%%%%%%%%%%%%%%%%%%%%%%%%%%%%%%%

\usepackage{geometry} % Required to modify the page layout
\usepackage{multicol}
\usepackage{amsmath}
\usepackage{amssymb}

\usepackage[pdftex]{graphicx}
\usepackage{wrapfig}
\usepackage[font=scriptsize, labelfont=bf]{caption}
\usepackage[utf8]{inputenc} % Required for including letters with accents
\usepackage[T1]{fontenc} % Use 8-bit encoding that has 256 glyphs

\usepackage{avant} % Use the Avantgarde font for headings
\usepackage{courier}
\usepackage{xparse}
\usepackage{xcolor}
\usepackage{listings}  % for code verbatim and console outputs

\setlength{\textwidth}{16cm} % Width of the text on the page
\setlength{\textheight}{23cm} % Height of the text on the page
\setlength{\oddsidemargin}{0cm} % Width of the margin - negative to move text left, positive to move it right
\setlength{\topmargin}{-1.25cm} % Reduce the top margin

\setlength{\parindent}{0mm} % Don't indent paragraphs
\setlength{\parskip}{2.5mm} % Whitespace between paragraphs
\renewcommand{\baselinestretch}{1.5}

\definecolor{green}{rgb}{0.18, 0.55, 0.34}

\graphicspath{ {figures/} }
\captionsetup[table]{skip=10pt}

\lstset{language=C, keywordstyle={\bfseries \color{black}}}

% defines algorithm counter for chapter-level
\newcounter{nalg}[section]

%defines appearance of the algorithm counter
\renewcommand{\thenalg}{\thesection .\arabic{nalg}}

% defines a new caption label as Algorithm x.y
\DeclareCaptionLabelFormat{algocaption}{Algorithm \thenalg}

% defines the algorithm listing environment
\lstnewenvironment{pseudocode}[1][] {
    \refstepcounter{nalg}  % increments algorithm number
    \captionsetup{font=normalsize, labelformat=algocaption, labelsep=colon}
    \lstset{
        breaklines=true,
        mathescape=true,
        numbers=left,
        numberstyle=\scriptsize,
        basicstyle=\footnotesize\ttfamily,
        keywordstyle=\color{black}\bfseries,
        keywords={input, output, return, parallel, function, for, to, in, if,
        else, foreach, while, and, or, new, print},
        xleftmargin=.04\textwidth,
        #1
    }
}{}

\renewcommand{\familydefault}{\sfdefault}  % default font for entire document
 % specifies the document layout and style

%------------------------------------------------------------------------------
% document info command
\newcommand{\documentinfo}[5]{
    \begin{centering}
        \parbox{2in}{
        \begin{spacing}{1}
            \begin{flushleft}
                \begin{tabular}{l l}
                    #1 \\
                    #2 \\
                    #3 \\
                \end{tabular}\\
                \rule{\textwidth}{1pt}
            \end{flushleft}
        \end{spacing}
        }
    \end{centering}
}

\newcommand{\ans}{\textbf{Answer} \ }
\newcommand\given[1][]{\:#1\vert\:}

\begin{document}

    \documentinfo{Sabbir Ahmed}{\textbf{DATE:} \today}{\textbf{CMPE 320} HW 01}
    \vspace{-0.2in}

    \begin{enumerate}

        \item Let $A$ and $B$ be two sets. Under what conditions is the set $A
        \cap (A \cup B)^c$ empty?

        \ans If $A = B$, then $A \cap (A \cup B)^c = A \cap (A \cup A)^c = A
        \cap A^c = \varnothing$.

        \item Prove:
        \begin{align*}
            P(A \cup B \cup C) & = P(A) + P(B) + P(C) - P(A \cap B) - P(B
            \cap C) - P(C \cap A) + P(A \cap B \cap C)
        \end{align*}
        \ans
        Let $D = B \cup C$ such that $A \cup B \cup C = A \cup D$. Then:
        \begin{align*}
            P(A \cup B \cup C) & = P(A \cup D) \\
            & = P(A) + P(D) - P(A \cap D) \\
            & = P(A) + P(B \cup C) - P(A \cap D) \\
            & = P(A) + P(B) + P(C) - P(B \cap C) - P(A \cap D)
        \end{align*}

        Since $P(A \cap D) = P(A \cap (B \cup C) ) = P((A \cap B) \cup (A \cap
        C))$ by the distributive law:
        \begin{align*}
        P(A \cup B \cup C) & = P(A) + P(B) + P(C) - P(B \cap C) - P(A \cap
        D) \\
        & = P(A) + P(B) + P(C) - P(B \cap C) - P((A \cap B) \cup (A \cap C))
        \end{align*}
        Since $P((A \cap B) \cup (A \cap C)) = P(A \cap B) + P(A \cap C) - P((A
        \cap B) \cap (A \cap C))$:
        \begin{align*}
        P(A \cup B \cup C) & = P(A) + P(B) + P(C) - P(B \cap C) - (P(A \cap B)
        + P(A \cap C) - P((A \cap B) \cap (A \cap C))) \\
        & = P(A) + P(B) + P(C) - P(B \cap C) - P(A \cap B) - P(A \cap C) + P(A
        \cap B \cap C)
        \end{align*}
        \begin{align*}
            \therefore P(A \cup B \cup C) & = P(A) + P(B) + P(C) - P(A \cap B)
            - P(B \cap C) - P(C \cap A) + P(A \cap B \cap C)
        \end{align*}

        \item We are given that $P(A^c) = 0.6$, $P(B) = 0.3$, and $P(A \cap B)
        = 0.2$. Determine $P(A \cup B)$.

        \ans
        \vspace{-0.4in}
        \begin{align*}
            \because P(A \cup B) & = P(A) + P(B) - P(A \cap B) \\
            \because P(A^c) & = 1 - P(A) \\
            \therefore P(A \cup B) & = (1 - 0.6) + 0.3 - 0.2 \\
            & = 0.5 \\
        \end{align*}

        \vspace{-0.4in}
        \item We roll a four-side die once and then we roll it as many times as
        is necessary to obtain a different face than the one obtained in the
        first roll. Let the outcome of the experiment be $(r_1, r_2)$ where
        $r_1$ and $r_2$ are the results of the first and the last rolls,
        respectively. Assume that all possible outcomes have equal probability.
        Find the probability that:

            \begin{enumerate}

                \item $r_1$ is even.

                \ans Since the possible even outcomes $\in \{1, 2, 3, 4\}$ are
                $\{2, 4\}$, then $P(\text{$r_1$ is even}) = 2/4 = 1/2$

                \item Both $r_1$ and $r_2$ are even.

                \ans Since the second roll $r_2$ is different from $r_1$, the
                cardinality of $(r_1, r_2)$ is $4 \cdot 3 = 12$. Therefore,
                $P(\text{$r_1$ and $r_2$ are even}) = 2/12 = 1/6$

                \item $r_1 + r_2 < 5$.

                \ans If $r_1 + r_2 < 5$ then the possible outcomes are only
                $\{1, 2, 3, 4\}$ with a cardinality of 4. Therefore, $P(r_1 +
                r_2 < 5) = 4/12 = 1/3$

            \end{enumerate}

        \item You enter a special kind of chess tournament, whereby you play
        one game with each of three opponents, but you get to choose the order
        in which you play your opponents. You win the tournament if you win two
        games in a row. You know your probability of a win against each of the
        three opponents. What is your probability of winning the tournament,
        assuming that you choose the optimal order of playing the opponents?

        \ans Let $W_i$ represent winning against player $i = 1, 2, 3$ and $p_i$
        represent the probability of winning against player $i = 1, 2, 3$.

        Then the sample space $S$:
            \[ S = \{ W_{1}^{c}W_{2}^{c}W_{3}^{c}, \ W_{1}^{c}W_{2}^{c}W_{3}, \
            W_{1}^{c}W_{2}W_{3}^{c}, \ W_{1}^{c}W_{2}W_{3}, \
            W_{1}W_{2}^{c}W_{3}^{c}, \ W_{1}W_{2}^{c}W_{3}, \
            W_{1}W_{2}W_{3}^{c}, \ W_{1}W_{2}W_{3} \} \]

        \item Alice and Bob each choose at random a number between zero and
        two. We assume a uniform probability law under which the probability of
        an event is proportional to its area. Consider the following events:

            \begin{enumerate}

                \item The magnitude of the difference of the two numbers is
                greater than 1/3.

                \ans Let $x, y \in [0, 2]$ represent the numbers chosen by
                Alice and Bob.
                \begin{align*}
                    x + y & > 3 \\
                    3x - 3y & > 1
                \end{align*}
                \begin{align*}
                    \text{area} & = \frac{1}{2} \cdot \frac{1}{3} \cdot
                    \frac{1}{3} \\
                    & = \frac{1}{18}
                \end{align*}
                \begin{align*}
                    \text{required area} & = 2 \cdot 2 - \frac{1}{18} \\
                    & = 4 - \frac{1}{18}
                \end{align*}
                \begin{align*}
                    \text{probability} & = \frac{4 - \frac{1}{18}}{4} \\
                    & = \frac{71}{72}
                \end{align*}

                \item At least one of the numbers is greater than 1/3.

                \ans

                \item The two numbers are equal.

                \ans

                \item Alice's number is greater than 1/3.

                \ans

            \end{enumerate}

        Find the probabilities $P(A)$, $P(B)$, $P(A \cap B)$, $P(C)$, $P(D)$,
        $P(A \cap D)$,

        \item The disc containing the only copy of your term project just got
        corrupted, and the disc got mixed up with three other corrupted discs
        that were lying around. It is equally likely that any of the four discs
        holds the corrupted remains of your term project. Your computer expert
        friend offers to have a look, and you know from past experience that
        his probability of finding your term project from any disc is 0.4
        (assuming the term project is there). Given that he searches on disc 1
        but cannot find your thesis, what is the probability that your thesis
        is on disc $i$ for $i = 1, 2, 3, 4$?

        \ans Let $D_i$ represent the event that the term project in the disc
        $i$ and $T$ represent the event the term project finds the thesis on
        disc 1.

            \begin{equation*}
                \therefore P(D_i) = 0.25, \text{ \ for \ } i = 1, 2, 3, 4 \\
            \end{equation*}
            \begin{align*}
                P(T \given D_1) & = 0.4 \\
                P(T \given D_2) & = 0 \\
                P(T \given D_3) & = 0 \\
                P(T \given D_4) & = 0
            \end{align*}
            \begin{align*}
                P(T \given D_1^c) & = 0.6 \\
                P(T \given D_2^c) & = 1 \\
                P(T \given D_3^c) & = 1 \\
                P(T \given D_4^c) & = 1
            \end{align*}

        Using Bayes Rule to calculate $P(D_i \given T^c)$ for $i = 1, 2, 3, 4$:
            \begin{align*}
                P(D_i \given T^c) & = \frac{P(D_i)P(F^c \given
                D_i)}{\sum_{k=1}^{4}P(D_k)P(F^c \given D_k)}
            \end{align*}
            \begin{align*}
                P(D_1 \given T^c) & = \frac{P(D_1)P(F^c \given
                D_1)}{\sum_{k=1}^{4}P(D_k)P(F^c \given D_k)} \\
                & = \frac{P(D_1)P(F^c \given D_1)}{P(D_1)P(F^c \given D_1) +
                P(D_2)P(F^c \given D_2) + P(D_3)P(F^c \given D_3) + P(D_4)P(F^c
                \given D_4)} \\
                & = \frac{(0.25 \cdot 0.6)}{(0.25 \cdot 0.6) + (0.25
                \cdot 1) + (0.25 \cdot 1) + (0.25 \cdot 1)} \\
                & = \frac{0.15}{0.90} \\
                & = 0.1667
            \end{align*}
        Similarly,
            \begin{align*}
                P(D_2 \given T^c) & = \frac{P(D_2)P(F^c \given
                D_2)}{\sum_{k=1}^{4}P(D_k)P(F^c \given D_k)} \\
                & = \frac{0.25}{0.90} \\
                & = 0.2778
            \end{align*}
            \begin{align*}
                P(D_3 \given T^c) & = \frac{P(D_3)P(F^c \given
                D_3)}{\sum_{k=1}^{4}P(D_k)P(F^c \given D_k)} \\
                & = \frac{0.25}{0.90} \\
                & = 0.2778
            \end{align*}
            \begin{align*}
                P(D_4 \given T^c) & = \frac{P(D_4)P(F^c \given
                D_4)}{\sum_{k=1}^{4}P(D_k)P(F^c \given D_k)} \\
                & = \frac{0.25}{0.90} \\
                & = 0.2778
            \end{align*}

        \item A person has forgotten the last digit of a telephone number, so
        he dials the number with the last digit randomly chosen. How many times
        does he have to dial (not counting repetitions) in order that the
        probability of dialing the correct number is more than 0.5?

        \ans The probability of correct digit: 1/10. Let $n$ be the number of
        dials. Then:
        \begin{align*}
            n \cdot \frac{1}{10} & > 0.5 \\
            n & > 0.5 \cdot 10 \\
            n & > 5
        \end{align*}
        Therefore, the minimum number of dials required is $n=6$.

        \item A new test has been developed to determine whether a given
        student is overstressed. This test is 95\% accurate if the student is
        not overstressed, but only 85\% accurate if the student is in fact
        overstressed. It is known that 99.5\% of all students are overstressed.
        Given that a particular student tests negative for stress, what is the
        probability that the test results are correct, and that this student is
        not overstressed?

        \ans Let $S$ represent a non-overstressed student, and $N$ represent a
        negative test. We need to find $P(S \given N)$.
        \begin{align*}
            P(S \given N) & = \frac{P(S^c)P(N \given S^c)}{P(N)} \\
            & = \frac{0.005 \cdot 0.95}{0.005 \cdot 0.95 + 0.995 \cdot 0.15 }\\
            & \approx 0.03
        \end{align*}

        \item A hiker starts by taking one of $n$ available trails, denoted $1,
        2, \ldots, n$. An hour into the hike, trail $i$ subdivides into 1 +
        $i$ subtrails, only one of which leads to the hiker’s destination. The
        hiker has no map and makes random choices of trail and subtrail. What
        is the probability of reaching the destination?

        \ans Let $T_i$ represent the event of starting with trail $i$.
        \begin{align*}
            P(T_i) & = \frac{1}{n} \text{ \ for \ } i = 1, 2, \ldots, n
        \end{align*}
        Let $D$ represent reaching the destination.
        \begin{align*}
            P(D \given T_i) & = \frac{1}{i+1} \text{ \ for \ } i = 1, 2,
            \ldots, n
        \end{align*}
        By the total probability theorem, the probability of reaching the
        destination is:
        \begin{align*}
            P(D) & = P(T_i)P(D \given T_i) \\
            & = \frac{1}{n} \cdot \frac{1}{i+1} \\
            & = \frac{1}{n(i+1)}
        \end{align*}


    \end{enumerate}

\end{document}
