%%%%%%%%%%%%%%%%%%%%%%%%%%%%%%%%%%%%%%%%%
%
% Sabbir Ahmed
%
%%%%%%%%%%%%%%%%%%%%%%%%%%%%%%%%%%%%%%%%%

\documentclass[paper=usletter, fontsize=12pt]{article}
%%%%%%%%%%%%%%%%%%%%%%%%%%%%%%%%%%%%%%%%%
% Contract Structural Definitions File Version 1.0 (December 8 2014)
%
% Created by: Vel (vel@latextemplates.com)
% 
% This file has been downloaded from: http://www.LaTeXTemplates.com
%
% License: CC BY-NC-SA 3.0 (http://creativecommons.org/licenses/by-nc-sa/3.0/)
%
%%%%%%%%%%%%%%%%%%%%%%%%%%%%%%%%%%%%%%%%%

\usepackage{geometry} % Required to modify the page layout
\usepackage{multicol}
\usepackage{amsmath}
\usepackage{amssymb}

\usepackage[pdftex]{graphicx}
\usepackage{wrapfig}
\usepackage[font=scriptsize, labelfont=bf]{caption}
\usepackage[utf8]{inputenc} % Required for including letters with accents
\usepackage[T1]{fontenc} % Use 8-bit encoding that has 256 glyphs

\usepackage{avant} % Use the Avantgarde font for headings
\usepackage{xparse}
\usepackage{xcolor}
\usepackage{listings}  % for code verbatim and console outputs

\setlength{\textwidth}{16cm} % Width of the text on the page
\setlength{\textheight}{23cm} % Height of the text on the page
\setlength{\oddsidemargin}{0cm} % Width of the margin - negative to move text left, positive to move it right
\setlength{\topmargin}{-1.25cm} % Reduce the top margin

\setlength{\parindent}{0mm} % Don't indent paragraphs
\setlength{\parskip}{2.5mm} % Whitespace between paragraphs
\renewcommand{\baselinestretch}{1.2}

\renewcommand\familydefault{\sfdefault}  % default font for entire document

\definecolor{green}{rgb}{0.18, 0.55, 0.34}

\graphicspath{ {figures/} }
\captionsetup[table]{skip=10pt}

\lstset{language=C, keywordstyle={\bfseries \color{black}}}

% defines algorithm counter for chapter-level
\newcounter{nalg}[section]

%defines appearance of the algorithm counter
\renewcommand{\thenalg}{\thesection .\arabic{nalg}}

% defines a new caption label as Algorithm x.y
\DeclareCaptionLabelFormat{algocaption}{Algorithm \thenalg}

%defines the algorithm listing environment
\lstnewenvironment{pseudocode}[1][] {
    \refstepcounter{nalg} %increments algorithm number

    \captionsetup{labelformat=algocaption,labelsep=colon}
    \lstset{
        mathescape=true,
        frame=tB,
        numbers=left,
        numberstyle=\tiny,
        basicstyle=\scriptsize,
        keywordstyle=\color{black}\bfseries\em,
        keywords={,input, output, return, datatype, function, in, if, else, foreach, while, begin, end, },
        xleftmargin=.04\textwidth,
        #1
    }
}{}
 % specifies the document layout and style

\begin{document}

    \documentinfo{\today}{12}

    \begin{itemize}

        \item[\textbf{4.1}]
        \begin{enumerate}

            \item[\textbf{1}] Let $f(x)$, $g(x)$, $g(x) \in F[x]$. Show that
            the following properties hold.
            \begin{enumerate}

                \item[\textbf{c}] If $g(x) \mid f(x)$, then $g(x) \cdot h(x)
                \mid f(x) \cdot h(x)$.
                \begin{proof}
                \end{proof}

                \item[\textbf{d}] If $g(x) \mid f(x)$ and $f(x) \mid g(x)$,
                then $f(x)=kg(x)$ for some $k\in F$.
                \begin{proof}
                \end{proof}

            \end{enumerate}

            \item[\textbf{5}] Over the given field $\mathbb{F}$, write
            $f(x)=q(x)(x-c)+f(c)$ for
            \begin{enumerate}

                \item[\textbf{b}] $f(x)=2x^3+x^2-4x+3$; $c=1$;
                $\mathbb{F}=\mathbb{Q}$;
                \begin{proof}
                \end{proof}

                \item[\textbf{d}] $f(x)=x^3+2x+3$; $c=2$;
                $\mathbb{F}=\mathbb{Z}_5$;
                \begin{proof}
                \end{proof}

            \end{enumerate}

            \item[\textbf{6}] Let $p$ be a prime number. Find all roots of
            $x^{p-1}-1$ in $\mathbb{Z}_p$.
            \begin{proof}
            \end{proof}

            \item[\textbf{7}] Show that if $c$ is any element of the field
            $\mathbb{F}$ and $k>2$ is an odd integer, then $x+c$ is a factor of
            $x^k+c^k$.
            \begin{proof}
            \end{proof}

            \item[\textbf{11}] Show that the set
            $\mathbb{Q}(\sqrt{3})=\{a+b\sqrt{3}\mid a,b \in \mathbb{Q}\}$ is
            closed under addition, subtraction, multiplication, and division.
            \begin{proof}
            \end{proof}

            \item[\textbf{13}] Show that the set of matrices of the form
            $\left[\begin{tabular}{LL}
                        a & b \\
                        -b & a
            \end{tabular}\right]$, where $a,b\in \mathbb{R}$, is a
            field under the operations of matrix addition and
            multiplication.
            \begin{proof}
            \end{proof}

            \item[\textbf{17}] Let $(x_0,y_0),(x_1,y_1),(x_2,y_2)$ be points in
            the Euclidean plane $\mathbb{R}^2$ such that $x_0,x_1,x_2$ are
            distinct. Show the formula
            \begin{equation*}
                f(x)=\frac{y_0(x-x_1)(x-x_2)}{(x_0-x_1)(x_0-x_2)}+\frac{y_1(x-x_0)(x-x_2)}{(x_1-x_0)(x_1-x_2)}+\frac{y_2(x-x_0)(x-x_1)}{(x_2-x_0)(x_2-x_1)}
            \end{equation*}
            defines a polynomial $f(x)$ such that $f(x_0)=y_0$, $f(x_1)=y_1$,
            and $f(x_2)=y_2$.
            \begin{proof}
            \end{proof}

            \item[\textbf{18}] Use Lagrange's interpolation formula to find a
            polynomial $f(x)$ such that $f(1)=0$, $f(2)=1$, and $f(3)=4$.
            \begin{proof}
            \end{proof}

        \end{enumerate}

    \end{itemize}

\end{document}
