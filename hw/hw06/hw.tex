%%%%%%%%%%%%%%%%%%%%%%%%%%%%%%%%%%%%%%%%%
% Template
% LaTeX Template
% Version 1.0 (December 8 2014)
%
% This template has been downloaded from:
% http://www.LaTeXTemplates.com
%
% Original author:
% Brandon Fryslie
% With extensive modifications by:
% Vel (vel@latextemplates.com)
%
% License:
% CC BY-NC-SA 3.0 (http://creativecommons.org/licenses/by-nc-sa/3.0/)
%
% Authors:
% Sabbir Ahmed
%
%%%%%%%%%%%%%%%%%%%%%%%%%%%%%%%%%%%%%%%%%

\documentclass[paper=usletter, fontsize=12pt]{article}
%%%%%%%%%%%%%%%%%%%%%%%%%%%%%%%%%%%%%%%%%
% Contract Structural Definitions File Version 1.0 (December 8 2014)
%
% Created by: Vel (vel@latextemplates.com)
% 
% This file has been downloaded from: http://www.LaTeXTemplates.com
%
% License: CC BY-NC-SA 3.0 (http://creativecommons.org/licenses/by-nc-sa/3.0/)
%
%%%%%%%%%%%%%%%%%%%%%%%%%%%%%%%%%%%%%%%%%

\usepackage{geometry} % Required to modify the page layout
\usepackage{multicol}
\usepackage{amsmath}
\usepackage{amssymb}

\usepackage[pdftex]{graphicx}
\usepackage{wrapfig}
\usepackage[font=scriptsize, labelfont=bf]{caption}
\usepackage[utf8]{inputenc} % Required for including letters with accents
\usepackage[T1]{fontenc} % Use 8-bit encoding that has 256 glyphs

\usepackage{avant} % Use the Avantgarde font for headings
\usepackage{courier}
\usepackage{xparse}
\usepackage{xcolor}
\usepackage{listings}  % for code verbatim and console outputs

\setlength{\textwidth}{16cm} % Width of the text on the page
\setlength{\textheight}{23cm} % Height of the text on the page
\setlength{\oddsidemargin}{0cm} % Width of the margin - negative to move text left, positive to move it right
\setlength{\topmargin}{-1.25cm} % Reduce the top margin

\setlength{\parindent}{0mm} % Don't indent paragraphs
\setlength{\parskip}{2.5mm} % Whitespace between paragraphs
\renewcommand{\baselinestretch}{1.5}

\definecolor{green}{rgb}{0.18, 0.55, 0.34}

\graphicspath{ {figures/} }
\captionsetup[table]{skip=10pt}

\lstset{language=C, keywordstyle={\bfseries \color{black}}}

% defines algorithm counter for chapter-level
\newcounter{nalg}[section]

%defines appearance of the algorithm counter
\renewcommand{\thenalg}{\thesection .\arabic{nalg}}

% defines a new caption label as Algorithm x.y
\DeclareCaptionLabelFormat{algocaption}{Algorithm \thenalg}

% defines the algorithm listing environment
\lstnewenvironment{pseudocode}[1][] {
    \refstepcounter{nalg}  % increments algorithm number
    \captionsetup{font=normalsize, labelformat=algocaption, labelsep=colon}
    \lstset{
        breaklines=true,
        mathescape=true,
        numbers=left,
        numberstyle=\scriptsize,
        basicstyle=\footnotesize\ttfamily,
        keywordstyle=\color{black}\bfseries,
        keywords={input, output, return, parallel, function, for, to, in, if,
        else, foreach, while, and, or, new, print},
        xleftmargin=.04\textwidth,
        #1
    }
}{}

\renewcommand{\familydefault}{\sfdefault}  % default font for entire document
 % specifies the document layout and style
\allowdisplaybreaks

%------------------------------------------------------------------------------
% document info command
\newcommand{\documentinfo}[5]{
    \begin{centering}
        \parbox{2in}{
        \begin{spacing}{1}
            \begin{flushleft}
                \begin{tabular}{l l}
                    #1 \\
                    #2 \\
                    #3 \\
                \end{tabular}\\
                \rule{\textwidth}{1pt}
            \end{flushleft}
        \end{spacing}
        }
    \end{centering}
}

\begin{document}

    \documentinfo{Sabbir Ahmed}{\textbf{DATE:} \today}{\textbf{CMPE 320:} HW 05}
    \vspace{-0.2in}

    \begin{enumerate}[label=\textbf{\arabic*}.]

        % 1
        \item
        A stock market trader buys 100 shares of stock A and 200 shares of
        stock B. Let $X$ and $Y$ be the price changes of A and B, respectively,
        over a certain time period. And assume that the joint PMF of $X$ and
        $Y$ is uniform over the set of integers $x$ and $y$ satisfying
        \begin{align*}
            -2 \le x \le 4 && -1 \le y-x \le 1.
        \end{align*}
        \begin{enumerate}[label=(\alph*)]

            \item Find the marginal PMFs and the means of $X$ and $Y$.
            \begin{proof}[\unskip\nopunct]
                Given:
                \begin{equation*}
                    X \in \{x:-2 \le x \le 4\}
                \end{equation*}
                \begin{equation*}
                    Y \in \{y : x-1 \le y \le x+1, \ x \in X\}
                \end{equation*}
                Therefore, the pairs $(x,y)$ consist of:
                \begin{align*}
                    (x,y) & \in \{ (-2, -3), (-2, -2), (-2, -1), \\
                    & \ \ \ \ \ \ (-1, -2), (-1, -1), (-1, 0), \ldots, \\
                    & \ \ \ \ \ \ (4, 3), (4, 4), (4, 5) \}
                \end{align*}
                Totalling in $7 \times 3 = 21$ pairs.\\
                Therefore, the joint PMF is\\
                \begin{align*}
                    p_{X,Y}(x,y) & =
                    \begin{cases}
                        1/21, & \text{if } -2 \le x \le 4, -1 \le y-x \le 1 \\
                        0, & \text{otherwise}
                    \end{cases}
                \end{align*}
                The marginal PMF are \\
                \begingroup
                \addtolength{\jot}{1em}
                \begin{align*}
                    p_{X}(x) & = \sum_{y}p_{X,Y}(x,y) \\
                    & = \begin{cases}
                        3/21, & \text{if } -2 \le x \le 4, \\
                        0, & \text{otherwise}
                    \end{cases}
                \end{align*}
                \endgroup
                \begingroup
                \addtolength{\jot}{1em}
                \begin{align*}
                    p_{Y}(y) & = \sum_{x}p_{X,Y}(x,y) \\
                    & = \begin{cases}
                        1/21, & \text{if } y=-3,5, \\
                        2/21, & \text{if } y=-2,4, \\
                        3/21, & \text{if } -1 \le x \le 3, \\
                        0, & \text{otherwise}
                    \end{cases}
                \end{align*}
                \endgroup
                The means,\\
                \begingroup
                \addtolength{\jot}{1em}
                \begin{align*}
                    E[X] & = \sum_{x}x \cdot p_{X}(x) \\
                    & = \frac{3}{21}((-2)+(-1)+0+1+2+3+4)\\
                    & = \frac{3}{21}(7)\\
                    & = 1
                \end{align*}
                \endgroup
                \begingroup
                \addtolength{\jot}{1em}
                \begin{align*}
                    E[Y] & = \sum_{y}y \cdot p_{Y}(y) \\
                    & = \frac{1}{21}((-3)+5) + \frac{2}{21}((-2)+4) + \frac{3}{21}((-1)+0+1+2+3)\\
                    & = 1 \qedhere
                \end{align*}
                \endgroup
            \end{proof}
            \vspace{0.2in}

            \item Find the mean of the trader's profit.
            \begin{proof}[\unskip\nopunct]
                \vspace{-0.2in}
                \begin{align*}
                    100E[X] + 200E[Y] & = 100(1) + 200(1) \\
                    &= 300 \qedhere
                \end{align*}
            \end{proof}
            \vspace{0.2in}

        \end{enumerate}

        % 2
        \item
        The MIT football team wins any one game with probability $p$, and loses
        it with probability $1 - p$. Its performance in each game is
        independent of its performance in other games. Let $L_1$ be the number
        of losses before its first win, and let $L_2$ be the number of losses
        after its first win and before its second win. Find the joint PMF of
        $L_1$ and $L_2$.
        \begin{proof}[\unskip\nopunct]
            For $L_1=0$, $L_2=0$,\\
            \begin{align*}
                P(L_1=0, L_2=0) & = p \cdot p \\
                & = p^2
            \end{align*}
            For $L_1=0$, $L_2=1$,\\
            \begin{align*}
                P(L_1=0, L_2=1) & = p \cdot ((1-p) \cdot p)  \\
                & = p^2(1-p)
            \end{align*}
            Similarly, for $L_1=1$, $L_2=0$,\\
            \begin{align*}
                P(L_1=1, L_2=0) & = ((1-p) \cdot p) \cdot p  \\
                & = p^2(1-p)
            \end{align*}
            For $L_1=0$, $L_2=2$,\\
            \begin{align*}
                P(L_1=0, L_2=2) & = p \cdot ((1-p) \cdot (1-p) \cdot p)  \\
                & = p^2(1-p)^2
            \end{align*}
            For $L_1=0$, $L_2=3$,\\
            \begin{align*}
                P(L_1=0, L_2=3) & = p \cdot ((1-p) \cdot (1-p) \cdot (1-p) \cdot p)  \\
                & = p^2(1-p)^3
            \end{align*}
            And so on. Therefore, the general expression is:
            \begin{equation*}
                p^2(1-p)^{L_1+L_2}
            \end{equation*}
            with the PMF:
            \begin{equation*}
                p_{L_1,L_2}(L_1,L_2) = p^2(1-p)^{L_1+L_2} \qedhere
            \end{equation*}
        \end{proof}
        \vspace{0.2in}

        % 3
        \item
        A class of $n$ students take a test in which each student gets an A
        with probability $p$, a B with probability $q$, and a grade below B
        with probability $1 - p - q$, independently of any other student. If
        $X$ and $Y$ are the numbers of students that get an A and a B,
        respectively. calculate the joint PMF $p_{x,y}$.
        \begin{proof}[\unskip\nopunct]
            Let $r = 1 - p - q$\\
            Then, the multinomial distribution
            \begin{align*}
            p_{X,Y}(x, y) & = \frac{n!}{x!y!(n-x-y)!} \cdot p^x \cdot q^y \cdot r^{(n-x-j)} \text{ for } y=0,1,2..., \ 0 \le x+y \le n \qedhere
            \end{align*}
        \end{proof}
        \vspace{0.2in}

        % 4
        \item
        Your probability class has 250 undergraduate students and 50 graduate
        students. The probability of an undergraduate (or graduate) student
        getting an A is 1/3 (or 1/2, respectively). Let $X$ be the number of
        students that get an A in your class.
        \begin{enumerate}[label=(\alph*)]

            \item Calculate $E[X]$ by first finding the PMF of $X$
            \begin{proof}[\unskip\nopunct]
                Let $x_i$ for $i = 1,2,\ldots,300$ represent the event\\
                where if $x_i=1$, student $i$ gets an A, and $x_i=0$ otherwise\\
                The PMF, $p_X$:
                \begin{align*}
                    p_X(x_i \mid \text{Undergraduate})
                    & = \begin{cases}
                        1/3, & \text{ if } x_i = 1, \\
                        2/3, & \text{ if } x_i = 0
                    \end{cases} \\
                    p_X(x_i \mid \text{Graduate})
                    & = \begin{cases}
                        1/2, & \text{ if } x_i = 1, \\
                        1/2, & \text{ if } x_i = 0
                    \end{cases}
                \end{align*}
                Therefore,
                \begingroup
                \addtolength{\jot}{1em}
                \begin{align*}
                    E[X] &= E \sum_{i=1}^{300} x_i \\
                    &= 300E[x_i] \\
                    &= 300 \bigg( \frac{1}{3} \cdot \frac{5}{6} + \frac{1}{2} \cdot \frac{1}{6} \bigg) \\
                    & \approx 109
                \end{align*} \qedhere
                \endgroup
            \end{proof}
            \vspace{0.2in}

            \item Calculate $E[X]$ by viewing $X$ as a sum of random variables,
            whose mean is easily calculated.
            \begin{proof}[\unskip\nopunct]
                Let $Y$ and $Z$ represent the number of undergraduate and
                graduate students who receive an A, respectively\\
                Therefore,
                \begin{equation*}
                    X = Y + Z
                \end{equation*}
                Thus, the expectation is
                \begingroup
                \addtolength{\jot}{1em}
                \begin{align*}
                    E[X] &= E[Y] + E[Z] \\
                    &= 250 \cdot \frac{1}{3} + 50 \cdot \frac{1}{2} \\
                    & \approx 109
                \end{align*} \qedhere
                \endgroup
            \end{proof}
            \vspace{0.2in}

        \end{enumerate}

        % 5
        \item
        A scalper is considering buying tickets for a particular game. The
        price of the tickets is \$75, and the scalper will sell them at \$150.
        However, if she can't sell them at \$150, she won't sell them at all.
        Given that the demand for tickets is a binomial random variable with
        parameters $n=10$ and $p=1/2$, how many tickets should she buy in order
        to maximize her expected profit?
        \begin{proof}[\unskip\nopunct]
            Let $i$ be the number of tickets,
            \begin{align*}
                i & = (n+1)p \\
                & = (10 + 1)(0.5) \\
                & = 5.5 \approx 6
            \end{align*}
            Therefore, she should buy 6 tickets in order to maximize her
            expected profit \qedhere
        \end{proof}
        \vspace{0.2in}

        % 6
        \item
        Suppose that $X$ and $Y$ are independent discrete random variables with
        the same geometric PMF:
        \begin{align*}
            p_X(k)=p_Y(k)=p(1-p)^{k-1}&& k=1,2,\ldots,
        \end{align*}
        where $p$ is a scalar with $0$, $p<1$. Show that for any integer $n \ge
        2$, the conditional PMF
        \begin{equation*}
            P(X=k \mid X+Y=n)
        \end{equation*}
        is uniform.
        \begin{proof}[\unskip\nopunct]
            \begingroup
            \addtolength{\jot}{1em}
            \begin{align*}
                P(X=k \mid X+Y=n) & = \frac{P(X=k, X+Y=n)}{P(X+Y=n)} \\
                & = \frac{P(X=k, Y=n-k)}{P(Y=n-k)}\\
                & = \frac{P(X=k, Y=n-k)}{\sum_{k=0}^{n}P(X=k,Y=n-k)}\\
                & = \frac{P(X=k) \cdot P(Y=n-k)}{\sum_{k=0}^{n}P(X=k,Y=n-k)}\\
                & = \frac{p \cdot (1-p)^{k-1} \cdot p \cdot (1-p)^{n-k}}{\sum_{k=0}^{n}p \cdot p^{k-1} \cdot p \cdot (1-p)^{n-k}}\\
                & = \frac{p^2 \cdot (1-p)^{n-1}}{p^2 \cdot (1-p)^{n-1} \cdot (n+1)}\\
                & = \frac{1}{n+1}
            \end{align*}
            \endgroup
            Therefore, $P(X=k \mid X+Y=n)$ is uniform \qedhere
        \end{proof}
        \vspace{0.2in}

        % 7
        \item
        Consider four independent rolls of a 6-sides die. Let $X$ be the number
        of 1s and let $Y$ be the number of 2s obtained. What is the joint PMF
        of $X$ and $Y$?
        \begin{proof}[\unskip\nopunct]
            Let $R$ represent the number yielded after a roll\\
            For a single trial,
            \begin{equation*}
                P(R=1) = P(R=2) = \frac{1}{6}
            \end{equation*}
            and
            \begin{equation*}
                P(R=3,4,5,6) = \frac{4}{6}
            \end{equation*}
            Therefore, the multinomial distribution
            \begingroup
            \addtolength{\jot}{1em}
            \begin{align*}
                p_{X,Y}(x,y) &= \frac{4!}{x!y!(4-(x+y))!} \cdot P(R=1)^{x} \cdot P(R=2)^{y} \cdot P(R=3)^{(4-(x+y))} \\
                & = \frac{4!}{x!y!(4-(x+y))!} \cdot \bigg( \frac{1}{6} \bigg)^{x} \cdot \bigg( \frac{1}{6} \bigg)^{y} \cdot \bigg( \frac{4}{6} \bigg)^{(4-x-y)} \qedhere
            \end{align*}
            \endgroup
        \end{proof}
        \vspace{0.2in}

        % 8
        \item
        Alvin shops for probability books for $K$ hours, where $K$ is a random
        variable that is equally likely to be 1, 2, 3, or 4. The number of
        books $N$ that he buys is random and depends on how long he shops
        according to the conditional PMF
        \begin{equation*}
            p_{N \mid K}(n \mid k) = \frac{1}{k}, \ \text{ for } n=1,\ldots,k
        \end{equation*}
        \begin{enumerate}[label=(\alph*)]

            \item Find the joint PMF of $K$ and $N$
            \begin{proof}[\unskip\nopunct]
                Since $K$ is equally likely to be 1, 2, 3, or 4
                \begin{equation*}
                    p_K(k) = \frac{1}{4}, \ \text{ for } k=1,2,3,4
                \end{equation*}
                Therefore
                \begingroup
                \addtolength{\jot}{1em}
                \begin{align*}
                    p_{N , K}(n , k) & = p_{N \mid K}(n \mid k) \cdot p_K(k)\\
                    & = \frac{1}{k}\cdot \frac{1}{4}, \ \text{ for } k=1,2,3,4, \ n=1,\ldots,k \\
                    & = \begin{cases}
                        \frac{1}{4k}, & \text{ if } k=1,2,3,4, \ n=1,\ldots,k,\\
                        0, & \text{ otherwise }
                    \end{cases}
                \end{align*} \qedhere
                \endgroup
            \end{proof}
            \vspace{0.2in}

            \item Find the marginal PMF of $N$
            \begin{proof}[\unskip\nopunct]
                The marginal PMF is given by
                \begin{equation*}
                    p_N(n) = \sum_{n}^{4}p_{N, K}(n,k)
                \end{equation*}
                Therefore,
                \begingroup
                \addtolength{\jot}{1em}
                \begin{align*}
                    p_N(n=1) & = \frac{1}{4} + \frac{1}{8} + \frac{1}{12} + \frac{1}{16} \\
                    & = \frac{25}{48} \\
                    p_N(n=2) & = \frac{1}{8} + \frac{1}{12} + \frac{1}{16} \\
                    & = \frac{13}{48} \\
                    p_N(n=3) & = \frac{1}{12} + \frac{1}{16} \\
                    & = \frac{7}{48} \\
                    p_N(n=4) & = \frac{1}{16} \\
                    p_N(n > 4) & = 0 \qedhere
                \end{align*}
                \endgroup
            \end{proof}
            \vspace{0.2in}

            \item Find the conditional PMF of $K$ given that $N=2$
            \begin{proof}[\unskip\nopunct]
                \begingroup
                \addtolength{\jot}{1em}
                \begin{align*}
                    p_{K \mid 2}(k \mid 2) & = \frac{p_{N,K}(2,k)}{p_N(2)} \\
                    & = \begin{cases}
                        \frac{6}{13}, & \text{ if } n = 2, \\
                        \frac{4}{13}, & \text{ if } n = 3, \\
                        \frac{3}{13}, & \text{ if } n = 4, \\
                        0, & \text{ otherwise }
                    \end{cases}
                \end{align*}
                \endgroup \qedhere
            \end{proof}
            \vspace{0.2in}

            \item Find the conditional mean and variance of $K$, given that he
            bought at least 2 but no more than 3 books.
            \begin{proof}[\unskip\nopunct]
                \begingroup
                \addtolength{\jot}{1em}
                \begin{align*}
                    p_{K \mid (2 \le n \le 3)}(k) & = \frac{P(K=k,(2 \le n \le 3))}{p_N(2) + p_N(3)} \\
                    \text{where } p_N(2) + p_N(3) & = \frac{5}{12} \\
                    P(K=k,(2 \le n \le 3)) & = \begin{cases}
                        \frac{1}{8}, & \text{ if } k = 2, \\
                        \frac{1}{6}, & \text{ if } k = 3, \\
                        \frac{1}{8}, & \text{ if } k = 4, \\
                        0, & \text{ otherwise }
                    \end{cases}
                \end{align*}
                \endgroup
                Therefore,
                \begingroup
                \addtolength{\jot}{1em}
                \begin{align*}
                    p_{K \mid (2 \le n \le 3)}(k) & = \begin{cases}
                        \frac{3}{10}, & \text{ if } k = 2, \\
                        \frac{4}{10}, & \text{ if } k = 3, \\
                        \frac{3}{10}, & \text{ if } k = 4, \\
                        0, & \text{ otherwise }
                    \end{cases}
                \end{align*}
                \endgroup
                Conditional mean
                \begin{equation*}
                    E[K \mid (2 \le n \le 3)]=3
                \end{equation*}
                Conditional variance
                \begingroup
                \addtolength{\jot}{1em}
                \begin{align*}
                    var(K \mid (2 \le n \le 3)) & = E[K-E[K \mid (2 \le n \le 3)]^2 \mid (2 \le n \le 3)] \\
                    & = \frac{3}{10}(2-3)^2 + \frac{2}{5}(0) + \frac{3}{10}(4-3)^2 \\
                    & = \frac{3}{5} \qedhere
                \end{align*}
                \endgroup
            \end{proof}
            \vspace{0.2in}

            \item The cost of each book is a random variable with mean \$30.
            What is the expected value of his total expenditure? \textit{Hint:}
            Condition on the events $\{N=1\},\ldots, \{N=4\}$, and use the
            total expectation theorem.
            \begin{proof}[\unskip\nopunct]
                Let $T=C_1+\ldots + C_N$, where $E[C_i]=30$ for $i = 1, \ldots, N$\\
                \begingroup
                \addtolength{\jot}{1em}
                \begin{align*}
                    E[T] & = E[E[T \mid N]] \\
                    & = E[N \cdot 30] \\
                    & = 30E[N] \\
                    & = 30 (1 \cdot \frac{25}{48} + 2 \cdot \frac{13}{48} + 3 \cdot \frac{7}{48} + 4 \cdot \frac{3}{48}) \\
                    & = 52.5 \qedhere
                \end{align*}
                \endgroup
            \end{proof}
            \vspace{0.2in}

        \end{enumerate}

    \end{enumerate}

\end{document}
