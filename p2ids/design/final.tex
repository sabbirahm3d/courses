%
% CMSC 421: Project 2 Final Design Document
% Template layout and preamble defined in structure.tex
%
% Author: Sabbir Ahmed
%

\documentclass[paper=usletter, fontsize=12pt]{article}
%%%%%%%%%%%%%%%%%%%%%%%%%%%%%%%%%%%%%%%%%
% Contract Structural Definitions File Version 1.0 (December 8 2014)
%
% Created by: Vel (vel@latextemplates.com)
% 
% This file has been downloaded from: http://www.LaTeXTemplates.com
%
% License: CC BY-NC-SA 3.0 (http://creativecommons.org/licenses/by-nc-sa/3.0/)
%
%%%%%%%%%%%%%%%%%%%%%%%%%%%%%%%%%%%%%%%%%

\usepackage{geometry} % Required to modify the page layout
\usepackage{multicol}
\usepackage{amsmath}
\usepackage{amssymb}

\usepackage[pdftex]{graphicx}
\usepackage{wrapfig}
\usepackage[font=scriptsize, labelfont=bf]{caption}
\usepackage[utf8]{inputenc} % Required for including letters with accents
\usepackage[T1]{fontenc} % Use 8-bit encoding that has 256 glyphs

\usepackage{avant} % Use the Avantgarde font for headings
\usepackage{xparse}
\usepackage{xcolor}
\usepackage{listings}  % for code verbatim and console outputs

\setlength{\textwidth}{16cm} % Width of the text on the page
\setlength{\textheight}{23cm} % Height of the text on the page
\setlength{\oddsidemargin}{0cm} % Width of the margin - negative to move text left, positive to move it right
\setlength{\topmargin}{-1.25cm} % Reduce the top margin

\setlength{\parindent}{0mm} % Don't indent paragraphs
\setlength{\parskip}{2.5mm} % Whitespace between paragraphs
\renewcommand{\baselinestretch}{1.2}

\renewcommand\familydefault{\sfdefault}  % default font for entire document

\definecolor{green}{rgb}{0.18, 0.55, 0.34}

\graphicspath{ {figures/} }
\captionsetup[table]{skip=10pt}

\lstset{language=C, keywordstyle={\bfseries \color{black}}}

% defines algorithm counter for chapter-level
\newcounter{nalg}[section]

%defines appearance of the algorithm counter
\renewcommand{\thenalg}{\thesection .\arabic{nalg}}

% defines a new caption label as Algorithm x.y
\DeclareCaptionLabelFormat{algocaption}{Algorithm \thenalg}

%defines the algorithm listing environment
\lstnewenvironment{pseudocode}[1][] {
    \refstepcounter{nalg} %increments algorithm number

    \captionsetup{labelformat=algocaption,labelsep=colon}
    \lstset{
        mathescape=true,
        frame=tB,
        numbers=left,
        numberstyle=\tiny,
        basicstyle=\scriptsize,
        keywordstyle=\color{black}\bfseries\em,
        keywords={,input, output, return, datatype, function, in, if, else, foreach, while, begin, end, },
        xleftmargin=.04\textwidth,
        #1
    }
}{}
 % specifies the document layout and style

\begin{document}

    \documentinfo{May 13, 2018}{Final}

    % import commmon sections on Introduction
    %
% CMSC 421: Project 2 Introduction
% Template layout and preamble defined in structure.tex
%
% Author: Sabbir Ahmed
%

\section{Introduction} This project implements a new version of the Linux
kernel that adds functionality to support a simple intrusion detection
system (IDS). This system will operate by logging the system calls made by
a process in the kernel, while analysis and intrusion detection will be
done in user space. This assignment is designed to teach a simple method of
intrusion detection, as well as to reinforce the idea of how user space and
kernel space interact through the use of system calls.

An intrusion detection system is a computer program that attempts to
identify (and thwart) attacks that might be performed on the system by
attackers. There are several time-tested approaches to the development of
an IDS. The project will keep track of the system calls made by a monitored
process and check for abnormalities in the sequences of system calls made.
When an attacker breaks into a process, they will need to make system calls
in order to attempt to access the resources of the system that are under
attack. As the system calls that the attacker will perform will likely be
different than those performed by a process that is not under attack, it
follows that by monitoring both healthy and broken processes, it is
possible to develop a scheme to identify those that might be under attack
for further action to be taken.

\section{Objective} The project will compare sequences of system calls made
by a monitored process to known good sequences.

\subsection{Kernel Space Requirements} The kernel-space program will
instrument the system call dispatcher of the Linux kernel with code that
logs each time a system call is made. Built-in system calls such as
\texttt{ptrace} are not allowed to trace the usage of system calls to
generate the logs for the project.

\subsection{User Space Requirements} The analysis of the logs will be
handled by the user-space program, which may be implemented in any
supported programming language. The user-space process should construct a
bit array for each process under monitoring showing which system calls have
been run in a window of the last $k$ system calls. If a particular system
call is made in the window, the bit for that system call will be set to 1.
The bit arrays will then be measured for their hamming distance with the
example of a healthy system call sequence for a process.


    \section{Design Approach} The final design of the project adopted a much
    simpler approach than what was previously assumed. The methods to achieve
    the requirements became apparent after extensive research.

    \subsection{Kernel Space} The kernel-space program will trace the system
    calls and dump them on logs consisting of the \texttt{pid} and the system
    call numbers of each processes spawned by a program. The log was intended
    to also contain the corresponding timestamps of the execution of the system
    calls, but the kernel-space did not appear to have nanosecond resolution of
    their timestamps. Initially, the kernel-space program intended to create
    the log files in $N$ chunks specified by the user to pass on to the user-
    space program for analysis. This parameter will be removed as an option to
    the user and instead be converted to a constant. The constant $N$ number of
    system calls per log files would provide a smooth communication method with
    the user-space program.

    Initially, \texttt{ptrace} appeared to be the only method of tracing system
    calls. Two different approaches were proposed in the initial design. The
    first approach suggested inserting breakpoints in the implementations for
    each system calls in the kernel. The second and more probable approach
    suggested a stripped down version of \texttt{ptrace} that behaved
    identically. After extensive research, a third approach was considered that
    merged aspects of both of the prior approaches. The conditional jump
    instruction, \texttt{jnz}, was updated to an unconditional jump,
    \texttt{jmp}, inside the system call entry point section in
    \texttt{entry\_64.S}. This update forces all the system calls to save their
    system call number to registers before disabling the interrupt request. The
    system call number is then accessible in the register \texttt{ORIG\_RAX} in
    \texttt{common.c} which can be provided as a macro to external files.

    The kernel-space program will require the user to provide the programs to
    be traced. The program will fork its own copy of the process, and gather
    the process IDs and the system call numbers. The \texttt{proc} filesystem
    will be utilized to provide a convenient interface between the kernels-
    space program and writing log files accessible to the user space. The
    common directory shared by the entire IDS, \texttt{idsdata}, will be used
    to store the log files.

    \subsection{User Space} The user-space program was developed before the
    kernel-space program. The program was tested using dummy data simulating
    log files created by the kernel-space program. An object oriented approach
    was taken to develop the interface with the user and the kernel-space
    logger. The user would be required to initialize the user-space program after the kernel-space program has been initiated to run concurrently. The user-space program would wait for the logger to generate a chunk of the system call logs, and then automatically store the first

\end{document}
