%%%%%%%%%%%%%%%%%%%%%%%%%%%%%%%%%%%%%%%%%
% Template
% LaTeX Template
% Version 1.0 (December 8 2014)
%
% This template has been downloaded from:
% http://www.LaTeXTemplates.com
%
% Original author:
% Brandon Fryslie
% With extensive modifications by:
% Vel (vel@latextemplates.com)
%
% License:
% CC BY-NC-SA 3.0 (http://creativecommons.org/licenses/by-nc-sa/3.0/)
%
% Authors:
% Sabbir Ahmed, Jeffrey Osazuwa, Howard To, Brian Weber
% 
%%%%%%%%%%%%%%%%%%%%%%%%%%%%%%%%%%%%%%%%%

\documentclass[paper=usletter, fontsize=12pt]{article}
%%%%%%%%%%%%%%%%%%%%%%%%%%%%%%%%%%%%%%%%%
% Contract Structural Definitions File Version 1.0 (December 8 2014)
%
% Created by: Vel (vel@latextemplates.com)
% 
% This file has been downloaded from: http://www.LaTeXTemplates.com
%
% License: CC BY-NC-SA 3.0 (http://creativecommons.org/licenses/by-nc-sa/3.0/)
%
%%%%%%%%%%%%%%%%%%%%%%%%%%%%%%%%%%%%%%%%%

\usepackage{geometry} % Required to modify the page layout
\usepackage{multicol}
\usepackage{amsmath}
\usepackage{amssymb}

\usepackage[pdftex]{graphicx}
\usepackage{wrapfig}
\usepackage[font=scriptsize, labelfont=bf]{caption}
\usepackage[utf8]{inputenc} % Required for including letters with accents
\usepackage[T1]{fontenc} % Use 8-bit encoding that has 256 glyphs

\usepackage{avant} % Use the Avantgarde font for headings
\usepackage{courier}
\usepackage{xparse}
\usepackage{xcolor}
\usepackage{listings}  % for code verbatim and console outputs

\setlength{\textwidth}{16cm} % Width of the text on the page
\setlength{\textheight}{23cm} % Height of the text on the page
\setlength{\oddsidemargin}{0cm} % Width of the margin - negative to move text left, positive to move it right
\setlength{\topmargin}{-1.25cm} % Reduce the top margin

\setlength{\parindent}{0mm} % Don't indent paragraphs
\setlength{\parskip}{2.5mm} % Whitespace between paragraphs
\renewcommand{\baselinestretch}{1.5}

\definecolor{green}{rgb}{0.18, 0.55, 0.34}

\graphicspath{ {figures/} }
\captionsetup[table]{skip=10pt}

\lstset{language=C, keywordstyle={\bfseries \color{black}}}

% defines algorithm counter for chapter-level
\newcounter{nalg}[section]

%defines appearance of the algorithm counter
\renewcommand{\thenalg}{\thesection .\arabic{nalg}}

% defines a new caption label as Algorithm x.y
\DeclareCaptionLabelFormat{algocaption}{Algorithm \thenalg}

% defines the algorithm listing environment
\lstnewenvironment{pseudocode}[1][] {
    \refstepcounter{nalg}  % increments algorithm number
    \captionsetup{font=normalsize, labelformat=algocaption, labelsep=colon}
    \lstset{
        breaklines=true,
        mathescape=true,
        numbers=left,
        numberstyle=\scriptsize,
        basicstyle=\footnotesize\ttfamily,
        keywordstyle=\color{black}\bfseries,
        keywords={input, output, return, parallel, function, for, to, in, if,
        else, foreach, while, and, or, new, print},
        xleftmargin=.04\textwidth,
        #1
    }
}{}

\renewcommand{\familydefault}{\sfdefault}  % default font for entire document
 % specifies the document layout and style


% document info command
\newcommand{\documentinfo}[5]{
    \begin{centering}
        \parbox{2in}{
        \begin{spacing}{1}
            \begin{flushleft}
                \begin{tabular}{l l}
                    #1 \\
                    #2 \\
                    #3 \\
                \end{tabular}\\
                \rule{\textwidth}{1pt}
            \end{flushleft}
        \end{spacing}
        }
    \end{centering}
}

\begin{document}


    \documentinfo{Sabbir Ahmed}{\textbf{DATE:} \today}{\textbf{CMSC 411} HW 03}
    \vspace{-0.3in}

    \begin{enumerate}

        \item
        Problems in this exercise refer to the following sequence of instructions:

        \begin{table}[h]
            \centering
            \begin{tabular*}{100pt}{@{\extracolsep{\fill}} llll}
            LW  & \$5, & -16(\$5) & \\
            SW  & \$5, & -16(\$5) & \\
            ADD & \$5, & \$5, & \$5 \\
            \end{tabular*}
        \end{table}

        \begin{enumerate}

            \item \textbf{Question}
            Indicate dependences and their type.

            \textbf{Answer} \\
            RAW: I1 -> I2 (\$5), I1 -> I3 (\$5) \\
            WAR: I2 -> I3 (\$5), I1 -> I3 (\$5) \\
            WAW: I1 -> I3 (\$5) \\

            \item \textbf{Question}
            Assume there is not forwarding in this pipeline processor, indicate hazards and add NOP instructions to eliminate them.

            \textbf{Answer} \\
            With no forwarding, the only hazards are the RAW.

            \begin{table}[h]
                \centering
                \begin{tabular*}{100pt}{@{\extracolsep{\fill}} llll}
                LW & \$5, & -16(\$5) & \\
                NOP & & & \\
                NOP & & & \\
                SW & \$5, & -16(\$5) & \\
                ADD & \$5, & \$5, & \$5 \\
                \end{tabular*}
            \end{table}

            \item \textbf{Question}
            Assume there is full forwarding, indicate hazards and add NOP instructions to eliminate unresolved cases.

            \textbf{Answer} \\
            The only hazard with full forwarding is the RAW: \\


            \begin{table}[h]
                \centering
                \begin{tabular*}{100pt}{@{\extracolsep{\fill}} llll}
                LW & \$5, & -16(\$5) & \\
                NOP & & & \\
                SW & \$5, & -16(\$5) & \\
                ADD & \$5, & \$5, & \$5 \\
                \end{tabular*}
            \end{table}

            The remaining problem in this exercise assumes the following clock cycle times:

            Without forwarding: 200 ps \\
            With full forwarding: 250 ps \\
            With ALU-ALU forwarding only: 220 ps \\

            \item \textbf{Question}
            What is the total execution time of this instruction sequence without forwarding and with full forwarding? What is the speed-up achieved by adding full forwarding to a pipeline that had no forwarding?

            \textbf{Answer} \\
            With full forwarding, the instructions need 4 cycles. With no forwarding, the instructions need 5 cycles.

                \[ \therefore \frac{200\cdot5}{250\cdot4}=1 \]

            \item \textbf{Question}
            What is the total execution time of this instruction sequence with only ALU-ALU forwarding? What is the speed-up over a no-forwarding pipeline?

            \textbf{Answer} \\
            Total time: 220 $\cdot$ 5 = 1100 ps \\
            Speed up: $\frac{1000}{1100}=0.91$

        \end{enumerate}

            \item For this problem, assume that all branches are resolved in ID stage and are perfectly predicted (this eliminates all control hazards). For the following fragment of MIPS code:

            \begin{table}[h]
                \centering
                \begin{tabular*}{200pt}{@{\extracolsep{\fill}} lllll}
                LW & \$5, & -16(\$5) & & \\
                SW & \$4, & -16(\$4) & & \\
                LW & \$3, & -20(\$4) & & \\
                BEQ & \$2, & \$0, & Label & ; assume \$2 $\neq$ \$0 \\
                ADD & \$1, & \$5, & \$4 \\
                Label: & SUB & \$2, & \$1, & \$3 \\
                \end{tabular*}
            \end{table}

            If we only have one memory (for both instruction and data), there is a structural hazard every time we need to fetch an instruction in the same cycle in which another instruction accesses data. To guarantee forward progress, this hazard must always be resolved in favor of the instruction that accesses data.

        \begin{enumerate}

            \item \textbf{Question}
            What is the total execution time of this instruction sequence in the 5-stage pipeline that only has one memory?

            \textbf{Answer} \\
            Labeling the instructions:

            \begin{table}[h]
                \centering
                \begin{tabular*}{200pt}{@{\extracolsep{\fill}} llllll}
                LW & \$5, & -16(\$5) & & & ; I1 \\
                SW & \$4, & -16(\$4) & & & ; I2 \\
                LW & \$3, & -20(\$4) & & & ; I3 \\
                BEQ & \$2, & \$0, & Label & & ; I4 \\
                ADD & \$1, & \$5, & \$4  & & ; I5 \\
                Label: & SUB & \$2, & \$1, & \$3 & \\
                \end{tabular*}
            \end{table}

            \begin{table}[h]
                \caption{12 Cycles Required for 5 Instructions}
                \centering
                \begin{tabular*}{450pt}{@{\extracolsep{\fill}} c|cccccccccccc}

                \textbf{Instruction} & \textbf{t0} & \textbf{t1} & \textbf{t2} & \textbf{t3} & \textbf{t4} & \textbf{t5} & \textbf{t6} & \textbf{t7} & \textbf{t8} & \textbf{t9} & \textbf{t10} & \textbf{t11} \\
                \hline
                I1 & IF & ID & EX & - & - & - & MEM & WB & & & & \\
                I2 & & IF & ID & EX & - & - & - & MEM & WB & & & \\
                I3 & & & IF & ID & EX & - & - & - & MEM & WB & & \\
                I4 & & & & IF & ID & EX & - & - & - & MEM & WB & \\
                I5 & & & & & IF & ID & EX & - & - & - & MEM & WB \\
                \end{tabular*}
            \end{table}

            \item \textbf{Question}
            How can the structural hazard be eliminated by adding NOP to the code? (Please show a modified version of the program with the added NOP instructions)

            \textbf{Answer} \\
            Inserting NOP instructions cannot resolve the structural hazard since NOPs requires access to memory.

        \end{enumerate}

        \item \textbf{Question}
        For following code, assume that the loop index (\$10) is a multiple of 8:

        \begin{table}[h]
            \centering
            \begin{tabular*}{200pt}{@{\extracolsep{\fill}} lllll}
            Loop:   & LW & \$2, & 0(\$10) & \\
                    & SUB & \$4, & \$2, & \$3 \\
                    & SW & \$4, & 0(\$10) & \\
                    & LW & \$5, & 4(\$10) & \\
                    & SUB & \$6, & \$5, & \$3 \\
                    & SW & \$6, & 4(\$10) & \\
                    & ADDI & \$10, & \$10, & 8 \\
                    & BNE & \$10, & \$30, & Loop \\
            \end{tabular*}
        \end{table}
        \newpage

        Schedule this code (reorder the instructions and make any necessary changes) for fast execution on the 5-stage MIPS pipeline. Assume data forwarding and not-taken prediction of conditional branching.

        \textbf{Answer} \\
        Labelling the instructions as the following:

        \begin{table}[h]
            \centering
            \begin{tabular*}{200pt}{@{\extracolsep{\fill}} llllll}
            Loop:   & LW & \$2, & 0(\$10) & & ; I1 \\
                    & SUB & \$4, & \$2, & \$3 & ; I2 \\
                    & SW & \$4, & 0(\$10) & & ; I3 \\
                    & LW & \$5, & 4(\$10) & & ; I4 \\
                    & SUB & \$6, & \$5, & \$3 & ; I5 \\
                    & SW & \$6, & 4(\$10) & & ; I6 \\
                    & ADDI & \$10, & \$10, & 8 & ; I7 \\
                    & BNE & \$10, & \$30, & Loop & ; I8 \\
            \end{tabular*}
        \end{table}

        RAW data dependencies exist in I2 (on I1) and I5 (on I4). Since the branches are assumed not-taken, finding the clock cycles for the current order of instructions:

        \begin{table}[h]
            \caption{13 Cycles Required for 8 Instructions}
            \centering
            \begin{tabular*}{500pt}{@{\extracolsep{\fill}} c|cccccccccccccc}

            \textbf{Instruction} & \textbf{t0} & \textbf{t1} & \textbf{t2} & \textbf{t3} & \textbf{t4} & \textbf{t5} & \textbf{t6} & \textbf{t7} & \textbf{t8} & \textbf{t9} & \textbf{t10} & \textbf{t11} & \textbf{t12} & \textbf{t13} \\
            \hline
            I1 & IF & ID & EX & MEM & WB & & & & & & & & & \\
            I2 & & IF & - & ID & EX & MEM & WB & & & & & & & \\
            I3 & & & & IF & ID & EX & MEM & WB & & & & & & \\
            I4 & & & & & IF & ID & EX & MEM & WB & & & & & \\
            I5 & & & & & & IF & - & ID & EX & MEM & WB & & & \\
            I6 & & & & & & & & IF & ID & EX & MEM & WB & & \\
            I7 & & & & & & & & & IF & ID & EX & MEM & WB & \\
            I8 & & & & & & & & & & IF & ID & EX & MEM & WB \\
            \end{tabular*}
        \end{table}
        \newpage

        Reordering the instructions may decrease the total number of cycles and result in faster execution on the 5-stage MIPS pipeline.

        The LW operations can be grouped together at the beginning, and the SUB at the end of the instructions. The reordered instructions would look similar to the following:

        \begin{table}[h]
            \caption{Rescheduled Instructions}
            \centering
            \begin{tabular*}{200pt}{@{\extracolsep{\fill}} llllll}
            Loop:   & LW & \$2, & 0(\$10) & & ; I1 \\
                    & LW & \$5, & 4(\$10) & & ; I4 \\
                    & ADDI & \$10, & \$10, & 8 & ; I7 \\
                    & SUB & \$4, & \$2, & \$3 & ; I2 \\
                    & SW & \$4, & 0(\$10) & & ; I3 \\
                    & SUB & \$6, & \$5, & \$3 & ; I5 \\
                    & SW & \$6, & 4(\$10) & & ; I6 \\
                    & BNE & \$10, & \$30, & Loop & ; I8 \\
            \end{tabular*}
        \end{table}

        The new schedule of the instructions would now require the following clock cycles:

        \begin{table}[h]
            \caption{11 Cycles Required for 8 Instructions}
            \centering
            \begin{tabular*}{500pt}{@{\extracolsep{\fill}} c|cccccccccccc}

            \textbf{Instruction} & \textbf{t0} & \textbf{t1} & \textbf{t2} & \textbf{t3} & \textbf{t4} & \textbf{t5} & \textbf{t6} & \textbf{t7} & \textbf{t8} & \textbf{t9} & \textbf{t10} & \textbf{t11} \\
            \hline
            I1 & IF & ID & EX & MEM & WB & & & & & & & \\
            I4 & & IF & ID & EX & MEM & WB & & & & & & \\
            I7 & & & IF & ID & EX & MEM & WB & & & & & \\
            I2 & & & & IF & ID & EX & MEM & WB & & & & \\
            I3 & & & & & IF & ID & EX & MEM & WB & & & \\
            I5 & & & & & & IF & ID & EX & MEM & WB & & \\
            I6 & & & & & & & IF & ID & EX & MEM & WB & \\
            I8 & & & & & & & & IF & ID & EX & MEM & WB \\
            \end{tabular*}
        \end{table}

    \end{enumerate}

\end{document}
