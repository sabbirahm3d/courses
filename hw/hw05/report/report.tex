%%%%%%%%%%%%%%%%%%%%%%%%%%%%%%%%%%%%%%%%%
% Template LaTeX Template Version 1.0 (December 8 2014)
%
% This template has been downloaded from: http://www.LaTeXTemplates.com
%
% Original author: Brandon Fryslie With extensive modifications by: Vel
% (vel@latextemplates.com)
%
% License: CC BY-NC-SA 3.0 (http://creativecommons.org/licenses/by-nc-sa/3.0/)
%
% Authors: Sabbir Ahmed
% 
%%%%%%%%%%%%%%%%%%%%%%%%%%%%%%%%%%%%%%%%%

\documentclass[paper=usletter, fontsize=12pt]{article}
%%%%%%%%%%%%%%%%%%%%%%%%%%%%%%%%%%%%%%%%%
% Contract Structural Definitions File Version 1.0 (December 8 2014)
%
% Created by: Vel (vel@latextemplates.com)
% 
% This file has been downloaded from: http://www.LaTeXTemplates.com
%
% License: CC BY-NC-SA 3.0 (http://creativecommons.org/licenses/by-nc-sa/3.0/)
%
%%%%%%%%%%%%%%%%%%%%%%%%%%%%%%%%%%%%%%%%%

\usepackage{geometry} % Required to modify the page layout
\usepackage{multicol}
\usepackage{amsmath}
\usepackage{amssymb}

\usepackage[pdftex]{graphicx}
\usepackage{wrapfig}
\usepackage[font=scriptsize, labelfont=bf]{caption}
\usepackage[utf8]{inputenc} % Required for including letters with accents
\usepackage[T1]{fontenc} % Use 8-bit encoding that has 256 glyphs

\usepackage{avant} % Use the Avantgarde font for headings
\usepackage{xparse}
\usepackage{xcolor}
\usepackage{listings}  % for code verbatim and console outputs

\setlength{\textwidth}{16cm} % Width of the text on the page
\setlength{\textheight}{23cm} % Height of the text on the page
\setlength{\oddsidemargin}{0cm} % Width of the margin - negative to move text left, positive to move it right
\setlength{\topmargin}{-1.25cm} % Reduce the top margin

\setlength{\parindent}{0mm} % Don't indent paragraphs
\setlength{\parskip}{2.5mm} % Whitespace between paragraphs
\renewcommand{\baselinestretch}{1.2}

\renewcommand\familydefault{\sfdefault}  % default font for entire document

\definecolor{green}{rgb}{0.18, 0.55, 0.34}

\graphicspath{ {figures/} }
\captionsetup[table]{skip=10pt}

\lstset{language=C, keywordstyle={\bfseries \color{black}}}

% defines algorithm counter for chapter-level
\newcounter{nalg}[section]

%defines appearance of the algorithm counter
\renewcommand{\thenalg}{\thesection .\arabic{nalg}}

% defines a new caption label as Algorithm x.y
\DeclareCaptionLabelFormat{algocaption}{Algorithm \thenalg}

%defines the algorithm listing environment
\lstnewenvironment{pseudocode}[1][] {
    \refstepcounter{nalg} %increments algorithm number

    \captionsetup{labelformat=algocaption,labelsep=colon}
    \lstset{
        mathescape=true,
        frame=tB,
        numbers=left,
        numberstyle=\tiny,
        basicstyle=\scriptsize,
        keywordstyle=\color{black}\bfseries\em,
        keywords={,input, output, return, datatype, function, in, if, else, foreach, while, begin, end, },
        xleftmargin=.04\textwidth,
        #1
    }
}{}
 % specifies the document layout and style

\begin{document}

    \documentinfo {\textbf{Homework 5: Snake Levels and Obstacles Report}}
    {\today} {Sabbir Ahmed}
    \vspace{-0.1in}

    \section{Background} For this assignment, the previous Snake Game from HW04
    was reimplemented with additional features such as levels and obstacles.

    The game was to conform to the following specifications:

        \begin{itemize}

            \item The obstacles should be drawn in magenta, but otherwise are
            like the fence and the game should stop once the snake (head)
            overlays an obstacle.

            \item The game field should be initiated with no obstacles in the
            field of play for Level 0, and proceed just as in HW4 until the 5th
            apple is eaten.

            \item Every time the player eats 5 apples within a level, a new
            level should be generated with 10 more obstacles than the previous
            level and play should restart on that level.

                \begin{itemize}

                    \item Level 0 has no obstacles, Level 1 has ten obstacles,
                    Level 2 has twenty obstacles, and so on...

                    \item The snake size is reinitialized to 1 to start each
                    level

                \end{itemize}

            \item The level score, and the level should be displayed on the LCD
            display. The display should be as follows:

                \[ \texttt{Lyy} \]

                with (yy) denoting the current level.

            \item When the game ends, append an E as follows:

                \[ \texttt{LyyE} \]

        \end{itemize}

    \section{Design Approach} Several discrete modules from the previous
    implementation were used in this version. The \texttt{direction},
    \texttt{food\_pos} and \texttt{pacemaker} modules were left unmodified. The
    \texttt{snake\_pos} module was modified to truncate the size of the snake
    from 32 to 5. The \texttt{vga\_layout} module was modified to include the
    new coordinate pairs of the obstacles. Additional submodules:
    \texttt{game\_state} and \texttt{lcd\_driver} were integrated into the
    design.
 
    These submodules were connected using a top level module that may be
    visualized with the schematic diagram configured as a block diagram in
    Figure \ref{fig:schematic}. All the modules implicitly accept clock cycles
    as inputs.

    \begin{figure}[ht]
        \begin{center}
            \includegraphics[width=1\textwidth]{top_level_design.png}
            \caption{Block Diagram of the Implementation of the Game}
            \label{fig:schematic}
        \end{center}
    \end{figure}
    \newpage

        \subsection{direction} The \texttt{direction} module is used to control
        the user inputs. The inputs are one-shotted, debounced and fed into the
        internal state machine to determine the direction the user intended.
        This module sets an enable to the \texttt{snake\_pos} module to notify
        a change in direction.

        % \begin{figure}[ht]
        %     \begin{center}
        %         \includegraphics[width=1\textwidth]{direction_wav.png}
        %         \caption{Waveform of the Testbench of \texttt{direction}}
        %         \label{fig:direction_wav}
        %     \end{center}
        % \end{figure}

        The sample output demonstrates the \texttt{dir} signal incrementing
        when \texttt{east} is high and decrementing when \texttt{west} is high.

        \subsection{food\_pos} This module generates the \texttt{food\_x} and
        \texttt{food\_y} coordinates of the food when enabled by the
        \texttt{collision} module. The module combinedly utilizes an internal
        counter and a linear feedback shift register to generate the pseudo-
        random coordinates.

        % \begin{figure}[ht]
        %     \begin{center}
        %         \includegraphics[width=1\textwidth]{food_pos_wav.png}
        %         \caption{Waveform of the Testbench of \texttt{food\_pos}}
        %         \label{fig:food_pos_wav}
        %     \end{center}
        % \end{figure}

        The sample output demonstrates the seemingly random coordinates
        generated for \texttt{food\_x} and \texttt{food\_y}.

        \subsection{snake\_pos} The \texttt{snake\_pos} module generates the
        coordinates for the 32 segments of the snake body, including its head.
        The module takes in the 2 bit direction from \texttt{direction} and 2
        enable control signals from \texttt{collision}. The control signals,
        \texttt{grow} and \texttt{dead} are used to indicate the state of the
        snake body. If \texttt{grow} is enabled, the module utilizes
        \texttt{dir} to shift the body segments. If \texttt{dead} is enabled,
        the body segments freeze to indicate end of the game. Its clock is
        timed by \texttt{pacemaker} to control the speed of the moving snake
        body.

        % \begin{figure}[ht]
        %     \begin{center}
        %         \includegraphics[width=1\textwidth]{snake_pos_wav.png}
        %         \caption{Waveform of the Testbench of \texttt{snake\_pos}}
        %         \label{fig:snake_pos_wav}
        %     \end{center}
        % \end{figure}

        The signals were reorganized to highlight the relevant waveforms. The
        sample output demonstrates the movement of the individual body segments
        by utilizing the internal shift register. Only the initial segments
        change due to the snake body's growth being restricted to a length of 2
        in the test bench.

        \subsection{collision} \texttt{collision} accepts the coordinates of
        the food and the snake segments and determines if a collision has been
        detected. If a collision has not been detected, it sends out an enable
        signal to the \texttt{snake\_pos} module. If a collision with the snake
        body, specifically the snake head, with the food is detected, the
        module sends a signal to \texttt{pacemaker} to determine the interval
        at which the snake should move. This module has an internal counter
        that speeds up when the snake head had made 32 collisions with the
        food. If a collision between the snake head and the fence is detected,
        the game is frozen.

        % \begin{figure}[ht]
        %     \begin{center}
        %         \includegraphics[width=1\textwidth]{collision_wav.png}
        %         \caption{Waveform of the Testbench of \texttt{collision}}
        %         \label{fig:collision_wav}
        %     \end{center}
        % \end{figure}

        The signals were reorganized to highlight the relevant waveforms. The
        sample output indicates the conditions to which a collision is
        classified as \texttt{grow} and \texttt{dead}.

        \subsection{vga\_layout} This module draws the fence of the game, and
        the snake and the randomly placed food on the VGA display.

        \subsection{Other Modules} Other minor modules have been utilized for
        the implementation. \texttt{pacemaker} is used to send out control
        signals to the other modules such that they update at reasonable rates.
        The module consists of an internal counter that speeds up the update
        rate once the game has registered over 31 bites of the food.
        \texttt{vga\_sync} is used to synchronize the outputs to the VGA
        display.

\end{document}
