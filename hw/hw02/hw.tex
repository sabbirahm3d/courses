%%%%%%%%%%%%%%%%%%%%%%%%%%%%%%%%%%%%%%%%%
% Template
% LaTeX Template
% Version 1.0 (December 8 2014)
%
% This template has been downloaded from:
% http://www.LaTeXTemplates.com
%
% Original author:
% Brandon Fryslie
% With extensive modifications by:
% Vel (vel@latextemplates.com)
%
% License:
% CC BY-NC-SA 3.0 (http://creativecommons.org/licenses/by-nc-sa/3.0/)
%
% Authors:
% Sabbir Ahmed
%
%%%%%%%%%%%%%%%%%%%%%%%%%%%%%%%%%%%%%%%%%

\documentclass[paper=usletter, fontsize=12pt]{article}
%%%%%%%%%%%%%%%%%%%%%%%%%%%%%%%%%%%%%%%%%
% Contract Structural Definitions File Version 1.0 (December 8 2014)
%
% Created by: Vel (vel@latextemplates.com)
% 
% This file has been downloaded from: http://www.LaTeXTemplates.com
%
% License: CC BY-NC-SA 3.0 (http://creativecommons.org/licenses/by-nc-sa/3.0/)
%
%%%%%%%%%%%%%%%%%%%%%%%%%%%%%%%%%%%%%%%%%

\usepackage{geometry} % Required to modify the page layout
\usepackage{multicol}
\usepackage{amsmath}
\usepackage{amssymb}

\usepackage[pdftex]{graphicx}
\usepackage{wrapfig}
\usepackage[font=scriptsize, labelfont=bf]{caption}
\usepackage[utf8]{inputenc} % Required for including letters with accents
\usepackage[T1]{fontenc} % Use 8-bit encoding that has 256 glyphs

\usepackage{avant} % Use the Avantgarde font for headings
\usepackage{courier}
\usepackage{xparse}
\usepackage{xcolor}
\usepackage{listings}  % for code verbatim and console outputs

\setlength{\textwidth}{16cm} % Width of the text on the page
\setlength{\textheight}{23cm} % Height of the text on the page
\setlength{\oddsidemargin}{0cm} % Width of the margin - negative to move text left, positive to move it right
\setlength{\topmargin}{-1.25cm} % Reduce the top margin

\setlength{\parindent}{0mm} % Don't indent paragraphs
\setlength{\parskip}{2.5mm} % Whitespace between paragraphs
\renewcommand{\baselinestretch}{1.5}

\definecolor{green}{rgb}{0.18, 0.55, 0.34}

\graphicspath{ {figures/} }
\captionsetup[table]{skip=10pt}

\lstset{language=C, keywordstyle={\bfseries \color{black}}}

% defines algorithm counter for chapter-level
\newcounter{nalg}[section]

%defines appearance of the algorithm counter
\renewcommand{\thenalg}{\thesection .\arabic{nalg}}

% defines a new caption label as Algorithm x.y
\DeclareCaptionLabelFormat{algocaption}{Algorithm \thenalg}

% defines the algorithm listing environment
\lstnewenvironment{pseudocode}[1][] {
    \refstepcounter{nalg}  % increments algorithm number
    \captionsetup{font=normalsize, labelformat=algocaption, labelsep=colon}
    \lstset{
        breaklines=true,
        mathescape=true,
        numbers=left,
        numberstyle=\scriptsize,
        basicstyle=\footnotesize\ttfamily,
        keywordstyle=\color{black}\bfseries,
        keywords={input, output, return, parallel, function, for, to, in, if,
        else, foreach, while, and, or, new, print},
        xleftmargin=.04\textwidth,
        #1
    }
}{}

\renewcommand{\familydefault}{\sfdefault}  % default font for entire document
 % specifies the document layout and style

%------------------------------------------------------------------------------
% document info command
\newcommand{\documentinfo}[5]{
    \begin{centering}
        \parbox{2in}{
        \begin{spacing}{1}
            \begin{flushleft}
                \begin{tabular}{l l}
                    #1 \\
                    #2 \\
                    #3 \\
                \end{tabular}\\
                \rule{\textwidth}{1pt}
            \end{flushleft}
        \end{spacing}
        }
    \end{centering}
}

\begin{document}


    \documentinfo{Sabbir Ahmed}{\textbf{DATE:} \today}{\textbf{MATH 407} HW 02}
    \vspace{-0.2in}

    \begin{itemize}

        \item[\textbf{A.1}] Let A, B, C be subsets of a given set S. Prove the
        following statements.

            \begin{itemize}

                \item[\textbf{10}] $(A - B) \cup (B - A) = (A \cup B) - (A \cap
                B)$
                \item[\textbf{Ans}]
                \begin{proof}[\unskip\nopunct]
                    Let $x \in (A - B) \cup (B - A)$. \\
                    Therefore, $x \in (A - B)$ or $x \in (B - A)$.\\

                    If $x \in (A - B)$, then $x \in A, x \not\in B$.\\
                    Therefore:
                    \begin{align*}
                        x \in (A \cup B), \ x \not\in (A \cap B) \\
                        \Rightarrow x \in (A \cup B) - (A \cap B)
                    \end{align*}
                    Similarly, if $x \in (B - A)$, then $x \in (A \cup B) - (A
                    \cap B)$. \\ Therefore, $(A - B) \cup (B - A) \subseteq (A
                    \cup B) - (A \cap B)$ \\

                    Conversely, if $x \in (A \cup B) - (A \cap B)$, then $x \in
                    (A \cup B), \ x \not\in (A \cap B)$ \\ If $x \in (A \cup
                    B)$, then $x \in A$ or $x \in B$. \\ If $x \in A$, then $x
                    \not\in B$. \\ Therefore:
                    \begin{align*}
                        x \in A, \ x \not\in B
                        \Rightarrow x \in (A - B) \\
                        \Rightarrow x \in (A - B) \cup (B - A)
                    \end{align*}
                    Similary, if $x \in B$, then $x \in (A - B) \cup (B - A)$.
                    \\ Therefore, $(A \cup B) - (A \cap B) \subseteq (A - B)
                    \cup (B - A)$ \\

                    $\therefore (A - B) \cup (B - A) = (A \cup B) - (A \cap B)$
                    \qedhere
                \end{proof}
                \vspace{0.2in}

                \item[\textbf{11}] $(A \cup B) \times C = (A \times C) \cup (B
                \times C)$
                \item[\textbf{Ans}]
                \begin{proof}[\unskip\nopunct]
                    Let $(x,y) \in (A \cup B) \times C$. \\
                    Then:
                    \begin{align*}
                        & x \in (A \cup B), \ y \in C \\
                        & \Rightarrow (x \in A, \ y \in C) \text{ \ or \ } (x
                        \in B, \ y \in C) \\
                        & \Leftrightarrow ((x, y) \in A \times C) \text{ \ or \
                        } ((x, y) \in B \times C) \\
                        & \Leftrightarrow (x, y) \in (A \times C) \cup (B
                        \times C)
                    \end{align*}

                $\therefore (A \cup B) \times C = (A \times C) \cup (B \times
                C)$ \qedhere
                \end{proof}
                \vspace{0.2in}

            \end{itemize}

        \item[\textbf{A.4}]

        \begin{itemize}

            \item[\textbf{9}] Let $a_1, \ldots, a_n$ be positive real numbers,
            $G_n = \sqrt[n]{a_{1}a_{2}\ldots a_{n}}$, and $A_n =
            \frac{1}{n}\sum_{i=1}^{n}a_i$. Then $G_n$ is called the geometric
            mean and $A_n$ is called the arithmetic mean. We wish to show that
            $G_n \le A_n$.

            \begin{enumerate}

                \item Show that $G_2 \le A_2$.
                \item[\textbf{Ans}]
                \begin{proof}[\unskip\nopunct]
                    Substituting $n = 2$:
                    \vspace{-0.1in}
                    \begin{align*}
                        G_2 & = \sqrt[2]{a_{1}a_{2}\ldots a_{2}} \\
                        & = \sqrt[2]{a_{1}a_{2}} \\
                        & = \sqrt{a_{1}a_{2}} \\
                        A_2 & = \frac{1}{2}\sum_{i=1}^{2}a_i \\
                        & = \frac{1}{2}(a_1 + a_2)
                    \end{align*}

                    If $a_1 = a_2$, then $G_2 = a_1 = A_2 = a_1$. \\

                    If $a_1 \neq a_2$, then let $a_1 = k, a_2 = nk+ 1$, $k
                    \in \mathbb{R^{+}}$. \\ Then, $G_2 = \sqrt{(k)(k+1)}$,
                    $G_2^2 = k^2+k$ \\ and $A_2 = \frac{1}{2}(2k + 1)$, $A_2^2
                    = k^2 + k + \frac{1}{4}$ \\

                    $\therefore G_2 \le A_2$ \qedhere
                \end{proof}
                \vspace{0.2in}

                \item Show that $G_{2^n} \le A_{2^n}$ by using induction on
                $n$.
                \item[\textbf{Ans}]
                \begin{proof}[\unskip\nopunct]
                    It was proven the proposition held for $n = 2^k \text{ for
                    \ } k = 1 \Rightarrow 2$. \\
                    Suppose the proposition holds for $n = 2^k, \text{ for \ }
                    k > 1$. Therefore:
                    \begin{align*}
                        A_{2^k} & = \frac{1}{2^k}\sum_{i=1}^{2^k}a_i \\
                        & = \frac{1}{2^k}(a_1 + a_2 + \ldots + a_{2^k}) \\
                        & = \frac{\frac{1}{2^{k-1}}(a_1 + a_2 + \ldots +
                        a_{2^{k-1}}) + \frac{1}{2^{k-1}}(a_{2^{k-1}+1} +
                        a_{2^{k-1}+2} + \ldots + a_{2^k})}{2} \\
                        & \ge \frac{\sqrt[2^{k-1}]{a_1 + a_2 + \ldots +
                        a_{2^{k-1}}} + \sqrt[2^{k-1}]{a_{2^{k-1}+1} +
                        a_{2^{k-1}+2} + \ldots + a_{2^k}}}{2} \\
                        & \ge \sqrt{\sqrt[2^{k-1}]{a_1 + a_2 + \ldots +
                        a_{2^{k-1}}} + \sqrt[2^{k-1}]{a_{2^{k-1}+1} +
                        a_{2^{k-1}+2} + \ldots + a_{2^k}}} \\
                        & \ge \sqrt[2^k]{a_{1}a_{2}\ldots a_{2^k}} \\
                        & = G_{2^k}
                    \end{align*}
                    $\therefore G_{2^k} \le A_{2^k}$ \qedhere
                \end{proof}
                \vspace{0.2in}

            \end{enumerate}

            \item[\textbf{10}] Let $a$ and $b$ be real numbers. Prove the
            binomial theorem , which states that

                \[ (a+b)^n = \sum_{i=0}^{n} \binom{n}{i}a^{i}b^{n-i} \text{\ \
                where \ } \binom{n}{i}=\frac{n!}{i!(n-i)!} \]

            and $n!$ = $n(n-1)\ldots 2 \cdot 1$ for $n \ge 1$ and $0! = 1$.

            \textit{Hint:} $\binom{m+1}{k}=\binom{m}{k}+\binom{m}{k-1}$.
            \item[\textbf{Ans}]
            \begin{proof}[\unskip\nopunct]
                Base case: For $n = k = 0$:
                \begin{align*}
                    (a + b)^0 & = \sum_{i=0}^{0} \binom{n}{i}a^{i}b^{n-i} \\
                    & = \binom{0}{0}a^{0}b^{0} \\
                    & = 1
                \end{align*}
                Assume the proposition holds for $n = k + 1$. Then:
                \begin{align*}
                    (a + b)^{k+1} & = (a + b)(a + b)^{k} \\
                    & = (a + b)\sum_{i=0}^{k} \binom{k}{i}a^{i}b^{k-i} \\
                    & = a\sum_{i=0}^{k} \binom{k}{i}a^{i}b^{k-i} +
                    b\sum_{i=0}^{k} \binom{k}{i}a^{i}b^{k-i} \\
                    & = \sum_{i=0}^{k} \binom{k}{i}a^{i+1}b^{k-i} +
                    \sum_{i=0}^{k} \binom{k}{i}a^{i}b^{k+1-i} \\
                    & = \sum_{i=1}^{k+1} \binom{k}{i-1}a^{(i-1)+1}b^{k-(i-1)} +
                    \sum_{i=0}^{k} \binom{k}{i}a^{i}b^{k+1-i} \\
                    & = \sum_{i=1}^{k+1} \binom{k}{i-1}a^{i}b^{k+1-i} +
                    \sum_{i=0}^{k} \binom{k}{i}a^{i}b^{k+1-i} \\
                    & = \binom{0}{0}a^{0}b^{k+1} + \sum_{i=1}^{k+1}
                    \binom{k}{i-1}a^{i}b^{k+1-i} + \binom{0}{0}a^{k+1}b^{0} +
                    \sum_{i=1}^{k} \binom{k}{i}a^{i}b^{k+1-i} \\
                    & = \binom{0}{0}a^{0}b^{k+1} + \binom{k+1}{k+1}a^{k+1}b^{0}
                    + \sum_{i=1}^{k}\bigg(\binom{k}{i} +
                      \binom{k}{i-1}\bigg)a^{i}b^{k+1-i} \\
                    & = b^{k+1} + a^{k+1} + \sum_{i=1}^{k}
                    \binom{k+1}{i}a^{i}b^{k+1-i} \\
                    \therefore (a + b)^{k+1} & =
                    \sum_{i=0}^{k+1}\binom{k+1}{i}a^{i}b^{k+1-i} \qedhere
                \end{align*}
            \end{proof}
            \vspace{0.2in}

            \item[\textbf{11}] Find a formula for the derivative of the product
            of $n$ functions, and give a detailed proof by induction (assuming
            the product rule for the derivative of two functions).
            \item[\textbf{Ans}]
            \begin{proof}[\unskip\nopunct]
                Let $h = \prod_{i=1}^{n} f_i$ be the product of $n$ functions.
                \\ We need to show:
                \begin{equation*}
                    h^\prime = f_1^\prime(f_2 \cdot f_3 \cdot \ldots \cdot f_n)
                    + f_2^\prime(f_1 \cdot f_3 \cdot \ldots \cdot f_n) + \ldots
                    + f_n^\prime(f_1 \cdot f_2 \cdot f_3 \cdot \ldots \cdot
                      f_{n-1}) \text{ \ (product rule) \ }
                \end{equation*}
                Base case: For $n = 2$, $h = \prod_{i=1}^{2} f_i$
                \begin{equation*}
                    h^\prime = f_1^\prime \cdot f_2 + f_2^\prime \cdot f_1
                    \text{ \ (product rule) \ }
                \end{equation*}
                Suppose the proposition holds for $n = k \ge 2$, with $h = f_1
                \cdot f_2 \cdot f_3 \cdot \ldots \cdot f_{k+1}$. Then:
                \begin{align*}
                    h^\prime & = (f_1 \cdot f_2 \cdot f_3 \cdot \ldots \cdot
                    f_{k})^\prime f_{k+1} + f_{k+1}^\prime(f_1 \cdot f_2 \cdot
                    f_3 \cdot \ldots \cdot f_{k}) \text{ \ (product rule) \ }
                    \\
                    & = \!\begin{multlined}[t][10.5cm]
                        (f_1^\prime(f_2 \cdot f_3 \cdot \ldots \cdot f_k) +
                        f_2^\prime(f_1 \cdot f_3 \cdot \ldots \cdot f_k) +
                        \ldots \\ + f_k^\prime(f_1 \cdot f_2 \cdot f_3 \cdot
                        \ldots \cdot f_{k-1})) f_{k+1} + f_{k+1}^\prime(f_1
                        \cdot f_2 \cdot f_3 \cdot \ldots \cdot f_{k})
                    \end{multlined} \\
                    \therefore h^\prime & = f_1^\prime(f_2 \cdot f_3 \cdot
                    \ldots \cdot f_{k+1}) + f_2^\prime(f_1 \cdot f_3 \cdot
                      \ldots \cdot f_{k+1}) + \ldots + f_{k+1}^\prime(f_1 \cdot
                      f_2 \cdot f_3 \cdot \ldots \cdot f_{k}) \qedhere
                \end{align*}
            \end{proof}
            \vspace{0.2in}

            \item[\textbf{12}] Find a formula for the $n$th derivative of the
            product of two functions, and give a detailed proof by induction.
            \item[\textbf{Ans}]
            \begin{proof}[\unskip\nopunct]
                Let $f, g$ be two functions. \\ Using the general Leibniz rule,
                the $n$th derivative of a product of two functions is given by:
                \begin{equation*}
                    (fg)^{(n)} = \sum_{i=0}^{n} \binom{n}{i}f^{(n-i)}g^{(i)}
                \end{equation*}
                Base case: For $n = k = 0$,
                \begin{align*}
                    (fg)^{(0)} & = fg \\
                    & = \binom{0}{0}f^{(0)}g^{(0)}
                \end{align*}
                Assume the proposition holds for $n = k + 1$, such that
                $(fg)^{(k+1)} = \big((fg)^{(k)}\big)^\prime$. \\ Then:
                \begin{align*}
                    (fg)^{(k+1)} & = \big((fg)^{(k)}\big)^\prime \\
                    & = \Bigg(\sum_{i=0}^{k} \binom{k}{i}f^{(k-i)}g^{(i)}\Bigg)^\prime \\
                    & = \sum_{i=0}^{k} \binom{k}{i}\big(f^{(k-i)}g^{(i)}\big)^\prime \\
                    & = \sum_{i=0}^{k} \binom{k}{i}\big(f^{(k+1-i)}g^{(i)} + f^{(k-i)}g^{(i+1)}\big) \\
                    & = \sum_{i=0}^{k} \binom{k}{i}f^{(k+1-i)}g^{(i)} + \sum_{i=0}^{k} \binom{k}{i}f^{(k-i)}g^{(i+1)} \\
                    & = \sum_{i=0}^{k} \binom{k}{i}f^{(k+1-i)}g^{(i)} + \sum_{i=1}^{k+1} \binom{k}{i-1}f^{(k-(i-1))}g^{((i-1)+1)} \\
                    & = \sum_{i=0}^{k} \binom{k}{i}f^{(k+1-i)}g^{(i)} + \sum_{i=1}^{k+1} \binom{k}{i-1}f^{(k+1-i)}g^{(i)} \\
                    & = \binom{0}{0}f^{(k+1)}g^{(0)} + \sum_{i=1}^{k} \binom{k}{i}f^{(k+1-i)}g^{(i)} + \binom{k}{k}f^{(0)}g^{(k+1)} + \sum_{i=1}^{k} \binom{k}{i-1}f^{(k+1-i)}g^{(i)} \\
                    & = \binom{0}{0}f^{(k+1)}g^{(0)} + \binom{k+1}{k+1}f^{(0)}g^{(k+1)} + \sum_{i=1}^{k} \bigg(\binom{k}{i} + \binom{k}{i-1}\bigg)f^{(k+1-i)}g^{(i)} \\
                    & = f^{(k+1)} + g^{(k+1)} + \sum_{i=1}^{k} \bigg(\binom{k}{i} + \binom{k}{i-1}\bigg)f^{(k+1-i)}g^{(i)}
                \end{align*}
                Since $\binom{m+1}{k}=\binom{m}{k}+\binom{m}{k-1}$,
                \begin{align*}
                    (fg)^{(k+1)} & = f^{(k+1)} + g^{(k+1)} + \sum_{i=1}^{k} \bigg(\binom{k+1}{i} + \binom{k}{i-1}\bigg)f^{(k+1-i)}g^{(i)} \\
                    \therefore (fg)^{(k+1)} & = \sum_{i=0}^{k+1} \binom{k+1}{i}f^{(k+1-i)}g^{(i)}
                    \qedhere
                \end{align*}
            \end{proof}
            \vspace{0.2in}

        \end{itemize}

    \end{itemize}

\end{document}
