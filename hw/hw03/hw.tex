%%%%%%%%%%%%%%%%%%%%%%%%%%%%%%%%%%%%%%%%%
% Template
% LaTeX Template
% Version 1.0 (December 8 2014)
%
% This template has been downloaded from:
% http://www.LaTeXTemplates.com
%
% Original author:
% Brandon Fryslie
% With extensive modifications by:
% Vel (vel@latextemplates.com)
%
% License:
% CC BY-NC-SA 3.0 (http://creativecommons.org/licenses/by-nc-sa/3.0/)
%
% Authors:
% Sabbir Ahmed
%
%%%%%%%%%%%%%%%%%%%%%%%%%%%%%%%%%%%%%%%%%

\documentclass[paper=usletter, fontsize=12pt]{article}
%%%%%%%%%%%%%%%%%%%%%%%%%%%%%%%%%%%%%%%%%
% Contract Structural Definitions File Version 1.0 (December 8 2014)
%
% Created by: Vel (vel@latextemplates.com)
% 
% This file has been downloaded from: http://www.LaTeXTemplates.com
%
% License: CC BY-NC-SA 3.0 (http://creativecommons.org/licenses/by-nc-sa/3.0/)
%
%%%%%%%%%%%%%%%%%%%%%%%%%%%%%%%%%%%%%%%%%

\usepackage{geometry} % Required to modify the page layout
\usepackage{multicol}
\usepackage{amsmath}
\usepackage{amssymb}

\usepackage[pdftex]{graphicx}
\usepackage{wrapfig}
\usepackage[font=scriptsize, labelfont=bf]{caption}
\usepackage[utf8]{inputenc} % Required for including letters with accents
\usepackage[T1]{fontenc} % Use 8-bit encoding that has 256 glyphs

\usepackage{avant} % Use the Avantgarde font for headings
\usepackage{courier}
\usepackage{xparse}
\usepackage{xcolor}
\usepackage{listings}  % for code verbatim and console outputs

\setlength{\textwidth}{16cm} % Width of the text on the page
\setlength{\textheight}{23cm} % Height of the text on the page
\setlength{\oddsidemargin}{0cm} % Width of the margin - negative to move text left, positive to move it right
\setlength{\topmargin}{-1.25cm} % Reduce the top margin

\setlength{\parindent}{0mm} % Don't indent paragraphs
\setlength{\parskip}{2.5mm} % Whitespace between paragraphs
\renewcommand{\baselinestretch}{1.5}

\definecolor{green}{rgb}{0.18, 0.55, 0.34}

\graphicspath{ {figures/} }
\captionsetup[table]{skip=10pt}

\lstset{language=C, keywordstyle={\bfseries \color{black}}}

% defines algorithm counter for chapter-level
\newcounter{nalg}[section]

%defines appearance of the algorithm counter
\renewcommand{\thenalg}{\thesection .\arabic{nalg}}

% defines a new caption label as Algorithm x.y
\DeclareCaptionLabelFormat{algocaption}{Algorithm \thenalg}

% defines the algorithm listing environment
\lstnewenvironment{pseudocode}[1][] {
    \refstepcounter{nalg}  % increments algorithm number
    \captionsetup{font=normalsize, labelformat=algocaption, labelsep=colon}
    \lstset{
        breaklines=true,
        mathescape=true,
        numbers=left,
        numberstyle=\scriptsize,
        basicstyle=\footnotesize\ttfamily,
        keywordstyle=\color{black}\bfseries,
        keywords={input, output, return, parallel, function, for, to, in, if,
        else, foreach, while, and, or, new, print},
        xleftmargin=.04\textwidth,
        #1
    }
}{}

\renewcommand{\familydefault}{\sfdefault}  % default font for entire document
 % specifies the document layout and style
\allowdisplaybreaks

%------------------------------------------------------------------------------
% document info command
\newcommand{\documentinfo}[5]{
    \begin{centering}
        \parbox{2in}{
        \begin{spacing}{1}
            \begin{flushleft}
                \begin{tabular}{l l}
                    #1 \\
                    #2 \\
                    #3 \\
                \end{tabular}\\
                \rule{\textwidth}{1pt}
            \end{flushleft}
        \end{spacing}
        }
    \end{centering}
}

\newcommand{\ans}{\textbf{Answer} \ }

\begin{document}

    \documentinfo{Sabbir Ahmed}{\textbf{DATE:} February 26, 2018}{\textbf{CMPE 320} HW 03}
    \vspace{-0.2in}

    \begin{enumerate}

        % 1
        \item
        \begin{proof}[\unskip\nopunct]
            With 3 $n$-sided rolls, there are $n^3$ possibilities. \\
            The probability that either of the pair of persons roll the same
            face of the die is therefore $n/n^3=1/n^2$. \\
            Therefore
            \begin{align*}
                P(A_{12}) = P(A_{13}) = P(A_{23}) & = \frac{n}{n^3} \\
                & = \frac{1}{n^2}
            \end{align*}
            But if both the events $A_{12}$ and $A_{13}$ takes place, that is
            both persons $1$ and $2$ and persons $1$ and $3$ roll the same
            face, then that yields $A_{23}$. \\
            That is, if both persons $1$ and $2$ and persons $1$ and $3$ roll
            the same face, then that implies persons $1$ and $3$ rolled the
            same face.\\
            But the outcome of person $3$'s roll is not dependent on the other
            persons.\\
            That is, pairwise $A_{12}$ and $A_{13}$, $A_{12}$ and $A_{23}$, and
            $A_{13}$ and $A_{23}$ are independent.\\
            But if considered individually, they are dependent. \qedhere
        \end{proof}
        \vspace{0.2in}

        % 2
        \item
        \begin{proof}[\unskip\nopunct]

            Consider the following counter-example with two independent tosses
            of a fair coin.\\
            Let events $B=\{HT,HH\}$ and $C=\{HT,TT\}$ represent tosses where
            they landed heads and tails respectively.\\
            Let $A=\{HT,TH\}$ be the event that exactly one toss resulted in
            heads. \\
            Then,
            \begin{equation*}
                P(A)=P(B)=P(C)=\frac{1}{2}
            \end{equation*}
            And,
            \begin{equation*}
                P(A \cup B)=P(A \cup C)=\frac{1}{4}
            \end{equation*}
            Therefore, $A$ and $B$ and $A$ and $C$ are both independent
            events.\\
            Therefore, $B$ and $C$ are also independent events.\\
            However,
            \begin{equation*}
                P(A \cap (B \cup C))=\frac{1}{4} \ne P(A)P(B \cup C) =
                \frac{1}{2}\cdot \frac{3}{4}
            \end{equation*}
            Therefore, $A$ and $B \cup C$ are dependent \qedhere

        \end{proof}
        \vspace{0.2in}

        % 3
        \item

            \begin{enumerate}

                \item
                \begin{proof}[\unskip\nopunct]
                    $A_1A_2A_4A_6$ and $A_1A_3A_5A_6$ \qedhere
                \end{proof}
                \vspace{0.2in}

                \item
                \begin{proof}[\unskip\nopunct]

                    Given,\\
                    Probability that $a_1$ is closed $=p$\\
                    Probability that $a_2$ and $a_4$ are closed $= p \cdot p = p^2$\\
                    Probability that $a_3$ and $a_5$ are closed $= p \cdot p = p^2$\\
                    Therefore, the probability that at least one closed path,
                    \begin{align*}
                        A_2A_4A_3A_5 & = 1-P(\text{neither paths are closed})\\
                        & = 1-(1-p^2)(1-p^2)\\
                        & = 1-(1-p^2)^2 \\
                        & = p^2(1-(1-p^2)^2) \qedhere
                    \end{align*}

                \end{proof}
                \vspace{0.2in}

            \end{enumerate}

        % 4
        \item
        \begin{proof}[\unskip\nopunct]
            Let $p_5$ denote the longer path of 5 links from $A$ to $B$,\\
            and $p_3$ denote the shorter path of 3 links. \\

            Given, the probability of links failing independently is $q$.\\
            Therefore, the probability of links not failing is $1-q$. \\

            For a successful transmission, all of the links have to not fail.\\
            Therefore, for path $p_5$ the probability is $P(p_5)=(1-q)^5$\\
            and for path $p_3$ the probability is $P(p_3) = (1-q)^3$. \\
            Since the paths are independent of each other,
            \begin{align*}
                P(p_5 \cap p_3) & = P(p_5)\cdot P(p_3) \\
                & = (1-q)^5 \cdot (1-q)^3\\
                & =(1-q)^8
            \end{align*}
            That is, the probability of both the paths not failing is
            $(1-q)^8$. \\
            Therefore, the probability of either the paths not failing for a
            successful transmission from terminal $A$ to $B$ is:
            \begin{align*}
                P(p_5 \cup p_3) & = P(p_5) + P(p_3) - P(p_5 \cap p_3) \\
                & = (1-q)^5 + (1-q)^3 - (1-q)^8 \qedhere
            \end{align*}
        \end{proof}
        \vspace{0.2in}

        % 5
        \item
        \begin{proof}[\unskip\nopunct]
            Possible combinations for the sum of the two rolls to be 7: \\
            \begin{equation*}
                \{\{1,6\}, \{2,5\}, \{3,4\}, \{4,3\}, \{5,2\}, \{6,1\}\}
            \end{equation*}
            6 combinations, therefore $P(\text{sum = 7})=6/36=1/6$.\\
            For 100 repetitions:
            \begin{align*}
                P(\text{sum = 7 10 times}) & = \binom{100}{10}\bigg(\frac{1}{6}\bigg)^{10}\bigg(1-\frac{1}{6}\bigg)^{100-10}\\
                & \approx 0.021 \qedhere
            \end{align*}
        \end{proof}
        \vspace{0.2in}

        % 6
        \item Since the probability of destroying one BM by both the AMMs is\\
        $1-P(\text{neither of the AMMs destroy}) = 1-0.2*0.2=0.96$
            \begin{enumerate}

                \item
                \begin{proof}[\unskip\nopunct]
                    With 6 BMs:
                    \begin{align*}
                        P(\text{all BMs are destroyed}) & = 0.96^6\\
                        & = 0.78275 \qedhere
                    \end{align*}
                \end{proof}

                \item
                \begin{proof}[\unskip\nopunct]
                    \begin{align*}
                        P(\text{at least one BM gets through}) & = 1 - P(\text{all BMs are destroyed}) \\
                        & = 1-0.96^6 \\
                        & = 0.21724 \qedhere
                    \end{align*}
                \end{proof}

                \item
                \begin{proof}[\unskip\nopunct]
                    \begin{align*}
                        P(\text{exactly one BM gets through}) & = 6\cdot 0.96^5\cdot 0.04\\
                        & = 0.19568 \qedhere
                    \end{align*}
                \end{proof}

            \end{enumerate}

        % 7
        \item
        \begin{proof}[\unskip\nopunct]
            Given: \\
            $P(\text{qualified}) = q$,\\
            $P(\text{not qualified}) = 1-q$,\\
            $P(\text{correct answer} \given \text{qualified}) = p$ \\
            $P(\text{incorrect answer} \given \text{not qualified}) = p$ \\
            Therefore,
            \begingroup
            \addtolength{\jot}{1em}
            \begin{align*}
                P(\text{>15 correct} \given \text{qualified}) & = \frac{P(\text{>15 correct} \cap \text{qualified})}{P(\text{qualified})} \\
                & = \frac{q \sum_{k=15}^{20} \binom{20}{15}(p)^{15}(1-p)^5}{q\sum_{k=15}^{20} \binom{20}{15}(p)^{15}(1-p)^5 + (1-q)\sum_{k=15}^{20} \binom{20}{15}(p)^{15}(1-p)^5} \qedhere
            \end{align*}
            \endgroup
        \end{proof}
        \vspace{0.2in}

        % 8
        \item Let $C$ represent the 20 random distinct cars chosen for a
        test drive.

            \begin{enumerate}

                \item
                \begin{proof}[\unskip\nopunct]
                    To find $P(K=0 \given C)$, without replacement:
                    \begingroup
                    \addtolength{\jot}{1em}
                    \begin{align*}
                        P(K=0 \given C) & = \frac{P(K=0)P(C \given K=0)}{\sum_{i=0}^{9}P(K=i)P(C \given K=i)} \\
                        & = \frac{\binom{100}{20}}{\sum_{i=0}^{9}\binom{100-i}{20}}\\
                        & \approx 0.227 \qedhere
                    \end{align*}
                    \endgroup
                \end{proof}
                \vspace{0.2in}

                \item
                \begin{proof}[\unskip\nopunct]
                    To find $P(K=0 \given C)$, with replacement:
                    \begingroup
                    \addtolength{\jot}{1em}
                    \begin{align*}
                        P(K=0 \given C) & = \frac{P(K=0)P(C \given K=0)}{\sum_{i=0}^{9}P(K=i)P(C \given K=i)} \\
                        & = \frac{100^{20}}{\sum_{i=0}^{9}(100-i)^{20}}\\
                        & \approx 0.213 \qedhere
                    \end{align*}
                    \endgroup
                \end{proof}
                \vspace{0.2in}

            \end{enumerate}

        % 9
        \item
        \begin{proof}[\unskip\nopunct]
            With 6 colors of jelly beans, without replacement, a negative binomial distribution can be simulated:
            \begin{align*}
                \binom{100+6-1}{6-1} & = \binom{105}{5} \\
                & = 96560646 \qedhere
            \end{align*}
        \end{proof}
        \vspace{0.2in}

        % 10
        \item
        The permutations of a word is given by:
        \begin{equation*}
            \frac{(\text{length of word})!}{(\text{repetitions of A})!(\text{repetitions of B})!\ldots (\text{repetitions of Z})!}
        \end{equation*}
        \begin{enumerate}

            % 10a
            \item
            \begin{proof}[\unskip\nopunct]
                Since there are no repeating characters, the permutation is
                simply:
                \begingroup
                \addtolength{\jot}{1em}
                \begin{align*}
                    \text{permutations} & = length(\text{children})! \\
                    & = 8! \\
                    & = 40320 \qedhere
                \end{align*}
                \endgroup
            \end{proof}
            \vspace{0.2in}

            % 10b
            \item
            \begin{proof}[\unskip\nopunct]
                Since the characters $o$ repeats 2 times, $k$ repeats 2 times,
                and $e$ repeats 3 times:
                \begingroup
                \addtolength{\jot}{1em}
                \begin{align*}
                    \text{permutations} & =
                    \frac{length(\text{bookkeeper})!}{(\text{repetitions of
                    o})!(\text{repetitions of k})!(\text{repetitions of e})!}
                    \\ & = \frac{10!}{2!2!3!} \\ & = 151200 \qedhere
                \end{align*}
                \endgroup
            \end{proof}
            \vspace{0.2in}

        \end{enumerate}

    \end{enumerate}

\end{document}
