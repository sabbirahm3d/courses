%%%%%%%%%%%%%%%%%%%%%%%%%%%%%%%%%%%%%%%%%
% Template
% LaTeX Template
% Version 1.0 (December 8 2014)
%
% This template has been downloaded from:
% http://www.LaTeXTemplates.com
%
% Original author:
% Brandon Fryslie
% With extensive modifications by:
% Vel (vel@latextemplates.com)
%
% License:
% CC BY-NC-SA 3.0 (http://creativecommons.org/licenses/by-nc-sa/3.0/)
%
% Authors:
% Sabbir Ahmed
%
%%%%%%%%%%%%%%%%%%%%%%%%%%%%%%%%%%%%%%%%%

\documentclass[paper=usletter, fontsize=12pt]{article}
%%%%%%%%%%%%%%%%%%%%%%%%%%%%%%%%%%%%%%%%%
% Contract Structural Definitions File Version 1.0 (December 8 2014)
%
% Created by: Vel (vel@latextemplates.com)
% 
% This file has been downloaded from: http://www.LaTeXTemplates.com
%
% License: CC BY-NC-SA 3.0 (http://creativecommons.org/licenses/by-nc-sa/3.0/)
%
%%%%%%%%%%%%%%%%%%%%%%%%%%%%%%%%%%%%%%%%%

\usepackage{geometry} % Required to modify the page layout
\usepackage{multicol}
\usepackage{amsmath}
\usepackage{amssymb}

\usepackage[pdftex]{graphicx}
\usepackage{wrapfig}
\usepackage[font=scriptsize, labelfont=bf]{caption}
\usepackage[utf8]{inputenc} % Required for including letters with accents
\usepackage[T1]{fontenc} % Use 8-bit encoding that has 256 glyphs

\usepackage{avant} % Use the Avantgarde font for headings
\usepackage{courier}
\usepackage{xparse}
\usepackage{xcolor}
\usepackage{listings}  % for code verbatim and console outputs

\setlength{\textwidth}{16cm} % Width of the text on the page
\setlength{\textheight}{23cm} % Height of the text on the page
\setlength{\oddsidemargin}{0cm} % Width of the margin - negative to move text left, positive to move it right
\setlength{\topmargin}{-1.25cm} % Reduce the top margin

\setlength{\parindent}{0mm} % Don't indent paragraphs
\setlength{\parskip}{2.5mm} % Whitespace between paragraphs
\renewcommand{\baselinestretch}{1.5}

\definecolor{green}{rgb}{0.18, 0.55, 0.34}

\graphicspath{ {figures/} }
\captionsetup[table]{skip=10pt}

\lstset{language=C, keywordstyle={\bfseries \color{black}}}

% defines algorithm counter for chapter-level
\newcounter{nalg}[section]

%defines appearance of the algorithm counter
\renewcommand{\thenalg}{\thesection .\arabic{nalg}}

% defines a new caption label as Algorithm x.y
\DeclareCaptionLabelFormat{algocaption}{Algorithm \thenalg}

% defines the algorithm listing environment
\lstnewenvironment{pseudocode}[1][] {
    \refstepcounter{nalg}  % increments algorithm number
    \captionsetup{font=normalsize, labelformat=algocaption, labelsep=colon}
    \lstset{
        breaklines=true,
        mathescape=true,
        numbers=left,
        numberstyle=\scriptsize,
        basicstyle=\footnotesize\ttfamily,
        keywordstyle=\color{black}\bfseries,
        keywords={input, output, return, parallel, function, for, to, in, if,
        else, foreach, while, and, or, new, print},
        xleftmargin=.04\textwidth,
        #1
    }
}{}

\renewcommand{\familydefault}{\sfdefault}  % default font for entire document
 % specifies the document layout and style

\begin{document}

    \documentinfo{\today}{06}
    \vspace{-0.2in}

    \begin{enumerate}

        % 1
        \item
        At his workplace, the first thing Oscar does every morning is to go to
        the supply room and pick up one, two, or three pens with equal
        probability 1/3. If he picks up three pens, he does not return to the
        supply room again that day. If he picks up one or two pens, he will
        make one additional trip to the supply room, where he again will pick
        up one, two, or three pens with equal probability 1/3. (The number of
        pens taken in one trip will not affect the number of pens taken in any
        other trip.)

        \begin{enumerate}

            \item The probability that Oscar gets a total of three pens on any
            particular day.
            \begin{cproof}

                The probability that he gets 3 pens on his first trip
                \begin{equation*}
                    P_1 = \frac{1}{3}
                \end{equation*}

                The probability that he gets 2 pens on his first trip and 1 pen
                on his second
                \begin{equation*}
                    P_2 = \frac{1}{3}\cdot\frac{1}{3}=\frac{1}{9}
                \end{equation*}

                The probability that he gets 1 pen on his first trip and 2 pens
                on his second
                \begin{equation*}
                    P_3 = \frac{1}{3}\cdot\frac{1}{3}=\frac{1}{9}
                \end{equation*}

                The probability that he gets 3 pens on any particular day
                \begin{equation*}
                    P(\text{3 pens}) = \frac{1}{3}+\frac{1}{9}+\frac{1}{9}=\frac{5}{9} \qedhere
                \end{equation*}

            \end{cproof}

            \item The conditional probability that he visited the supply room
            twice on a given day, given that it is a day in which he got a
            total of three pens.
            \begin{cproof}

                The probability that he makes two trips
                \salign{1}
                \begin{align*}
                    P(\text{two trips} \mid \text{3 pens}) & =
                    \frac{P(\text{two trips} \cap \text{3 pens})}{P(\text{3
                    pens})} \\
                    & = \frac{P_2 + P_3}{P(\text{3 pens})} \\
                    & = \frac{\frac{1}{9} + \frac{1}{9}}{\frac{5}{9}} \\
                    & = \frac{2}{5} \qedhere
                \end{align*}
                \endgroup

            \end{cproof}

            \item \textbf{E}$[N]$ and \textbf{E}$[N \mid C]$, where
            \textbf{E}$[N]$ is the unconditional expectation of $N$, the total
            number of pens Oscar gets on any given day, and \textbf{E}$[N \mid
            C]$ is the conditional expectation of $N$ given the event
            $C=\{N>3\}$.
            \begin{cproof}

                \salign{1}
                \begin{align*}
                    \textbf{E}[N] & = 2P(\text{2 pens}) + 3P(\text{3 pens}) + 4P(\text{4 pens}) + 5P(\text{5 pens}) \\
                    & = 2P_2 + 3P(\text{3 pens}) + 4(P_1P_1P_1P_1) + 5(P_1P_1) \\
                    & = 2\cdot\frac{1}{9} + 3\cdot\frac{5}{9} + 4\cdot\bigg(\frac{1}{9}\cdot\frac{1}{9}\bigg) + 5\cdot\bigg(\frac{1}{3}\cdot\frac{1}{3}\bigg)\\
                    & = \frac{30}{9}
                \end{align*}
                \endgroup

                \salign{1}
                \begin{align*}
                    \textbf{E}[N \mid C] & = \textbf{E}[N \mid \{\text{4 pens,
                    5 pens}\}] \\
                    & = 4\cdot\frac{P(\text{4 pens})}{P(\text{4 pens}) + P(\text{5 pens})} + 5\cdot\frac{P(\text{5 pens})}{P(\text{4 pens}) + P(\text{5 pens})} \\
                    & = 4\cdot\frac{\frac{1}{9}+\frac{1}{9}}{\frac{1}{9}+\frac{1}{9} + \frac{1}{3}\cdot\frac{1}{3}} + 5\cdot\frac{\frac{1}{3}\cdot\frac{1}{3}}{\frac{1}{9}+\frac{1}{9} + \frac{1}{3}\cdot\frac{1}{3}} \\
                    & = \frac{13}{3}  \qedhere
                \end{align*}
                \endgroup

            \end{cproof}

            \item $\sigma_{N \mid C}$, the conditional standard deviation of
            the total number of pens Oscar gets on a particular day, where $N$
            and $C$ are as in part (c).
            \begin{cproof}

                Since $\textbf{E}[N \mid C]= 13/3$,\\
                $\textbf{E}[N^2 \mid C]$
                \salign{1}
                \begin{align*}
                    \textbf{E}[N^2 \mid C] & = \textbf{E}[N^2 \mid \{\text{4 pens, 5 pens}\}] \\
                    & = 4^2\cdot\frac{P(\text{4 pens})}{P(\text{4 pens}) + P(\text{5 pens})} + 5^2\cdot\frac{P(\text{5 pens})}{P(\text{4 pens}) + P(\text{5 pens})} \\
                    & = 4^2\cdot\frac{\frac{1}{9}+\frac{1}{9}}{\frac{1}{9}+\frac{1}{9} + \frac{1}{3}\cdot\frac{1}{3}} + 5^2\cdot\frac{\frac{1}{3}\cdot\frac{1}{3}}{\frac{1}{9}+\frac{1}{9} + \frac{1}{3}\cdot\frac{1}{3}} \\
                    & = 19
                \end{align*}
                \endgroup

                Therefore, $\sigma_{N \mid C}$
                \salign{1}
                \begin{align*}
                    \sigma_{N \mid C} &= \sqrt{\textbf{E}[N^2 \mid C]-(\textbf{E}[N^2 \mid C])^2} \\
                    & = \sqrt{19-\bigg(\frac{13}{3}\bigg)^2}\\
                    & = \frac{\sqrt{2}}{3} \qedhere
                \end{align*}
                \endgroup

            \end{cproof}

            \item The probability that he gets more than three pens on each of
            the next 16 days.
            \begin{cproof}
                \salign{1}
                \begin{align*}
                    P(N > 3)^{16} & = \bigg(\frac{1}{3}\bigg)^{16}\\
                    & = \frac{1}{3^{16}} \qedhere
                \end{align*}
                \endgroup
            \end{cproof}

            \item The conditional standard deviation of the total number of
            pens he gets in the next 16 days given that he gets more than three
            pens on each of those days.
            \begin{cproof}

                \salign{1}
                \begin{align*}
                    \sigma_{N_1,N_2,\ldots,N_{16} \mid C} &= \sqrt{16\cdot(\textbf{E}[N^2 \mid C]-(\textbf{E}[N^2 \mid C])^2)} \\
                    & = \sqrt{16\cdot\bigg(19-\bigg(\frac{13}{3}\bigg)^2\bigg)}\\
                    & = \frac{4\sqrt{2}}{3} \qedhere
                \end{align*}
                \endgroup

            \end{cproof}

        \end{enumerate}

        % 2
        \item Your computer has been acting very strangely lately, and you
        suspect that it might have a virus on it. Unfortunately, all 12 of the
        different virus detection programs you own are outdated. You know that
        if your computer does have a virus, each of the programs, independently
        of the others, has a 0.8 chance of believing that your computer as
        infected, and a 0.2 chance of thinking your computer is fine. On the
        other hand, if your computer does not have a virus, each program has a
        0.9 chance of believing that your computer is fine, and a 0.1 chance of
        wrongly thinking your computer is infected. Given that your computer
        has a 0.65 chance of being infected with some virus, and given that you
        will believe your virus protection programs only if 9 or more of them
        agree, find the probability that your detection programs will lead you
        to the right answer.
        \begin{cproof}
        \end{cproof}

        % 3
        \item Joe Lucky plays the lottery on any given week with probability
        $p$, independently of whether he played on any other week. Each time he
        plays, he has a probability $q$ of winning, again independently of
        everything else. During a fixed time period of $n$ weeks, let $X$ be
        the number of weeks that he played the lottery and $Y$ be the number of
        weeks that he won.
        \begin{enumerate}

            \item What is the probability that he played the lottery on any
            particular week, given that he did not win on that week?
            \begin{cproof}

                Let $L_i$ represent the event that Joe played the lottery, \\
                and $W_i$ represent the event that he won on week $i$ \\
                Therefore,
                \salign{1}
                \begin{align*}
                    P(L_i \mid W_i^c) &= \frac{P(W_i^c \mid L_i)P(L_i)}{P(W_i^c \mid L_i)P(L_i) + P(W_i^c \mid L_i^c)P(L_i^c)}\\
                    & = \frac{(1-q)p}{(1-q)p+(1-p)} \qedhere
                \end{align*}
                \endgroup

            \end{cproof}

            \item Find the conditional PMF $p_{Y \mid X}(y \mid x)$.
            \begin{cproof}

                The binomial PMF:
                \salign{1}
                \begin{align*}
                    p_{Y \mid X}(y \mid x) & =
                    \begin{cases}
                        \binom{x}{y}q^y(1-q)^{x-y} &, \text{ if } 0 \le y \le x,\\
                        0 &, \text{ otherwise }
                    \end{cases} \qedhere
                \end{align*}
                \endgroup

            \end{cproof}

            \item Find the joint PMF $p_{X, Y}(x, y)$.
            \begin{cproof}

                Since $p_{X, Y}(x, y) = p_{Y \mid X}(y \mid x)p_X(x)$
                \salign{1}
                \begin{align*}
                    p_{Y \mid X}(y \mid x) & =
                    \begin{cases}
                        \binom{x}{y}q^y(1-q)^{x-y}\binom{n}{x}p^x(1-p)^{n-x} &, \text{ if } 0 \le y \le x \le n,\\
                        0 &, \text{ otherwise }
                    \end{cases} \qedhere
                \end{align*}
                \endgroup

            \end{cproof}

            \item Find the marginal PMF $p_{Y}(y)$. \\ \textit{Hint}: One
            possibility is to start with the answer to part (c), but the
            algebra can be messy. However, if you think intuitively about the
            procedure that generates $Y$, you may be able to guess the answer.
            \begin{cproof}

                Let $Y_i$ be a Bernoulli random variable that takes 1 when Joe wins, and 0 otherwise
                \begin{align*}
                    p_{Y_i}(y) & =
                    \begin{cases}
                        pq &, \text{ if } y = 1,\\
                        1 - pq &, \text{ otherwise }
                    \end{cases}
                \end{align*}
                Therefore, the marginal PMF
                \salign{1}
                \begin{align*}
                    p_{Y}(y) & =
                    \begin{cases}
                        \binom{n}{y}(pq)^y(1-(pq))^{n-y} &, \text{ if } 0 \le y \le n,\\
                        0 &, \text{ otherwise }
                    \end{cases} \qedhere
                \end{align*}
                \endgroup

            \end{cproof}

            \item Find the conditional PMF $p_{X \mid Y}(x \mid y)$. Do this
            algebraically using the preceding answers.
            \begin{cproof}

                Since $p_{X \mid Y}(x \mid y) = \frac{p_{X, Y}(x, y)}{p_Y(y)}$
                \salign{1}
                \begin{align*}
                    p_{X \mid Y}(x \mid y) & =
                    \begin{cases}
                        \frac{\binom{x}{y}q^y(1-q)^{x-y}\binom{n}{x}p^x(1-p)^{n-x}}{\binom{n}{y}(pq)^y(1-(pq))^{n-y}} &, \text{ if } 0 \le y \le x \le n,\\
                        0 &, \text{ otherwise }
                    \end{cases} \qedhere
                \end{align*}
                \endgroup

            \end{cproof}

            \item Rederive the answer to part (e) by thinking as follows: for
            each one of the $n - Y$ weeks that he did not win, the answer to
            part (a) should tell you something.
            \begin{cproof}
            \end{cproof}

        \end{enumerate}

        % 4
        \item The runner-up in a road race is given a reward that depends on
        the difference between his time and the winner's time. He is given 10
        dollars for being one minute behind, 6 dollars for being one to three
        minutes behind, 2 dollars for being 3 to 6 minutes behind, and nothing
        otherwise. Given that the difference between his time and the winner's
        time is uniformly distributed between 0 and 12 minutes, find the mean
        and variance of the reward of the runner-up.
        \begin{cproof}
        \end{cproof}

        % 5
        \item Let $X$ be a random variable with PDF
        \begin{equation*}
            f_X(x) = \begin{cases}
                2x/3 & \text{, if } 1 < x \le 2,\\
                0 & \text{, otherwise, }
            \end{cases}
        \end{equation*}
        and let $Y=X^2$. Calculate \textbf{E}$[Y]$ and $var(Y)$.
        \begin{cproof}

            To compute \textbf{E}$[Y]$
            \salign{1}
            \begin{align*}
                \textbf{E}[Y] & = \textbf{E}[X^2] \\
                & = \int_{1}^{2}x^2\cdot\frac{2x}{3}dx \\
                & = \frac{2}{3}\cdot\frac{x^4}{4}\bigg\vert_1^2\\
                & = \frac{5}{2}
            \end{align*}
            \endgroup

            And
            \salign{1}
            \begin{align*}
                \textbf{E}[Y^2] & = \textbf{E}[X^4] \\
                & = \int_{1}^{2}x^4\cdot\frac{2x}{3}dx \\
                & = \frac{2}{3}\cdot\frac{x^6}{6}\bigg\vert_1^2\\
                & = 7
            \end{align*}
            \endgroup

            Therefore, $var(Y)$
            \salign{1}
            \begin{align*}
                var(Y) & = \textbf{E}[Y^2] - (\textbf{E}[Y])^2 \\
                & = 7 - \bigg(\frac{5}{2}\bigg)^2 \\
                & = \frac{3}{4} \qedhere
            \end{align*}
            \endgroup

        \end{cproof}

        % 6
        \item The pdf of a RV $X$ is shown in Figure 6. The numbers in
        parentheses indicate area. Compute the value of $A$.
        \begin{cproof}
        \end{cproof}

        % 7
        \item Find the PDF, the mean, and the variance of the random variable
        $X$ with CDF
        \salign{1}
        \begin{equation*}
            F_X(x) = \begin{cases}
                1-\frac{a^3}{x^3} & \text{, if } x \ge a,\\
                0 & \text{, if } x < a,
            \end{cases}
        \end{equation*}
        \endgroup
        where $a$ is a positive constant.
        \begin{cproof}
        \end{cproof}

        % 8
        \item You are allowed to take a certain test three times, and your
        final score will be the maximum of the test scores. Your score in test
        $i$, where $i$ = 1, 2, 3, takes one of the values from $i$ to 10 with
        equal probability $1/(11-i)$, independently of the scores in other
        tests. What is the PMF of the final score?
        \begin{cproof}
        \end{cproof}

        % 9
        \item The \textbf{median} of a random variable $X$ is a number $\mu$
        that satisfies $F_X(\mu)=1/2$. Find the median of the exponential
        random variable with parameter $\lambda$.
        \begin{cproof}

            Let $m$ = median
            \salign{1}
            \begin{align*}
                P(X < m) & = \int_{0}^{m}e^{-\lambda x}dx\\
                & = (-\lambda e^{x})\vert_{0}^{m}\\
                \Rightarrow 1 - \lambda e^{-m} & = 0.5\\
                \Rightarrow m & = \ln(2)\lambda \qedhere
            \end{align*}
            \endgroup

        \end{cproof}

    \end{enumerate}

\end{document}
