%%%%%%%%%%%%%%%%%%%%%%%%%%%%%%%%%%%%%%%%%
% Template
% LaTeX Template
% Version 1.0 (December 8 2014)
%
% This template has been downloaded from:
% http://www.LaTeXTemplates.com
%
% Original author:
% Brandon Fryslie
% With extensive modifications by:
% Vel (vel@latextemplates.com)
%
% License:
% CC BY-NC-SA 3.0 (http://creativecommons.org/licenses/by-nc-sa/3.0/)
%
% Authors:
% Sabbir Ahmed
%
%%%%%%%%%%%%%%%%%%%%%%%%%%%%%%%%%%%%%%%%%

\documentclass[paper=usletter, fontsize=12pt]{article}
%%%%%%%%%%%%%%%%%%%%%%%%%%%%%%%%%%%%%%%%%
% Contract Structural Definitions File Version 1.0 (December 8 2014)
%
% Created by: Vel (vel@latextemplates.com)
% 
% This file has been downloaded from: http://www.LaTeXTemplates.com
%
% License: CC BY-NC-SA 3.0 (http://creativecommons.org/licenses/by-nc-sa/3.0/)
%
%%%%%%%%%%%%%%%%%%%%%%%%%%%%%%%%%%%%%%%%%

\usepackage{geometry} % Required to modify the page layout
\usepackage{multicol}
\usepackage{amsmath}
\usepackage{amssymb}

\usepackage[pdftex]{graphicx}
\usepackage{wrapfig}
\usepackage[font=scriptsize, labelfont=bf]{caption}
\usepackage[utf8]{inputenc} % Required for including letters with accents
\usepackage[T1]{fontenc} % Use 8-bit encoding that has 256 glyphs

\usepackage{avant} % Use the Avantgarde font for headings
\usepackage{courier}
\usepackage{xparse}
\usepackage{xcolor}
\usepackage{listings}  % for code verbatim and console outputs

\setlength{\textwidth}{16cm} % Width of the text on the page
\setlength{\textheight}{23cm} % Height of the text on the page
\setlength{\oddsidemargin}{0cm} % Width of the margin - negative to move text left, positive to move it right
\setlength{\topmargin}{-1.25cm} % Reduce the top margin

\setlength{\parindent}{0mm} % Don't indent paragraphs
\setlength{\parskip}{2.5mm} % Whitespace between paragraphs
\renewcommand{\baselinestretch}{1.5}

\definecolor{green}{rgb}{0.18, 0.55, 0.34}

\graphicspath{ {figures/} }
\captionsetup[table]{skip=10pt}

\lstset{language=C, keywordstyle={\bfseries \color{black}}}

% defines algorithm counter for chapter-level
\newcounter{nalg}[section]

%defines appearance of the algorithm counter
\renewcommand{\thenalg}{\thesection .\arabic{nalg}}

% defines a new caption label as Algorithm x.y
\DeclareCaptionLabelFormat{algocaption}{Algorithm \thenalg}

% defines the algorithm listing environment
\lstnewenvironment{pseudocode}[1][] {
    \refstepcounter{nalg}  % increments algorithm number
    \captionsetup{font=normalsize, labelformat=algocaption, labelsep=colon}
    \lstset{
        breaklines=true,
        mathescape=true,
        numbers=left,
        numberstyle=\scriptsize,
        basicstyle=\footnotesize\ttfamily,
        keywordstyle=\color{black}\bfseries,
        keywords={input, output, return, parallel, function, for, to, in, if,
        else, foreach, while, and, or, new, print},
        xleftmargin=.04\textwidth,
        #1
    }
}{}

\renewcommand{\familydefault}{\sfdefault}  % default font for entire document
 % specifies the document layout and style

\begin{document}

    \documentinfo{\today}{08}
    \vspace{-0.2in}

    \begin{itemize}

        \item[\textbf{3.4}]
        \begin{itemize}

            \item[\textbf{4}] Show that $\mathbb{Z}_{5}^{\times}$ is not
            isomorphic to $\mathbb{Z}_{8}^{\times}$ by showing that the first
            group has an element of order 4 but the second group does not
            \begin{cproof}

                The elements in each of the groups
                \begin{align*}
                    \{[1], [2], [3], [4]\} & \in \mathbb{Z}_{5}^{\times}\\
                    \{[1], [3], [5], [7]\} & \in \mathbb{Z}_{8}^{\times}
                \end{align*}

                For $\mathbb{Z}_{5}^{\times}$\\
                \begin{equation*}
                    [5]^2 = [1]
                \end{equation*}
                Therefore, $o()$

            \end{cproof}

            \item[\textbf{7}] Let $G$ be a group. Show that the group $(G, *)$
            defined in Exercise 3 of Section 3. 1 is isomorphic to $G$.
            \begin{cproof}
            \end{cproof}

            \item[\textbf{11}] Let $G$ be the set of all matrices in
            $GL_2(\mathbb{Z}_3)$ of the form $\left[\begin{tabular}{cc}
                    m & b \\
                    0 & 1
                \end{tabular}\right]$. That is, $m, b \in \mathbb{Z}_3$ and
            $m\neq [0]_3$. Show that $G$ is a subgroup of
            $GL_2(\mathbb{Z}_3)$ that is isomorphic to $S_3$.
            \begin{cproof}
            \end{cproof}

            \item[\textbf{14}] Let $G=\{x \in \mathbb{R} \mid x > 0 \text{ and
            } x \neq 1\}$, and define $*$ on $G$ by $a * b = a^{\ln{b}}$. Show
            that $G$ is isomorphic to the multiplicative group
            $\mathbb{R}^{\times}$. (See Exercise 9 of Section 3.1.)
            \begin{cproof}
            \end{cproof}

            \item[\textbf{17}] Let $\phi: G_1 \rightarrow G_2$ be a group
            isomorphism. Prove that if $H$ is asubgroup of $G_1$, then
            $\phi(H)=\{y\in G_2 \mid y = \phi(h) \text{ for some } h \in H\}$
            is a subgroup of $G_2$.
            \begin{cproof}

                Since $\phi:G_1 \rightarrow G_2$ is a group isomorphism,
                $\phi(e_1)=e_2$\\
                Since $H$ is a subgroup,
                \begin{align*}
                    e_1 & \in H\\
                    \Rightarrow e_2 & \in \phi(H)
                \end{align*}

                A non-empty set $G$ is a subgroup if $xy^{-1}\in G$, $\forall \
                x,y\in G$\\
                Let $x,y \in \phi(H)$\\
                Then, there exists $h_1,h_2 \in H$, such that
                \begin{align*}
                    \phi(h_1) & = x\\
                    \phi(h_2) & = y
                \end{align*}
                Also, since $\phi$ is homomorphic,
                \begin{align*}
                    \phi(h_2^{-1}) & = (\phi(h_2))^{-1}\\
                    & = y^{-1}\\
                    \phi(h_1h_2^{-1}) & = \phi(h_1)\phi(h_2^{-1})\\
                    & = xy^{-1}
                \end{align*}

                Since $H$ is a subgroup, $h_1h_2^{-1}\in H$, $\forall \ h_1,h_2
                \in H$\\
                Therefore,
                \begin{align*}
                    \phi(h_1h_2^{-1}) & = xy^{-1}\\
                    & \in \phi(H)
                \end{align*}

                That is, $\phi(h_1h_2^{-1}) \in \phi(H)$, $\forall \ x,y \in
                \phi(H)$ \qedhere

            \end{cproof}

            \item[\textbf{24}] Let $G = \mathbb{R} - {-1}$. Define $*$ on $G$
            by $a*b = a+b+ab$. Show that $G$ is isomorphic to the
            multiplicative group $\mathbb{R}^{\times}$. (See Exercise 13 of
            Section 3.1.) \\ \textit{Hint}: Remember that an isomorphism maps
            identity to identity. Use this fact to help find the necessary
            mapping.
            \begin{cproof}
            \end{cproof}

            \item[\textbf{26}] Let $G_1$ and $G_2$ be groups. A function from
            $G$ into $G_2$ that preserves products but is not necessarily a
            one-to-one correspondence will be called a group homomorphism, from
            the Greek word \textit{homos} meaning same. Show that $\phi:
            \text{GL}_2(\mathbb{R}) \rightarrow \mathbb{R}^{\times}$ defined by
            $\phi(A)=\text{det}(A)$ for all matrices $A \in
            \text{GL}_2(\mathbb{R})$ is a group homomorphism.
            \begin{cproof}
            \end{cproof}

        \end{itemize}

        \item[\textbf{3.5}]

        \begin{itemize}

            \item[\textbf{2}] Let $G$ be a group and let $a \in G$ be an
            element of order 30. List the powers of $a$ that have order 2,
            order 3 or order 5.
            \begin{cproof}
            \end{cproof}

            \item[\textbf{3}] Give the subgroup diagrams of the following
            groups.
            \begin{enumerate}

                \item[\textbf{a}] $\mathbb{Z}_{24}$
                \begin{cproof}
                \end{cproof}

                \item[\textbf{b}] $\mathbb{Z}_{36}$
                \begin{cproof}
                \end{cproof}

            \end{enumerate}

            \item[\textbf{10}] Find all cyclic subgroups of $\mathbb{Z}_{6}
            \times \mathbb{Z}_{3}$
            \begin{cproof}
            \end{cproof}

            \item[\textbf{12}] Let $a,b$ be positive integers, and let
            $d=\gcd(a,b)$ and $m=\text{lcm}(a,b)$. Use Proposition 3.5.5 to
            prove that $\mathbb{Z}_{a}\times \mathbb{Z}_{b} \cong
            \mathbb{Z}_{d} \times \mathbb{Z}_{m}$
            \begin{cproof}
            \end{cproof}

            \item[\textbf{13}] Show that in a finite cyclic group of order $n$,
            the equation $x^m=e$ has exactly $m$ solutions, for each positive
            integer $m$ that is a divisor of $n$.
            \begin{cproof}
            \end{cproof}

            \item[\textbf{17}] Let $G$ be the set of all $3\times 3$ matrices of the form $\left[\begin{tabular}{ccc}
                    1 & b & c \\
                    0 & 1 & c \\
                    0 & 0 & 1
                \end{tabular}\right]$.

            \begin{itemize}

                \item[\textbf{a}] Show that if $a,b,c\in \mathbb{Z}_3$, the $G$
                is a group with exponent 3.
                \begin{cproof}
                \end{cproof}

                \item[\textbf{b}] Show that if $a,b,c\in \mathbb{Z}_2$, the $G$
                is a group with exponent 4.
                \begin{cproof}
                \end{cproof}

            \end{itemize}

            \item[\textbf{19}] Let $n=2^k$ for $k>2$. Prove that
            $\mathbb{Z}_{n}^{\times}$ is not cyclic. \\ \textit{Hint}: Show
            that $\pm 1$ satisfy the equation $x^2=1$, and that this is
            impossible in any cyclic group.
            \begin{cproof}
            \end{cproof}

        \end{itemize}

    \end{itemize}

\end{document}
