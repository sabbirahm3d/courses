%%%%%%%%%%%%%%%%%%%%%%%%%%%%%%%%%%%%%%%%%
%
% Sabbir Ahmed
%
%%%%%%%%%%%%%%%%%%%%%%%%%%%%%%%%%%%%%%%%%

\documentclass[paper=usletter, fontsize=12pt]{article}
%%%%%%%%%%%%%%%%%%%%%%%%%%%%%%%%%%%%%%%%%
% Contract Structural Definitions File Version 1.0 (December 8 2014)
%
% Created by: Vel (vel@latextemplates.com)
% 
% This file has been downloaded from: http://www.LaTeXTemplates.com
%
% License: CC BY-NC-SA 3.0 (http://creativecommons.org/licenses/by-nc-sa/3.0/)
%
%%%%%%%%%%%%%%%%%%%%%%%%%%%%%%%%%%%%%%%%%

\usepackage{geometry} % Required to modify the page layout
\usepackage{multicol}
\usepackage{amsmath}
\usepackage{amssymb}

\usepackage[pdftex]{graphicx}
\usepackage{wrapfig}
\usepackage[font=scriptsize, labelfont=bf]{caption}
\usepackage[utf8]{inputenc} % Required for including letters with accents
\usepackage[T1]{fontenc} % Use 8-bit encoding that has 256 glyphs

\usepackage{avant} % Use the Avantgarde font for headings
\usepackage{courier}
\usepackage{xparse}
\usepackage{xcolor}
\usepackage{listings}  % for code verbatim and console outputs

\setlength{\textwidth}{16cm} % Width of the text on the page
\setlength{\textheight}{23cm} % Height of the text on the page
\setlength{\oddsidemargin}{0cm} % Width of the margin - negative to move text left, positive to move it right
\setlength{\topmargin}{-1.25cm} % Reduce the top margin

\setlength{\parindent}{0mm} % Don't indent paragraphs
\setlength{\parskip}{2.5mm} % Whitespace between paragraphs
\renewcommand{\baselinestretch}{1.5}

\definecolor{green}{rgb}{0.18, 0.55, 0.34}

\graphicspath{ {figures/} }
\captionsetup[table]{skip=10pt}

\lstset{language=C, keywordstyle={\bfseries \color{black}}}

% defines algorithm counter for chapter-level
\newcounter{nalg}[section]

%defines appearance of the algorithm counter
\renewcommand{\thenalg}{\thesection .\arabic{nalg}}

% defines a new caption label as Algorithm x.y
\DeclareCaptionLabelFormat{algocaption}{Algorithm \thenalg}

% defines the algorithm listing environment
\lstnewenvironment{pseudocode}[1][] {
    \refstepcounter{nalg}  % increments algorithm number
    \captionsetup{font=normalsize, labelformat=algocaption, labelsep=colon}
    \lstset{
        breaklines=true,
        mathescape=true,
        numbers=left,
        numberstyle=\scriptsize,
        basicstyle=\footnotesize\ttfamily,
        keywordstyle=\color{black}\bfseries,
        keywords={input, output, return, parallel, function, for, to, in, if,
        else, foreach, while, and, or, new, print},
        xleftmargin=.04\textwidth,
        #1
    }
}{}

\renewcommand{\familydefault}{\sfdefault}  % default font for entire document
 % specifies the document layout and style
\usepackage{polynom}

\begin{document}

    \documentinfo{\today}{13}

    \begin{enumerate}

        \item[\textbf{4.2}]
        \begin{enumerate}

            \item[\textbf{1}] Use the division algorithm to find the quotient
            and remainder when $f(x)$ is divided by $g(x)$ over the field of
            rational numbers $\mathbb{Q}$.
            \begin{enumerate}

                \item[\textbf{c}] $f(x) = x^5+1$, \ \ \ $g(x) = x+1$
                \begin{proof}

                    \polylongdiv{x^5+1}{x+1}

                    Therefore,
                    \begin{align*}
                        f(x) & = g(x)(x^4-x^3+x^2-x+1) + (0)\\
                        & = (x+1)(x^4-x^3+x^2-x+1) + (0) \Mod{\mathbb{Q}} \qedhere
                    \end{align*}

                \end{proof}

            \end{enumerate}

            \item[\textbf{2}] Use the division algorithm to find the quotient
            and remainder when $f(x)$ is divided by $g(x)$ over the indicated
            field.
            \begin{enumerate}

                \item[\textbf{c}] $f(x) = x^5+2x^3+3x^2+x-1$, \ \ \ $g(x) =
                x^2+5$ over $\mathbb{Z}_7$
                \begin{proof}

                    \begin{align*}
                        f(x)&=x^5+2x^3+3x^2+x-1 \\
                        & \equiv x^5+2x^3+3x^2+x+6 \Mod{\mathbb{Z}_7}\\
                        g(x) &= x^2+5\\
                        &\equiv x^2+5 \Mod{\mathbb{Z}_7}
                    \end{align*}

                    \begin{equation*}
                        \begin{array}{*2r @{\hskip\arraycolsep}c@{\hskip\arraycolsep} *7r}
                            &    &&  &&  +x^3  && +4x & +3 \\
                            \cline{3-9}
                            & x^2 + 5 &\big)& +x^5 & +0x^4 & +2x^3 &+3x^2 &+x &+6 \\
                            &    &&  -(+x^5 && +5x^3) \\
                            \cline{4-8}
                            &    &&   &&  +4x^3 & +3x^2 & +x \\
                            &    &&   &&  -(+4x^3 & & +6x)\\
                            \cline{6-8}
                            &    &&    &    & & +3x^2 & +2x \\
                            &    &&    &    & &-( +3x^2 & +2x) \\
                            \cline{7-9}
                            &    &&    &    &  & & & +6\\
                        \end{array}
                    \end{equation*}

                    Therefore,
                    \begin{align*}
                        f(x) & = g(x)(x^3+4x+3) + 6\\
                        & = (x^2+5)(x^3+4x+3) + 6 \Mod{\mathbb{Z}_7} \qedhere
                    \end{align*}

                \end{proof}

            \end{enumerate}

            \item[\textbf{3}] Find the greatest common divisor of $f(x)$ and
            $f^\prime$, over $\mathbb{Q}$.
            \begin{enumerate}

                \item[\textbf{d}] $f(x) = x^4+2x^3+3x^2+2x+1$
                \begin{proof}

                    Given $f(x) = x^4+2x^3+3x^2+2x+1$\\
                    and $f^\prime = 4x^3+6x^2+6x+2$\\
                    And, $\frac{f(x)}{f^\prime}$,\\
                    \begin{center}

                        \polylongdiv{x^4+2x^3+3x^2+2x+1}{4x^3+6x^2+6x+2}\\

                    \end{center}

                    Multiplying the remainder with a non-zero constant keeps it
                    unchanged, and therefore,
                    \begin{align*}
                        \text{remainder} &= \bigg(\frac{3}{4}x^2+\frac{3}{4}x+\frac{3}{4}\bigg)\frac{4}{3}\\
                        & = x^2+x+1
                    \end{align*}
                    Thus,
                    \begin{align*}
                        \gcd(x^4+2x^3+3x^2+2x+1,4x^3+6x^2+6x+2) \\ = \gcd(4x^3+6x^2+6x+2,x^2+x+1)
                    \end{align*}
                    Dividing as before,\\
                    \begin{center}

                        \polylongdiv{4x^3+6x^2+6x+2}{x^2+x+1}\\

                    \end{center}
                    Therefore,
                    \begin{equation*}
                        \gcd(x^4+2x^3+3x^2+2x+1,4x^3+6x^2+6x+2) = x^2+x+1 \qedhere
                    \end{equation*}

                \end{proof}

            \end{enumerate}

            \item[\textbf{5}] Find the greatest common divisor of the given
            polynomials, over the given field.
            \begin{enumerate}

                \item[\textbf{c}] $x^5+4x^4+6x^3+6x^2+5x+2$ and $x^4+3x^2+3x+6$
                over $\mathbb{Z}_7$
                \begin{proof}

                    Doing long division until remainder is $0$,\\
                    \begin{equation*}
                        \begin{array}{*2r @{\hskip\arraycolsep}c@{\hskip\arraycolsep} *7r}
                            &    &&    x & +4 \\
                            \cline{3-9}
                            & x^4+3x^2+3x+6 &\big)& +x^5 &+4x^4 &+6x^3 &+6x^2 &+5x &+2 \\
                            &    &&  -(x^5 &&+3x^3 &+3x^2 &+6x) \\
                            \cline{4-8}
                            &    &&   &  +4x^4 &+3x^3 &+3x^2 &+x \\
                            &    &&   &  -(4x^4&& +5x^2& +5x& +3) \\
                            \cline{5-9}
                            &    &&    &    & +3x^3 & +5x^2& +3x& +6 \\
                        \end{array}
                    \end{equation*}
                    \begin{equation*}
                        \begin{array}{*2r @{\hskip\arraycolsep}c@{\hskip\arraycolsep} *5r}
                            &    &&    5x & +1 \\
                            \cline{3-8}
                            & 3x^3+5x^2+3x+6 &\big)& +x^4 &&+3x^2 &+3x &+6 \\
                            &    &&  -(x^4 &+4x^3 &+x^2 &+2x) \\
                            \cline{4-8}
                            &    &&   &  +3x^3 &+2x^2 &+x &+6 \\
                            &    &&   &  -(3x^3 &+5x^2 &+3x &+6) \\
                            \cline{5-8}
                            &    &&    &    & +4x^2 & +5x \\
                        \end{array}
                    \end{equation*}
                    \begin{equation*}
                        \begin{array}{*2r @{\hskip\arraycolsep}c@{\hskip\arraycolsep} *5r}
                            &    &&    6x & +6 \\
                            \cline{3-7}
                            & 4x^2 +5x &\big)& +3x^3 &+5x^2 &+3x &+6 \\
                            &    &&  -(3x^3 &+2x^2) \\
                            \cline{4-7}
                            &    &&   &  +3x^2 &+3x &+6 \\
                            &    &&   &  -(3x^2 &+2x) \\
                            \cline{5-7}
                            &    &&    &    & +x & +6 \\
                        \end{array}
                    \end{equation*}
                    \begin{equation*}
                        \begin{array}{*2r @{\hskip\arraycolsep}c@{\hskip\arraycolsep} *4r}
                            &    &&    4x & +2 \\
                            \cline{3-5}
                            & x+6 &\big)& +4x^2 &+5x \\
                            &    &&  -(4x^2 &+3x) \\
                            \cline{4-5}
                            &    &&   &  +2x \\
                            &    &&   &  -(2x &+5) \\
                            \cline{5-6}
                            &    &&    &    & +2 \\
                        \end{array}
                    \end{equation*}
                    \begin{equation*}
                        \begin{array}{*2r @{\hskip\arraycolsep}c@{\hskip\arraycolsep} *3r}
                            &    &&    4x & +3 \\
                            \cline{3-5}
                            & 2 &\big)& +x &+6 \\
                            &    &&  -(x) \\
                            \cline{4-5}
                            &    &&   &  +6 \\
                            &    &&   &  -(6) \\
                            \cline{4-5}
                            &    &&    & 0 \\
                        \end{array}
                    \end{equation*}

                    Therefore,
                    $\gcd(x^5+4x^4+6x^3+6x^2+5x+2,x^4+3x^2+3x+6)=2 \in
                    \mathbb{Z}_7$ \qedhere

                \end{proof}

            \end{enumerate}

            \item[\textbf{9}] Let $a \in \mathbb{R}$, and let $f(x)\in
            \mathbb{R}[x]$, with derivative $f^\prime(x)$. Show that the
            remainder when $f(x)$ is divided by $(x-a)^2$ is
            $f^\prime(a)(x-a)+f(a)$.
            \begin{proof}

                By the division algorithm, there exists unique polynomials
                $q(x), r(x) \in F[x]$,\\ such that $f(x) = q(x)(x-a)^2+r(x)$,
                where $\deg(x) < 2$\\

                Let $r(x) = bx + c$\\
                Then, $f(a) = 0 + r(a) = ba + c$\\
                Deriving,
                \begin{equation*}
                    f^\prime(x)=(q^\prime(x)(x-a)+2q(x))(x-a)+b
                \end{equation*}
                So, $f^\prime(a)=b$\\
                Also,
                \begin{align*}
                    c &= f(a) - ba\\
                    & = f(a) - f^\prime(a)a
                \end{align*}
                Therefore,
                \begin{align*}
                    r(x) &= f^\prime(a)x + f(a) - f(a)a \\
                    & = f^\prime(a)(x - a) + f(a) \qedhere
                \end{align*}

            \end{proof}

            \item[\textbf{11}] Find the irreducible factors of $x^6-1$ over
            $\mathbb{R}$.
            \begin{proof}

                Factoring $x^6-1$,
                \begin{align*}
                    x^6-1 & = (x^3)^2-1^2\\
                    & = (x^3-1)(x^3+1)\\
                    & = (x^3-1^3)(x^3+1^3)\\
                    & = (x^3-1^3)(x+1)(x^2-x+1)\\
                    & = (x^3-1^3)(x+1)(x^2-x+1)\\
                    & = (x-1)(x^2+x+1)(x+1)(x^2-x+1)
                \end{align*}
                Factors of degree $1$, $(x-1)$ and $(x+1)$ are irreducible\\
                Also, both factors $(x^2+x+1)$ and $(x^2-x+1)$ have no roots in $\mathbb{R}$ since their discriminant $(b^2-4ac)$ are less than zero.\\
                Therefore, all factors of $x^6-1$; $(x-1)$,$(x^2+x+1)$,$(x+1)$,
                and $(x^2-x+1)$ are irreducible over $\mathbb{R}$ \qedhere

            \end{proof}

            \item[\textbf{18}] Compute the following products.
            \begin{enumerate}

                \item[\textbf{b}] $(a+bx)(c+dx) \equiv \text{???}
                \Mod{x^2-2}$ over $\mathbb{Q}$.
                \begin{proof}

                    Since $x^2 \equiv 2 \Mod{x^2-2}$
                    \begin{align*}
                        (a+bx)(c+dx) & = ac+adx+cbx+bdx^2\\
                        & = ac+adx+cbx+2bd \Mod{x^2-2}\\
                        & = (ac+2bd) + (ad+cb)x \Mod{x^2-2} \qedhere
                    \end{align*}

                \end{proof}

            \end{enumerate}

        \end{enumerate}

    \end{enumerate}

\end{document}
